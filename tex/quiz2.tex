\input preamble

%\input ../../preamble

\usepackage{amsmath,amssymb,amsthm,graphicx,color}
\newcommand{\pop}{\mathrm{Pop}}
\newcommand{\train}{\mathrm{Train}}
\DeclareMathOperator*{\argmin}{argmin}
\DeclareMathOperator*{\argmax}{argmax}
\newcommand{\tuple}[1]{{\mbox{$\langle#1\rangle$}}}

\parindent = 0em

%\newcommand{\solution}[1]{}
\newcommand{\solution}[1]{\bigskip {\color{red} {\bf Solution:} #1}}

\begin{document}


\centerline{\bf TTIC 31230 Fundamentals of Deep Learning}
\centerline{\bf Quiz 2}

\bigskip
{\bf Problem 1: Running Averages.}  Consider a sequence of vectors $x_0$, $x_1$, $\ldots$ and two running averages $y_t$ and $z_t$ defined by
as follows for $0 < \beta < 1$ and $\gamma > 0$.
\begin{eqnarray*}
  y_0 & = & 0 \\
  y_{t+1} & = & \beta y_t + (1-\beta) x_t
  \\
  \\
  z_0 & = & 0 \\
  z_{t+1} & = & \beta z_t + \gamma x_t
\end{eqnarray*}

(a) Suppose that the values $x_t$ are drawn IID from a distribution with mean vector $\overline{x} = E\;x_t$.  Give values for
$$\overline{y} = \lim_{t \rightarrow \infty} E \;y_t$$
and
$$\overline{z} = \lim_{t \rightarrow \infty} E \;z_t$$
as functions of $\beta$, $\gamma$ and $\overline{x}$

Hint: Solve for $E\;y_{t+1}$ as a function of $E\;y_t$ and assume that a limiting expectation exists.

\solution{
  \begin{eqnarray*}
   E\; y_{t+1} & = & \beta \;E\;y_t + (1-\beta) \; E\;x_t \\
   \overline{y} & = & \beta \;\overline{y} + (1-\beta)\; \overline{x} \\
   (1-\beta)\;\overline{y} & = & (1-\beta)\;\overline{x} \\
   \overline{y} & = & \overline{x}
  \end{eqnarray*}

  \begin{eqnarray*}
   E\; z_{t+1} & = & \beta \;E\;z_t + \gamma\; E\;x_t \\
   \overline{z} & = & \beta \;\overline{z} + \gamma\; \overline{x} \\
   (1-\beta)\;\overline{z} & = & \gamma\;\overline{x} \\
   \overline{z} & = & \frac{\gamma}{1-\beta}\;\overline{x}
  \end{eqnarray*}

}


(b) Express $z_t$ as a function of $y_t$, $\beta$ and $\gamma$.

\solution{
  \begin{eqnarray*}
    z_{t+1} & = & \beta \;z_t + \gamma\;x_t \\
    & = & \sum_{t'= 0}^t \gamma \beta^{t-t'} x_{t'} \\
    & = & \frac{\gamma}{1-\beta}\;\sum_{t'=0}^t (1-\beta)\beta^{t-t'} x_t \\
    & = & \frac{\gamma}{1-\beta}\;y_{t+1}
  \end{eqnarray*}
}


\bigskip

{\bf Problem 2. Adaptive SGD.}  This problem considers the question of whether the convergence theorem hold for adaptive methods ---
in the limit as the learning rate goes to zero do adaptive methods converge to a local minimum of the loss.

Consider a generalization of RMSProp where we allow an arbitrary adaptation with with different learning rates for
ifferent parameter values.  More specifically consider the SGD update equation

$$(1)\;\;\;\;\Phi_{t+1} = \Phi_t - \eta\left(A(\Phi_t,x_t,y_t)\odot \nabla_\Phi {\cal L}(\Phi_t,x_t,y_t)\right)$$

where $\tuple{x_t,y_t}$ is the $t$th training pair, $A(\Phi_t,x_t,y_t)$ is an adaptation vector, and $\odot$ is the Haddamard product $(x \odot y)[i] = x[i]\;y[i]$.

Consider the special case given by
\begin{eqnarray*}
  A(\Phi,x,y)[i] & = & \frac{1}{\sqrt{s(\Phi,x,y)} + \epsilon} \\
  \\
  s(\Phi,x,y) & = & \frac{1}{d} ||\nabla_\Phi\;{\cal L}(\Phi,x,y)||^2 \\
\end{eqnarray*}
where $d$ is the dimension of $\Phi$.
\medskip

(a) For the given interpretation of $A(\Phi,x,y)$, let $\Phi^*$ be a parameter setting that is a stationary point of the update equation (1)
in the sense that expected update over a random draw from the population is zero.  Write this stationary condition
on $\Phi^*$ explicitly as an expectation equaling zero under the given interpretation of $A(\Phi,x,y)$.

\solution{
$$E_{\tuple{x,y} \sim \pop}\;\frac{1}{\sqrt{s(\Phi^*,x,y)+ \epsilon}}\;\nabla_\Phi\;{\cal L}(\Phi,x,y) = 0$$
}

\medskip
(b) Is $\Phi^*$ as defined in part (b) a stationary point of the original loss --- a point where the expected gradient of ${\cal L}(\Phi^*,x,y)$ equal to zero?

\solution{
  No, the average a weighted sum is different from the average of an unweighted sum
  and hence the fact that the weighted average is zero does not imply that the average is zero.
  }
  
\medskip
(c) Do these observations have implications for the adaptive methods described in this class.  Explain your answer.

\solution{Yes, the example considered here is just a special case of RMSProp or Adam which are in fact not guaranteed to converge to a stationary point (or local optimum) of the loss function.}
    

\end{document}
