\input preamble


\begin{document}

{\Huge

  \centerline{\bf TTIC 31230, Fundamentals of Deep Learning}
  \bigskip
  \centerline{David McAllester, Winter 2020}

  \vfill
  \centerline{\bf Representing Functions with Shallow Circuits}
  \vfill
  \centerline{\bf The Classical Universality Theorems}
  \vfill
  \vfill

\slide{Function Representations}

Consider continuous functions $f: [0,1]^N \rightarrow \reals$

\vfill
\vfill
\centerline{\includegraphics[width = 2in]{../images/n-cube} 
\begin{minipage}[b]{2.0in}
  $\stackrel{f}{\rightarrow} \;\;\;\reals$ \newline
  \vspace{2ex}
\end{minipage}}

\vfill
Given the corner values, the interior can be filled.

$$f(x_1,\ldots,x_N) = E_{y_1,\ldots,y_N \sim \mathrm{Round}(x_1,\ldots,x_N)}\;f(y_1,\ldots,y_n)$$

\vfill
Hence each of the $2^N$ corners has an independent value.

\slide{The Kolmogorov-Arnold representation theorem (1956)}

For continuous $f: [0,1]^N \rightarrow \reals$ there exists continuous
``activation functions'' $\sigma_i: \mathbb{R} \rightarrow \mathbb{R}$ and continuous $w_{i,j} : \reals \rightarrow \reals$ such that


\vfill
$$f(x_1,\;\ldots,\;x_N)=\sum _{{i=1}}^{{2N+1}} \sigma_i \left(\sum_{j=1}^N\;w_{i,j}(x_j)\right)$$



\slide{A Simpler, Similar Theorem}

For (possibly discontinuous) $f: [0,1]^N \rightarrow \reals$ there exists (possibly discontinuous)
$\sigma, w_i: \reals \rightarrow \reals$.

\vfill
$$f(x_1,\;\ldots,\;x_N) = \sigma\left(\sum_i w_i(x_i)\right)$$

\vfill
Proof: Select $w_i$ to spread out the digits of its argument so that $\sum_i w_i(x_i)$ contains all the digits of all the $x_i$.

\slide{Cybenko's Universal Approximation Theorem (1989)}

For continuous $f: [0,1]^N \rightarrow \reals$ and $\varepsilon >0$ there exists

\vfill
\begin{eqnarray*}
  F(x) &= & \alpha^\top \sigma(Wx + \beta) \\
  \\
  & = & \sum_i \alpha_i \sigma\left(\sum_j W_{i,j} \;x_j + \beta_i\right)
\end{eqnarray*}


\vfill
such that for all $x$ in $[0,1]^N$ we have $| F( x ) - f ( x ) | < \varepsilon$.

\slide{How Many Hidden Units?}

Consider Boolean functions $f:\;\{0,1\}^N \rightarrow \{0,1\}$.

\vfill
For Boolean functions we can simply list the inputs $x^0,\;\ldots,\;x^k$ where the function is true.

\begin{eqnarray*}
  f(x) & = & \sum_k \mathbf{1}[x=x^k] \\
  \\
  \mathbf{1}[x = x^k] & \approx & \sigma\left(\sum_i W_{k,i} x_i + b_k\right)
\end{eqnarray*}

\vfill
A simpler statement is that any Boolean function can be put in disjunctive normal form.

\slide{Representing Functions as IO Tables}

These universality theorems implicitly treat functions as tables of intput-output pairs.

\vfill
\centerline{\includegraphics[height = 1in]{../images/Frown}}

\slide{END}

}
\end{document}
