\input preamble


\begin{document}

{\Huge

  \centerline{\bf TTIC 31230, Fundamentals of Deep Learning}
  \bigskip
  \centerline{David McAllester, Winter 2020}

  \vfill
  \centerline{\bf Representing Functions with Deep Circuits}
  \vfill
  \centerline{\bf Circuit Complexity Theory}
  \vfill
  \vfill

\slide{Representing Functions by Circuits}

We can define functions of Boolean variables (the corners of $[0,1]^d$) with Boolean circuits or linear threshold circuits.

\vfill
We can define functions of $[0,1]^d$ with feed-forward real-valued networks (Deep models).

\slide{Circuit Complexity Theory}

Building on work of Ajtai, Sipser and others, Hastad proved (1987) that any bounded-depth Boolean circuit computing the parity function must have exponential size. 

\vfill
Matus Telgarsky recently gave some formal conditions under which shallow networks provably require exponentially more parameters than deeper networks (COLT 2016).


\slideplain{The Kaurnaugh Model of DNNs}

The Karnaugh map, also known as the K-map, is a method to simplify boolean algebra expressions.

\vfil
\centerline{\includegraphics[width = 1.5in]{../images/Kmap1} \hspace{1.0in} \includegraphics[width=3.0in]{../images/Kmap2}}

\begin{eqnarray*}
  F(A,B,C,D) & = & AC' + AB' + BCD' + AD' \\
  & = & (A+B)(A+C)(B' + C' + D')(A+D')
\end{eqnarray*}

\slideplain{The Kaurnaugh Model of DNNs}

\vfil
\centerline{\includegraphics[width = 1.5in]{../images/Kmap1} \hspace{1.0in} \includegraphics[width=3.0in]{../images/Kmap2}}

Many very different circuits compute the same function.

\slideplain{A Kaurnaugh Person Detector}

{\color{red}
  
\centerline{Wheel or Face}

\vfill
\centerline{Hand or Flower \hspace{2.5in} Hand or Flower}

\vfill
\centerline{Leg or Tree  \includegraphics[height=1.5in]{../images/StickFig} Leg or Tree}
}

\vfill
The set of locally minimal models (circuits) could be vast (exponential) without damaging performance.

\vfill
Is a Boolean circuit a distributed representation?

\slide{END}

}
\end{document}
