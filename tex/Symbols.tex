\input preamble



\begin{document}

{\Huge

  \centerline{\bf TTIC 31230, Fundamentals of Deep Learning}
  \bigskip
  \centerline{David McAllester, Autumn 2020}
  \vfill
  \vfil
  \centerline{The Case for Symbols}
  \vfill
  \vfill
  
\slide{Vector Quantization (Emergent Symbols)}

Vector quantization represents a distribution (or density) on vectors with a discrete set of embedded symbols.

\vfill
Vector quantization optimizes a rate-distortion tradeoff for vector compression.

\vfill
The VQ-VAE uses vector quantization to construct a discrete representation of images and hence a measurable image compression rate-distortion trade-off.

\slide{Symbols: A Better Learning Bias}

Do the objects of reality fall into categories?

\vfill
If so, shouldn't a learning architecture be designed to categorize?

\vfill
Whole image symbols would yield emergent whole image classification.

\slide{Symbols: Improved Interpretability}

Vector quantization shifts interpretation from linear threshold units to the emergent symbols.

\vfill
This seems related to the use of t-SNE as a tool in interpretation.


\slide{Symbols: Unifying Vision and Language}

Modern language models use word vectors.

\vfill
Word vectors are embedded symbols.

\vfill
Vector quantization also results in models based on embedded symbols.

\slide{Symbols: Addressing the ``Forgetting'' Problem}
When we learn to ski we do not forget how to ride a bicycle.

\vfill
However, when a model is trained on a first task, retraining on a second tasks degrades performance on the first (the model ``forgets'').

\vfill
But embedded symbols can be task specific.

\vfill
The embedding of a task-specific symbol will not change when training on a different task.


\slide{Symbols: Improved Transfer Learning.}

Embedded symbols can be domain specific.

\vfill
Separating domain-general parameters from domain-specific parameters may improve transfer between domains.

\slide{END}

\end{document}
