\input preamble


\begin{document}

{\Huge

  \centerline{\bf TTIC 31230, Fundamentals of Deep Learning}
  \bigskip
  \centerline{David McAllester, Autumn 2020}
  \vfill
  \centerline{\bf The REINFORCE Algorithm}
  \vfill
  \vfill  


\slide{The REINFORCE Algorithm}

\centerline{\bf Williams, 1992}

\vfill
REINFORCE is a Policy Gradient Algorithm

\vfill
We assume a parameterized policy $\pi_\Phi(a|s)$.

\vfill
$\pi_\Phi(a|s)$ is normalized while $Q_\Phi(s,a)$ is not.

\slide{Policy Gradient Theorem (Episodic Case)}

$$\Phi^* = \argmax_\Phi\; E_{\pi_\Phi} \; R$$

{\huge
\begin{eqnarray*}
  \nabla_\Phi \;E_{\pi_\Phi}\;R & = & \sum_{s_0,a_0,s_1,a_1,\ldots,s_T,a_T}\;\;\nabla_\Phi P(s_0,a_0,s_1,a_1,\ldots,s_T,a_T)\;R \\
  \\
  \nabla_\Phi \;P(\ldots)R & = & P(S_0){\color{red} \nabla_\Phi \;\pi(a_0)}P(s_1)\pi(a_1) \cdots P(s_T)\pi(a_T)\;R \\
  & & + P(S_0)\pi(a_0)P(s_1){\color{red} \nabla_\Phi\;\pi(a_1)} \cdots P(s_T)\pi(a_T)\;R \\
  & & \vdots \\
  & & + P(S_0)\pi(a_0)P(s_1)\pi(a_1) \cdots P(s_T){\color{red} \nabla_\Phi\;\pi(a_T)}\;R \\
  \\
  & = & P(\ldots) \left(\sum_t\;\frac{\nabla_\Phi\;\pi_\Phi(a_t)}{\pi_\Phi(a_t)}\right) R
\end{eqnarray*}
}


\slide{Policy Gradient Theorem (Episodic Case)}
\begin{eqnarray*}
  \nabla_\Phi \;P(\ldots)R  & = & P(\ldots) \left(\sum_t\;\frac{\nabla_\Phi\;\pi_\Phi(a_t|s_t)}{\pi_\Phi(a_t|s_t)}\right) R \\
  \\
  \nabla_\Phi \;E_{\pi_\Phi}\;R & = & E_{\pi_\Phi}\;\left(\sum_t\;\nabla_\Phi\;\ln \pi_\Phi(a_t|s_t)\right) R
\end{eqnarray*}

\slide{Policy Gradient Theorem}
\begin{eqnarray*}
 & &   \nabla_\Phi \; E_{\pi_\Phi}\; R \\
  \\
  & = & E_{\pi_\Phi}\;\left(\sum_t\;\nabla_\Phi\;\ln \pi_\Phi(a_t|s_t)\right)\;\;R \\
  \\
  & = & E_{\pi_\Phi}\; \left(\sum_t\;\nabla_\Phi\;\ln \pi_\Phi(a_t|s_t)\right)\left(\sum_{t} r_{t}\right)  \\
  \\
  & = & E_{\pi_\Phi}\; \sum_{t,t'} \nabla_\Phi\;\ln \pi_\Phi(a_{t}|s_{t}) \;\;r_{t'}
\end{eqnarray*}


\slide{Policy Gradient Theorem}

\begin{eqnarray*}
    \nabla_\Phi \; E_{\pi_\Phi}\; R  & = & \sum_{t,t'}\;E_{s_t,a_t,r_{t'}}\;\; \nabla_\Phi\;\ln \pi_\Phi(a_{t}|s_{t})\;\;r_{t'}
    \end{eqnarray*}
For $t' < t$ we have
{\huge
\begin{eqnarray*}
    E_{r_{t'},s_t,a_t}\;\; r_{t'} \nabla_\Phi\;\ln \pi_\Phi(a_t|s_t)  & =  & E_{r_{t'},s_t}\; r_{t'}  \;\sum_{a_t} \; \pi_\Phi(a_t|s_t)\; \nabla_\Phi\;\ln \pi_\Phi(a_t|s_t) \\
    \\
    & =  & E_{r_{t'},s_t}\; r_{t'}  \;\sum_{a_t}\; \nabla_\Phi\; \pi_\Phi(a_t|s_t) \\
    \\
    & =  & E_{r_{t'},s_t}\; r_{t'}  \;\nabla_\Phi\; \sum_{a_t}\;\pi_\Phi(a_t|s_t) \\
    \\
    & = & 0
\end{eqnarray*}
}

\slide{REINFORCE}
$$\nabla_\Phi \;E_{\pi_\Phi}\;R \;\;\; = \;\;\; E_{\pi_\Phi}\; \sum_{t,\;t' \geq t} \; \left(\nabla_\Phi\;\ln \pi_\Phi(a_t|s_t)\right) \;r_{t'}$$

\vfill
Sampling runs and computing the above sum over $t$ and $t'$ is Williams' REINFORCE algorithm.

\slidetwo{Optimizing Discrete Decisions}{with Non-Differentiable Loss}

The REINFORCE algorithm is used generally for non-differentiable loss functions.

\vfill
For example error rate and BLEU score are non-differentiable --- they are defined on the result of discrete decisions.

\vfill
$$\Phi^* = \argmax_\Phi \;E_{w_1,\ldots,w_n \sim P_\Phi}\;\mathrm{BLEU}$$



\slide{END}

}
\end{document}


