\input preamble

\parindent = 0em

%\newcommand{\solution}[1]{}
\newcommand{\solution}[1]{\bigskip {\color{red} {\bf Solution}: #1}}

\begin{document}


\centerline{\bf TTIC 31230 Fundamentals of Deep Learning}
\bigskip
\centerline{\bf AlphaZero Problems.}

\bigskip
\bigskip
{\bf Backgound.}
A version of AlphaZero can be defined as follows.

\medskip
To select a move in the game, first construct a search tree over possible moves to evaluate options.

\medskip
The tree is grown by running ``simulations''.  Each simulation descends into the tree from the root selecting a move from each position
until the selected move has not been explored before.  When the simulation reaches an unexplored move it expands the tree by adding a node for that move.
Each simulation returns the value $V_\Phi(s)$ for the newly added node $s$.

\medskip
Each node in the search tree represents a board position $s$ and stores the following information which can be initialized
by running the value and policy networks on position $s$.

\begin{itemize}
\item $V_\Phi(s)$ --- the value network value for the position $s$.
\item For each legal move $a$ from $s$, the policy network probability $\pi_\Phi(s,a)$.
\item For each legal move $a$ from $s$, the number $N(s,a)$ of simulations that have taken move $a$ from $s$. This is initially zero.
\item For each legal move $a$ from $s$ with $N(s,a) > 0$, the average $\hat{\mu}(s,a)$ of the values of the simulations that have
  taken move $a$ from position $s$.
\end{itemize}

\medskip
In descending into the tree, simulations select the move $\argmax_a U(s,a)$ where we have
$$U(s,a) =  \left\{\begin{array}{ll}\lambda_u \pi_\Phi(s,a) &\mbox{if $N(s,a) = 0$} \\ \hat{\mu}(s,a) + \lambda_u \pi_\Phi(s,a)/N(s,a) & \mbox{otherwise} \end{array}\right.\;\;\;\;\;(1)$$

\medskip
When the search is completed, we must select a move from the root position.  For this we use a post-search stochastic policy
$$\pi_{s_{\mathrm{root}}}(a) \propto N(s_{\mathrm{root}},a)^\beta\;\;\;\;\;(2)$$
where $\beta$ is a temperature hyperparameter.

\medskip
For training we construct

\medskip
\centerline{a replay buffer of triples $(s_{\mathrm{root}},\pi_{s_{\mathrm{root}}},z)$ \hspace{2em}$(3)$}

\medskip
accumulated from self play where $s_{\mathrm{root}}$ is a root position from a search during a game,
$\pi_{s_{\mathrm{root}}}$ is the post-search policy constructed for $s_{\mathrm{root}}$, and $z$ is the outcome of that game.

\medskip
Training is then done by SGD on the following objective function.

$$\Phi^*\; = \;\argmin_\Phi \;E_{(s,\pi,z) \sim \mathrm{Replay},\;a \sim \pi}\;\left(\begin{array}{l} (V_\Phi(s) - z)^2 \\ \\ - \lambda_1\log \pi_\Phi(a|s) \\ \\ + \lambda_2||\Phi||^2\end{array}\right)\;\;\;\;(4)$$

  
~{\bf Problem 1. AlphaZero for BLEU Translation Score.}

A version of AlphaZero can be defined as follows.

\medskip
To select a move in the game, first construct a search tree over possible moves to evaluate options.

\medskip
The tree is grown by running ``simulations''.  Each simulation descends into the tree from the root selecting a move from each position
until the selected move has not been explored before.  When the simulation reaches an unexplored move it expands the tree by adding a node for that move.
Each simulation returns the value $V_\Phi(s)$ for the newly added node $s$.

\medskip
Each node in the search tree represents a board position $s$ and stores the following information which can be initialized
by running the value and policy networks on position $s$.

\begin{itemize}
\item $V_\Phi(s)$ --- the value network value for the position $s$.
\item For each legal move $a$ from $s$, the policy network probability $\pi_\Phi(s,a)$.
\item For each legal move $a$ from $s$, the number $N(s,a)$ of simulations that have taken move $a$ from $s$. This is initially zero.
\item For each legal move $a$ from $s$ with $N(s,a) > 0$, the average $\hat{\mu}(s,a)$ of the values of the simulations that have
  taken move $a$ from position $s$.
\end{itemize}

\medskip
In descending into the tree, simulations select the move $\argmax_a U(s,a)$ where we have
$$U(s,a) =  \left\{\begin{array}{ll}\lambda_u \pi_\Phi(s,a) &\mbox{if $N(s,a) = 0$} \\ \hat{\mu}(s,a) + \lambda_u \pi_\Phi(s,a)/N(s,a) & \mbox{otherwise} \end{array}\right.\;\;\;\;\;(1)$$

\medskip
When the search is completed, we must select a move from the root position.  For this we use a post-search stochastic policy
$$\pi_{s_{\mathrm{root}}}(a) \propto N(s_{\mathrm{root}},a)^\beta\;\;\;\;\;(2)$$
where $\beta$ is a temperature hyperparameter.

\medskip
For training we construct

\medskip
\centerline{a replay buffer of triples $(s_{\mathrm{root}},\pi_{s_{\mathrm{root}}},z)$ \hspace{2em}$(3)$}

\medskip
accumulated from self play where $s_{\mathrm{root}}$ is a root position from a search during a game,
$\pi_{s_{\mathrm{root}}}$ is the post-search policy constructed for $s_{\mathrm{root}}$, and $z$ is the outcome of that game.

\medskip
Training is then done by SGD on the following objective function.

$$\Phi^*\; = \;\argmin_\Phi \;E_{(s,\pi,z) \sim \mathrm{Replay},\;a \sim \pi}\;\left(\begin{array}{l} (V_\Phi(s) - z)^2 \\ \\ - \lambda_1\log \pi_\Phi(a|s) \\ \\ + \lambda_2||\Phi||^2\end{array}\right)\;\;\;\;(4)$$

\bigskip


Reformulate this algorithm to optimizing BLEU score in machine translation. More specifically,

\medskip
(a) Define the optimization of the BLEU score as a tree search problem.  What are the nodes and what are the moves?

\solution{
    The tree is defined by sequentially choosing words. Each node correspond to a sequences of moves --- a node at depth $d$ in the tree corresponds to a prefix of length $d$ (the first $d$ words)
    of a possible translation.  The moves from a given node correspond to the choice of the next word.
    }

\medskip
(b) AlphaZero has three levels --- complete games resulting in a final outcome $z$ --- move selection at each position in a complete game based on UTC algorithm ---
and simulations within the UTC algorithm.  Describe how each of these can be interpreted in an algorithm for optimizing BLEU score in machine translation.

\solution{
  A ``complete game'' consists of selecting a move (word) at each position in the game ending one specific translation.

  \medskip
  Move selection corresponds to selecting the next word using athe UCT tree search algorithm.

  \medskip
  We can use the same simulation algorithm as in AlphaZero.  It might also be possible to let each simulation generate a full translation, although there is a danger of the simulation getting very deep.
}

\medskip
(c) What do we take $z$ to be in the replay buffer and in equation (4)?

\solution{$z$ should be taken to be the BLEU score of the final translation generated in the ``game''.}
  

\bigskip
{\bf Problem 2. Tuning $\lambda_u$}
    
\medskip
(a) If we increase $\lambda_u $ encourage more diversity or less diversity in the actions selected by simulations? Explain your answer.

\solution{
  Increasing $\lambda_u$ leads to more diversity because larger values of $\lambda_u$ cause $U(s,a)$ to decline
  as $N(s,a)$ increases leading to the simulation moving away from $a$ as $N(s,a)$ increases.
}

\medskip
(b) Consider the case of very large $\lambda_u$ so that the term $\lambda_u \pi_\Phi(a|s)/N(s,a) >> \hat{\mu}(s,a)$.  Equvalently we can change the definition
of $U(s,a)$ to be
$$U(s,a) = \frac{\pi_\Phi(a|s)}{\min(1,N(s,a))} \;\;(5)$$
After running a some number of simulations from $s$ define $a^*$ by
$$a^* = \argmax_a U(s,a)$$
In other words $a^*$ is the move that would be expanded in the next simulation visiting $S$.
Consider a move $a$ othr than $a^*$.  Give a lower bound on $N(s,a)$ in terms of $U(s,a^*)$ and $\pi_\Phi(s,a)$ where $U(s,a)$ is defined by (5).

\solution{
  \begin{eqnarray*}
    U(s,a^*) & \geq & U(s,a) \\
    \\
    U(s,a^*) & \geq & \frac{\pi_\Phi(a|s)}{N(s,a)} \\
    \\
    N(s,a) & \geq & \frac{\pi_\Phi(s|a)}{U(s,a^*)} \;\;\;(6)
  \end{eqnarray*}
  Note that (6) holds for all $a$.  So for very large $\lambda_u$ the number of simulations visiting an action $a$ is
  proportional to $\pi_\Phi(a|s)$ for some top $k$ actions.
}


\bigskip
{\bf Problem 2.  Replacing the Policy with a $Q$-function.}
We consider replacing the policy network $\pi_\Phi$ with a $Q$-value network $Q_\Phi$ so that each node $s$ stores the $Q$-values $Q_\Phi(s,a)$ rather than
the policy probabilities $\pi_\Phi(s,a)$.  We then replace (1) with
$$U(s,a) =  \left\{\begin{array}{ll}\lambda_u Q_\Phi(s,a) &\mbox{if $N(s,a) = 0$} \\ \hat{\mu}(s,a) + \lambda_u Q_\Phi(s,a)/N(s,a) & \mbox{otherwise} \end{array}\right.\;\;\;\;\;(1')$$
and leave (2) and (3) unchanged.  Rewrite (4) to train $Q_\Phi(s,a)$ by minimizing a squared ``Bellman Error'' between $Q_\Phi(s,a)$ and the outcome $z$ over actions drawn from the replay buffer's stored policy.
Presumably this version does not work as well.

\solution{
  
  $$\Phi^*\; = \;\argmin_\Phi \;E_{(s,\pi,z) \sim \mathrm{Replay},\;a \sim \pi}\;
  \left(\begin{array}{l} (V_\Phi(s) - z)^2 \\ \\ - \lambda_1(Q(s,a) - z)^2 \\ \\ + \lambda_2||\Phi||^2\end{array}\right)$$
}

\bigskip
{\bf Problem 3. An advantage actor-critic version.}
We consider replacing the policy network $\pi_\Phi$ with an advantage network $A_\Phi$ so that each node $s$ stores the $A$-values $A_\Phi(s,a)$ rather than
the policy probabilities $\pi_\Phi(s,a)$.  We now have each node $s$ also store $\hat{\mu}(s)$ which equals the average value of the simulations that go through state $s$.
We then replace (1) with
$$U(s,a) =  \left\{\begin{array}{ll}\lambda_u(A_\Phi(s,a) + V_\Phi(s)) &\mbox{if $N(s,a) = 0$} \\ \hat{\mu}(s,a) + \lambda_u(A_\Phi(s,a) + V_\Phi(s))/N(s,a) & \mbox{otherwise} \end{array}\right.\;\;\;\;\;(1'')$$
and leave (2) unchanged and replace (3) by

\medskip
\centerline{a replay buffer of tuples $(s_{\mathrm{root}},\pi_{s_{\mathrm{root}}},z,\hat{\mu}(s_{\mathrm{root}}),\hat{\mu}(s_{\mathrm{root}},a))$ \hspace{2em}$(3')$}

\medskip
Rewrite (4) to train $A_\Phi(s,a)$ by minimizing a squared ``Bellman Error'' between $A_\Phi(s,a)$ and $\hat{A}(s,a)$ defined as follows
$$\hat{A}(s,a) =  \left\{\begin{array}{ll}A_\Phi(s,a) &\mbox{if $N(s,a) = 0$} \\ \hat{\mu}(s,a) - \hat{\mu}(s) & \mbox{otherwise} \end{array}\right.\;\;\;\;\;(5)$$

\solution{

  $$\Phi^*\; = \;\argmin_\Phi \;E_{(s,\pi,R) \sim \mathrm{Replay},\;a \sim \pi}\;\left(\begin{array}{l} (V_\Phi(s) - z)^2
    \\ \\
    \lambda_A\;(A_\Phi(s,a) - \hat{A}(s,a))^2 \\ \\ + \lambda_R\;||\Phi||^2\end{array}\right)\;\;\;\;(2)$$

}

\medskip
Although a valiant attempt, this version also presumably also does not work as well.

\end{document}
