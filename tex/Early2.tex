\input preamble



\begin{document}

{\Huge


\centerline{\bf TTIC 31230, Fundamentals of Deep Learning}
\bigskip
\centerline{David McAllester, Autumn 2020}

\vfill
\centerline{\bf Early Stopping meets Shrinkage}
\vfill
\centerline{\bf $L_1$ Regularization and Sparsity}
\vfill
\centerline{\bf Ensembles}
\vfill
\vfill

\slide{Shrinkage meets Early Stopping}

Early stopping can limit $||\Phi||$.

\vfill
But early stopping more directly limits $||\Phi - \Phi_\mathrm{init}||$.

\vfill
It seems better to take the prior on $\Phi$ to be

\vfill
{\color{red} $$p(\Phi) \propto \exp\left(-\frac{||\Phi - \Phi_{\mathrm{init}}||^2}{2\sigma^2}\right)$$}

\vfill
giving

\vfill
$$\Phi_{t+1} = \Phi_t - \eta\hat{g} - \gamma(\Phi_t - \Phi_{\mathrm{init}})$$

\slide{$L_1$ Regularization}

\begin{eqnarray*}
p(\Phi) & \propto & e^{-\lambda ||\Phi||_1} \;\;\;\;\;\;\;\;||\Phi||_1 = \sum_i |\Phi_i| \\
\\
\Phi^* & = & \argmax_\Phi \; \;\;p(\Phi) \prod_i P_\Phi(y_i|x_i) \\
\\
\Phi^* & = & \argmin_\Phi \; \;\;\left(\sum_i\; -\ln P_\Phi(y_i|x_i)\right) \;+ \; \;\lambda||\Phi||_1 \\
\\
\Phi^* & = & \argmin_\Phi \; \;\;\hat{\cal L}(\Phi) \;+ \; \;\frac{\lambda}{N_{\mathrm{Train}}}||\Phi||_1
\end{eqnarray*}

\slide{$L_1$ Regularization}

\begin{eqnarray*}
\Phi^* & = & \argmin_\Phi \; \;\;\hat{\cal L}(\Phi) \;+ \; \;\frac{\lambda}{N_{\mathrm{Train}}}||\Phi||_1 \\
\\
\\
\Phi_i & \minuseq & \eta \left(\hat{g}_i + \frac{\lambda}{N_{\mathrm{Train}}}\;\mathrm{sign}(\Phi_i)\right) \\
\\
\\
\eta & = & (1-\mu)B\eta_0
\end{eqnarray*}

\slide{Sparsity}

$$\Phi_i \;\;  \minuseq \;\;  \eta \left(\hat{g}_i + \frac{\lambda}{N_{\mathrm{Train}}}\;\mathrm{sign}(\Phi_i)\right)$$
\vfill

\vfill
For $\Phi^*$ the gradient of the objective, and hence the average update, must be zero:

\vfill
$$\begin{array}{rcll}
\Phi^*_i &  = & 0  & \;\;\;\;\;\mbox{if} \;|g_i| <  \lambda/N_{\mathrm{Train}} \\
\\
g_i & = &  -(\lambda/N_{\mathrm{Train}}) \mathrm{sign}(\Phi_i) &\;\;\;\;\; \mbox{otherwise}
\end{array}$$

\vfill
But in practice $\Phi_i$ will never be exactly zero.

\slide{Ensembles}

Train several models $\mathrm{Ens} = (\Phi_1,\;\ldots,\; \Phi_K)$ from different initializations and/or under different meta parameters.

\vfill
We define the ensemble model by

$$P_\mathrm{Ens}(y|x) = \frac{1}{K} \sum_k\; P_{\Phi_k}(y|x) = E_k \;P_k(y|x)$$

\vfill
Ensemble models almost always perform better than any single model.


\vfill
\slide{Ensembles Under Cross Entropy Loss}

\begin{eqnarray*}
{\cal L}\left(P_\mathrm{Ens}\right) & = & E_{\tuple{x,y} \sim \pop}\;\;-\ln P_\mathrm{Ens}(y|x) \\
\\
 & = & E_{\tuple{x,y} \sim \pop}\;\;-\ln E_k P_k(y|x) \\
\\
& \leq & E_{\tuple{x,y} \sim \pop} \;E_k\;\;-\ln P_k(y|x) \\
\\
& = & E_k\; {\cal L}(P_k)
\end{eqnarray*}

\slide{Ensembles Under Cross Entropy Loss}

It is important to note that

\vfill
$$-\ln E_k \;P_k(y|x) \;\;\leq E_k -\ln P_k(y|x)$$

\vfill
for each individual pair $\tuple{x,y}$.

\vfill
This may explain why in practice an ensemble model is typically better than any single component model.

\slide{END}

}
\end{document}
