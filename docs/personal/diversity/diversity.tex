% Source: http://tex.stackexchange.com/a/5374/23931
\documentclass{article}
\usepackage[T1]{fontenc}
\usepackage[utf8]{inputenc}
\usepackage[margin=1in]{geometry}

\newcommand{\HRule}{\rule{\linewidth}{0.5mm}}
\newcommand{\Hrule}{\rule{\linewidth}{0.3mm}}

\makeatletter% since there's an at-sign (@) in the command name
\renewcommand{\@maketitle}{%
  \parindent=0pt% don't indent paragraphs in the title block
  \centering
  {\Large \bfseries\textsc{\@title}}
  \HRule\par%
  \textit{\@author \hfill \@date}
  \par
}
\makeatother% resets the meaning of the at-sign (@)

\title{Diversity Essay}
\author{Clayton W. Seitz}
\date{25 July, 2021}

\begin{document}
  \maketitle% prints the title block
\vspace{0.4in}

Describe your leadershi\textit{p}, work experience, service experience, or other significant involvement with racial, ethnic, socio-economic, or educational communities that have traditionally been underrepresented in higher education, and how these experiences would promote a diversity of views, experiences, and ideas in the pursuit of research, scholarship, and creative excellence.\\\\\

	Growing up I spent a significant amount of my time in my mexican grandmother’s household. My grandmother taught english as a second language to immigrants from all over the word and felt very strongly about topics related to diversity and inclusion. She felt that we should strive to give everyone equal opportunity and made a considerable effort to instill this value in her grandchildren. I believe my sympathy for those who are underrepresented in academia traces back to these teachings. Since then, I have learned about the nature of exclusion and its effects formally and informally, all the while developing perspectives of my own and building upon the foundations laid during childhood.
	
As an undergraduate at Indiana University Bloomington, I became a member of the Hudson and Holland Scholars Program (HHSP) for high achieving underrepresented minorities. As a member of Hudson and Holland, I took part in the Leadership, Engagement, Academics, and Diversity (LEAD) program where I attended seminars on diversity and inclusion and took a course covering current issues in undergraduate life for minorities. My experience as a member of HHSP instilled in me a yearning to see underrepresented women and minorities succeed, especially in STEM. In the future, I intend to remain an advocate for this form of equality wherever my graduate education takes me. 
	While pursuing my second undergraduate degree at IUPUI, I worked part time at the Indianapolis Public Library. There, I assisted individuals from a wide variety of racial, ethnic, and socio-economic backgrounds in finding jobs and opportunities for higher education. A number of those individuals were impoverished primarily due to their lack of education and the skills that are valued by many employers and academic institutions today. In some cases, it was obvious that these individuals had undergone some sort of discrimination for their race or ethnic heritage and had not been granted the same opportunities as their more highly represented counterparts. As a member of HHSP as an undergraduate, we discussed situations like this; however, these conceptual discussions can only approximate the actual effect of neglecting those values. While a library employee, I experienced firsthand what those individuals who are underrepresented go through and the contributions they could have made given the opportunity.
	Undoubtedly, these experiences have shaped my perspective on diversity and inclusion of underrepresented minorities for the better. I hold that the academic institution should be a place for free and open inquiry where a discussion is had by individuals from a wide range of backgrounds. This is not only fair, but scientific progress often comes about due to collective efforts and when constituents are forced to entertain a wide variety of perspectives. During my graduate education, I intend to continue realizing these points of view in order to promote the diversity of views, experiences, and ideas that Purdue advocates for. 





\end{document}