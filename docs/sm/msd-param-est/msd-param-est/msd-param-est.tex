% Latex template: mahmoud.s.fahmy@students.kasralainy.edu.eg
% For more details: https://www.sharelatex.com/learn/Beamer

\documentclass{beamer}					% Document class

\setbeamertemplate{footline}[text line]{%
  \parbox{\linewidth}{\vspace*{-8pt}Langevin Dynamics\hfill\insertshortauthor\hfill\insertpagenumber}}
\setbeamertemplate{navigation symbols}{}

\usepackage[english]{babel}				% Set language
\usepackage[utf8x]{inputenc}			% Set encoding
\usepackage{bm}

\mode<presentation>						% Set options
{
  \usetheme{default}					% Set theme
  \usecolortheme{default} 				% Set colors
  \usefonttheme{default}  				% Set font theme
  \setbeamertemplate{caption}[numbered]	% Set caption to be numbered
}

% Uncomment this to have the outline at the beginning of each section highlighted.
%\AtBeginSection[]
%{
%  \begin{frame}{Outline}
%    \tableofcontents[currentsection]
%  \end{frame}
%}

\usepackage{graphicx}					% For including figures
\usepackage{booktabs}					% For table rules
\usepackage{hyperref}					% For cross-referencing

\title{Optimal Parameter Estimation for Mean Squared Displacement}	% Presentation title
\author{Clayton W. Seitz}								% Presentation author
\date{\today}									% Today's date	

\begin{document}

% Title page
% This page includes the informations defined earlier including title, author/s, affiliation/s and the date
\begin{frame}
  \titlepage
\end{frame}

% Outline
% This page includes the outline (Table of content) of the presentation. All sections and subsections will appear in the outline by default.
\begin{frame}{Outline}
  \tableofcontents
\end{frame}

\begin{frame}{Langevin Dynamics}

Originally a reformulation of Einsteins theory of Brownian motion (BM) using stochastic differential equations (SDEs)

\begin{equation*}
\frac{dx}{dt} = \eta(t), \;\;\; \eta(t) \sim T(x,t|x',t')
\end{equation*}

For BM, $T(x,t|x',t') = \mathcal{N}(x',\sigma^{2})$ where $\langle \eta(t)\eta(t')\rangle = \delta(t-t')$. If we have many x's, and $\eta(t)$ is uncorrelated over the ensemble we may write 

\begin{equation*}
\langle \eta(t)\eta(t')\rangle = \sigma^{2}\delta_{ij}\delta(t-t')
\end{equation*}

\end{frame}

\begin{frame}{Application to Brownian Motion}

The solution to an SDE is a probability distribution $P(x,t)$ which obeys the Markov property

\begin{equation*}
P(x,t') = \int T(x,t|x',t')P(x',t')dx'
\end{equation*}


With some effort this can be transformed into the Fokker-Planck equation

\begin{equation*}
\frac{dP}{dt} = \frac{\sigma^{2}}{2}\frac{\partial^{2}P}{\partial x^{2}} = D\frac{\partial^{2}P}{\partial x^{2}}
\end{equation*}

which has a familiar non-stationary solution for $P(x,t)$ in BM:

\begin{equation*}
P(x,t) = \frac{1}{\sqrt{4\pi Dt}}\exp\left(-\frac{x^{2}}{4Dt}\right)
\end{equation*}

\end{frame}

\begin{frame}{Time-averaged mean squared displacement}

A common dynamical quantity measured for a single particle trajectory is the mean-squared displacement (MSD)

\begin{align*}
\bm{\mathrm{MSD}}(\Delta t) &= \langle |\tilde{\bm{r}}(t+\Delta t) - \tilde{\bm{r}}(t)|^{2}\rangle\\
&=  \frac{1}{M}\sum_{n=1}^{M} |\tilde{\bm{r}}(t+n\tau) - \tilde{\bm{r}}(t)|^{2}
\end{align*}

Each lag time $\Delta t = n\tau$ has an associated histogram $T(d^{2})$ with $M = {N\choose n}$ samples. The $\bm{\mathrm{MSD}}$ is essentially the variance over $M$ samples

\end{frame}

\begin{frame}{Time averaged MSD: Brownian motion}

$\bm{\mathrm{MSD}} = 4D\Delta t$ for Brownian motion. However our measurement $\tilde{\bm{r}}(t)$ is generally not equal to the true value $\bm{r}(t)$ due to localization error\\
\vspace{0.2in}

\end{frame}

\begin{frame}{Time averaged MSD: Brownian motion}


$\tilde{\bm{r}}(t) = \bm{r}(t) + \epsilon$ where $\epsilon$ is normally distributed $\epsilon \sim \mathcal{N}(0,\sigma^{2})$\\
\vspace{0.1in}
Our uncertainty $\sigma$ of the particle position is related to experimental parameters by

\begin{equation*}
\sigma = a\sqrt{\left(1+\frac{\tilde{D}t_{E}}{s^{2}}\right)\cdot \frac{1}{2\pi I_{0}}}
\end{equation*}
\\
\vspace{0.2in}
where $s$ and $I_{0}$ parameterize a symmetric Gaussian PSF and $t_{E}$ is the exposure time

\end{frame}

\section{References}

% Adding the option 'allowframebreaks' allows the contents of the slide to be expanded in more than one slide.
\begin{frame}[allowframebreaks]{References}
	\tiny\bibliography{references}
	\bibliographystyle{apalike}
\end{frame}

\end{document}