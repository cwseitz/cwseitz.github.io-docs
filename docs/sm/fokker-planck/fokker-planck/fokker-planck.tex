

\documentclass{article}
\title{The Fokker-Planck Equation}
\author{C.W. Seitz}
\date{\today}

\usepackage{graphicx}
\usepackage{subfigure,epsfig,amsfonts}
\usepackage{amsmath}
\usepackage{siunitx}
\usepackage{float}
\usepackage{bm}

\begin{document}
\maketitle

\section{The Fokker-Planck Equation}

\subsection{Kramers-Moyal Expansion}

Consider the following Ito stochastic differential equation 

\begin{align*}
d\vec{x} = F(\vec{x},t) + G(\vec{x},t)dW
\end{align*}

The SDE given above corresponds to the Kramers-Moyal expansion (KME) of a transition density $T(x',t'|x,t)$ see (Risken 1989) for a full derivation.

\begin{align}
\frac{\partial P(x,t)}{\partial t}  &= \sum_{n=1}^{\infty} \frac{1}{n!}\left(-\frac{\partial}{\partial x}\right)^{n} \left[M_{n}(x,t)P(x,t)\right]
\end{align}

where $M_{n}$ is the $n$th moment of the transition density. In the diffusion approximation, the KME becomes the Fokker-Planck equation (FPE) (Risken 1989). For the sake of demonstration, consider the univariate case with random variable $x$ and the form of $T(x',t'|x,t)$ is a Gaussian with mean $\mu(t)$ and variance $\sigma^{2}(t)$. In this scenario, the FPE applies because $M_{n} = 0$ for all $n > 2$. Given that $M_{1}(x,t) = \mu(t)$ (drift) and $M_{2}(x,t) = \sigma^{2}(t)$ (diffusion), the FPE reads

\begin{align}
\frac{\partial P(x,t)}{\partial t}  &= \left(-\frac{\partial}{\partial x}M^{(1)}(t) + \frac{1}{2}\frac{\partial^{2}}{\partial x^{2}}M^{(2)}(t)\right)P(x,t)
\end{align}

We can additionally define the term in parentheses as a differential operator acting on $P(x,t)$

\begin{align}
\hat{\mathcal{L}}_{FP} = \left(-\frac{\partial}{\partial x}M^{(1)}(t) + \frac{1}{2}\frac{\partial^{2}}{\partial x^{2}}M^{(2)}(t)\right)
\end{align}

It is common to additionally define the probability current $J(x,t)$ as 

\begin{align}
J(x,t)  &= \left(M^{(1)}(t) - \frac{1}{2}\frac{\partial}{\partial x}M^{(2)}(t)\right)P(x,t)
\end{align}

This definition provides some useful intuition. The value of $J(x,t)$ is the net probability flux into the interval between $x$ and $x+dx$ at at time $t$. This also allows us to write the FPE as a continuity equation

\begin{align}
\frac{\partial P(x,t)}{\partial t} = -\frac{\partial J(x,t)}{\partial x}
\end{align}

\subsection{The Heat Equation}

The well-known heat equation is a special case of the FPE where $M^{(1)}(t) = 0$ and $M^{(2)}(t) = \sigma^{2}$

\begin{align}
\frac{\partial P(x,t)}{\partial t}  &= \frac{\sigma^{2}}{2}\frac{\partial^{2}P(x,t)}{\partial x^{2}}
\end{align}

The FPE is a parabolic partial differential equation which are difficult to solve directly. We will show a first example of solving the FPE on the above heat equation by Fourier transformation

\begin{align}
\frac{\partial}{\partial t} \int  P(x,t) e^{i\omega t}dx = \frac{\sigma^{2}}{2}\int \frac{\partial^{2}P(x,t)}{\partial x^{2}} e^{i\omega t}dx
\end{align}

We know that $\mathcal{F}[\partial_{x}f] = -i\omega \mathcal{F}[f]$ and $\mathcal{F}[\partial_{x}^{2}f] = \omega^{2} \mathcal{F}[f]$ which allows us to write the heat equation as a first order equation

\begin{align}
\frac{\partial \tilde{P}(\omega,t)}{\partial t}  &= -\frac{\sigma^{2}\omega^{2}}{2}\tilde{P}(\omega,t)
\end{align}

which suggests the solution $c\cdot\exp\left(-\alpha\omega^{2}t\right)$. We can find the solution in the spatial domain by inverse Fourier transformation

\subsection{The Multivariate Case}

If we now generalize the above equation to a case where we are faced with many variables $\bm{x} = (x_{1},x_{2},...,x_{n})$. The continuity equation becomes 

\begin{align}
\frac{\partial P(\vec{x},t)}{\partial t} = -\vec{\nabla} \cdot J(\vec{x},t)
\end{align}

where the multivariate probability current now has the interpretation of the net flux into or out of a volume $dx^{n}$ centered around $\bm{x}$. If we consider each dimension, 

\begin{align}
J(x_{i},t)  &= \left(M_{i}^{(1)}(t) - \sum_{j}\frac{\partial}{\partial x_{j}}M_{ij}^{(2)}(t) \right)P(\vec{x},t)
\end{align}

The full Fokker-Planck equation then reads

\begin{align}
\frac{\partial P(\vec{x},t)}{\partial t}  &= \vec{\nabla} \cdot J(\vec{x},t)\\
&= \sum_{i=1}^{N}\left(-\frac{\partial}{\partial x_{i}}M_{i}^{(1)}(t) + \sum_{j=1}^{N} \frac{\partial^{2}}{\partial x_{i}\partial x_{j}}M_{ij}^{(2)}(t)\right)P(\vec{x},t)
\end{align}

It proves quite useful in this form because we can see that the Fokker-Planck equation represents a differentiation operator acting on the distribution $P(\vec{x},t)$

\begin{align}
\hat{\mathcal{L}}_{FP} = \sum_{i=1}^{N}\left(-\frac{\partial}{\partial x_{i}}M_{i}^{(1)}(t) + \sum_{j=1}^{N} \frac{\partial^{2}}{\partial x_{i}\partial x_{j}}M_{ij}^{(2)}(t)\right)
\end{align}

\subsection{Ornstein-Uhlenbeck Process}

If the transition density is Gaussian then the density is fully specified by the first two moments $M^{(1)}(t) = \vec{\mu}(t)$ and $M^{(2)}(\vec{x},t) = \Sigma(t)$. The moments can also be functions of $\vec{x}$. Both of these possibilities are evident in the Ornstein-Uhlenbeck (OU) process. Let the drift vector be a linear function of the state $\vec{x}$ and the diffusion matrix the square of the Gaussian covariances

\begin{align*}
M^{(1)}(t) = \Gamma \vec{x}\;\;\;\;\;M^{(2)}(t) = 2D
\end{align*}

with $D = \Sigma\Sigma^{T}$ which is assumed to be independent of time.

\begin{align}
\hat{\mathcal{L}}_{FP} = \sum_{i=1}^{N}\left(-\frac{\partial}{\partial x_{i}}\Gamma\vec{x} + \sum_{j=1}^{N} \frac{\partial^{2}}{\partial x_{i}\partial x_{j}}D\right)
\end{align}



\end{document}