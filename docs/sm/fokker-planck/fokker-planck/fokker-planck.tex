

\documentclass{article}
\title{The Fokker-Planck Equation}
\author{C.W. Seitz}
\date{\today}

\usepackage{graphicx}
\usepackage{subfigure,epsfig,amsfonts}
\usepackage{amsmath}
\usepackage{siunitx}
\usepackage{float}
\usepackage{bm}

\begin{document}
\maketitle

\section{The Fokker-Planck Equation}

\subsection{Kramers-Moyal Expansion}

Given many instantiations of a stochastic variable $x$, we can construct a normalize histogram oxer all observations as a function of time $P(x,t)$. However, in order to systematically explore the relationship between the parameterization of the process and $P(x,t)$ we require an expression for $\dot{P}(x,t)$. If we make a fundamental assumption that the evolution of $P(x,t)$ follows a Markov process i.e. its evolution has the memoryless property, then we can write

\begin{equation}
P(x', t) = \int T(x', t | x, t-\tau)P(x, t-\tau)dx
\end{equation} 

which is known at the Chapman-Kolmogorov equation. The factor $T(x', t | x, t-\tau)$ is known as the \emph{transition operator} in a Markov process and determines the evolution of $P(x,t)$ in time. We proceed by writing $T(x', t | x, t-\tau)$ in a form referred to as the Kramers-Moyal expansion

\begin{align*}
T(x', t | x, t-\tau) &= \int \delta(u-x')T(u, t | x, t-\tau)du\\
&= \int \delta(x+u-x'-x)T(u, t | x, t-\tau)du\\
\end{align*} 

If we use the Taylor expansion of the $\delta$-function 

\begin{equation*}
\delta(x+u-x'-x) = \sum_{n=0}^{\infty} \frac{(u-x)^{n}}{n!}\left(-\frac{\partial}{\partial x}\right)^{n}\delta(x-x')
\end{equation*}

Inserting this into the result from above, pulling out terms independent of $u$ and swapping the order of the sum and integration gixes

\begin{align}
T(x', t | x, t-\tau) &= \sum_{n=0}^{\infty} \frac{1}{n!}\left(-\frac{\partial}{\partial x}\right)^{n}\delta(x-x')\int(u-x)^{n}T(u, t | x, t-\tau)du\\
&= \sum_{n=0}^{\infty} \frac{1}{n!}\left(-\frac{\partial}{\partial x}\right)^{n}\delta(x-x')M_{n}(x,t)
\end{align} 

noticing that $M_{n}(x,t) = \int(u-x)^{n}T(u, t | x, t-\tau)du$ is just the $n$th moment of the transition operator $T$. Plugging (2.6) back in to (2.4) gixes 

\begin{align}
P(x, t) &= \int \left(1 + \sum_{n=1}^{\infty} \frac{1}{n!}\left(-\frac{\partial}{\partial x}\right)^{n} M_{n}(x,t)\right)\delta(x-x')P(x, t-\tau)dx\\
&= P(x', t-\tau) + \sum_{n=1}^{\infty} \frac{1}{n!}\left(-\frac{\partial}{\partial x}\right)^{n} \left[M_{n}(x,t)P(x,t)\right]
\end{align} 

Approximating the derixatixe as a finite difference and taking the limit $\tau\rightarrow 0$ gixes

\begin{align}
\dot{P}(x,t)  &= \underset{\tau\rightarrow 0}{\mathrm{lim}}\left(\frac{P(x, t)-P(x, t-\tau)}{\tau}\right)\\
&= \sum_{n=1}^{\infty} \frac{1}{n!}\left(-\frac{\partial}{\partial x}\right)^{n} \left[M_{n}(x,t)P(x,t)\right]
\end{align} 

which is formally known as the Kramers-Moyal (KM) expansion. The Fokker-Planck equation is a special case of (2.10) where we neglect terms $n>2$ in the \emph{diffusion approximation}.


Consider the following Ito stochastic differential equation 

\begin{align*}
d\vec{x} = F(\vec{x},t) + G(\vec{x},t)dW
\end{align*}

The SDE gixen aboxe corresponds to the Kramers-Moyal expansion (KME) of a transition density $T(x',t'|x,t)$ see (Risken 1989) for a full derixation.

\begin{align}
\frac{\partial P(x,t)}{\partial t}  &= \sum_{n=1}^{\infty} \frac{1}{n!}\left(-\frac{\partial}{\partial x}\right)^{n} \left[M_{n}(x,t)P(x,t)\right]
\end{align}

where $M_{n}$ is the $n$th moment of the transition density. In the diffusion approximation, the KME becomes the Fokker-Planck equation (FPE) (Risken 1989). For the sake of demonstration, consider the univariate case with random variable $x$ and the form of $T(x',t'|x,t)$ is a Gaussian with mean $\mu(t)$ and variance $\sigma^{2}(t)$. In this scenario, the FPE applies because $M_{n} = 0$ for all $n > 2$. Given that the drift $M_{1}(x,t) = \mu(t)$ and the diffusion $M_{2}(x,t) = \sigma^{2}(t)$, the FPE reads

\begin{align}
\frac{\partial P(x,t)}{\partial t}  &= \left(-\frac{\partial}{\partial x}M^{(1)}(t) + \frac{1}{2}\frac{\partial^{2}}{\partial x^{2}}M^{(2)}(t)\right)P(x,t)
\end{align}

We can additionally define the term in parentheses as a differential operator acting on $P(x,t)$

\begin{align}
\hat{\mathcal{L}}_{FP} = \left(-\frac{\partial}{\partial x}M^{(1)}(t) + \frac{1}{2}\frac{\partial^{2}}{\partial x^{2}}M^{(2)}(t)\right)
\end{align}

It is common to additionally define the probability current $J(x,t)$ as 

\begin{align}
J(x,t)  &= \left(M^{(1)}(t) - \frac{1}{2}\frac{\partial}{\partial x}M^{(2)}(t)\right)P(x,t)
\end{align}

This definition provides some useful intuition. The value of $J(x,t)$ is the net probability flux into the interval between $x$ and $x+dx$ at at time $t$. This also allows us to write the FPE as a continuity equation

\begin{align}
\frac{\partial P(x,t)}{\partial t} = -\frac{\partial J(x,t)}{\partial x}
\end{align}

\subsection{Solving the FPE: Heat (Diffusion) Equation}

The well-known heat equation (it has several names: diffusion equation, heat equation, Brownian motion, Wiener process) is a special case of the FPE where $M^{(1)}(t) = 0$ and $M^{(2)}(t) = \sigma^{2} = \mathrm{const}$. 

\begin{align}
\frac{\partial P(x,t)}{\partial t}  &= D\frac{\partial^{2}P(x,t)}{\partial x^{2}}
\end{align}

with $D = \sigma^{2}/2$. We would like to solve the above equation, but it is a PDE which usually require some tricks to solve e.g., integral transforms. Generally a transform can reduce a differential equation to a simpler form, like an ODE. Upon Fourier tranformation, spatial derivatives turn into factors of $ik$. That is, 

\begin{align*}
\frac{\partial}{\partial x} \psi(x,t) \rightarrow ik\tilde{\psi}(k,t)\;\;\;\;\;\; \frac{\partial^{2}}{\partial x^{2}} \psi(x,t) \rightarrow -k^{2}\tilde{\psi}(k,t)
\end{align*}


\subsubsection{Fourier Transform of the Heat Equation}

Recall the general definition of a Fourier pair

\begin{align*}
\tilde{\psi}(k) &= \mathcal{F}[\psi] = \int_{-\infty}^{\infty} \psi(x) e^{-2\pi ikx}dx\\
\psi(x) &= \mathcal{F}^{-1}[\tilde{\psi}] = \int_{-\infty}^{\infty} \tilde{\psi}(k) e^{2\pi ikx}dk
\end{align*}


Let's see the Fourier transformation of Eq. (6)

\begin{align}
\frac{\partial}{\partial t} \int_{-\infty}^{\infty}  P(x,t) e^{-2\pi ikx}dx = D\int_{-\infty}^{\infty} \frac{\partial^{2}P(x,t)}{\partial x^{2}} e^{-2\pi ikx}dx
\end{align}

As mentioned above, $\mathcal{F}[\partial_{x}\psi] = ik \mathcal{F}[\psi]$ and $\mathcal{F}[\partial_{x}^{2}\psi] = -k^{2} \mathcal{F}[\psi]$ which allows us to write the heat equation as a first order equation

\begin{align}
\frac{\partial \tilde{P}(k,t)}{\partial t}  &= -Dk^{2}\tilde{P}(k,t)
\end{align}

which suggests the solution $\tilde{p}_{0}(k) \exp\left(-D k^{2}t\right)$, which is Gaussian in $k$-space. Let's say our initial condition satisfies $\tilde{P}(x,t_{0}) = \delta (x-x_{0})$ which in the Fourier domain is $P(k,t_{0}) = \exp (-ikx_{0})$. The inverse transform is

\begin{align}
\int_{-\infty}^{\infty} \tilde{p}_{0}(k) \exp\left(ikx-D k^{2}t\right)dk &= \int_{-\infty}^{\infty} \exp\left(ik(x-x_{0})-D k^{2}t\right)dk
\end{align}

which we can rewrite as

\begin{align*}
\frac{1}{\sqrt{2\pi}}\int_{-\infty}^{\infty} \exp\left(-\left(D k^{2}t - ik(x-x_{0}\right)\right)dk
= \frac{1}{\sqrt{2\pi}}\int_{-\infty}^{\infty} \exp\left(-Dt\left(k^{2} - \frac{ik(x-x_{0})}{Dt}\right)\right)dk
\end{align*}

Now we would like to complete the square in the exponential, since we know how to do Gaussian integrals. This can be done as follows:

\begin{align*}
\frac{1}{\sqrt{2\pi}}\int_{-\infty}^{\infty} \exp\left(-Dt\left(k^{2} - \frac{ik(x-x_{0})}{Dt} + \frac{(x-x_{0})^{2}}{4D^{2}t^{2}} - \frac{(x-x_{0})^{2}}{4D^{2}t^{2}}\right)\right)dk
\end{align*}

We are then left to simplify,

\begin{align*}
\frac{1}{\sqrt{2\pi}}\int_{-\infty}^{\infty} \exp\left(-Dt\left(k-\frac{i(x-x_{0})}{2Dt}\right)^{2}\right)dk &= \frac{1}{\sqrt{2\pi}}\exp\left(- \frac{(x-x_{0})^{2}}{4Dt}\right)\int_{-\infty}^{\infty} \exp\left(-Dtk'^{2}\right)dk'\\
&= \frac{1}{\sqrt{2Dt}}\exp\left(- \frac{(x-x_{0})^{2}}{4Dt}\right)
\end{align*}

which is a Gaussian distribution will time-dependent variance $\sigma=4Dt$, given originally by Einstein in his famous paper on Brownian motion in 1905. 

\subsection{Solving the FPE: Ornstein-Uhlenbeck}

The Ornstein-Uhlenbeck process is another special case of the FPE where $M^{(1)}(t) = -\gamma$ and $M^{(2)}(t) = \sigma^{2} = \mathrm{const}$. It is a stationary Gauss–Markov process, which means that it is a Gaussian process, a Markov process, and is temporally homogeneous. The Ito SDE for this process reads

\begin{align}
dx = -\gamma xdt + \sigma dW
\end{align}

which of course has a corresponding Fokker-Planck equation

\begin{align}
\frac{\partial P(x,t)}{\partial t} = -\gamma\frac{\partial}{\partial x} x P(x,t) + D\frac{\partial^{2}P(x,t)}{\partial x^{2}}
\end{align}

In this form, the solution is slightly complicated by the presence of the first order spatial derivative. However, we can still find a solution via a Fourier transform:

\begin{align}
\frac{\partial \tilde{P}(k,t)}{\partial t}  =  -\gamma k \frac{\partial \tilde{P}(k,t)}{\partial k} -k^{2}D \tilde{P}(k,t)
\end{align}

Notice that this is a partial differential equation with the general form 

\begin{align}
a(\tilde{P},k,t)\partial_{k}\tilde{P} + b(\tilde{P},k,t)\partial_{t}\tilde{P} - c(\tilde{P},k,t) = 0
\end{align}

Therefore can solve the above equation using the method of characteristics. As a brief review, suppose we know a solution surface $\tilde{P}$. A vector normal to this surface has the form $\vec{u} = \langle \partial_{k}\tilde{P}, \partial_{t}\tilde{P}, -1 \rangle$. If this vector is normal to the surface, then the vector field 

\begin{align}
\vec{v} = \langle a(\tilde{P},k,t),b(\tilde{P},k,t),c(\tilde{P},k,t) \rangle
\end{align}

is tangent to the surface at every point. In other words, we would like to find a surface $\tilde{P}(k,t)$ for which the vector field above lies in the tangent plane to $\tilde{P}(k,t)$ and therefore $\vec{u}\cdot \vec{v} = 0$. The task that remains then is to find a $\tilde{P}(k,t)$ s.t. the vector $\vec{u}$ is orthogonal to $\vec{v}$. Now, if we construct a curve $\mathcal{C}$ which is an integral curve of $\vec{v}$, then this curve lies on the solution surface $\tilde{P}(k,t)$. Such a curve satisfies the ODEs

\begin{align*}
\frac{dk}{ds} &= \gamma k\\
\frac{dt}{ds} &= 1\\
\frac{d\tilde{P}}{ds} &= -k^{2}D\tilde{P}
\end{align*}

since the vector field given by the Fokker-Planck equation we have is $\vec{v} = \langle \gamma k, 1, -k^{2}D\rangle$. Clearly $t=s$ and $k = k_{0}\exp(\gamma t)$ and thus

\begin{align}
\frac{d\tilde{P}}{dt} &= -k^{2}D\tilde{P}\\
&= -Dk_{0}^{2}\exp(2\gamma t)\tilde{P}
\end{align}

and we have the solution in the Fourier domain

\begin{align}
\tilde{P}(k,t) &= \tilde{P}(k,0)\exp\left(-\frac{Dk_{0}^{2}}{2\gamma}(\exp(2\gamma t)-1)\right)\\
&= \exp\left(-ik_{0} x_{0} -\frac{Dk_{0}^{2}}{2\gamma}(\exp(2\gamma t)-1)\right)\\
&= \exp\left(-i k e^{-\gamma t} x_{0} -\frac{Dk^{2}}{2\gamma}(1-\exp(-2\gamma t))\right)
\end{align}

Let $\mu(t) = x_{0}\exp(-\gamma t)$ and $\sigma^{2}(t) = \frac{D}{\gamma}(1-e^{-2\gamma t})$

\begin{align}
\tilde{P}(k,t) &= \exp\left(-i k \mu(t) -\frac{k^{2}}{2}\sigma^{2}(t)\right)
\end{align}

Taking the inverse Fourier transform of this equation gives 


\begin{align}
P(x,t) &= \frac{1}{\sqrt{2\sigma^{2}(t)}}\exp\left(-\frac{(x-\mu(t))^{2}}{2\sigma^{2}(t)}\right)
\end{align}

\subsection{The Multivariate Case}

If we now generalize the above equation to a case where we are faced with many variables $\bm{x} = (x_{1},x_{2},...,x_{n})$. The continuity equation becomes 

\begin{align}
\frac{\partial P(\vec{x},t)}{\partial t} = -\vec{\nabla} \cdot J(\vec{x},t)
\end{align}

where the multivariate probability current now has the interpretation of the net flux into or out of a volume $dx^{n}$ centered around $\bm{x}$. If we consider each dimension, 

\begin{align}
J(x_{i},t)  &= \left(M_{i}^{(1)}(t) - \sum_{j}\frac{\partial}{\partial x_{j}}M_{ij}^{(2)}(t) \right)P(\vec{x},t)
\end{align}

The full Fokker-Planck equation then reads

\begin{align}
\frac{\partial P(\vec{x},t)}{\partial t}  &= \vec{\nabla} \cdot J(\vec{x},t)\\
&= \sum_{i=1}^{N}\left(-\frac{\partial}{\partial x_{i}}M_{i}^{(1)}(t) + \sum_{j=1}^{N} \frac{\partial^{2}}{\partial x_{i}\partial x_{j}}M_{ij}^{(2)}(t)\right)P(\vec{x},t)
\end{align}

It proves quite useful in this form because we can see that the Fokker-Planck equation represents a differentiation operator acting on the distribution $P(\vec{x},t)$

\begin{align}
\hat{\mathcal{L}}_{FP} = \sum_{i=1}^{N}\left(-\frac{\partial}{\partial x_{i}}M_{i}^{(1)}(t) + \sum_{j=1}^{N} \frac{\partial^{2}}{\partial x_{i}\partial x_{j}}M_{ij}^{(2)}(t)\right)
\end{align}

\subsection{Ornstein-Uhlenbeck Process}

If the transition density is Gaussian then the density is fully specified by the first two moments $M^{(1)}(t) = \vec{\mu}(t)$ and $M^{(2)}(\vec{x},t) = \Sigma(t)$. The moments can also be functions of $\vec{x}$. Both of these possibilities are evident in the Ornstein-Uhlenbeck (OU) process. Let the drift vector be a linear function of the state $\vec{x}$ and the diffusion matrix the square of the Gaussian covariances

\begin{align*}
M^{(1)}(t) = \Gamma \vec{x}\;\;\;\;\;M^{(2)}(t) = 2D
\end{align*}

with $D = \Sigma\Sigma^{T}$ and $\Gamma$ are assumed to be independent of time (non-volatile).


The Fourier transform of (34) is fairly simple, since the FT is linear. We will switch to matrix notation to drop the summations

\begin{align}
\frac{\partial \tilde{P}(\vec{x},t)}{\partial t}= -\bm{\Gamma}\bm{k}\frac{\partial}{\partial \bm{k}} \tilde{P}(\vec{k},t) + \bm{\mathrm{D}}\bm{k}\bm{k}^{T}\tilde{P}(\vec{k},t)
\end{align}

This is in a very similar form to the single variable case (18). We then make the ansatz in analogy with (27)

\begin{align}
\tilde{P}(\bm{k},t) &= \exp\left(-i \bm{k}^{T} \bm{\mu}(t) -\frac{\bm{k}\bm{k}^{T}}{2}\bm{\Sigma}(t)\right)
\end{align}

Plugging this into (34) we have

\begin{align}
\frac{\partial \tilde{P}(\vec{x},t)}{\partial t}+ \bm{\Gamma}\bm{k}\frac{\partial}{\partial \bm{k}} \tilde{P}(\vec{k},t) + \bm{\mathrm{D}}\bm{k}\bm{k}^{T}\tilde{P}(\vec{k},t)\\
= -i\bm{k}\dot{\bm\mu}(t) - \frac{1}{2}\bm{k}\bm{k}^{T}\dot{\bm\Sigma}-i\bm\Gamma \bm\mu^{T} - \bm\Gamma
\end{align}

\end{document}