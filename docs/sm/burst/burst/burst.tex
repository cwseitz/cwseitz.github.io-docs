

\documentclass{article}
\title{Interferon-stimulated transcriptional bursting in space and time}
\author{C.W. Seitz}
\date{\today}

\usepackage{graphicx}
\usepackage{subfigure,epsfig,amsfonts}
\usepackage{amsmath}
\usepackage{siunitx}
\usepackage{float}
\usepackage{bm}

\begin{document}
\maketitle

\section{Introduction: Inferferon-stimulated gene expression}

Motivate the study of inferferon stimulated transcriptional bursting. Why is interferon stimulated gene expression important and what are the potential consequences of non-constitutive gene expression? 

\section{Transcriptional bursting}

Summarize statistical signatures of transcriptional bursting (Poisson versus non-Poisson gene expression)

\section{Promoter switching in mammalian nuclei}

Summarize the origins of bursting at the promoter sequence

\section{Testing the ergodic hypothesis}

Discuss ergodic and non-ergodic regimes of transcriptional bursting. This determines the statistical equivalence of ensemble snapshots and time-series data

\section{Diffusion of nascent RNA in live-cell nuclei}

If RNA copy number is an ergodic process, is the diffusion also ergodic? What type of diffusion do we observe and why? 

\section{Diffusion-based compartment models of gene expression}

Compartment models define promoter states and export of RNA out of the nucleus. Inference on compartment models is performed by using the finite state projection of the chemical master equation to compute the likelihood of parameters. RNA export rates are not diffusion-based in existing models, can we do this?   


\appendix

\section{Photon statistics for fluorescence microscopy}


\section{Single molecule localization: The Cramer-Rao bound}

Discuss parameter uncertainty, Fisher information and CR bound for RNA localization. This is important for achieving accuracy in particle detection and diffusion models

\end{document}