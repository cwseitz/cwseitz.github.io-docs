% Latex template: mahmoud.s.fahmy@students.kasralainy.edu.eg
% For more details: https://www.sharelatex.com/learn/Beamer

\documentclass{beamer}					% Document class

\setbeamertemplate{footline}[text line]{%
  \parbox{\linewidth}{\vspace*{-8pt}A (very) brief introduction to graphical models\hfill\insertshortauthor\hfill\insertpagenumber}}
\setbeamertemplate{navigation symbols}{}

\usepackage[english]{babel}				% Set language
\usepackage[utf8x]{inputenc}			% Set encoding

\mode<presentation>						% Set options
{
  \usetheme{default}					% Set theme
  \usecolortheme{default} 				% Set colors
  \usefonttheme{default}  				% Set font theme
  \setbeamertemplate{caption}[numbered]	% Set caption to be numbered
}

% Uncomment this to have the outline at the beginning of each section highlighted.
%\AtBeginSection[]
%{
%  \begin{frame}{Outline}
%    \tableofcontents[currentsection]
%  \end{frame}
%}

\usepackage{graphicx}					% For including figures
\usepackage{booktabs}					% For table rules
\usepackage{hyperref}					% For cross-referencing

\title{A (very) brief introduction to graphical models}	% Presentation title
\author{Clayton W. Seitz}								% Presentation author
\date{\today}									% Today's date	

\begin{document}

% Title page
% This page includes the informations defined earlier including title, author/s, affiliation/s and the date
\begin{frame}
  \titlepage
\end{frame}

% Outline
% This page includes the outline (Table of content) of the presentation. All sections and subsections will appear in the outline by default.
\begin{frame}{Outline}
  \tableofcontents
\end{frame}

% The following is the most frequently used slide types in beamer
% The slide structure is as follows:
%
%\begin{frame}{<slide-title>}
%	<content>
%\end{frame}


\section{Introduction to graphical models}

\begin{frame}{The logic of generative modeling}

Say we have a set of variables $\mathbf{x} = (x_{1},x_{2},...,x_{n})$ which might have some statistical dependence\\
\vspace{0.1in}
The variable $\mathbf{x}$ might be an amino acid sequence, gene expression data, microscopy image, etc.\\
\vspace{0.1in}
\begin{itemize}
\item Often we are handed a batch of empirical samples $\{\mathbf{x}_{i}\}_{i=1}^{N}$
\item We want to know the generating distribution $p(\mathbf{x})$
\end{itemize}

In supervised \textcolor{red}{generative learning}, we try to explicity learn the joint distribution $p(\mathbf{x}) = \prod_{i=1}^{N-1}p(x_{i}|x_{i+1:N})p(x_{N})$, which is generally more difficult than discriminative learning. 

\end{frame}

\begin{frame}{Primary types of graphical models}

\end{frame}



\begin{frame}{Perks of generative modeling}

\begin{itemize}
\item Fitting complete multivariate distributions $p(\mathbf{x})$ goes beyond correlation-based or clustering approaches
\item Correlations cannot discover partial correlation in the context of other neighbors
\item Fitting $p(\mathbf{x})$ permits sampling based inference
\end{itemize}

\end{frame}

\begin{frame}{Why generative modeling is difficult}

When describing a distribution over multiple variables, we may not know the proper normalization $Z$. That is,

\begin{equation*}
p(\mathbf{x}) = \frac{1}{Z}\tilde{p}(\mathbf{x})
\end{equation*}

\vspace{0.1in}
This \textcolor{red}{very important} situation arises in several contexts:\\
\vspace{0.1in}
1. In \textcolor{red}{Bayesian inference} where $p(x_{1}|x_{2}) = p(x_{2}|x_{1})p(x_{1})/p(x_{2})$ is intractable due to $Z = p(x_{2}) = \int p(x_{2}|x_{1})p(x_{1})dx_{1}$. This integral can be very difficult or impossible to
compute.\\
\vspace{0.1in}
2. In models from statistical physics, e.g. the Ising model, we only know
$\tilde{p}(\mathbf{x}) = e^{−H(\mathbf{x})}$ where $H(\mathbf{x})$ is the Hamiltonian

\end{frame}

\begin{frame}

In \textcolor{red}{image processing}, graphical models are a standard technique for:

\begin{itemize}
\item Image denoising
\item Image interpolation (resolution enhancement)
\item Semantic image segmentation
\end{itemize}

\vspace{0.2in}
In biology, graphical models are used widely in \textcolor{red}{omics} i.e.

\begin{itemize}
\item Genomics
\item Transcriptomics
\item Proteomics
\end{itemize}

\end{frame}

\begin{frame}{Bayesian image reconstruction}

Say a fluorophore emits photons at a rate $\lambda_{n}$. This is the best we can do according to QM
\vspace{0.1in}

For a CMOS array with quantum efficiency $\gamma\;\;[e^{-}/p]$ we have

\begin{equation*}
I_{n} = \gamma g_{n}P_{n}(\lambda_{n}) + G_{n}(\mu_{n};\sigma_{n}^{2}) + \beta
\end{equation*}

where $\mu_{n} \;\;[\mathrm{ADU}]$ is the detector offset and $g_{n}\;\; [\mathrm{ADU}/e^{-}]$ is the gain. \\
\vspace{0.2in}
All we know is $\lambda_{n}$, so both the true signal $I_{n}$ and the detected signal $\hat{I}_{n}$ are stochastic processes. 

\begin{equation*}
P_{\lambda}(I_{n},\hat{I}_{n}) = \frac{1}{Z}\frac{\exp\left({-\lambda_{n}}\right)\lambda_{n}^{p}}{p!}\exp\left(-\frac{(D-g_{n}p-\mu_{n})^{2}}{\sigma_{n}^{2}}\right)
\end{equation*}

\end{frame}


\begin{frame}{Bayesian image reconstruction}

Marginalizing over $p$ gives the noise model as a function of the rate $\lambda_{n}$

\begin{equation*}
P_{\lambda}(I_{n}) = \frac{1}{Z}\sum_{p}\frac{\exp\left({-\lambda_{n}}\right)\lambda_{n}^{p}}{p!}\exp\left(-\frac{(D-g_{n}p-\mu_{n})^{2}}{\sigma_{n}^{2}}\right)
\end{equation*}



\end{frame}




\section{Graphical models of gene expression}




\section{Graphical models in image processing}



\section{References}

% Adding the option 'allowframebreaks' allows the contents of the slide to be expanded in more than one slide.
\begin{frame}[allowframebreaks]{References}
	\tiny\bibliography{references}
	\bibliographystyle{apalike}
\end{frame}

\end{document}