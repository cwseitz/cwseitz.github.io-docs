

\documentclass{article}
\title{Eigenvectors and Eigenvalues}
\author{C.W. Seitz}
\date{\today}

\usepackage{graphicx}
\usepackage{subfigure,epsfig,amsfonts}
\usepackage{amsmath}
\usepackage{siunitx}
\usepackage{float}
\usepackage{bm}

\begin{document}
\maketitle

\begin{section}{Eigenvectors and Eigenvalues}

For a $N\times N$ matrix we have a maximum of $N$ eigenvectors and $N$ eigenvalues. The definition of an eigenvector is that it satisfies the matrix equation

\begin{align*}
A\vec{e} = \lambda \vec{e}
\end{align*}

Say we would like to find the $\vec{x}$ and $\lambda$ satisfy this equation for a particular $A$. For small $N$, we can find these eigenvectors and their corresponding eigevalues relatively easily. For example, when $N=2$ 

\begin{align*}
A\vec{e} = \begin{bmatrix}
    a & b\\
    c & d\\
\end{bmatrix}
\begin{bmatrix}
    e_{1}\\
    e_{2}\\
\end{bmatrix}
= \lambda  \begin{bmatrix}
    e_{1}\\
    e_{2}\\
\end{bmatrix}
\end{align*}

This matrix equation represents the following system of equations

\begin{align*}
ae_{1}+be_{2} &= \lambda e_{1}\\
ce_{1}+de_{2} &= \lambda e_{2}
\end{align*}

We have two equations and two unknowns so we can solve it easily. We want $e_{1}$ and $e_{2}$ in terms of $a,b,c,d$

\begin{align*}
ae_{1}+be_{2} &= \lambda e_{1}\\
c\frac{be_{2}}{\lambda - a} + de_{2} &= \lambda e_{2}
\end{align*}


\end{section}

\end{document}