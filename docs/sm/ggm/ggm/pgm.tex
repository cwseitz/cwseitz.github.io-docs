% Latex template: mahmoud.s.fahmy@students.kasralainy.edu.eg
% For more details: https://www.sharelatex.com/learn/Beamer

\documentclass{beamer}					% Document class

\setbeamertemplate{footline}[text line]{%
  \parbox{\linewidth}{\vspace*{-8pt}Gaussian Graphical Model Demo\hfill\insertshortauthor\hfill\insertpagenumber}}
\setbeamertemplate{navigation symbols}{}

\usepackage[english]{babel}				% Set language
\usepackage[utf8x]{inputenc}			% Set encoding

\mode<presentation>						% Set options
{
  \usetheme{default}					% Set theme
  \usecolortheme{default} 				% Set colors
  \usefonttheme{default}  				% Set font theme
  \setbeamertemplate{caption}[numbered]	% Set caption to be numbered
}

% Uncomment this to have the outline at the beginning of each section highlighted.
%\AtBeginSection[]
%{
%  \begin{frame}{Outline}
%    \tableofcontents[currentsection]
%  \end{frame}
%}

\usepackage{graphicx}					% For including figures
\usepackage{booktabs}					% For table rules
\usepackage{hyperref}					% For cross-referencing

\title{Gaussian Graphical Model Demo}	% Presentation title
\author{Clayton W. Seitz}								% Presentation author
\date{\today}									% Today's date	

\begin{document}

% Title page
% This page includes the informations defined earlier including title, author/s, affiliation/s and the date
\begin{frame}
  \titlepage
\end{frame}

% Outline
% This page includes the outline (Table of content) of the presentation. All sections and subsections will appear in the outline by default.
\begin{frame}{Outline}
  \tableofcontents
\end{frame}

% The following is the most frequently used slide types in beamer
% The slide structure is as follows:
%
%\begin{frame}{<slide-title>}
%	<content>
%\end{frame}



\begin{frame}{One-dimensional case}

Let's first consider a toy example and work our way to the Gaussian graphical model. The standard definition of the 1D Gaussian is

\begin{equation}
P(x|\theta) = \frac{1}{\sqrt{2\pi \sigma^{2}}}\exp\left(-\frac{(x-\mu)^{2}}{2\sigma^{2}}\right)
\end{equation}

We'd like to maximize the log-likelihood $\mathcal{L}_{\theta}$ of the parameters $\theta=(\mu,\sigma)$ i.e.

\begin{equation*}
\theta^{*} = \underset{\theta}{\mathrm{argmin}} -\log P(\theta|X)
\end{equation*}
\end{frame}

\begin{frame}{Bayesian Inference}

We can use Bayesian inference to estimate the optimal parameters $\theta$ given a sample of data drawn from $P(x)$.

\begin{equation}
P(\theta|x) = \frac{P(x|\theta)P(\theta)}{\int_{\theta} P(x|\theta)P(\theta)d\theta}
\end{equation}

In MAP estimation, we try to maximize the numerator as a function of $\theta$. In MLE we assume a uniform prior and try to maximize $P(x|\theta) = \prod_{i=1}^{N} P(x_{i}|\theta)$.\\
\vspace{0.1in}
Those are standard methods (which work nicely in this simple case) but let's try and use MCMC instead

\end{frame}

\begin{frame}{Metropolis MCMC}

Let $\tilde{P}(\theta|x) = P(x|\theta)P(\theta)$\\
\vspace{0.1in}
In the Metropolis algorithm, we randomly choose starting parameter values $\mu_{0}$ and $\sigma_{0}$. We will define two proposal distributions $T_{\mu}(\mu'|\mu) = \mathcal{N}(\mu,\sigma_{\mu}^{2})$ and $T_{\sigma}(\sigma'|\sigma) = \mathcal{N}(\sigma,\sigma_{\sigma}^{2})$\\
\vspace{0.2in}
Iterate:
\begin{itemize}
\item Draw $\mu'\sim T_{\mu}(\mu'|\mu)$, $\sigma'\sim T_{\sigma}(\sigma'|\sigma)$
\item Compute $a_{\mu} = \mathrm{min}\left(1,\frac{P(\mu)}{P(\mu')}\right)$, $a_{\sigma} = \mathrm{min}\left(1,\frac{P(\sigma)}{P(\sigma')}\right)$
\item Accept $\mu'$ w.p. $a_{\mu}$ and $\sigma'$ w.p. $a_{\sigma}$ 
\end{itemize}

\end{frame}

\begin{frame}{MCMC Run: 1D Gaussian Parameters}
\end{frame}


\section{References}

% Adding the option 'allowframebreaks' allows the contents of the slide to be expanded in more than one slide.
\begin{frame}[allowframebreaks]{References}
	\tiny\bibliography{references}
	\bibliographystyle{apalike}
\end{frame}

\end{document}