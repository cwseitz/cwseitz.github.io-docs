% Latex template: mahmoud.s.fahmy@students.kasralainy.edu.eg
% For more details: https://www.sharelatex.com/learn/Beamer

\documentclass{beamer}					% Document class

\setbeamertemplate{footline}[text line]{%
  \parbox{\linewidth}{\vspace*{-8pt}Bayesian image reconstruction\hfill\insertshortauthor\hfill\insertpagenumber}}
\setbeamertemplate{navigation symbols}{}

\usepackage[english]{babel}				% Set language
\usepackage[utf8x]{inputenc}			% Set encoding

\mode<presentation>						% Set options
{
  \usetheme{default}					% Set theme
  \usecolortheme{default} 				% Set colors
  \usefonttheme{default}  				% Set font theme
  \setbeamertemplate{caption}[numbered]	% Set caption to be numbered
}

% Uncomment this to have the outline at the beginning of each section highlighted.
%\AtBeginSection[]
%{
%  \begin{frame}{Outline}
%    \tableofcontents[currentsection]
%  \end{frame}
%}

\usepackage{graphicx}					% For including figures
\usepackage{booktabs}					% For table rules
\usepackage{hyperref}					% For cross-referencing

\title{Bayesian image reconstruction}	% Presentation title
\author{Clayton W. Seitz}								% Presentation author
\date{\today}									% Today's date	

\begin{document}

% Title page
% This page includes the informations defined earlier including title, author/s, affiliation/s and the date
\begin{frame}
  \titlepage
\end{frame}

% Outline
% This page includes the outline (Table of content) of the presentation. All sections and subsections will appear in the outline by default.
\begin{frame}{Outline}
  \tableofcontents
\end{frame}

% The following is the most frequently used slide types in beamer
% The slide structure is as follows:
%
%\begin{frame}{<slide-title>}
%	<content>
%\end{frame}


\begin{frame}{Bayesian image reconstruction}

Say a fluorophore emits photons at a rate $\lambda_{n}$. This is the best we can do according to QM
\vspace{0.1in}

For a CMOS array with quantum efficiency $\gamma\;\;[e^{-}/p]$ we have

\begin{equation*}
I_{n} = \gamma g_{n}P_{n}(\lambda_{n}) + G_{n}(\mu_{n};\sigma_{n}^{2}) + \beta
\end{equation*}

where $\mu_{n} \;\;[\mathrm{ADU}]$ is the detector offset and $g_{n}\;\; [\mathrm{ADU}/e^{-}]$ is the gain. \\
\vspace{0.2in}
All we know is $\lambda_{n}$, so both the true signal $I_{n}$ and the detected signal $\hat{I}_{n}$ are stochastic processes. 

\begin{equation*}
P_{\lambda}(I_{n},\hat{I}_{n}) = \frac{1}{Z}\frac{\exp\left({-\lambda_{n}}\right)\lambda_{n}^{p}}{p!}\exp\left(-\frac{(D-g_{n}p-\mu_{n})^{2}}{\sigma_{n}^{2}}\right)
\end{equation*}

\end{frame}


\begin{frame}{Bayesian image reconstruction}

Marginalizing over $p$ gives the noise model as a function of the rate $\lambda_{n}$

\begin{equation*}
P_{\lambda}(I_{n}) = \frac{1}{Z}\sum_{p}\frac{\exp\left({-\lambda_{n}}\right)\lambda_{n}^{p}}{p!}\exp\left(-\frac{(D-g_{n}p-\mu_{n})^{2}}{\sigma_{n}^{2}}\right)
\end{equation*}



\end{frame}


\section{References}

% Adding the option 'allowframebreaks' allows the contents of the slide to be expanded in more than one slide.
\begin{frame}[allowframebreaks]{References}
	\tiny\bibliography{references}
	\bibliographystyle{apalike}
\end{frame}

\end{document}