% Latex template: mahmoud.s.fahmy@students.kasralainy.edu.eg
% For more details: https://www.sharelatex.com/learn/Beamer

\documentclass[aspectratio=1610]{beamer}					% Document class

\setbeamertemplate{footline}[text line]{%
  \parbox{\linewidth}{\vspace*{-8pt}Bayesian image reconstruction\hfill\insertshortauthor\hfill\insertpagenumber}}
\setbeamertemplate{navigation symbols}{}

\usepackage[english]{babel}				% Set language
\usepackage[utf8x]{inputenc}			% Set encoding

\mode<presentation>						% Set options
{
  \usetheme{default}					% Set theme
  \usecolortheme{default} 				% Set colors
  \usefonttheme{default}  				% Set font theme
  \setbeamertemplate{caption}[numbered]	% Set caption to be numbered
}

% Uncomment this to have the outline at the beginning of each section highlighted.
%\AtBeginSection[]
%{
%  \begin{frame}{Outline}
%    \tableofcontents[currentsection]
%  \end{frame}
%}

\usepackage{graphicx}					% For including figures
\usepackage{booktabs}					% For table rules
\usepackage{hyperref}					% For cross-referencing

\title{Bayesian image reconstruction}	% Presentation title
\author{Clayton W. Seitz}								% Presentation author
\date{\today}									% Today's date	

\begin{document}

% Title page
% This page includes the informations defined earlier including title, author/s, affiliation/s and the date
\begin{frame}
  \titlepage
\end{frame}

% Outline
% This page includes the outline (Table of content) of the presentation. All sections and subsections will appear in the outline by default.
\begin{frame}{Outline}
  \tableofcontents
\end{frame}

% The following is the most frequently used slide types in beamer
% The slide structure is as follows:
%
%\begin{frame}{<slide-title>}
%	<content>
%\end{frame}


\begin{frame}{Photon statistics of CMOS cameras}
\begin{itemize}
\item Imaging noise consists of shot noise, thermal noise, and readout noise
\vspace{0.1in}
\item Shot noise is Poisson, thermal noise and readout noise are Gaussian
\end{itemize}
\vspace{0.2in}
For a CMOS pixel $n$, the true signal $S_{n}\; [\mathrm{ADU}]$ is a Poisson process with rate parameter $\lambda_{n}$

\begin{equation*}
S_{n} = \gamma g_{n}P_{n}(\lambda_{n})
\end{equation*}

where $\gamma\;\;[e^{-}/p]$ is the quantum efficiency and $g_{n}\;\; [\mathrm{ADU}/e^{-}]$ is the pixel's gain
\vspace{0.1in}

\begin{equation*}
P(S_{n}) = \frac{\exp\left({-\lambda_{n}}\right)\lambda_{n}^{p}}{p!}
\end{equation*}

\textcolor{blue}{But what is the distribution over the corrupted signal $P(\hat{S}_{n})$?}


\end{frame}


\begin{frame}{Photon statistics of CMOS cameras}

To find $P(\hat{S}_{n})$, we first evaluate the joint density $P(S_{n},\hat{S}_{n})$
\vspace{0.1in}
\begin{align*}
P(S_{n},\hat{S}_{n}) &= P(\hat{S}_{n}|S_{n}=s)P(S_{n}=s)\\
&= \frac{1}{Z}\exp\left(-\frac{(\hat{S}_{n}-g_{n}s-\mu_{n})^{2}}{\sigma_{n}^{2}}\right)\frac{\exp\left({-\lambda_{n}}\right)\lambda_{n}^{s}}{s!}
\end{align*}
\vspace{0.1in}

Marginalizing over $S_{n}$ gives the desired distribution over $\hat{S}_{n}$

\begin{equation*}
P(\hat{S}_{n}) = \frac{1}{Z}\sum_{s=0}^{\infty}\frac{\exp\left({-\lambda_{n}}\right)\lambda_{n}^{s}}{s!}\exp\left(-\frac{(\hat{S}_{n}-g_{n}s-\mu_{n})^{2}}{\sigma_{n}^{2}}\right)
\end{equation*}

\end{frame}

\begin{frame}{Bayesian parameter inference for CMOS photon statistics}

The parameters in our model $\theta = (\lambda_{n},g_{n},\mu_{n},\sigma^{2}_{n})$ are unknown apriori


\begin{equation*}
P(\theta|\hat{S}_{n}) \propto P(\hat{S}_{n}|\theta)P(\theta)
\end{equation*}

\vspace{0.2in}
We can just computed the likelihood $P(\hat{S}_{n}|\theta)$ on the last slide. Samples from the posterior can be found for example by MCMC or we could use MAP estimation\\
\vspace{0.2in}
Either of these approaches only make sense for stationary statistics, which means the physical locations and photophysics of the sample remain unchanged in time\\
\vspace{0.2in}
For example photostable fluorophores like quantum dots would be a good choice

\end{frame}

\begin{frame}{Fisher Information and the Cramer-Rao Bound}

Consider maximum likelihood estimation (MLE) where the objective is to find an optimal parameter(s) that best explains the data

\begin{align*}
\theta^{*} = \underset{\theta}{\mathrm{argmax}}\; \ell(\mathcal{D}|\theta)
\end{align*}

where $\ell = \log\mathcal{L}$ is the log-likelihood function. We can define the sensitivity of $\ell$ with respect to $\theta$

\begin{align*}
\ell'(x|\theta) = \frac{\partial}{\partial\theta} \ell(x|\theta) = \frac{\mathcal{L}'(x|\theta)}{\mathcal{L}(x|\theta)}
\end{align*}

for $x\in\mathcal{D}$. Intuitively, if the likelihood is insensitive to changes in $\theta$, then $\mathcal{D}$ does not provide very much information about $\theta$

\end{frame}

\begin{frame}{Fisher Information and the Cramer-Rao Bound}

Since $x$ is a continuous random variable, we have to consider the average sensitivity\\
\vspace{0.2in}
That is, for each $x\sim P(x)$ we can compute $\ell'(x|\theta)$ for all $\theta$

\begin{align*}
I(\theta) = \mathbb{E}\left[\frac{\partial}{\partial\theta} \left(\ell(x|\theta)\right)^{2}\right] = \int \left(\ell(x|\theta)\right)^{2} \mathcal{L}(x|\theta)dx
\end{align*}

for $x\in\mathcal{D}$. Intuitively, if the likelihood is insensitive to changes in $\theta$, then $\mathcal{D}$ does not provide very much information about $\theta$\\
\vspace{0.1in}
The Cramer-Rao Bound places a lower bound on the variance in our parameter estimate in iterms of $I(\theta)$ :

\begin{equation*}
\mathrm{Var}(\hat{\theta}) \geq \frac{1}{I(\theta)}
\end{equation*}

\end{frame}

\begin{frame}{Consequences for single molecule localization}

The point spread function

\end{frame}

\section{References}

% Adding the option 'allowframebreaks' allows the contents of the slide to be expanded in more than one slide.
\begin{frame}[allowframebreaks]{References}
	\tiny\bibliography{references}
	\bibliographystyle{apalike}
\end{frame}

\end{document}