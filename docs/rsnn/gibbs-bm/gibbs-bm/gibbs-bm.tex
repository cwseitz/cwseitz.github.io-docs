

\documentclass{article}
\title{Gibbs sampling and the Boltzmann machine}
\author{C.W. Seitz}
\date{\today}

\usepackage{graphicx}
\usepackage{subfigure,epsfig,amsfonts}
\usepackage{amsmath}
\usepackage{siunitx}

\begin{document}
\maketitle

\begin{abstract}
\noindent Sinds enkele jaren ben ik op zoek naar eenvoudige wiskundige en fysische problemen die onverwacht gerelateerd zijn met het getal $\pi$. In \emph{The bouncing balls and pi} beschreef ik eerder al hoe de opeenvolgende decimalen van $\pi$ kunnen berekend worden door twee ballen volledig elastisch tegen elkaar en tegen een muur te laten botsen. In dit artikel zal ik aantonen hoe het getal $\pi$ tevoorschijn komt door een oneindige serie rechthoeken met oppervlakte 1 spiraalsgewijze aan elkaar te kleven. In een veralgemening van dit probleem duikt op een natuurlijke wijze de gammafunctie en de formule van Stirling op.    
\end{abstract}

\section{Probleemstelling}

Het volgend probleem ontleende ik aan MindYourDecisions, het wiskundige videokanaal van Presh Talwalkar op YouTube. Hoewel de formulering van het probleem enkel eenvoudige meetkunde bevat, werd de oplossing op de website aangekondigd als zeer pittig. In vrije vertaling kan het probleem als volgt geformuleerd worden.



Om deze limietverhouding te berekenen, gaan we eerst proefondervindelijk te werk. Aangezien het zijdelings aankleven van een rechthoek wiskundig een beetje afwijkt van het boven- of onderaan aankleven, zullen we in één stap simultaan stwee rechthoeken aankleven. Met de nummering van figuur \ref{spiraal} nemen we eerst de rechthoeken 1 en 2 samen, vervolgens de rechthoeken 3 en 4, daarna 5 en 6 enz.  

Rechthoek 1 heeft de vorm van een eenheidsvierkant. Rechthoek 2 heeft breedte 2. Om een oppervlakte van 1 $\rm{dm}^2$ te verkrijgen moet de hoogte gelijk zijn aan $\frac{1}{2}$. Na uitbreiding met deze twee rechthoeken, vinden we een totale breedte die gelijk is aan 2 en een totale hoogte die gelijk is aan $\frac{3}{2}$. De breedte-hoogte-verhouding van de samengestelde rechthoek is dan  $\frac{4}{3}$.

Rechthoek 3 heeft een hoogte die gelijk is aan $\frac{3}{2}$. De breedte is bijgevolg gelijk aan $\frac{2}{3}$ want de oppervlakte van elke rechthoek is 1. Hierna kunnen we de breedte van rechthoek 4 berekenen. Die is gelijk aan $\frac{8}{3}$. Omdat ook deze rechthoek oppervlakte 1 heeft, moet de hoogte gelijk zijn aan $\frac{3}{8}$. De samengestelde rechthoek heeft na twee uitbereidingsronden een breedte van $\frac{8}{3}$ en een hoogte van $\frac{15}{8}$. De breedte-hoogte-verhouding is nu gelijk aan $\frac{64}{45}$.

Zonder de gave van de helderziendheid is het vrijwel onmogelijk om het rijtje dat begint met de breuken $\frac{4}{3}$ en $\frac{64}{45}$ verder te zetten. Zelfs na de berekening van de volgende term, $\frac{256}{175}$, zal de regelmaat wellicht niet duidelijk worden. Daarom proberen we in dit verslag eerst een recursieformule op te stellen voor de opeenvolgende termen van deze rij van breedte-hoogte-verhoudingen. Deze betrekking zal volstaan om de breedte-hoogteverhouding in de limietsituatie te berekenen.

\end{document}