
\documentclass[aspectratio=169]{beamer}
\setbeamertemplate{navigation symbols}{}

\usepackage{subfigure,epsfig,amsfonts}
\usepackage{beamerthemeshadow}
\usepackage{amsmath}
\usepackage{bm}
\usepackage{siunitx}

\setbeamertemplate{footline}{}
\setbeamertemplate{navigation symbols}{}

\begin{document}
\title{Stochastic computation in recurrent networks of spiking neurons}  
\author{Clayton Seitz}
\date{\today} 

\maketitle

\begin{frame}{The squid giant axon}

Hodkin and Huxley developed a mathematical model for nerve cell communication in 1952 using voltage data from the giant axon of a squid

\begin{figure}
\centering
\includegraphics[width=95mm]{figure-20}
\end{figure}

\end{frame}

\begin{frame}{$Na^{+}$ and $K^{+}$ are the major charge carriers}

\begin{figure}
\centering
\includegraphics[width=105mm]{figure-19}
\end{figure}

\end{frame}

\begin{frame}{$Ca^{2+}$ sensors enable high-speed two-photon imaging}

Animal models and experimental technologies have improved drastically

\begin{figure}
\centering
\includegraphics[width=140mm]{figure-16}
\end{figure}
Scale bars: b, 250 um; c, d, 100 um

$4\si{mm}^{2}$ FOV at $\sim 8\si{Hz}$

\end{frame}

\begin{frame}{Spiking neural networks (SNN): integrate and fire models}


\begin{figure}
\centering
\includegraphics[width=75mm]{figure-15}
\end{figure}

\begin{figure}
\centering
\includegraphics[width=115mm]{figure-14}
\end{figure}

\begin{equation*}
\tau\dot{V(t)} = -g_{L}V(t) + \sum_{n} w_{n}\theta_{n}(t)
\end{equation*}

\end{frame}


\begin{frame}{Synaptic strengths are dynamic}

$w_{n}$ represents the change in the post-synaptic membrane potential induced by an action potential at the presynaptic cell ($\sim 1-4 \si{mV}$)

\begin{figure}
\centering
\includegraphics[width=95mm]{figure-12}
\end{figure}

$w_{n}$ is a result of complex biochemical pathways and is not necessarily a constant (synaptic plasticity)
\end{frame}

\begin{frame}{Synaptic current as a stochastic process}


\begin{figure}
\centering
\includegraphics[width=105mm]{figure-23}
\end{figure}

\begin{equation*}
\tau\dot{V(t)} = -g_{L}V(t) + \mu(t) + \sqrt{2D}\eta(t)
\end{equation*}

\end{frame}

\begin{frame}{An example simulation}

\begin{figure}
\centering
\includegraphics[width=105mm]{figure-24}
\end{figure}

\end{frame}


\begin{frame}{Fokker-Planck equation for Brownian motion}

\begin{figure}
\centering
\subfigure[]{\label{fig:a}\includegraphics[width=65mm]{figure-21}}
\subfigure[]{\label{fig:b}\includegraphics[width=65mm]{figure-22}}
\end{figure}

Predicting $I_{n}(t)$ is hard in complex networks. We instead solve for $P(V,t)$

\begin{equation*}
\tau\frac{\partial P}{\partial t} = (\mu(t)-V)\frac{\partial P}{\partial V} + \sqrt{2D}\frac{\partial^{2}P}{\partial V^{2}}
\end{equation*}

\end{frame}

\begin{frame}{Potential applications: neuromorphic computing}

\begin{itemize}
\item Very low power consumption
\item Dynamically remaps the synapses between artificial neurons
\item Useful in embedded applications
\end{itemize}


\begin{figure}
\centering
\includegraphics[width=130mm]{figure-26}
\end{figure}


\end{frame}

\end{document}