\ProvidesFile{thesis.tex}[2022-10-05 PurdueThesis thesis.tex file]

\def\ZZinstitution{Purdue University}
\def\ZZcampus{Indianapolis}
\def\ZZprogram{Physics}
\def\ZZdegree{Doctor of Philosophy}
\def\ZZauthor{Clayton Seitz}
\def\ZZdocument{A Dissertation}
\def\ZZgraduation{December 2024}
\def\ZZtitle{Advancing super resolution microscopy for quantitative in-vivo imaging of chromatin nanodomains}
\def\ZZshowcolophon{false}
\def\ZZshowdiagonalline{false}
\def\ZZshowgridlines{false}
\def\ZZshowmarginlines{false}
\def\ZZshowtimestamp{false}
\def\ZZtodonotes{false}
\InputIfFileExists{optional-debugging-code.tex}{}{}

\documentclass{PurdueThesis}


\def\ZZatinformation{}
% If you are at the Hammond or Westville campus
% remove the "%" from the begining of the next line.
%\def\ZZatinformation{~at~Purdue~Northwest}

% If the title contains commas, do, for example,
% \def\ZZtitle{WIRELESS POWER TRANSFER:
% EFFICIENCY, FAR FIELD, DIRECTIVITY, AND PHASED ARRAY ANTENNAS}



% PurdueThesis.cls loads the rotating package which loads the graphicx
% package.  From page 12 of "Packages in the `graphics' bundle", 2021-03-05,
% retrieved 2021-06-16, at https://texdoc.org/serve/grfguide.pdf/0
%     \graphicspath{<dir-list>}
%
%         This optional declaration may be used to specify a list of
%         directories in which to search for graphics files.  The
%         format is the same as for the LaTeX 2e primitive \input@path.
%         A list of directories, each in a {} group (even if there is
%         only one in the list).  For example:
%             \graphicspath{{eps/}{tiff/}}
%         would cause the system to look in the subdirectories eps and
%         tiff of the current directory.  (All modern TeX systems use /
%         as the directory separator, even on Windows.)
%
%         The default setting of this path is \input@path that is:
%         graphics files will be found whereever TeX files are found.
%
% Look in the "graphics" subfolder for graphics files.
% This is done to reduce the number of files in the main thesis folder
% so the ones in there are easier to find.
\graphicspath{{graphics/}}

% Look in the "packages" subfolder for packages.
% This is done to reduce the number of files in the main thesis folder
% so the ones in there are easier to find.
\makeatletter
  \def\input@path{{packages/}}
\makeatother

%
% Configure bibliography.
%
% Automatically configure the bibliography.  Based on the
% institution, campus, and program listed in the \documentclass
% command \ZZBibProcessor is set to "BibLaTeX" or "BibTeX".
% For BibLaTeX, a
%    \usepackage[...]{biblatex}
% is done.  Put your bibliography entries in all-biblatex.bib.
% For BibTeX, a
%     \bibliographystyle{...}
% command is done.  Put your bibliography entries in all-bibtex.bib.
%
% All combinations of institution, campus, and program use BibLaTeX.
% Exceptions that use BibTeX:
%     o  "Purdue University", "West Lafayette", "Earth, Atmospheric,
%        and Planetary Sciences" uses the ametsoc2014 bibliography style.
%     o  "Purdue University", "West Lafayette", "Veterinary Clinical
%        Sciences" uses the ama bibliography style.
%
% To override the default choices picked by \ConfigureBibliography, change,
% for example,
%     \ConfigureBibliography
% to
%     % \ConfigureBibliography
%     \newcommand{\ZZBibProcessor}{BibLaTeX}
%     \usepackage[backend=biber, citestyle=apa, dashed=false, sortcites=true, style=apa]{biblatex}
%     \addbibresource{all-biblatex.bib}

\ConfigureBibliography

%
% This is only done if you are using BibLaTeX.
%
%
% If you don't want to ignore urldate fields,
% comment out (put "%" before) the next ten lines.
%
\DeclareSourcemap
  {
    \maps[datatype=bibtex]
    {
      % Ignore "urldate = {...}" in .bib files.
      % See the first complete example on page 201 of
      %     https://mirrors.rit.edu/CTAN/macros/latex/contrib/biblatex/doc/biblatex.pdf
      \map
        {
          \step[fieldset=urldate, null]
        }
        % Enter approximate (circa) dates using, for example,
        % "year = c2020"  See
        %     https://tex.stackexchange.com/questions/224617/what-is-the-correct-way-to-handle-approximate-dates-in-biblatex
      \map[overwrite=false]
        {
          \step[fieldsource=year]
          \step[fieldset=sortyear, origfieldval, final]
          \step[fieldsource=sortyear, match={c}, replace={}]
        }
    }
  }

% To let {\bfseries\scshape text} work as expected.
% See
%     https://tex.stackexchange.com/questions/27411/small-caps-and-bold-face
\usepackage{bold-extra}

% For chemical figures.
\usepackage{chemfig}
\usepackage{bm}
\usepackage{braket}
\usepackage{amsmath}

% For typesetting cryptography pseudocode, algorithms, and protocols.
% See
%     https://mirror.las.iastate.edu/tex-archive/macros/latex/contrib/cryptocode/cryptocode.pdf
\usepackage
[
  n,            % or lambda
  advantage,
  operators,
  sets,
  adversary,
  landau,
  probability,
  notions,
  logic,
  ff,
  mm,
  primitives,
  events,
  complexity,
  oracles,
  asymptotics,
  keys,
]
{cryptocode}

% Define
%    \VerbatimInput[options]{filename}
%    \begin{VerbatimOut}{filename} ... \end{VerbatimOut}.
\usepackage{fancyvrb}
  \DefineShortVerb{\|}  % so "|verbatim|" will be verbatim

% For \InpuutIfFileExists.
\usepackage{filehook}

% So "_" will work in URLs when using BibTeX.
\usepackage[T1]{fontenc}

% For nlui testing.
\usepackage{listings}

% For chemical equations.
% See
%     https://ctan.org/pkg/mhchem?lang=en
% From the "Package documentation" linked-to document
%     mhchem needs a couple of other packages.
%     For instance, expl3, amsmath and calc.
\usepackage[version=4]{mhchem}
  % If I'm loading the package to just define a few new commands I'll indent
  % two spaces right after loading the package and define the few new
  % commands here.  If I'm defining more than a few commands I usually do it
  % after loading all the packages.
  % Define "\nitrate" to be the chemical symbol for nitrate.
  \newcommand{\nitrate}{\ce{NO3{-}}}
  % Define "\pnitrate" (short for "parenthesized nitrate") to be the chemical
  % symbol for nitrate surrounded by parentheses.
  \newcommand{\pnitrate}{(\nitrate)}
  % "Define \vpnitrate" (short for "verbose parenthesized nitrate") to be
  % the word "nitrate" followed by a space followed by the chemical symbol
  % for nitrate with parentheses around it.
  \newcommand{\vpnitrate}{nitrate (\nitrate)}

% For
%     \cancel
%     \highlight
% See
%     http://ftp.math.purdue.edu/mirrors/ctan.org/macros/latex/contrib/siunitx/siunitx.pdf
% pages 11--12.
\usepackage{cancel}


% Redefine description, enumerate, and itemize lists.
% See
%     https://mirrors.concertpass.com/tex-archive/macros/latex/contrib/enumitem/enumitem.pdf
% \usepackage{enumitem}
% \setlist[itemize]{leftmargin=7pt,rightmargin=24pt}



% This gets rid of
%     [5] (./thesis.toc
%     ! Undefined control sequence.
%     \vbox_set:Nn ...box:D {\color_group_begin: #2\par
%                                                       \color_group_end: }
%     l.32 ...}Basic Circuit Components}{31}{section.67}
%                                                       %
%     ?
% and
%     [6]
%     ! Undefined control sequence.
%     \vbox_set:Nn ...box:D {\color_group_begin: #2\par
%                                                       \color_group_end: }
%     l.61 ...rline {P.1}Frenchspacing}{67}{section.445}
%                                                       %
%     ?
% errors.
% See
%     https://github.com/latex3/latex2e/issues/73
\usepackage{etoc}

% Define \setmaxprintline{number_of_columns}.
% \usepackage{hardwrap}

% For indexing.  Making an index is optional.
% Make these commands available:
%     COMMAND           DESCRIPTION
%     \index{string}    put "string" in index information
%     \makeindex        save information to make the index
%     \printindex       print the index
% See
%     https://ctan.org/pkg/makeidx?lang=en
% for more information.
\usepackage{makeidx}
  % By default \index ignores its argument.
  % This activates indexing.
  \makeindex
  % The "chapter name" for the index.
  \renewcommand{\indexname}{INDEX}

% The mathtools package
% (see http://mirror.utexas.edu/ctan/macros/latex/required/amsmath/amsmath.pdf)
% loads the amsmath package which defines the
%     align
%     align*
%     alignat
%     alignat*
%     equation
%     equation*
%     flalign
%     flalign*
%     gather
%     gather*
%     multitaper
%     multitaper*
%     split
% environments and extends amsmath by defining many other commands.
% See
%     https://ctan.org/pkg/amsmath
% for information about amsmath and
%     http://ctan.math.washington.edu/tex-archive/macros/latex/contrib/mathtools/mathtools.pdf
% for information about mathtools.
\usepackage{mathtools}

% Define \includemedia.
\usepackage{media9}

% Define \begin{multicols}{number_of_columns} ... \end{multicolumns}.
% Used in ap-text.tex.
\usepackage{multicol}

% Define \ditto.
\usepackage{pa-ditto}

% Define \FigureDash.
% \FigureDash is a dash the width of a digit in the current font.
\usepackage{pa-figure-dash}

% For PurdueThesis, PuTh, TeX, LaTeX, METAFONT, METAPOST, etc. related logos.
\usepackage{pa-logos}

% (Or maybe use isomath instead?  -mark  2021-06-20)
% Follow ISO 80000-2:2019
%     o   put e, i, j, and pi in upright font automatically
%     o   use, for example, "\di x" to get "\,mathrm{d}\/x"
% This loads
%     o   amsmath.sty (which is already loaded above)
%     o   mathtools.sty
%     o   upgreek.sty
% Load the package.
\usepackage{pa-mismath}
  % Tell mismath to put e, i, j, and pi in upright font automatically.
  \enumber
  \inumber
  \jnumber
  \pinumber
  % To typeset math italic e, i, j, and pi use
  %     \mathit e
  %     \mathit i
  %     \mathit j
  %     \itpi

% Define \MyRepeat{what}{repeat}.
% Do "what" "repeat" number of times.
\usepackage{pa-repeat}

% Define \FloatBarrier.
% \FloatBarrier process all unproccesed floats (tables, figures, etc.).
\usepackage{placeins}

% Define \hl.
% Undefine \st so soul will load without an error.
% I hope \st wasn't used for something important!
\let\st\relax
\usepackage{soul}

% Define \textcent.
\usepackage{textcomp}

% !!! This doesn't work yet, figure it out later.
% For \textprimstress.
% \usepackage{tipa}

% Needed for chapter "Graphics", section "TikZ and PGF".
\usepackage{tikz}
  % Needed to customize arrows.
  \usetikzlibrary{arrows.meta}
  % For electrical diagrams.
  % Uses the TikZ package.
  % The circuitikz name is short for "circuit TikZ".
  \usepackage{circuitikz}
  %
  \usepackage{menukeys}
  %
  % Needed for 3D TikZ stuff.
  \usetikzlibrary{3d}
  %
  % Needed for pa-typographic-conventions package.
  \usetikzlibrary{calc,shadows,shapes.misc,shapes.symbols}
  %
  % Needed for putting text along a path.
  \usetikzlibrary{decorations.text}
  %
  % Draw TikZ decorations.
  % Needed for at least the Kalman filter system model graphic.
  \usetikzlibrary{decorations.pathmorphing} % noisy shapes
  %
  % Fit shapes to coordinates.
  % Needed for at least the Kalman filter system model graphic.
  \usetikzlibrary{fit}
  %
  % Draw the background after the foreground.
  \usetikzlibrary{backgrounds}	% drawing the background after the foreground

% Needed for the Feynman diagram in ap-physics.tex.
% Tikz-feynman requires LuaLaTeX instead of pdflatex be run.
% LuaLaTeX screws up spacing in the list of figures so this
% is not loaded and LuaLaTeX should not be used.
\usepackage[compat=1.1.0]{tikz-feynman}

% The vertical space between a table heading and the table contents
% in a tabular environment.
\newcommand{\tabularspace}{\noalign{\vspace*{2pt}}}

% For \sfrac, used to do slanted fractions, similar to, e.g., 1/2,
% but 1 is small and high and 2 is small and low.
\usepackage{xfrac}


% Define \I.
% \I1 does \indent once, \I2 does \indent twice, etc.
\newcommand{\I}[1]{\MyRepeat{\indent}{#1}}

% Define \MyI.
% Typeset my input.
\long\def\MyI#1%
  {%
    {%
      \fontsize{8}{10}\tt
      \VerbatimInput
        [
          firstnumber = 1,
          numbers     = left,
          xleftmargin = 0.33in,
        ]%
        {#1}
    }%
  }

% Define \MyIO.
% Typeset my input and output.
% The input will all be on the same page.
% The output may be split over multiple pages.
\newcommand{\MyIO}
  {%
    \input{z.out}

    {%
      \fontsize{8}{10}\tt
      \VerbatimInput
        [
          firstnumber = 1,
          numbers     = left,
          xleftmargin = 0.33in,
        ]
        {z.out}
    }
    \FloatBarrier
  }

% Define \MyIOS.
% Typeset my input and output.
% The input may be split over multiple pages.
% The output may be split over multiple pages.
% This doesn't work right:
%     o  Putting a \vbox around the input and output
%        does not allow todoindex entries to be listed.
%     o  Using \vfilneg at beginning and end of definition
%        screws up vertical spacing.
% \newcommand{\MyIOS}
% {%
%   \input{z.out}
%
%   {%
%     \fontsize{8}{10}\tt
%     \VerbatimInput
%     [
%       firstnumber = 1,
%       numbers     = left,
%       xleftmargin = 0.33in,
%     ]{z.out}%
%   }
% }

% Define \MyIOT.
% Typset my input and output together on the same page.
% This doesn't work right:
%     o  Putting a \vbox around the input and output
%        does not allow todoindex entries to be listed.
%     o  Using \vfilneg at beginning and end of definition
%        screws up vertical spacing.
% \def\MyIOT
% {%
%   \vfilneg
%   % \vbox
%   {%
%     \input{z.out}%
%     \fontsize{8}{10}\tt
%     \VerbatimInput[
%       firstnumber = 1,
%       numbers     = left,
%       xleftmargin = 0.33in,
%     ]{z.out}%
%   }%
%   \FloatBarrier
%   \vfilneg
% }

% Define \NL (newline) so LaTeX goes to the next output line.
% Just doing \\ complains
%     ! LaTeX Error: There's no line here to end.
% \mbox{} is an empty math box.
\newcommand{\NL}{\mbox{}\\}

% Print a list of files used and their version numbers in the log file.
\listfiles


% \def\bibindent{0em}
% Customize the bibliography.
% \DefineBibliographyStrings{english}{
%   urlfrom = {URLFROM},
%   urlseen = {URLSEEN}
% }

% For typographical conventions stuff including
%     \Emph{...}
%     \First{...}
%     \Keys{...}
%     \Literal{...}
%     \Menu{...}
%     \Place{...}
%     \Shell{...}
% This must be after
%     \usepackage{tikz}
\usepackage{pa-typographic-conventions}


% For the \begin{example} ... \end{example} environment
% used in ap-linguistics.tex.
\usepackage{covington}
\usepackage{slgloss}

% "CTAN---Comprehensive" did not get hyphenated and extended
% into the right margin when using BibLaTeX and the apa style.
% These did not change it:
%     \hyphenation{Com-pre-hen-sive}
%     \hyphenation{CTAN---Com-pre-hen-sive}
% I changed    publisher = {CTAN---Comprehensive TeX Archive Network},
% to           publisher = {CTAN---Com\-pre\-hen\-sive TeX Archive Network},
% in my all-biblatex.bib file and it worked as expeceted.
% If you need to change the hyphenation points of a word in the text
% you can do, for example,
%     \hyphenation{ve-ry-od-dly-hy-phen-at-ed}


\begin{document}

\setcounter{tocdepth}{3}

\maketitle

% Define front matter
%     dedication
%     acknowledgments
%     preface
%     table of contents
%     list of tables
%     list of figures
%     list of symbols
%     abbreviations
%     nomenclature
%     glossary
%     abstract
\ProvidesFile{ch-front.tex}[2022-10-05 front matter chapter]
%
%  This is the ``front matter'' for the thesis.
%
%  REFERENCES
%
%    TCMOS17
%      The Chicago Manual of Style Online, 17th edition.
%      https://www.chicagomanualofstyle.org/home.html
%      retrieved on 2020-02-29
%
%    TEMPL
%      Thesis and Disertation Office Templates.
%      https://www.purdue.edu/gradschool/research/thesis/templates.html
%      retrieved on 2020-02-29
%
%    WNNCD
%    Webster's Ninth New Collegiate Dictionary.
%

%
%   Only Purdue University uses this page
%
%   Comment out \begin{statement} through \end{statement}
%   if you are not at Purdue University.
%
% Statement of Thesis/Dissertation Approval Page
% This page is REQUIRED.  The page should be numbered "2"
% and should NOT be listed in your TABLE OF CONTENTS.
\begin{statement}
  % Delete or add \entry commands as needed for all committe members.
  \entry{Dr. Gautam Vemuri, Chair}{Department of Physics}
  \entry{Dr. Jing Liu}{Department of Physics}
  \entry{Dr. Ruihua Cheng}{Department of Physics}
  \entry{Dr. Stephen Wassall}{Department of Physics}
  \entry{Dr. Horia Petrache}{Department of Physics}
  % There should be one \approvedby command containing the
  % "FORM 9 THESIS FORM HEAD NAME HERE" (from TEMPL, retrieved on 2020-03-01).
  \approvedby{TBD}
\end{statement}

% Dedication page is optional.
% A name and often a message in tribute to a person or cause.
% References: WEB9 332.
\begin{dedication}
I dedicate this thesis to Alexandra, who has steadily encouraged my pursuit of a doctoral degree. I am forever indebted for your patience, understanding, and proofreading.\\
\vspace{2in}
\textit{What we observe is not nature itself, but nature exposed to our method of questioning} \\Werner Heisenberg \\
\vspace{1in}
\textit{To deal with a 14-dimensional space, visualize a 3-dimensional space and say ``fourteen'' to yourself very loudly. Everyone does it}\\ Geoffrey Hinton \\
\vspace{1in}
\textit{Information is the resolution of uncertainty} \\Claude Shannon
\vspace{1in}

\end{dedication}

% Acknowledgements page is optional but most theses include
% a brief statement of appreciation or recognition of special
% assistance.
\begin{acknowledgments}

\end{acknowledgments}

% The preface is optional.
% References: TCMOS17 1.49, WEB9 927.
%\begin{preface}
%  This is the preface.
%\end{preface}

% The Table of Contents is required.
% The Table of Contents will be automatically created for you
% using information you supply in
%     \chapter
%     \section
%     \subsection
%     \subsubsection
%     commands.
\pdfbookmark{TABLE OF CONTENTS}{Contents}
\tableofcontents

% If your thesis has tables, a list of tables is required.
% The List of Tables will be automatically created for you using
% information you supply in
%     \begin{table} ... \end{table}
% environments.
\listoftables

% If your thesis has figures, a list of figures is required.
% The List of Figures will be automatically created for you using
% information you supply in
%     \begin{figure} ... \end{figure}
% environments.
\listoffigures

% If your thesis has listings, a list of listings is required.
% The List of Listings will be automatically created for you using
% information you supply in
%     \begin{ZZlisting} ... \end{ZZlisting}
% environments.
%\ZZlistoflistings

% If your thesis has protocols, you may want to do a list of protocols.
% The List of Protocols will be automatically created for you using
% information you supply in
%     \begin{protocol} ... \end{protocol}
% environments.
%\listofprotocols

% If your thesis has schemes, you may want to do a list of schemes.
% The List of Schemes will be automatically created for you using
% information you supply in
%     \begin{scheme} ... \end{scheme}
% environments.
%\listofschemes

% List of Symbols is optional.
\begin{symbols}
  $\bold{x}$& An image at base resolution\cr
  $\bold{y}$& An image at higher resolution\cr
  $\delta$& Pixel lateral width\cr
  $k$& Pixel index\cr
  $\theta$& A parameter\cr
  $u,v$& Cartesian coordinates in two-dimensions\cr
  $g_k$& Pixel-wise gain\cr
  $o_k$& Pixel-wise offset\cr
  $w_k$& Pixel-wise readout noise standard deviation\cr
  $\mu_k$& Pixel-wise expected value\cr
  $s_k$&  Pixel-wise  measured signal\cr
  $\xi_k$&  Pixel-wise measured readout noise\cr
  $O$& Point spread function\cr
  $\mathrm{erf}$& Error function\cr
  $\sigma_{\bold{x}}$& Gaussian PSF width\cr
  $\sigma_{\bold{y}}$& Kernel width for kernel density estimate\cr
  $\epsilon$& An image of pure Gaussian noise\cr
  $\beta$& Diffusion model noise variance\cr
  $\mathrm{SNR}$& Diffusion model signal to noise ratio\cr
  $\psi$& Diffusion model parameters\cr
  $\phi$& Augmentation network parameters\cr
  $\mathcal{L}$& An objective function\cr
  $D_{KL}$& KL-divergence\cr
  $p$& A discrete or continuous probability distribution\cr
  $\mathbb{E}_p$& Expectation with respect to a distribution $p$\cr
  $I$& Fisher information matrix\cr
  $\lambda$& Expected number of background counts per frame\cr
  $d$& Lateral dimension of a region of interest\cr
  $\zeta$& Photon detection probability\cr
  $\tau$& Delay time\cr
  $g^{(2)}(\tau)$& Second order coherence function\cr
  $B$& Expected number of signal-background coincidences per frame\cr
  $G^{(2)}(m)$& Measured number of signal-signal coincidences at lag time $m$\cr
  $N_{\mathrm{frames}}$& Number of frames\cr
  $N$& Number of active fluorescent emitters\cr
  $N^{*}$& Maximum a posteriori estimate of number of active emitters\cr
  $n$& Number of photon counts in a frame\cr
  $\ell$& Log-likelihood\cr
  $\hat{a}$& Ladder operator\cr
  $\hat{E}$& Electric field operator\cr
  $\rho$& Density matrix or number density for molecular dynamics\cr
  $L(r)$& Besag's L-function\cr
  $K(r)$& Ripley's K-function\cr
  $G(r)$& Nearest neighbor distribution function\cr
  $\lambda$& (Point Pattern Analysis) Intensity of a point process\cr
  $\gamma$& Friction tensor\cr
  $\xi$& A delta-correlated Gaussian noise\cr
  $k_{B}$& Boltzmann's constant\cr
  $T$& Temperature\cr
  $U$& Potential energy\cr
  $D$& Diffusion coefficient\cr
  $\epsilon$& (Molecular Dynamics) Energy\cr
  $r_{0}$& Harmonic bond equilibrium length\cr
  $R_{0}$& Binder potential equilibrium length\cr
  $\alpha$& Anomalous diffusion exponent\cr
  
  
\end{symbols}

% List of Abbreviations is optional.
%\begin{abbreviations}
%  abbr& abbreviation\cr
%  bcf& billion cubic feet\cr
%  BMOC& big man on campus\cr
%\end{abbreviations}



% Abstract is required.
% Note that the information for the first paragraph of the output
% doesn't need to be input here...it is put in automatically from
% information you supplied earlier using \title, \author, \degree,
% and \majorprof.
% Reference: PU 17.
\begin{abstract}%

Single-molecule localization microscopy (SMLM) techniques, such as direct stochastic optical reconstruction microscopy (dSTORM), can be used to produce a pointillist representation of fluorescently-labeled biological structures at diffraction-unlimited precision. Direct STORM approaches leverage the deactivation of standard fluorescent tags, followed by spontaneous or photoinduced reactivation, allowing resolution of fluorophores at distances below the diffraction limit. This dissertation introduces single molecule localization microscopy and covers its application as discussed in the following papers:

\textit{BRD4 phosphorylation regulates structure of chromatin nanodomains} [1] describes the role of the BRD4 phosphoswitch in the maintenance of chromatin nanodomains via super resolution microscopy and molecular dynamics simulation. We build on the notion that chromatin binding activity of BRD4 is regulated by phosphorylation by demonstrating that BRD4 phosphorylation regulated chromatin packing and mobility in mammalian nuclei.

\textit{Uncertainty-aware localization microscopy by variational diffusion} [2] describes a novel algorithm that leverages a diffusion model in order to model a posterior distribution on super-resolution localization microscopy images. 
Fast extraction of physically relevant information from images using deep neural networks has led to significant advances in fluorescence microscopy and its application to the study of biological systems. For example, the application of deep networks for kernel density (KD) estimation in single molecule localization microscopy (SMLM) has accelerated super-resolution imaging of densely-labeled structures in the cell. However, simple and interpretable uncertainty quantification is lacking in these applications, and remains a necessary modeling component in high-risk research. We propose a generative modeling framework for KD estimation in SMLM based on variational diffusion. This approach allows us to probe the structure of the posterior on KD estimates, creating an additional avenue toward quality control. We demonstrate that data augmentation with traditional SMLM architectures followed by a diffusion process permits simultaneous high-fidelity super-resolution with uncertainty estimation of regressed KDEs. 


\end{abstract}


%
% Put chapter \include commands here.
%

% Introductions may precede the first chapters or major divisions of theses.
% Reference: TM2017, page 31.

\ProvidesFile{ch1.tex}[Chapter1]

\chapter{Single molecule localization microscopy}
\ix{physics//Physics appendix}

\section{Introduction}

In the quest to understand cellular function, biologists aim to directly observe the processes enabling cells to maintain homeostasis and respond dynamically to internal and environmental cues at the molecular level. Super-resolution (SR) microscopy techniques have emerged as a pathway to this aim, surpassing the classical diffraction limit of optical resolution—roughly half the wavelength of emitted light. Fluorescence microscopy techniques continually push the resolution boundary towards nanometer scales, facilitating imaging of cellular structures with a level of detail previously achievable only with electron microscopy (EM). Concurrently, SR techniques retain optical microscopy advantages in biological experiments, including sample preservation, imaging flexibility, and target specificity. SR enables extraction of quantitative information on spatial distributions and often absolute numbers of proteins, nucleic acids, or other macromolecules within subcellular compartments.

Many SR methods are based on wide-field (WF), total internal reflection fluorescence (TIRF) or confocal microscope setups and fundamentally differ in how fluorescently labeled samples are excited and how the emitted photons are detected. Here, I focus on single-molecule localization microscopy (SMLM) techniques – a class of SR diffraction-unlimited SR methods which leverage fluorescence intermittency to resolve fluorophores in the sample who’s spatially overlapping point spread functions would otherwise render them unresolvable at the detector. SMLM approaches, such as direct-STORM (dSTORM) have become quite popular because they can be implemented at low cost on conventional, camera-based, wide-field setups, shifting the complexity to biological sample preparation and image post processing. Common strategies for the temporal separation of molecules involve transient intramolecular rearrangements to switch from dark to fluorescent states or the exploitation of non-emitting molecular radicals. For example, in dSTORM, rhodamine derivatives can undergo intersystem crossing to a triplet state, which can be reduced by thiols to form a dark radical species. The dark state can then be quenched by oxidative processes, driving the fluorophore back to its ground state. 



\subsection{Elementary theory of SMLM}


\begin{equation}
\omega = i_{0}\int O(u)du\int O(v)dv
\end{equation}

where $i_{0} = \eta N_{0}\Delta$. The optical impulse response $O(u,v)$ is often approximated as a 2D isotropic Gaussian with standard deviation $\sigma$ (Zhang 2007). The parameter $\eta$ is the photon detection probability of the sensor and $\Delta$ is the exposure time. $N_{0}$ represents the number of photons emitted.

Using the common definition $\mathrm{erf}(z) = \frac{2}{\sqrt{\pi}}\int_{0}^{t}e^{-t^{2}}dt$,

\begin{equation}
\int O(u)du = \frac{1}{2}\left(\mathrm{erf}\left(\frac{u_{k}+\frac{1}{2}-u_{0}}{\sqrt{2}\sigma}\right) -\mathrm{erf}\left(\frac{u_{k}-\frac{1}{2}-u_{0}}{\sqrt{2}\sigma}\right)\right)
\end{equation}

For the sake of generality, the number of photoelectrons at a pixel $k$, $\bold{s}_k$, is  multiplied by a gain factor $g_k \;[\mathrm{ADU}/e^{-}]$, which is often unity. The readout noise per pixel $\zeta_{k}$ can be Gaussian with some pixel-specific offset $o_{k}$ and variance $\sigma_{k}^{2}$. Ultimately, we have a Poisson component of the signal, which scales with $N_{0}$ and may have Gaussian component, which does not. Therefore, in a single exposure, we measure: 

\begin{equation}
\bold{x}_t = \bold{s}_t + \bold{\zeta}
\end{equation}

What we are after is the likelihood $p(\bold{x}_{t}\lvert\theta)$ where $\theta$ are the molecular coordinates. Fundamental probability theory states that the distribution of $\bold{x}_{k}$ is the convolution of the distributions of $\bold{s}_{k}$ and $\zeta_{k}$,

\begin{equation}
p(\bold{x}_{t}\lvert\theta) = A\sum_{q=0}^{\infty} \frac{1}{q!}e^{-\omega_{k}}\omega_{k}^{q}\frac{1}{\sqrt{2\pi}\sigma_{k}}e^{-\frac{(\bold{x}_{k}-g_{k}q-o_{k})}{2\sigma_{k}^{2}}}
\end{equation}

where $P(\zeta_{k}) = \mathcal{N}(o_{k},\sigma_{k}^{2})$ and $P(S_{k}) = \mathrm{Poisson}(g_{k}\omega_{k})$,  $A$ is some normalization constant. In practice, (4) is difficult to work with, so we look for an approximation. We will use a Poisson-Normal approximation for simplification. Consider,

\begin{equation}
\zeta_{k} - o_{k} + \sigma_{k}^{2} \sim \mathcal{N}(\sigma_{k}^{2},\sigma_{k}^{2}) \approx \mathrm{Poisson}(\sigma_{k}^{2})
\end{equation}

Since $\bold{x}_{k} = \bold{s}_{k} + \zeta_{k}$, we transform $\bold{x}_{k}' = \bold{x}_{k} - o_{k} + \sigma_{k}^{2}$, which is distributed according to 

\begin{equation}
\bold{x}_{k}' \sim \mathrm{Poisson}(\omega_{k}')
\end{equation}

where $\omega_{k}' = g_{k}\omega_{k} + \sigma_{k}^{2}$. This result can be seen from the fact the the convolution of two Poisson distributions is also Poisson. The quality of this approximation will degrade with decreasing signal level, since the Poisson distribution does not retain its Gaussian shape at low expected counts. Nevertheless, the quality of the approximation can be predicted by the Komogonov distance between the convolution distribution (4).

\subsection{The definition of resolution in SMLM}

The distribution of a particular biomolecule in the cell can be described as a probability density over a two-dimensional space, casting super-resolution as a density estimation problem. Intuitively, the spatial resolution of SMLM images then increases as we draw more samples from this density - a concept which is made mathematically precise by the so-called Fourier ring correlation or FRC. Using FRC, one can compute image resolution as the spatial frequency at which a correlation function in the frequency domain drops below a threshold, typically taken to be $1/7$ (See Supplement). According to this theory, reducing localization uncertainty while increasing the number of samples, results in an increase in image resolution (Nieuwenhuizen 2013). However, there remains a fundamental limit to the the minimal localization uncertainty which can be obtained.


Localization uncertainty, typically the RMSE of a maximum likelihood or similar statistical estimator, is bounded from below by the inverse of the Fisher information matrix, known as the Cramer-Rao lower bound (Chao 2016). Localization uncertainties in sparse conditions are often tens of nanometers, although recent work on integration of Bayesian priors with modulation enhanced SMLM (meSMLM) or structured illumination with MINFLUX, has reduced spatial resolution below to a few nanometers (Kalisvaart 2022, Gwosh 2020). Nevertheless, managing the increase in localization uncertainty at high labeling density remains a major bottleneck to SMLM. Static uncertainty due to molecular crowding can be partially amelioriated by using pairwise or higher-order temporal correlations within a pixel neighborhood, known as stochastic optical fluctuation imaging or SOFI (Dertinger 2009). Other approaches such as stimulated emission and depletion (STED) imaging bring control over the photophysical state of a chosen subset of the sample, yet the need for laser scanning prevents widespread application in live-cell studies. The spatial resolution and relative simplicity of SMLM techniques remains unmatched, inciting an effort to increase the resolution of SMLM techniques and explore avenues towards time resolved SMLM.



\subsection{The Cramer-Rao lower bound}

The Poisson approximation is also convenient for computing the Fisher information matrix for $\theta_{\mathrm{MLE}}$ and thus the Cramer-Rao lower bound, which bounds the variance of a statistical estimator of $\theta_{\mathrm{MLE}}$, from below (Chao 2016). The Fisher information is

\begin{equation}
I_{ij}(\theta) = \underset{\theta}{\mathbb{E}}\left(\frac{\partial \ell}{\partial\theta_{i}}\frac{\partial\ell}{\partial\theta_{j}}\right) 
\end{equation}

Let $\mu_{k}' = g_{k}\mu_{k} + \sigma_{k}^{2}$. For an arbitrary parameter,

\begin{align*}
\frac{\partial \ell}{\partial \theta_{i}} &= \frac{\partial}{\partial \theta_{i}} \sum_{k}  x_{k}\log x_{k} + \mu_{k}' - x_{k}\log\left(\mu_{k}'\right)\\
&= \sum_{k} \frac{\partial \mu_{k}'}{\partial\theta_{i}} \left(\frac{\mu_{k}'-x_{k}}{\mu_{k}'}\right)
\end{align*}

\begin{equation*}
I_{ij}(\theta) = \underset{\theta}{\mathbb{E}}\left(\sum_{k}\frac{\partial \mu_{k}'}{\partial\theta_{i}}\frac{\partial \mu_{k}'}{\partial\theta_{j}} \left(\frac{\mu_{k}'-x_{k}}{\mu_{k}'}\right)^{2}\right) = \sum_{k}\frac{1}{\mu_{k}'}\frac{\partial \mu_{k}'}{\partial\theta_{i}}\frac{\partial \mu_{k}'}{\partial\theta_{j}}
\end{equation*}

\section{Optical fluctuation microscopy}

\subsection{Spatial coherence for an isolated emitter}

Photoswitching fluorescent molecules are described in the density matrix formalism

\begin{equation*}
\rho = \sum_{k}\xi_{k}\ket{\alpha_{k}}\bra{\alpha_{k}}\;\; \sum_{k}\xi_{k} = 1
\end{equation*}


where $\ket{\alpha_{k}}$ is a coherent state with amplitude $\alpha_{k}$ i.e., $\langle n\rangle = \bra{\alpha_{k}} n\ket{\alpha_{k}} = \lvert\alpha_{k}^{2}\rvert$. Typically $\xi_{k}$ and $\langle n_{k}\rangle$ are heterogeneous. We consider a simplified model consisting of a single mode field 

\begin{equation*}
E^{+}(r_{i}) = h(r_{i}-s_{0})\hat{a}_{n}
\end{equation*}

\begin{equation*}
g^{(2)}_{ij}(0) = \frac{\langle E^{-}(r_{i})E^{-}(r_{j})E^{+}(r_{i})E^{+}(r_{j}) \rangle}{\langle E^{-}(r_{i})E^{+}(r_{i})\rangle\langle E^{-}(r_{j})E^{+}(r_{j})\rangle} = \frac{\mathrm{Tr}(E^{-}(r_{i})E^{-}(r_{j})E^{+}(r_{i})E^{+}(r_{j})\rho)}{\mathrm{Tr}(E^{-}(r_{i})E^{+}(r_{i})\rho)\mathrm{Tr}(E^{-}(r_{j})E^{+}(r_{j})\rho)}
\end{equation*}

Terms related to point spread function will cancel. It is instructive to compute

\begin{align*}
\mathrm{Tr}(a^{\dagger}a^{\dagger}aa \left(\xi_{k}\ket{\alpha_{k}}\bra{\alpha_{k}}\right) &= \mathrm{Tr}\left(\xi_{k} e^{-\lvert\alpha\rvert^{2}}\sum_{n,m}^{\infty}\frac{\alpha^{n}}{n!}\ket{n}\bra{m}\right)\\
&= \mathrm{Tr}\left(\xi_{k} e^{-\lvert\alpha\rvert^{2}}\sum_{n}^{\infty}\frac{\lvert\alpha\rvert^{2n}}{n!}n(n-1)\right)\\
&= \mathrm{Tr}\left(\xi_{k} e^{-\lvert\alpha\rvert^{2}}\sum_{n=2}^{\infty}\frac{\lvert\alpha\rvert^{2n}}{(n-2)!}\right)\\
&= \xi_{k}\lvert\alpha_{k}\rvert^{4}
\end{align*}

Similarly,

\begin{align*}
\mathrm{Tr}(a^{\dagger}a \left(\xi \ket{\alpha}\bra{\alpha}\right)) &= \mathrm{Tr}\left(\xi e^{-\lvert\alpha\rvert^{2}}\sum_{n,m}^{\infty}\frac{\alpha^{n}(\alpha^{m})^{*}}{\sqrt{n!}\sqrt{m!}}a^{\dagger}a\ket{n}\bra{m} \right)\\
&= \xi e^{-\lvert\alpha\rvert^{2}}\sum_{n=0}^{\infty}\frac{(\lvert\alpha\rvert^{2})^{n}}{n!}n\\
&= \xi e^{-\lvert\alpha\rvert^{2}}\sum_{n=1}^{\infty}\frac{(\lvert\alpha\rvert^{2})^{n}}{(n-1)!}\\
&= \xi e^{-\lvert\alpha\rvert^{2}}\left(\lvert\alpha\rvert^{2} + \frac{\lvert\alpha\rvert^{4}}{1!} + \frac{\lvert\alpha\rvert^{6}}{2!}+...\right)\\
&= \xi e^{-\lvert\alpha\rvert^{2}}\lvert\alpha\rvert^{2}\left(1 + \frac{\lvert\alpha\rvert^{2}}{1!} + \frac{\lvert\alpha\rvert^{3}}{2!}+...\right)\\
&= \xi e^{-\lvert\alpha\rvert^{2}}e^{\lvert\alpha\rvert^{2}}\lvert\alpha\rvert^{2} = \xi\lvert\alpha\rvert^{2}
\end{align*}

\begin{align*}
\mathrm{Tr}(a a^{\dagger} \left(\xi \ket{\alpha}\bra{\alpha}\right)) &= \mathrm{Tr}\left(\xi e^{-\lvert\alpha\rvert^{2}}\sum_{n,m}^{\infty}\frac{\alpha^{n}(\alpha^{m})^{*}}{\sqrt{n!}\sqrt{m!}}a a^{\dagger}\ket{n}\bra{m} \right)\\
&= \xi e^{-\lvert\alpha\rvert^{2}}\sum_{n=0}^{\infty}\frac{(\lvert\alpha\rvert^{2})^{n}}{n!}(n+1)\\
&= \xi e^{-\lvert\alpha\rvert^{2}}\left(\sum_{n=1}^{\infty}\frac{(\lvert\alpha\rvert^{2})^{n}}{(n-1)!} + e^{\lvert\alpha\rvert^{2}}\right)\\
&= \xi e^{-\lvert\alpha\rvert^{2}}\left(\lvert\alpha\rvert^{2}e^{\lvert\alpha\rvert^{2}} + e^{\lvert\alpha\rvert^{2}}\right) = \xi(\lvert\alpha\rvert^{2} + 1)
\end{align*}

Putting it all together yields a simple expression for the two-point coherence function

\begin{equation*}
g^{(2)}_{ij}(0) = \frac{\sum_{k}\xi_{k}\lvert\alpha_{k}\rvert^{4}}{\left(\sum_{k}\xi_{k}\lvert\alpha_{k}\rvert^{2}\right)\left(\sum_{k}\xi_{k}\lvert\alpha_{k}\rvert^{2}\right)}
\end{equation*}

For example, if we have a two-level system consisting of a fluorescent state with amplitude $\alpha$ and the vacuum state, this becomes

\begin{equation*}
g^{(2)}_{ij}(0) = \frac{\xi\lvert\alpha\rvert^{4}}{\xi^{2}\lvert\alpha\rvert^{4}} = \frac{1}{\xi}
\end{equation*}

As $\xi\rightarrow 1$ (always on) we recover a coherent state. As $\xi\rightarrow 0$ we observe $g^{(2)}_{ij}(0) > 1$ i.e., bunching.

\subsection{Generalization to nonzero background}

\begin{equation*}
E_{0}^{+}\sim \sum_{j=1}^{M}\delta(s-s_{j})a_{j} \;\; E^{+}(r_{i}) = \int d^{2}s E_{0}^{+} = \sum_{n}h(r_{i}-s_{n})a_{n}
\end{equation*}

\begin{equation*}
\rho_{S} = \xi\ket{\alpha}\bra{\alpha} + (1-\xi)\ket{0}\bra{0}\;\;\rho_{B} = \ket{\beta}\bra{\beta}\;\;\rho = \rho_{S}\otimes\rho_{B}
\end{equation*}

\begin{equation*}
E(r_{i})^{+} = E_{S}(r_{i})^{+} + E_{B}(r_{i})^{+} = h(r_{i}-s_{n})a_{S} + a_{B}
\end{equation*}

\begin{align*}
G^{2}_{ij}(0) &= \langle(E_{S}^{\dagger} + E_{B}^{\dagger}) (E_{S}^{\dagger} + E_{B}^{\dagger})( E_{S} + E_{B}) (E_{S} + E_{B})\rangle \\
&= h_{i}^{2}h_{j}^{2}\langle a_{S}^{\dagger}a_{S}^{\dagger}a_{S}a_{S}\rangle + h_{i}^{2}\langle a_{S}^{\dagger}a_{B}^{\dagger}a_{S}a_{B}\rangle + h_{j}^{2}\langle a_{B}^{\dagger}a_{S}^{\dagger}a_{B}a_{S}\rangle  + \langle a_{B}^{\dagger}a_{B}^{\dagger}a_{B}a_{B}\rangle  \\
&= \xi(h_{i}^{2}h_{j}^{2}\lvert\alpha\rvert^{4}+ h_{i}^{2}\lvert\alpha\rvert^{2}\lvert\beta\rvert^{2} + h_{j}^{2}\lvert\alpha\rvert^{2}\lvert\beta\rvert^{2}\rangle  + \lvert\beta\rvert^{4} ) \\
&= \xi(h_{i}^{2}h_{j}^{2}\lvert\alpha\rvert^{4}+ \lvert\alpha\rvert^{2}\lvert\beta\rvert^{2}(h_{i}^{2} + h_{j}^{2})  + \lvert\beta\rvert^{4}) \\
\end{align*}

The normalized second order coherence function then reads

\begin{align*}
g^{2}_{ij}(0) &= \frac{\xi h_{i}^{2}h_{j}^{2}N_{0}^{2} + \xi N_{0}B_{0}(h_{i}^{2} + h_{j}^{2}) + B_{0}^{2}}{\xi^{2} h_{i}^{2}h_{j}^{2}N_{0}^{2} + \xi N_{0}B_{0}(h_{i}^{2}+h_{j}^{2}) +  B_{0}^{2}}
\end{align*}

Notice the PSF factor $h_{i}$ appears squared. This squared value can be seen as the probability of photon detection at a point $s_i$, while $h_{i}$ is the amplitude of the electric field. 

Note that, even though Markov jump processes are non-ergodic, a set of occupancy probabilities $\xi_k$ are sufficient remains sufficient to compute zero lag second order coherence. This is because the temporal structure of the hidden state dynamics is not considered when computing the zero-lag coherence and the jump processes are independent.

\section{Appendix}

We will derive the gradients for the integrated astigmatic Gaussian, since it is the more general case. As before, define $i_{0} = g_{k}\gamma\Delta t N_{0}$ such that $\mu_{k}' = i_{0}\lambda_{k}$

\begin{equation*}
J_{x_{0}} = \beta_{k}\lambda_{y}\frac{\partial \lambda_{x}}{\partial x_{0}} \;\; J_{y_{0}} = \beta_{k}\lambda_{x}\frac{\partial \lambda_{y}}{\partial y_{0}}\;\;\; J_{z_{0}}  = \frac{\partial \mu_{k}'}{\partial \sigma_{x}}\frac{\partial \sigma_{x}}{\partial z_{0}} + \frac{\partial \mu_{k}'}{\partial \sigma_{y}}\frac{\partial \sigma_{y}}{\partial z_{0}}
\end{equation*}

\begin{align*}
J_{x_{0}} &= \beta_{k}\lambda_{y}\frac{\partial \lambda_{x}}{\partial x_{0}} \\
&= \frac{\beta_{k}\lambda_{y}}{2}\frac{\partial}{\partial x_{0}}\left(\mathrm{erf}\left(\frac{x_{k}+\frac{1}{2}-x_{0}}{\sqrt{2}\sigma_{x}}\right) -\mathrm{erf}\left(\frac{x_{k}-\frac{1}{2}-x_{0}}{\sqrt{2}\sigma_{x}}\right)\right)\\
&= \frac{\beta_{k}\lambda_{y}}{\sqrt{2\pi}\sigma_{x}}\left(\mathrm{exp}\left(\frac{(x_{k}-\frac{1}{2}-x_{0})^{2}}{2\sigma_{x}^{2}}\right) -\mathrm{exp}\left(\frac{(x_{k}+\frac{1}{2}-x_{0})^{2}}{2\sigma_{x}^{2}}\right)\right)
\end{align*}

\begin{align*}
J_{y_{0}} &= \beta_{k}\lambda_{x}\frac{\partial \lambda_{y}}{\partial y_{0}} \\
&= \frac{\beta_{k}\lambda_{x}}{2}\frac{\partial}{\partial y_{0}}\left(\mathrm{erf}\left(\frac{y_{k}+\frac{1}{2}-y_{0}}{\sqrt{2}\sigma_{y}}\right) -\mathrm{erf}\left(\frac{y_{k}-\frac{1}{2}-y_{0}}{\sqrt{2}\sigma_{y}}\right)\right)\\
&= \frac{\beta_{k}\lambda_{x}}{\sqrt{2\pi}\sigma_{y}}\left(\mathrm{exp}\left(\frac{(y_{k}-\frac{1}{2}-y_{0})^{2}}{2\sigma_{y}^{2}}\right) -\mathrm{exp}\left(\frac{(y_{k}+\frac{1}{2}-y_{0})^{2}}{2\sigma_{y}^{2}}\right)\right)
\end{align*}

\begin{align*}
J_{\sigma_{x}} &= \beta_{k}\lambda_{y}\frac{\partial \lambda_{x}}{\partial \sigma_{x}} \\
&= \frac{\beta_{k}\lambda_{y}}{2}\frac{\partial}{\partial \sigma_{x}}\left(\mathrm{erf}\left(\frac{x_{k}+\frac{1}{2}-x_{0}}{\sqrt{2}\sigma_{x}}\right) -\mathrm{erf}\left(\frac{x_{k}-\frac{1}{2}-x_{0}}{\sqrt{2}\sigma_{x}}\right)\right)\\
&= \frac{\beta_{k}\lambda_{y}}{\sqrt{2\pi}}\left(\frac{\left(x-x_{0}-\frac{1}{2}\right) e^{-\frac{\left(x-x_{0}-\frac{1}{2}\right)^2}{2 \sigma_{x} ^2}}}{\sigma_{x} ^2}-\frac{ \left(x-x_{0}+\frac{1}{2}\right) e^{-\frac{\left(x-x_{0}+\frac{1}{2}\right)^2}{2 \sigma_{x} ^2}}}{\sigma_{x} ^2}\right)
\end{align*}

\begin{align*}
J_{\sigma_{y}} &= \beta_{k}\lambda_{x}\frac{\partial \lambda_{y}}{\partial \sigma_{y}} \\
&= \frac{\beta_{k}\lambda_{x}}{2}\frac{\partial}{\partial \sigma_{y}}\left(\mathrm{erf}\left(\frac{y_{k}+\frac{1}{2}-y_{0}}{\sqrt{2}\sigma_{y}}\right) -\mathrm{erf}\left(\frac{y_{k}-\frac{1}{2}-y_{0}}{\sqrt{2}\sigma_{y}}\right)\right)\\
&= \frac{\beta_{k}\lambda_{x}}{\sqrt{2\pi}}\left(\frac{\left(y-y_{0}-\frac{1}{2}\right) e^{-\frac{\left(y-y_{0}-\frac{1}{2}\right)^2}{2 \sigma_{y} ^2}}}{\sigma_{y} ^2}-\frac{ \left(y-y_{0}+\frac{1}{2}\right) e^{-\frac{\left(y-y_{0}+\frac{1}{2}\right)^2}{2 \sigma_{y} ^2}}}{\sigma_{y} ^2}\right)
\end{align*}

Luckily, computing the Hessian matrix for (2.9) is tractable, and is actually quite simple when one takes advantage of the chain rule for Hessian matrices. Looking at (2.9), the likelihood is a hierarchical function that maps a vector space $\Theta$ to a vector space $\Lambda$ to a scalar value. Formally, we define $T: \Theta \rightarrow \Lambda$ and $W: \Lambda \rightarrow \mathbb{R}$. The parameter vector $(x_{0},y_{0},z_{0}, \sigma_{0}, N_{0})\in \Theta$, the Poisson rate vector $\vec{\lambda} \in \Lambda$ and $\ell \in \mathbb{R}$. Note that we choose to optimize $\sigma_{x}$ and $\sigma_{y}$ directly and compute $z_{0}$ to simplify the computation of the Hessian. To get the Hessian, we need the chain-rule for Hessian matrices, which can be quickly computed in terms of the jacobian and hessian of $T$ and $W$.


\begin{equation*}
H_{\ell} = J_{\mu}^{T} H_{\ell} J_{\mu} + (J_{\ell}\otimes I_{n})H_{\mu}
\end{equation*}

where we have used $J_{\mu}$ to represent the jacobian of $T$ and $J_{\ell}$ for the jacobian of $W$. Similar notation is used for the corresponding Hessian matrices. 
In the 3D case, the Hessian matrix is not directly separable since $\mu \propto \lambda_{x}(x_{0},\sigma_{0},\sigma_{x})\lambda_{y}(y_{0},\sigma_{0},\sigma_{y})$. To see this, an abstract representation of the Hessian reads 


\subsection{Fisher information for 2D integrated gaussian}

For the 2D integrated gaussian point spread function, the Hessian only contains separable second order derivatives, so the Fisher information matrix takes on a convenient form

\begin{equation}
I_{ij}(\theta) = \underset{\theta}{\mathbb{E}}\left(\frac{\partial \ell}{\partial\theta_{i}}\frac{\partial\ell}{\partial\theta_{j}}\right) 
\end{equation}

For an arbitrary parameter then we have

\begin{align*}
\frac{\partial \ell}{\partial \theta_{i}} &= \frac{\partial}{\partial \theta_{i}} \sum_{k}  x_{k}\log x_{k} + \mu_{k}' - x_{k}\log\left(\mu_{k}'\right)\\
&= \sum_{k} \frac{\partial \mu_{k}'}{\partial\theta_{i}} \left(\frac{\mu_{k}'-x_{k}}{\mu_{k}'}\right)
\end{align*}

\begin{equation*}
I_{ij}(\theta) = \underset{\theta}{\mathbb{E}}\left(\sum_{k}\frac{\partial \mu_{k}'}{\partial\theta_{i}}\frac{\partial \mu_{k}'}{\partial\theta_{j}} \left(\frac{\mu_{k}'-x_{k}}{\mu_{k}'}\right)^{2}\right) = \sum_{k}\frac{1}{\mu_{k}'}\frac{\partial \mu_{k}'}{\partial\theta_{i}}\frac{\partial \mu_{k}'}{\partial\theta_{j}}
\end{equation*}

To compute the bound, it turns out all we need is the jacobian $\frac{\partial \mu_{k}'}{\partial\theta_{j}} $.



\ProvidesFile{ch2.tex}[Chapter2]

\chapter{Bromodomain protein 4 and chromatin organization}
\ix{physics//Physics appendix}

The interplay between chromatin structure and phase-separating proteins is an emerging topic in cell biology with implications for understanding disease states. Here, we investigate the functional relationship between bromodomain protein 4 (BRD4) and chromatin architecture. By combining molecular dynamics simulations with live-cell imaging, we demonstrate that BRD4, when phosphorylated at specific N-terminus sites, significantly impacts nucleosome nanodomain (NN) organization and dynamics. Our findings reveal that enhanced chromatin binding activity of BRD4 condenses NNs, while both loss or gain of BRD4 chromatin binding reduced diffusion of single nucleosomes, suggesting a role for BRD4 in the regulation of nanoscale chromatin architecture and the chromatin microenvironment. These observations shed light on the nuanced regulation of chromatin structure by BRD4, offering insights into its role in maintaining the nuclear architecture and transcriptional activity.

\section{Introduction}


\ProvidesFile{ap-physics.tex}[2022-10-05 Physics appendix]

\chapter{Denoising diffusion probabilistic models for blind deconvolution}
\ix{physics//Physics appendix}




%\include{ch-do-not-use-these-packages}

% Summary and/or conclusions are optional but often used.
% The summary and/or conclusions often are the last
% the last major division(s) of the text.
% Reference: TM2017 page 32.
%\include{ch-summary}

% Recommendations are optional.
% You may include recommendations as a major division if your
% subject matter and research dictate.
% Reference: TM2017 page 32.
%\include{ch-recommendations}

% Test \begin{refsection}...\end{refsection}.
%\include{ch-test}

% \immediate\setlength{\bibhang}{-3in}
% \immediate\setlength{\itemindent}{3in}
% \immediate\setlength{\rightmargin}{3in}

%
% This is only done if you are using BibLaTeX.
%
\makeatletter  % commented out on 2022-01-26
  \defbibenvironment{bibliography}
    {%
      \list
        {%
          \printtext[labelnumberwidth]%
          {%
            \printfield{prefixnumber}%
            \printfield{labelnumber}%
          }%
        }%
        {%
          \setlength{\bibhang}{1in} %%%%% was 0pt
          \setlength{\itemindent}{1in}%  -\leftmargin} %%%%% was 0pt
          \setlength{\itemsep}{\bibitemsep}%
          \setlength{\leftmargin}{0pt}%  .22in} % 0.42in}
          \setlength{\parsep}{\bibparsep}%
           \setlength{\rightmargin}{0.33in}%
        }%
    }
    {\endlist}
    {\item}
\makeatother  % commented out on 2022-01-26

% \immediate\setlength{\labelnumberwidth}{1.5in} %%%%% was commented out
\setlength{\labelwidth}{1.5in}
\def\sllnsez{[999] }

{%
  % Make _ in URLs visible.
  % \def\t{\char'137}%
  \catcode`*=\active
  \def*{\char'137}%  \char'137 is _
  \PrintBibliography
}

% Appendices are optional.  Not all theses contain appendices.
% An appendix is used for supplementary illustrative material,
% original data, computer programs, and other material that is not
% necessarily appropriate for inclusion within the text of your
% thesis.
% Reference: TM2017 page 33.
%
% Use ``\appendix'' for one appendix or ``\appendices'' for more than
% one appendix.
\appendices

% My filename conventions:
%     FILE THAT START WITH    ARE
%     ap-                     appendices
%     ch-                     chapters
%     gr-                     graphics
%     pa-                     packages
%     z                       temporary files

  % "About Appendices" appendix.
  %\include{ap-about-appendices}

  % "Bugs" appendix.
  %\include{ap-bugs}

  % Check margins.
  %\include{ap-check-margins}

  % Demonstrate how to do separate appendices per chapter.
  %\include{ap-chapter-appendices}

  % Demonstrate how to do separate references per chapter.
  %\include{ap-chapter-references}

  % Citations and references.
  %\ProvidesFile{appendix.tex}[Appendix]

\chapter{Appendix A}
\ix{physics//Physics appendix}

\subsection{Interpolation by second order coherence}

Photoswitching fluorescent molecules are described in the density matrix formalism

\begin{equation*}
\rho = \sum_{k}\xi_{k}\ket{\alpha_{k}}\bra{\alpha_{k}}\;\; \sum_{k}\xi_{k} = 1
\end{equation*}


where $\ket{\alpha_{k}}$ is a coherent state with amplitude $\alpha_{k}$ i.e., $\langle n\rangle = \bra{\alpha_{k}} n\ket{\alpha_{k}} = \lvert\alpha_{k}^{2}\rvert$. Typically $\xi_{k}$ and $\langle n_{k}\rangle$ are heterogeneous. We consider a simplified model consisting of a single mode field 

\begin{equation*}
E^{+}(r_{i}) = h(r_{i}-s_{0})\hat{a}_{n}
\end{equation*}

\begin{equation*}
g^{(2)}_{ij}(0) = \frac{\langle E^{-}(r_{i})E^{-}(r_{j})E^{+}(r_{i})E^{+}(r_{j}) \rangle}{\langle E^{-}(r_{i})E^{+}(r_{i})\rangle\langle E^{-}(r_{j})E^{+}(r_{j})\rangle} = \frac{\mathrm{Tr}(E^{-}(r_{i})E^{-}(r_{j})E^{+}(r_{i})E^{+}(r_{j})\rho)}{\mathrm{Tr}(E^{-}(r_{i})E^{+}(r_{i})\rho)\mathrm{Tr}(E^{-}(r_{j})E^{+}(r_{j})\rho)}
\end{equation*}

Terms related to point spread function will cancel. It is instructive to compute

\begin{align*}
\mathrm{Tr}(a^{\dagger}a^{\dagger}aa \left(\xi_{k}\ket{\alpha_{k}}\bra{\alpha_{k}}\right) &= \mathrm{Tr}\left(\xi_{k} e^{-\lvert\alpha\rvert^{2}}\sum_{n,m}^{\infty}\frac{\alpha^{n}}{n!}\ket{n}\bra{m}\right)\\
&= \mathrm{Tr}\left(\xi_{k} e^{-\lvert\alpha\rvert^{2}}\sum_{n}^{\infty}\frac{\lvert\alpha\rvert^{2n}}{n!}n(n-1)\right)\\
&= \mathrm{Tr}\left(\xi_{k} e^{-\lvert\alpha\rvert^{2}}\sum_{n=2}^{\infty}\frac{\lvert\alpha\rvert^{2n}}{(n-2)!}\right)\\
&= \xi_{k}\lvert\alpha_{k}\rvert^{4}
\end{align*}

Similarly,

\begin{align*}
\mathrm{Tr}(a^{\dagger}a \left(\xi \ket{\alpha}\bra{\alpha}\right)) &= \mathrm{Tr}\left(\xi e^{-\lvert\alpha\rvert^{2}}\sum_{n,m}^{\infty}\frac{\alpha^{n}(\alpha^{m})^{*}}{\sqrt{n!}\sqrt{m!}}a^{\dagger}a\ket{n}\bra{m} \right)\\
&= \xi e^{-\lvert\alpha\rvert^{2}}\sum_{n=0}^{\infty}\frac{(\lvert\alpha\rvert^{2})^{n}}{n!}n\\
&= \xi e^{-\lvert\alpha\rvert^{2}}\sum_{n=1}^{\infty}\frac{(\lvert\alpha\rvert^{2})^{n}}{(n-1)!}\\
&= \xi e^{-\lvert\alpha\rvert^{2}}\left(\lvert\alpha\rvert^{2} + \frac{\lvert\alpha\rvert^{4}}{1!} + \frac{\lvert\alpha\rvert^{6}}{2!}+...\right)\\
&= \xi e^{-\lvert\alpha\rvert^{2}}\lvert\alpha\rvert^{2}\left(1 + \frac{\lvert\alpha\rvert^{2}}{1!} + \frac{\lvert\alpha\rvert^{3}}{2!}+...\right)\\
&= \xi e^{-\lvert\alpha\rvert^{2}}e^{\lvert\alpha\rvert^{2}}\lvert\alpha\rvert^{2} = \xi\lvert\alpha\rvert^{2}
\end{align*}

\begin{align*}
\mathrm{Tr}(a a^{\dagger} \left(\xi \ket{\alpha}\bra{\alpha}\right)) &= \mathrm{Tr}\left(\xi e^{-\lvert\alpha\rvert^{2}}\sum_{n,m}^{\infty}\frac{\alpha^{n}(\alpha^{m})^{*}}{\sqrt{n!}\sqrt{m!}}a a^{\dagger}\ket{n}\bra{m} \right)\\
&= \xi e^{-\lvert\alpha\rvert^{2}}\sum_{n=0}^{\infty}\frac{(\lvert\alpha\rvert^{2})^{n}}{n!}(n+1)\\
&= \xi e^{-\lvert\alpha\rvert^{2}}\left(\sum_{n=1}^{\infty}\frac{(\lvert\alpha\rvert^{2})^{n}}{(n-1)!} + e^{\lvert\alpha\rvert^{2}}\right)\\
&= \xi e^{-\lvert\alpha\rvert^{2}}\left(\lvert\alpha\rvert^{2}e^{\lvert\alpha\rvert^{2}} + e^{\lvert\alpha\rvert^{2}}\right) = \xi(\lvert\alpha\rvert^{2} + 1)
\end{align*}

Putting it all together yields a simple expression for the two-point coherence function

\begin{equation*}
g^{(2)}_{ij}(0) = \frac{\sum_{k}\xi_{k}\lvert\alpha_{k}\rvert^{4}}{\left(\sum_{k}\xi_{k}\lvert\alpha_{k}\rvert^{2}\right)\left(\sum_{k}\xi_{k}\lvert\alpha_{k}\rvert^{2}\right)}
\end{equation*}

For example, if we have a two-level system consisting of a fluorescent state with amplitude $\alpha$ and the vacuum state, this becomes

\begin{equation*}
g^{(2)}_{ij}(0) = \frac{\xi\lvert\alpha\rvert^{4}}{\xi^{2}\lvert\alpha\rvert^{4}} = \frac{1}{\xi}
\end{equation*}

As $\xi\rightarrow 1$ (always on) we recover a coherent state. As $\xi\rightarrow 0$ we observe $g^{(2)}_{ij}(0) > 1$ i.e., bunching.

\subsection{Generalization to nonzero background}

\begin{equation*}
E_{0}^{+}\sim \sum_{j=1}^{M}\delta(s-s_{j})a_{j} \;\; E^{+}(r_{i}) = \int d^{2}s E_{0}^{+} = \sum_{n}h(r_{i}-s_{n})a_{n}
\end{equation*}

\begin{equation*}
\rho_{S} = \xi\ket{\alpha}\bra{\alpha} + (1-\xi)\ket{0}\bra{0}\;\;\rho_{B} = \ket{\beta}\bra{\beta}\;\;\rho = \rho_{S}\otimes\rho_{B}
\end{equation*}

\begin{equation*}
E(r_{i})^{+} = E_{S}(r_{i})^{+} + E_{B}(r_{i})^{+} = h(r_{i}-s_{n})a_{S} + a_{B}
\end{equation*}

\begin{align*}
G^{2}_{ij}(0) &= \langle(E_{S}^{\dagger} + E_{B}^{\dagger}) (E_{S}^{\dagger} + E_{B}^{\dagger})( E_{S} + E_{B}) (E_{S} + E_{B})\rangle \\
&= h_{i}^{2}h_{j}^{2}\langle a_{S}^{\dagger}a_{S}^{\dagger}a_{S}a_{S}\rangle + h_{i}^{2}\langle a_{S}^{\dagger}a_{B}^{\dagger}a_{S}a_{B}\rangle + h_{j}^{2}\langle a_{B}^{\dagger}a_{S}^{\dagger}a_{B}a_{S}\rangle  + \langle a_{B}^{\dagger}a_{B}^{\dagger}a_{B}a_{B}\rangle  \\
&= \xi(h_{i}^{2}h_{j}^{2}\lvert\alpha\rvert^{4}+ h_{i}^{2}\lvert\alpha\rvert^{2}\lvert\beta\rvert^{2} + h_{j}^{2}\lvert\alpha\rvert^{2}\lvert\beta\rvert^{2}\rangle  + \lvert\beta\rvert^{4} ) \\
&= \xi(h_{i}^{2}h_{j}^{2}\lvert\alpha\rvert^{4}+ \lvert\alpha\rvert^{2}\lvert\beta\rvert^{2}(h_{i}^{2} + h_{j}^{2})  + \lvert\beta\rvert^{4}) \\
\end{align*}

The normalized second order coherence function then reads

\begin{align*}
g^{2}_{ij}(0) &= \frac{\xi h_{i}^{2}h_{j}^{2}N_{0}^{2} + \xi N_{0}B_{0}(h_{i}^{2} + h_{j}^{2}) + B_{0}^{2}}{\xi^{2} h_{i}^{2}h_{j}^{2}N_{0}^{2} + \xi N_{0}B_{0}(h_{i}^{2}+h_{j}^{2}) +  B_{0}^{2}}
\end{align*}

Notice the PSF factor $h_{i}$ appears squared. This squared value can be seen as the probability of photon detection at a point $s_i$, while $h_{i}$ is the amplitude of the electric field. 

\subsection{Generalization to multi-level systems}

\begin{align*}
G^{2}_{ij}(0) &= \frac{h_{i}^{2}h_{j}^{2}\sum_{k}\xi_{k}\lvert\alpha_{k}\lvert^{4} + \lvert\beta\lvert^{2}(h_{i}^{2} + h_{j}^{2})\sum_{k}\xi_{k}\lvert\alpha_{k}\lvert^{2} + \lvert\beta\lvert^{4}}{\left(h_{i}^{2}\sum_{k}\xi_{k}\lvert\alpha_{k}\lvert^{2} + \lvert\beta\lvert^{2}\right)\left(h_{j}^{2}\sum_{k}\xi_{k}\lvert\alpha_{k}\lvert^{2} + \lvert\beta\lvert^{2}\right)}
\end{align*}



\subsection{Details of the Gaussian PSF}\

We will derive the gradients for the integrated astigmatic Gaussian, since it is the more general case. As before, define $i_{0} = g_{k}\gamma\Delta t N_{0}$ such that $\mu_{k}' = i_{0}\lambda_{k}$

\begin{equation*}
J_{x_{0}} = \beta_{k}\lambda_{y}\frac{\partial \lambda_{x}}{\partial x_{0}} \;\; J_{y_{0}} = \beta_{k}\lambda_{x}\frac{\partial \lambda_{y}}{\partial y_{0}}\;\;\; J_{z_{0}}  = \frac{\partial \mu_{k}'}{\partial \sigma_{x}}\frac{\partial \sigma_{x}}{\partial z_{0}} + \frac{\partial \mu_{k}'}{\partial \sigma_{y}}\frac{\partial \sigma_{y}}{\partial z_{0}}
\end{equation*}

\begin{align*}
J_{x_{0}} &= \beta_{k}\lambda_{y}\frac{\partial \lambda_{x}}{\partial x_{0}} \\
&= \frac{\beta_{k}\lambda_{y}}{2}\frac{\partial}{\partial x_{0}}\left(\mathrm{erf}\left(\frac{x_{k}+\frac{1}{2}-x_{0}}{\sqrt{2}\sigma_{x}}\right) -\mathrm{erf}\left(\frac{x_{k}-\frac{1}{2}-x_{0}}{\sqrt{2}\sigma_{x}}\right)\right)\\
&= \frac{\beta_{k}\lambda_{y}}{\sqrt{2\pi}\sigma_{x}}\left(\mathrm{exp}\left(\frac{(x_{k}-\frac{1}{2}-x_{0})^{2}}{2\sigma_{x}^{2}}\right) -\mathrm{exp}\left(\frac{(x_{k}+\frac{1}{2}-x_{0})^{2}}{2\sigma_{x}^{2}}\right)\right)
\end{align*}

\begin{align*}
J_{y_{0}} &= \beta_{k}\lambda_{x}\frac{\partial \lambda_{y}}{\partial y_{0}} \\
&= \frac{\beta_{k}\lambda_{x}}{2}\frac{\partial}{\partial y_{0}}\left(\mathrm{erf}\left(\frac{y_{k}+\frac{1}{2}-y_{0}}{\sqrt{2}\sigma_{y}}\right) -\mathrm{erf}\left(\frac{y_{k}-\frac{1}{2}-y_{0}}{\sqrt{2}\sigma_{y}}\right)\right)\\
&= \frac{\beta_{k}\lambda_{x}}{\sqrt{2\pi}\sigma_{y}}\left(\mathrm{exp}\left(\frac{(y_{k}-\frac{1}{2}-y_{0})^{2}}{2\sigma_{y}^{2}}\right) -\mathrm{exp}\left(\frac{(y_{k}+\frac{1}{2}-y_{0})^{2}}{2\sigma_{y}^{2}}\right)\right)
\end{align*}

\begin{align*}
J_{\sigma_{x}} &= \beta_{k}\lambda_{y}\frac{\partial \lambda_{x}}{\partial \sigma_{x}} \\
&= \frac{\beta_{k}\lambda_{y}}{2}\frac{\partial}{\partial \sigma_{x}}\left(\mathrm{erf}\left(\frac{x_{k}+\frac{1}{2}-x_{0}}{\sqrt{2}\sigma_{x}}\right) -\mathrm{erf}\left(\frac{x_{k}-\frac{1}{2}-x_{0}}{\sqrt{2}\sigma_{x}}\right)\right)\\
&= \frac{\beta_{k}\lambda_{y}}{\sqrt{2\pi}}\left(\frac{\left(x-x_{0}-\frac{1}{2}\right) e^{-\frac{\left(x-x_{0}-\frac{1}{2}\right)^2}{2 \sigma_{x} ^2}}}{\sigma_{x} ^2}-\frac{ \left(x-x_{0}+\frac{1}{2}\right) e^{-\frac{\left(x-x_{0}+\frac{1}{2}\right)^2}{2 \sigma_{x} ^2}}}{\sigma_{x} ^2}\right)
\end{align*}

\begin{align*}
J_{\sigma_{y}} &= \beta_{k}\lambda_{x}\frac{\partial \lambda_{y}}{\partial \sigma_{y}} \\
&= \frac{\beta_{k}\lambda_{x}}{2}\frac{\partial}{\partial \sigma_{y}}\left(\mathrm{erf}\left(\frac{y_{k}+\frac{1}{2}-y_{0}}{\sqrt{2}\sigma_{y}}\right) -\mathrm{erf}\left(\frac{y_{k}-\frac{1}{2}-y_{0}}{\sqrt{2}\sigma_{y}}\right)\right)\\
&= \frac{\beta_{k}\lambda_{x}}{\sqrt{2\pi}}\left(\frac{\left(y-y_{0}-\frac{1}{2}\right) e^{-\frac{\left(y-y_{0}-\frac{1}{2}\right)^2}{2 \sigma_{y} ^2}}}{\sigma_{y} ^2}-\frac{ \left(y-y_{0}+\frac{1}{2}\right) e^{-\frac{\left(y-y_{0}+\frac{1}{2}\right)^2}{2 \sigma_{y} ^2}}}{\sigma_{y} ^2}\right)
\end{align*}

Luckily, computing the Hessian matrix for (2.9) is tractable, and is actually quite simple when one takes advantage of the chain rule for Hessian matrices. Looking at (2.9), the likelihood is a hierarchical function that maps a vector space $\Theta$ to a vector space $\Lambda$ to a scalar value. Formally, we define $T: \Theta \rightarrow \Lambda$ and $W: \Lambda \rightarrow \mathbb{R}$. The parameter vector $(x_{0},y_{0},z_{0}, \sigma_{0}, N_{0})\in \Theta$, the Poisson rate vector $\vec{\lambda} \in \Lambda$ and $\ell \in \mathbb{R}$. Note that we choose to optimize $\sigma_{x}$ and $\sigma_{y}$ directly and compute $z_{0}$ to simplify the computation of the Hessian. To get the Hessian, we need the chain-rule for Hessian matrices, which can be quickly computed in terms of the jacobian and hessian of $T$ and $W$.


\begin{equation*}
H_{\ell} = J_{\mu}^{T} H_{\ell} J_{\mu} + (J_{\ell}\otimes I_{n})H_{\mu}
\end{equation*}

where we have used $J_{\mu}$ to represent the jacobian of $T$ and $J_{\ell}$ for the jacobian of $W$. Similar notation is used for the corresponding Hessian matrices. 
In the 3D case, the Hessian matrix is not directly separable since $\mu \propto \lambda_{x}(x_{0},\sigma_{0},\sigma_{x})\lambda_{y}(y_{0},\sigma_{0},\sigma_{y})$. To see this, an abstract representation of the Hessian reads 


\subsection{Fisher information for 2D integrated gaussian}

For the 2D integrated gaussian point spread function, the Hessian only contains separable second order derivatives, so the Fisher information matrix takes on a convenient form

\begin{equation}
I_{ij}(\theta) = \underset{\theta}{\mathbb{E}}\left(\frac{\partial \ell}{\partial\theta_{i}}\frac{\partial\ell}{\partial\theta_{j}}\right) 
\end{equation}

For an arbitrary parameter then we have

\begin{align*}
\frac{\partial \ell}{\partial \theta_{i}} &= \frac{\partial}{\partial \theta_{i}} \sum_{k}  x_{k}\log x_{k} + \mu_{k}' - x_{k}\log\left(\mu_{k}'\right)\\
&= \sum_{k} \frac{\partial \mu_{k}'}{\partial\theta_{i}} \left(\frac{\mu_{k}'-x_{k}}{\mu_{k}'}\right)
\end{align*}

\begin{equation*}
I_{ij}(\theta) = \underset{\theta}{\mathbb{E}}\left(\sum_{k}\frac{\partial \mu_{k}'}{\partial\theta_{i}}\frac{\partial \mu_{k}'}{\partial\theta_{j}} \left(\frac{\mu_{k}'-x_{k}}{\mu_{k}'}\right)^{2}\right) = \sum_{k}\frac{1}{\mu_{k}'}\frac{\partial \mu_{k}'}{\partial\theta_{i}}\frac{\partial \mu_{k}'}{\partial\theta_{j}}
\end{equation*}

To compute the bound, it turns out all we need is the jacobian $\frac{\partial \mu_{k}'}{\partial\theta_{j}} $.



  %\ProvidesFile{ap-citations-references.tex}[]
\begin{refsection}
\chapter{REFERENCES}
\PrintBibliography
\end{refsection}
\MyIO


  % Common mistakes.
  %\include{ap-common-mistakes}

  % Defining commands.
  %\include{ap-defining-commands}

  % Figures.
  %\include{ap-figures}

  % Frequently Asked Questions.
  %\include{ap-frequently-asked-questions}

  % Graphics.
  %\include{ap-graphics}

  % Ignore these references.
  %\include{ap-ignore-these-references}

  % Logos.
  %\include{ap-logos}

  % Miscellaneous.
  %\include{ap-miscellaneous}
  
  % Numbers and Units.
  %\include{ap-numbers-and-units}

  % Resources.
  %\include{ap-resources}

  % Tables.
  %\include{ap-tables}

  % Special characters.
  %\include{ap-special-characters}

  % Testing.
  %\include{ap-testing}

  % Text.
  %\usepackage{amsmath}
\ProvidesFile{ap-text.tex}[2022-10-05 text appendix]


\subsection{Photoswitching induced spatial coherence}

Photoswitching fluorescent molecules are described in the density matrix formalism

\begin{equation*}
\rho = \xi\ket{\alpha}\bra{\alpha} + (1-\xi)\ket{0}\bra{0}
\end{equation*}

where $\ket{\alpha}$ is a coherent state with amplitude $\alpha$ i.e., $\langle n\rangle = \bra{\alpha} n\ket{\alpha} = |\alpha|^{2}$. We consider a simplified model consisting of a single mode field 

\begin{equation*}
E_{0}^{+}\sim \sum_{j=1}^{M}\delta(s-s_{j})a_{j} \;\; E^{+}(r_{i}) = \int d^{2}s E_{0}^{+} h(r-s) = h(r_{i}-s)\hat{a}
\end{equation*}

\begin{equation*}
g^{(2)}_{ij}(0) = \frac{\langle E^{-}(r_{i})E^{-}(r_{j})E^{+}(r_{i})E^{+}(r_{j}) \rangle}{\langle E^{-}(r_{i})E^{+}(r_{i})\rangle\langle E^{-}(r_{j})E^{+}(r_{j})\rangle} = \frac{\mathrm{Tr}(a^{\dagger}a^{\dagger}aa\rho)}{\mathrm{Tr}(a^{\dagger}a\rho)^{2}}
\end{equation*}

Notice that terms related to point spread function will cancel. Now,

\begin{align*}
\mathrm{Tr}(a^{\dagger}a^{\dagger}aa\rho) &= \mathrm{Tr}(a^{\dagger}a^{\dagger}aa \left(\xi\ket{\alpha}\bra{\alpha} + (1-\xi)\ket{0}\bra{0}\right))\\
&= \mathrm{Tr}\left(\xi e^{-|\alpha|^{2}}\sum_{n,m}^{\infty}\frac{\alpha^{n}}{n!}\ket{n}\bra{m}\right)\\
&= \mathrm{Tr}\left(\xi e^{-|\alpha|^{2}}\sum_{n}^{\infty}\frac{|\alpha|^{2n}}{n!}n(n-1)\right)\\
&= \mathrm{Tr}\left(\xi e^{-|\alpha|^{2}}\sum_{n=2}^{\infty}\frac{|\alpha|^{2n}}{(n-2)!}\right)\\
&= \xi|\alpha|^{4}
\end{align*}

The second trace in the denominator proceeds similarly to the first

\begin{align*}
\mathrm{Tr}(a^{\dagger}a\rho) &= \mathrm{Tr}(a^{\dagger}a \left(\xi\ket{\alpha}\bra{\alpha} + (1-\xi)\ket{0}\bra{0}\right))\\
&= \mathrm{Tr}\left(\xi e^{-|\alpha|^{2}}\sum_{n,m}^{\infty}\frac{\alpha^{n}}{n!}\ket{n}\bra{m} \right)\\
&= \mathrm{Tr}\left(\xi e^{-|\alpha|^{2}}\sum_{n}^{\infty}\frac{|\alpha|^{2n}}{n!}n\right)\\
&= \mathrm{Tr}\left(\xi e^{-|\alpha|^{2}}\sum_{n=2}^{\infty}\frac{|\alpha|^{2n}}{(n-1)!}\right)\\
&= \xi|\alpha|^{2}
\end{align*}

As expected, this gives $\langle n\rangle$. Putting it all together yields a simple expression for the two-point coherence function

\begin{equation*}
g^{(2)}_{ij}(0) = \frac{\xi|\alpha|^{4}}{\xi^{2}|\alpha|^{4}} = \frac{1}{\xi}
\end{equation*}

Notice that as $\xi\rightarrow 1$ (always on) we recover the coherent state. As $\xi\rightarrow 0$ we observe $g^{(2)}_{ij}(0) > 1$ i.e., bunching. This is a critical result: photoswitching results in non-trivial correlations between pixels $i$ and $j$. Introducing more than one photoswitching emitter gives? In practice, we can estimate of $g^{(2)}_{ij}(0)$ in a finite time interval. I guess that $\langle n_{i}\rangle = \xi |\alpha|^{2}\Delta = 0.5$ is reasonable; however this is best addressed by Monte Carlo simulation. The total interval $T$ constrained by the super-resolution frame rate e.g., $T=10\mathrm{ms}$. 



\subsection{Details of the Gaussian PSF}

We will derive the gradients for the integrated astigmatic Gaussian, since it is the more general case. As before, define $i_{0} = g_{k}\gamma\Delta t N_{0}$ such that $\mu_{k}' = i_{0}\lambda_{k}$

\begin{equation*}
J_{x_{0}} = \beta_{k}\lambda_{y}\frac{\partial \lambda_{x}}{\partial x_{0}} \;\; J_{y_{0}} = \beta_{k}\lambda_{x}\frac{\partial \lambda_{y}}{\partial y_{0}}\;\;\; J_{z_{0}}  = \frac{\partial \mu_{k}'}{\partial \sigma_{x}}\frac{\partial \sigma_{x}}{\partial z_{0}} + \frac{\partial \mu_{k}'}{\partial \sigma_{y}}\frac{\partial \sigma_{y}}{\partial z_{0}}
\end{equation*}

\begin{align*}
J_{x_{0}} &= \beta_{k}\lambda_{y}\frac{\partial \lambda_{x}}{\partial x_{0}} \\
&= \frac{\beta_{k}\lambda_{y}}{2}\frac{\partial}{\partial x_{0}}\left(\mathrm{erf}\left(\frac{x_{k}+\frac{1}{2}-x_{0}}{\sqrt{2}\sigma_{x}}\right) -\mathrm{erf}\left(\frac{x_{k}-\frac{1}{2}-x_{0}}{\sqrt{2}\sigma_{x}}\right)\right)\\
&= \frac{\beta_{k}\lambda_{y}}{\sqrt{2\pi}\sigma_{x}}\left(\mathrm{exp}\left(\frac{(x_{k}-\frac{1}{2}-x_{0})^{2}}{2\sigma_{x}^{2}}\right) -\mathrm{exp}\left(\frac{(x_{k}+\frac{1}{2}-x_{0})^{2}}{2\sigma_{x}^{2}}\right)\right)
\end{align*}

\begin{align*}
J_{y_{0}} &= \beta_{k}\lambda_{x}\frac{\partial \lambda_{y}}{\partial y_{0}} \\
&= \frac{\beta_{k}\lambda_{x}}{2}\frac{\partial}{\partial y_{0}}\left(\mathrm{erf}\left(\frac{y_{k}+\frac{1}{2}-y_{0}}{\sqrt{2}\sigma_{y}}\right) -\mathrm{erf}\left(\frac{y_{k}-\frac{1}{2}-y_{0}}{\sqrt{2}\sigma_{y}}\right)\right)\\
&= \frac{\beta_{k}\lambda_{x}}{\sqrt{2\pi}\sigma_{y}}\left(\mathrm{exp}\left(\frac{(y_{k}-\frac{1}{2}-y_{0})^{2}}{2\sigma_{y}^{2}}\right) -\mathrm{exp}\left(\frac{(y_{k}+\frac{1}{2}-y_{0})^{2}}{2\sigma_{y}^{2}}\right)\right)
\end{align*}

\begin{align*}
J_{\sigma_{x}} &= \beta_{k}\lambda_{y}\frac{\partial \lambda_{x}}{\partial \sigma_{x}} \\
&= \frac{\beta_{k}\lambda_{y}}{2}\frac{\partial}{\partial \sigma_{x}}\left(\mathrm{erf}\left(\frac{x_{k}+\frac{1}{2}-x_{0}}{\sqrt{2}\sigma_{x}}\right) -\mathrm{erf}\left(\frac{x_{k}-\frac{1}{2}-x_{0}}{\sqrt{2}\sigma_{x}}\right)\right)\\
&= \frac{\beta_{k}\lambda_{y}}{\sqrt{2\pi}}\left(\frac{\left(x-x_{0}-\frac{1}{2}\right) e^{-\frac{\left(x-x_{0}-\frac{1}{2}\right)^2}{2 \sigma_{x} ^2}}}{\sigma_{x} ^2}-\frac{ \left(x-x_{0}+\frac{1}{2}\right) e^{-\frac{\left(x-x_{0}+\frac{1}{2}\right)^2}{2 \sigma_{x} ^2}}}{\sigma_{x} ^2}\right)
\end{align*}

\begin{align*}
J_{\sigma_{y}} &= \beta_{k}\lambda_{x}\frac{\partial \lambda_{y}}{\partial \sigma_{y}} \\
&= \frac{\beta_{k}\lambda_{x}}{2}\frac{\partial}{\partial \sigma_{y}}\left(\mathrm{erf}\left(\frac{y_{k}+\frac{1}{2}-y_{0}}{\sqrt{2}\sigma_{y}}\right) -\mathrm{erf}\left(\frac{y_{k}-\frac{1}{2}-y_{0}}{\sqrt{2}\sigma_{y}}\right)\right)\\
&= \frac{\beta_{k}\lambda_{x}}{\sqrt{2\pi}}\left(\frac{\left(y-y_{0}-\frac{1}{2}\right) e^{-\frac{\left(y-y_{0}-\frac{1}{2}\right)^2}{2 \sigma_{y} ^2}}}{\sigma_{y} ^2}-\frac{ \left(y-y_{0}+\frac{1}{2}\right) e^{-\frac{\left(y-y_{0}+\frac{1}{2}\right)^2}{2 \sigma_{y} ^2}}}{\sigma_{y} ^2}\right)
\end{align*}

Luckily, computing the Hessian matrix for (2.9) is tractable, and is actually quite simple when one takes advantage of the chain rule for Hessian matrices. Looking at (2.9), the likelihood is a hierarchical function that maps a vector space $\Theta$ to a vector space $\Lambda$ to a scalar value. Formally, we define $T: \Theta \rightarrow \Lambda$ and $W: \Lambda \rightarrow \mathbb{R}$. The parameter vector $(x_{0},y_{0},z_{0}, \sigma_{0}, N_{0})\in \Theta$, the Poisson rate vector $\vec{\lambda} \in \Lambda$ and $\ell \in \mathbb{R}$. Note that we choose to optimize $\sigma_{x}$ and $\sigma_{y}$ directly and compute $z_{0}$ to simplify the computation of the Hessian. To get the Hessian, we need the chain-rule for Hessian matrices, which can be quickly computed in terms of the jacobian and hessian of $T$ and $W$.


\begin{equation*}
H_{\ell} = J_{\mu}^{T} H_{\ell} J_{\mu} + (J_{\ell}\otimes I_{n})H_{\mu}
\end{equation*}

where we have used $J_{\mu}$ to represent the jacobian of $T$ and $J_{\ell}$ for the jacobian of $W$. Similar notation is used for the corresponding Hessian matrices. 
In the 3D case, the Hessian matrix is not directly separable since $\mu \propto \lambda_{x}(x_{0},\sigma_{0},\sigma_{x})\lambda_{y}(y_{0},\sigma_{0},\sigma_{y})$. To see this, an abstract representation of the Hessian reads 


\subsection{Fisher information for 2D integrated gaussian}

For the 2D integrated gaussian point spread function, the Hessian only contains separable second order derivatives, so the Fisher information matrix takes on a convenient form

\begin{equation}
I_{ij}(\theta) = \underset{\theta}{\mathbb{E}}\left(\frac{\partial \ell}{\partial\theta_{i}}\frac{\partial\ell}{\partial\theta_{j}}\right) 
\end{equation}

For an arbitrary parameter then we have

\begin{align*}
\frac{\partial \ell}{\partial \theta_{i}} &= \frac{\partial}{\partial \theta_{i}} \sum_{k}  x_{k}\log x_{k} + \mu_{k}' - x_{k}\log\left(\mu_{k}'\right)\\
&= \sum_{k} \frac{\partial \mu_{k}'}{\partial\theta_{i}} \left(\frac{\mu_{k}'-x_{k}}{\mu_{k}'}\right)
\end{align*}

\begin{equation*}
I_{ij}(\theta) = \underset{\theta}{\mathbb{E}}\left(\sum_{k}\frac{\partial \mu_{k}'}{\partial\theta_{i}}\frac{\partial \mu_{k}'}{\partial\theta_{j}} \left(\frac{\mu_{k}'-x_{k}}{\mu_{k}'}\right)^{2}\right) = \sum_{k}\frac{1}{\mu_{k}'}\frac{\partial \mu_{k}'}{\partial\theta_{i}}\frac{\partial \mu_{k}'}{\partial\theta_{j}}
\end{equation*}

To compute the bound, it turns out all we need is the jacobian $\frac{\partial \mu_{k}'}{\partial\theta_{j}} $.

\section{The Fokker-Planck Equation}

The Fokker-Planck equation is a central tool in non-equilibrium statistical mechanics, analagous to the master equation for discrete systems. It allows us to determine the time evolution of probability densities over continuous state spaces. Important examples in biophysics are the phase space of a particle or the membrane potential of a nerve cell.

Suppose we have a random variable $\bm{x}$ and its joint distribution $P(\bm{x},t)$, which is not necessarily stationary. Define a vector field $\vec{J}(\bm{x},t)$ which is the probability current, which we will specify in a moment. The Fokker-Planck equation is by starting with a continuity equation for probability 

\begin{align*}
\frac{d}{dt}\int_{V_{0}} P(\bm{x},t)dV &= \int_{S}P(\bm{x},t)(\vec{J}\cdot\hat{n})dS\\
&= -\int_{V_{0}}P(\bm{x},t)(\nabla\cdot \vec{J})dV
\end{align*}

Clearly this implies that

\begin{equation*}
\frac{dP(\bm{x},t)}{dt} = -\left(\nabla\cdot \vec{J}\right)P(\bm{x},t)
\end{equation*}

We often call the divergence term, the Fokker-Planck operator $\mathcal{L}_{FP}=-\nabla\cdot \vec{J}$. A more rigorous derivation is given in the appendix, which tells us that, to second order

\begin{equation*}
J(x_{i},t)  = \left(M_{i}^{(1)}(t) - \sum_{j}\frac{\partial}{\partial x_{j}}M_{ij}^{(2)}(t) \right)P(\bm{x},t)
\end{equation*}

where $M_{i}^{n}(t)$ is the $n$th moment of a transition kernel $T(x_{i}',t'|x_{i},t)$ for variable $i$. The first moment is essentially just the deterministic part of the Langevin dynamics. The second and higher moments will depend on these higher moments in the stochastic forcing terms. As proven more completely in the appendix, the full multi-dimensional Fokker-Planck equation reads

\begin{align}
\frac{\partial P(\vec{x},t)}{\partial t}  &= \vec{\nabla} \cdot J(\vec{x},t)\\
&= \sum_{i=1}^{N}\left(-\frac{\partial}{\partial x_{i}}M_{i}^{(1)}(t) + \sum_{j=1}^{N} \frac{\partial^{2}}{\partial x_{i}\partial x_{j}}M_{ij}^{(2)}(t)\right)P(\vec{x},t)
\end{align}

If we make a further constraint that the moments of the transition operator are stationary $M_{i}^{(1)}(t) = \Upsilon_{ij}$ and $M_{ij}^{(2)}(t) = D_{ij}$ 

\begin{align}
\frac{\partial P(\vec{x},t)}{\partial t}  &= \sum_{ij}\left(\Upsilon_{ij}\frac{\partial}{\partial x_{i}} + D_{ij}\frac{\partial^{2}}{\partial x_{i}\partial x_{j}}\right)P(\vec{x},t)
\end{align}

\begin{equation*}
D = \begin{pmatrix}0&0 \\ 0& \gamma k_{B}T/m \end{pmatrix}\;\;\Upsilon = \begin{pmatrix}0 & -1\\ 0 & \gamma\end{pmatrix}
\end{equation*}

\section{Free Brownian particle}

Consider a familiar Langevin dynamics on phase space $\bm{x} = (x,v)$, where a free particle ($V(x)=0\; \forall x$) experiences a viscous drag force and stochastic forcing $\xi(t)$ where $\xi(t)\sim\mathcal{N}(\mu,\sigma^{2})$ and $\langle \xi(t)\xi(t+\tau)\rangle = \delta(t-\tau)$. 

\begin{align*}
\dot{x} &= v\\
\dot{v} &= -\frac{\gamma}{m}v + \frac{1}{m}\xi(t)
\end{align*}

The moments of the transition kernel must be

\begin{equation*}
M_{x}^{(1)} = v  \;\; M_{v}^{(1)} = -\frac{\gamma}{m}v + \mu \;\; M_{v}^{(v)} = \sigma^{2}
\end{equation*}

To simplify the notation let us define $\nabla\cdot \vec{J} = \frac{\partial J_{x}}{\partial x} + \frac{\partial J_{x}}{\partial v}= \mathcal{L}_{x} + \mathcal{L}_{v} = \mathcal{L}_{FP}$. This gives the full Fokker-Planck equation $\frac{dP(\bm{x},t)}{dt} = -\mathcal{L}_{FP}P(\bm{x},t)$. 

\begin{align*}
\mathcal{L}_{x}P(\bm{x},t) &= \frac{\partial}{\partial x}\left(vP(\bm{x},t)\right)\\
\mathcal{L}_{v}P(\bm{x},t) &= \frac{\partial}{\partial v}\left(-\frac{\gamma}{m}v + \frac{1}{m}F(x)\right)P(\bm{x},t) + \sigma^{2}\frac{\partial^{2}}{\partial v^{2}}P(\bm{x},t)
\end{align*}


\section{The Brownian Harmonic oscillator}

Consider a familiar Langevin dynamics on phase space $\bm{x} = (x,v)$, where a particle in a potential $V(x)$ experiences a viscous drag force and stochastic forcing $\xi(t)$ where $\xi(t)\sim\mathcal{N}(\mu,\sigma^{2})$ and $\langle \xi(t)\xi(t+\tau)\rangle = \delta(t-\tau)$. 

\begin{align*}
\dot{x} &= v\\
\dot{v} &= -\frac{\gamma}{m}v + \frac{1}{m}F(x) + \frac{1}{m}\xi(t)
\end{align*}

The moments of the transition kernel must be

\begin{equation*}
M_{x}^{(1)} = v  \;\; M_{v}^{(1)} = -\frac{\gamma}{m}v + \frac{1}{m}F(x) + \mu \;\; M_{v}^{(v)} = \sigma^{2}
\end{equation*}

To simplify the notation let us define $\nabla\cdot \vec{J} = \frac{\partial J_{x}}{\partial x} + \frac{\partial J_{x}}{\partial v}= \mathcal{L}_{x} + \mathcal{L}_{v} = \mathcal{L}_{FP}$. This gives the full Fokker-Planck equation $\frac{dP(\bm{x},t)}{dt} = -\mathcal{L}_{FP}P(\bm{x},t)$. 

\begin{align*}
\mathcal{L}_{x}P(\bm{x},t) &= \frac{\partial}{\partial x}\left(vP(\bm{x},t)\right)\\
\mathcal{L}_{v}P(\bm{x},t) &= \frac{\partial}{\partial v}\left(-\frac{\gamma}{m}v + \frac{1}{m}F(x)\right)P(\bm{x},t) + \sigma^{2}\frac{\partial^{2}}{\partial v^{2}}P(\bm{x},t)
\end{align*}


\begin{VerbatimOut}{z.out}
\chapter{TEXT}

\end{VerbatimOut}

\MyIO


  % Video.
  % \include{ap-video}

  % Astronomy.
  %\include{ap-astronomy}

  % Biology.
  %\include{ap-biology}

  % Chemistry.
  %\include{ap-chemistry}

  % Computer Science.
  %\include{ap-computer-science}

  % Electrical Engineering.
  %\include{ap-electrical-engineering}

  % Linguistics.
  %\include{ap-linguistics}

  % Mathematics.
  %\include{ap-mathematics}

  % Music.
  %\include{ap-music}

  % The examples in ap-physics require LuaLaTeX but LuaLaTeX
  % screws up the spacing in the List of Figures.  So, the
  % ap-physics file is not included.
  %
  % For some reason, ap-physics doesn't work when using BibTeX.
  % Just enclosing \ProvidesFile{ap-physics.tex}[2022-10-05 Physics appendix]

\chapter{PHYSICS}
\ix{physics//Physics appendix}

\subsection{Photoswitching induced spatial coherence}

Photoswitching fluorescent molecules are described in the density matrix formalism

\begin{equation*}
\rho = \xi\ket{\alpha}\bra{\alpha} + (1-\xi)\ket{0}\bra{0}
\end{equation*}

where $\ket{\alpha}$ is a coherent state with amplitude $\alpha$ i.e., $\langle n\rangle = \bra{\alpha} n\ket{\alpha} = \lvert\alpha^{2}\rvert$. We consider a simplified model consisting of a single mode field 

\begin{equation*}
E_{0}^{+}\sim \sum_{j=1}^{M}\delta(s-s_{j})a_{j} \;\; E^{+}(r_{i}) = \int d^{2}s E_{0}^{+} h(r-s) = h(r_{i}-s)\hat{a}
\end{equation*}

\begin{equation*}
g^{(2)}_{ij}(0) = \frac{\langle E^{-}(r_{i})E^{-}(r_{j})E^{+}(r_{i})E^{+}(r_{j}) \rangle}{\langle E^{-}(r_{i})E^{+}(r_{i})\rangle\langle E^{-}(r_{j})E^{+}(r_{j})\rangle} = \frac{\mathrm{Tr}(a^{\dagger}a^{\dagger}aa\rho)}{\mathrm{Tr}(a^{\dagger}a\rho)^{2}}
\end{equation*}

Notice that terms related to point spread function will cancel. Now,

\begin{align*}
\mathrm{Tr}(a^{\dagger}a^{\dagger}aa\rho) &= \mathrm{Tr}(a^{\dagger}a^{\dagger}aa \left(\xi\ket{\alpha}\bra{\alpha} + (1-\xi)\ket{0}\bra{0}\right))\\
&= \mathrm{Tr}\left(\xi e^{-\lvert\alpha\rvert^{2}}\sum_{n,m}^{\infty}\frac{\alpha^{n}}{n!}\ket{n}\bra{m}\right)\\
&= \mathrm{Tr}\left(\xi e^{-\lvert\alpha\rvert^{2}}\sum_{n}^{\infty}\frac{\lvert\alpha\rvert^{2n}}{n!}n(n-1)\right)\\
&= \mathrm{Tr}\left(\xi e^{-\lvert\alpha\rvert^{2}}\sum_{n=2}^{\infty}\frac{\lvert\alpha\rvert^{2n}}{(n-2)!}\right)\\
&= \xi\lvert\alpha\rvert^{4}
\end{align*}

The second trace in the denominator proceeds similarly to the first

\begin{align*}
\mathrm{Tr}(a^{\dagger}a\rho) &= \mathrm{Tr}(a^{\dagger}a \left(\xi\ket{\alpha}\bra{\alpha} + (1-\xi)\ket{0}\bra{0}\right))\\
&= \mathrm{Tr}\left(\xi e^{-\lvert\alpha\rvert^{2}}\sum_{n,m}^{\infty}\frac{\alpha^{n}}{n!}\ket{n}\bra{m} \right)\\
&= \mathrm{Tr}\left(\xi e^{-\lvert\alpha\rvert^{2}}\sum_{n}^{\infty}\frac{\lvert\alpha\rvert^{2n}}{n!}n\right)\\
&= \mathrm{Tr}\left(\xi e^{-\lvert\alpha\rvert^{2}}\sum_{n=2}^{\infty}\frac{\lvert\alpha\rvert^{2n}}{(n-1)!}\right)\\
&= \xi\lvert\alpha\rvert^{2}
\end{align*}

As expected, this gives $\langle n\rangle$. Putting it all together yields a simple expression for the two-point coherence function

\begin{equation*}
g^{(2)}_{ij}(0) = \frac{\xi\lvert\alpha\rvert^{4}}{\xi^{2}\lvert\alpha\rvert^{4}} = \frac{1}{\xi}
\end{equation*}

Notice that as $\xi\rightarrow 1$ (always on) we recover the coherent state. As $\xi\rightarrow 0$ we observe $g^{(2)}_{ij}(0) > 1$ i.e., bunching. This is a critical result: photoswitching results in non-trivial correlations between pixels $i$ and $j$. Introducing more than one photoswitching emitter gives? In practice, we can estimate of $g^{(2)}_{ij}(0)$ in a finite time interval. I guess that $\langle n_{i}\rangle = \xi \lvert\alpha\rvert^{2}\Delta = 0.5$ is reasonable; however this is best addressed by Monte Carlo simulation. The total interval $T$ constrained by the super-resolution frame rate e.g., $T=10\mathrm{ms}$. 


\subsection{Details of the Gaussian PSF}

We will derive the gradients for the integrated astigmatic Gaussian, since it is the more general case. As before, define $i_{0} = g_{k}\gamma\Delta t N_{0}$ such that $\mu_{k}' = i_{0}\lambda_{k}$

\begin{equation*}
J_{x_{0}} = \beta_{k}\lambda_{y}\frac{\partial \lambda_{x}}{\partial x_{0}} \;\; J_{y_{0}} = \beta_{k}\lambda_{x}\frac{\partial \lambda_{y}}{\partial y_{0}}\;\;\; J_{z_{0}}  = \frac{\partial \mu_{k}'}{\partial \sigma_{x}}\frac{\partial \sigma_{x}}{\partial z_{0}} + \frac{\partial \mu_{k}'}{\partial \sigma_{y}}\frac{\partial \sigma_{y}}{\partial z_{0}}
\end{equation*}

\begin{align*}
J_{x_{0}} &= \beta_{k}\lambda_{y}\frac{\partial \lambda_{x}}{\partial x_{0}} \\
&= \frac{\beta_{k}\lambda_{y}}{2}\frac{\partial}{\partial x_{0}}\left(\mathrm{erf}\left(\frac{x_{k}+\frac{1}{2}-x_{0}}{\sqrt{2}\sigma_{x}}\right) -\mathrm{erf}\left(\frac{x_{k}-\frac{1}{2}-x_{0}}{\sqrt{2}\sigma_{x}}\right)\right)\\
&= \frac{\beta_{k}\lambda_{y}}{\sqrt{2\pi}\sigma_{x}}\left(\mathrm{exp}\left(\frac{(x_{k}-\frac{1}{2}-x_{0})^{2}}{2\sigma_{x}^{2}}\right) -\mathrm{exp}\left(\frac{(x_{k}+\frac{1}{2}-x_{0})^{2}}{2\sigma_{x}^{2}}\right)\right)
\end{align*}

\begin{align*}
J_{y_{0}} &= \beta_{k}\lambda_{x}\frac{\partial \lambda_{y}}{\partial y_{0}} \\
&= \frac{\beta_{k}\lambda_{x}}{2}\frac{\partial}{\partial y_{0}}\left(\mathrm{erf}\left(\frac{y_{k}+\frac{1}{2}-y_{0}}{\sqrt{2}\sigma_{y}}\right) -\mathrm{erf}\left(\frac{y_{k}-\frac{1}{2}-y_{0}}{\sqrt{2}\sigma_{y}}\right)\right)\\
&= \frac{\beta_{k}\lambda_{x}}{\sqrt{2\pi}\sigma_{y}}\left(\mathrm{exp}\left(\frac{(y_{k}-\frac{1}{2}-y_{0})^{2}}{2\sigma_{y}^{2}}\right) -\mathrm{exp}\left(\frac{(y_{k}+\frac{1}{2}-y_{0})^{2}}{2\sigma_{y}^{2}}\right)\right)
\end{align*}

\begin{align*}
J_{\sigma_{x}} &= \beta_{k}\lambda_{y}\frac{\partial \lambda_{x}}{\partial \sigma_{x}} \\
&= \frac{\beta_{k}\lambda_{y}}{2}\frac{\partial}{\partial \sigma_{x}}\left(\mathrm{erf}\left(\frac{x_{k}+\frac{1}{2}-x_{0}}{\sqrt{2}\sigma_{x}}\right) -\mathrm{erf}\left(\frac{x_{k}-\frac{1}{2}-x_{0}}{\sqrt{2}\sigma_{x}}\right)\right)\\
&= \frac{\beta_{k}\lambda_{y}}{\sqrt{2\pi}}\left(\frac{\left(x-x_{0}-\frac{1}{2}\right) e^{-\frac{\left(x-x_{0}-\frac{1}{2}\right)^2}{2 \sigma_{x} ^2}}}{\sigma_{x} ^2}-\frac{ \left(x-x_{0}+\frac{1}{2}\right) e^{-\frac{\left(x-x_{0}+\frac{1}{2}\right)^2}{2 \sigma_{x} ^2}}}{\sigma_{x} ^2}\right)
\end{align*}

\begin{align*}
J_{\sigma_{y}} &= \beta_{k}\lambda_{x}\frac{\partial \lambda_{y}}{\partial \sigma_{y}} \\
&= \frac{\beta_{k}\lambda_{x}}{2}\frac{\partial}{\partial \sigma_{y}}\left(\mathrm{erf}\left(\frac{y_{k}+\frac{1}{2}-y_{0}}{\sqrt{2}\sigma_{y}}\right) -\mathrm{erf}\left(\frac{y_{k}-\frac{1}{2}-y_{0}}{\sqrt{2}\sigma_{y}}\right)\right)\\
&= \frac{\beta_{k}\lambda_{x}}{\sqrt{2\pi}}\left(\frac{\left(y-y_{0}-\frac{1}{2}\right) e^{-\frac{\left(y-y_{0}-\frac{1}{2}\right)^2}{2 \sigma_{y} ^2}}}{\sigma_{y} ^2}-\frac{ \left(y-y_{0}+\frac{1}{2}\right) e^{-\frac{\left(y-y_{0}+\frac{1}{2}\right)^2}{2 \sigma_{y} ^2}}}{\sigma_{y} ^2}\right)
\end{align*}

Luckily, computing the Hessian matrix for (2.9) is tractable, and is actually quite simple when one takes advantage of the chain rule for Hessian matrices. Looking at (2.9), the likelihood is a hierarchical function that maps a vector space $\Theta$ to a vector space $\Lambda$ to a scalar value. Formally, we define $T: \Theta \rightarrow \Lambda$ and $W: \Lambda \rightarrow \mathbb{R}$. The parameter vector $(x_{0},y_{0},z_{0}, \sigma_{0}, N_{0})\in \Theta$, the Poisson rate vector $\vec{\lambda} \in \Lambda$ and $\ell \in \mathbb{R}$. Note that we choose to optimize $\sigma_{x}$ and $\sigma_{y}$ directly and compute $z_{0}$ to simplify the computation of the Hessian. To get the Hessian, we need the chain-rule for Hessian matrices, which can be quickly computed in terms of the jacobian and hessian of $T$ and $W$.


\begin{equation*}
H_{\ell} = J_{\mu}^{T} H_{\ell} J_{\mu} + (J_{\ell}\otimes I_{n})H_{\mu}
\end{equation*}

where we have used $J_{\mu}$ to represent the jacobian of $T$ and $J_{\ell}$ for the jacobian of $W$. Similar notation is used for the corresponding Hessian matrices. 
In the 3D case, the Hessian matrix is not directly separable since $\mu \propto \lambda_{x}(x_{0},\sigma_{0},\sigma_{x})\lambda_{y}(y_{0},\sigma_{0},\sigma_{y})$. To see this, an abstract representation of the Hessian reads 


\subsection{Fisher information for 2D integrated gaussian}

For the 2D integrated gaussian point spread function, the Hessian only contains separable second order derivatives, so the Fisher information matrix takes on a convenient form

\begin{equation}
I_{ij}(\theta) = \underset{\theta}{\mathbb{E}}\left(\frac{\partial \ell}{\partial\theta_{i}}\frac{\partial\ell}{\partial\theta_{j}}\right) 
\end{equation}

For an arbitrary parameter then we have

\begin{align*}
\frac{\partial \ell}{\partial \theta_{i}} &= \frac{\partial}{\partial \theta_{i}} \sum_{k}  x_{k}\log x_{k} + \mu_{k}' - x_{k}\log\left(\mu_{k}'\right)\\
&= \sum_{k} \frac{\partial \mu_{k}'}{\partial\theta_{i}} \left(\frac{\mu_{k}'-x_{k}}{\mu_{k}'}\right)
\end{align*}

\begin{equation*}
I_{ij}(\theta) = \underset{\theta}{\mathbb{E}}\left(\sum_{k}\frac{\partial \mu_{k}'}{\partial\theta_{i}}\frac{\partial \mu_{k}'}{\partial\theta_{j}} \left(\frac{\mu_{k}'-x_{k}}{\mu_{k}'}\right)^{2}\right) = \sum_{k}\frac{1}{\mu_{k}'}\frac{\partial \mu_{k}'}{\partial\theta_{i}}\frac{\partial \mu_{k}'}{\partial\theta_{j}}
\end{equation*}

To compute the bound, it turns out all we need is the jacobian $\frac{\partial \mu_{k}'}{\partial\theta_{j}} $.


 in braces, i..e.,
  %     {
  %       \ProvidesFile{ap-physics.tex}[2022-10-05 Physics appendix]

\chapter{PHYSICS}
\ix{physics//Physics appendix}

\subsection{Photoswitching induced spatial coherence}

Photoswitching fluorescent molecules are described in the density matrix formalism

\begin{equation*}
\rho = \xi\ket{\alpha}\bra{\alpha} + (1-\xi)\ket{0}\bra{0}
\end{equation*}

where $\ket{\alpha}$ is a coherent state with amplitude $\alpha$ i.e., $\langle n\rangle = \bra{\alpha} n\ket{\alpha} = \lvert\alpha^{2}\rvert$. We consider a simplified model consisting of a single mode field 

\begin{equation*}
E_{0}^{+}\sim \sum_{j=1}^{M}\delta(s-s_{j})a_{j} \;\; E^{+}(r_{i}) = \int d^{2}s E_{0}^{+} h(r-s) = h(r_{i}-s)\hat{a}
\end{equation*}

\begin{equation*}
g^{(2)}_{ij}(0) = \frac{\langle E^{-}(r_{i})E^{-}(r_{j})E^{+}(r_{i})E^{+}(r_{j}) \rangle}{\langle E^{-}(r_{i})E^{+}(r_{i})\rangle\langle E^{-}(r_{j})E^{+}(r_{j})\rangle} = \frac{\mathrm{Tr}(a^{\dagger}a^{\dagger}aa\rho)}{\mathrm{Tr}(a^{\dagger}a\rho)^{2}}
\end{equation*}

Notice that terms related to point spread function will cancel. Now,

\begin{align*}
\mathrm{Tr}(a^{\dagger}a^{\dagger}aa\rho) &= \mathrm{Tr}(a^{\dagger}a^{\dagger}aa \left(\xi\ket{\alpha}\bra{\alpha} + (1-\xi)\ket{0}\bra{0}\right))\\
&= \mathrm{Tr}\left(\xi e^{-\lvert\alpha\rvert^{2}}\sum_{n,m}^{\infty}\frac{\alpha^{n}}{n!}\ket{n}\bra{m}\right)\\
&= \mathrm{Tr}\left(\xi e^{-\lvert\alpha\rvert^{2}}\sum_{n}^{\infty}\frac{\lvert\alpha\rvert^{2n}}{n!}n(n-1)\right)\\
&= \mathrm{Tr}\left(\xi e^{-\lvert\alpha\rvert^{2}}\sum_{n=2}^{\infty}\frac{\lvert\alpha\rvert^{2n}}{(n-2)!}\right)\\
&= \xi\lvert\alpha\rvert^{4}
\end{align*}

The second trace in the denominator proceeds similarly to the first

\begin{align*}
\mathrm{Tr}(a^{\dagger}a\rho) &= \mathrm{Tr}(a^{\dagger}a \left(\xi\ket{\alpha}\bra{\alpha} + (1-\xi)\ket{0}\bra{0}\right))\\
&= \mathrm{Tr}\left(\xi e^{-\lvert\alpha\rvert^{2}}\sum_{n,m}^{\infty}\frac{\alpha^{n}}{n!}\ket{n}\bra{m} \right)\\
&= \mathrm{Tr}\left(\xi e^{-\lvert\alpha\rvert^{2}}\sum_{n}^{\infty}\frac{\lvert\alpha\rvert^{2n}}{n!}n\right)\\
&= \mathrm{Tr}\left(\xi e^{-\lvert\alpha\rvert^{2}}\sum_{n=2}^{\infty}\frac{\lvert\alpha\rvert^{2n}}{(n-1)!}\right)\\
&= \xi\lvert\alpha\rvert^{2}
\end{align*}

As expected, this gives $\langle n\rangle$. Putting it all together yields a simple expression for the two-point coherence function

\begin{equation*}
g^{(2)}_{ij}(0) = \frac{\xi\lvert\alpha\rvert^{4}}{\xi^{2}\lvert\alpha\rvert^{4}} = \frac{1}{\xi}
\end{equation*}

Notice that as $\xi\rightarrow 1$ (always on) we recover the coherent state. As $\xi\rightarrow 0$ we observe $g^{(2)}_{ij}(0) > 1$ i.e., bunching. This is a critical result: photoswitching results in non-trivial correlations between pixels $i$ and $j$. Introducing more than one photoswitching emitter gives? In practice, we can estimate of $g^{(2)}_{ij}(0)$ in a finite time interval. I guess that $\langle n_{i}\rangle = \xi \lvert\alpha\rvert^{2}\Delta = 0.5$ is reasonable; however this is best addressed by Monte Carlo simulation. The total interval $T$ constrained by the super-resolution frame rate e.g., $T=10\mathrm{ms}$. 


\subsection{Details of the Gaussian PSF}

We will derive the gradients for the integrated astigmatic Gaussian, since it is the more general case. As before, define $i_{0} = g_{k}\gamma\Delta t N_{0}$ such that $\mu_{k}' = i_{0}\lambda_{k}$

\begin{equation*}
J_{x_{0}} = \beta_{k}\lambda_{y}\frac{\partial \lambda_{x}}{\partial x_{0}} \;\; J_{y_{0}} = \beta_{k}\lambda_{x}\frac{\partial \lambda_{y}}{\partial y_{0}}\;\;\; J_{z_{0}}  = \frac{\partial \mu_{k}'}{\partial \sigma_{x}}\frac{\partial \sigma_{x}}{\partial z_{0}} + \frac{\partial \mu_{k}'}{\partial \sigma_{y}}\frac{\partial \sigma_{y}}{\partial z_{0}}
\end{equation*}

\begin{align*}
J_{x_{0}} &= \beta_{k}\lambda_{y}\frac{\partial \lambda_{x}}{\partial x_{0}} \\
&= \frac{\beta_{k}\lambda_{y}}{2}\frac{\partial}{\partial x_{0}}\left(\mathrm{erf}\left(\frac{x_{k}+\frac{1}{2}-x_{0}}{\sqrt{2}\sigma_{x}}\right) -\mathrm{erf}\left(\frac{x_{k}-\frac{1}{2}-x_{0}}{\sqrt{2}\sigma_{x}}\right)\right)\\
&= \frac{\beta_{k}\lambda_{y}}{\sqrt{2\pi}\sigma_{x}}\left(\mathrm{exp}\left(\frac{(x_{k}-\frac{1}{2}-x_{0})^{2}}{2\sigma_{x}^{2}}\right) -\mathrm{exp}\left(\frac{(x_{k}+\frac{1}{2}-x_{0})^{2}}{2\sigma_{x}^{2}}\right)\right)
\end{align*}

\begin{align*}
J_{y_{0}} &= \beta_{k}\lambda_{x}\frac{\partial \lambda_{y}}{\partial y_{0}} \\
&= \frac{\beta_{k}\lambda_{x}}{2}\frac{\partial}{\partial y_{0}}\left(\mathrm{erf}\left(\frac{y_{k}+\frac{1}{2}-y_{0}}{\sqrt{2}\sigma_{y}}\right) -\mathrm{erf}\left(\frac{y_{k}-\frac{1}{2}-y_{0}}{\sqrt{2}\sigma_{y}}\right)\right)\\
&= \frac{\beta_{k}\lambda_{x}}{\sqrt{2\pi}\sigma_{y}}\left(\mathrm{exp}\left(\frac{(y_{k}-\frac{1}{2}-y_{0})^{2}}{2\sigma_{y}^{2}}\right) -\mathrm{exp}\left(\frac{(y_{k}+\frac{1}{2}-y_{0})^{2}}{2\sigma_{y}^{2}}\right)\right)
\end{align*}

\begin{align*}
J_{\sigma_{x}} &= \beta_{k}\lambda_{y}\frac{\partial \lambda_{x}}{\partial \sigma_{x}} \\
&= \frac{\beta_{k}\lambda_{y}}{2}\frac{\partial}{\partial \sigma_{x}}\left(\mathrm{erf}\left(\frac{x_{k}+\frac{1}{2}-x_{0}}{\sqrt{2}\sigma_{x}}\right) -\mathrm{erf}\left(\frac{x_{k}-\frac{1}{2}-x_{0}}{\sqrt{2}\sigma_{x}}\right)\right)\\
&= \frac{\beta_{k}\lambda_{y}}{\sqrt{2\pi}}\left(\frac{\left(x-x_{0}-\frac{1}{2}\right) e^{-\frac{\left(x-x_{0}-\frac{1}{2}\right)^2}{2 \sigma_{x} ^2}}}{\sigma_{x} ^2}-\frac{ \left(x-x_{0}+\frac{1}{2}\right) e^{-\frac{\left(x-x_{0}+\frac{1}{2}\right)^2}{2 \sigma_{x} ^2}}}{\sigma_{x} ^2}\right)
\end{align*}

\begin{align*}
J_{\sigma_{y}} &= \beta_{k}\lambda_{x}\frac{\partial \lambda_{y}}{\partial \sigma_{y}} \\
&= \frac{\beta_{k}\lambda_{x}}{2}\frac{\partial}{\partial \sigma_{y}}\left(\mathrm{erf}\left(\frac{y_{k}+\frac{1}{2}-y_{0}}{\sqrt{2}\sigma_{y}}\right) -\mathrm{erf}\left(\frac{y_{k}-\frac{1}{2}-y_{0}}{\sqrt{2}\sigma_{y}}\right)\right)\\
&= \frac{\beta_{k}\lambda_{x}}{\sqrt{2\pi}}\left(\frac{\left(y-y_{0}-\frac{1}{2}\right) e^{-\frac{\left(y-y_{0}-\frac{1}{2}\right)^2}{2 \sigma_{y} ^2}}}{\sigma_{y} ^2}-\frac{ \left(y-y_{0}+\frac{1}{2}\right) e^{-\frac{\left(y-y_{0}+\frac{1}{2}\right)^2}{2 \sigma_{y} ^2}}}{\sigma_{y} ^2}\right)
\end{align*}

Luckily, computing the Hessian matrix for (2.9) is tractable, and is actually quite simple when one takes advantage of the chain rule for Hessian matrices. Looking at (2.9), the likelihood is a hierarchical function that maps a vector space $\Theta$ to a vector space $\Lambda$ to a scalar value. Formally, we define $T: \Theta \rightarrow \Lambda$ and $W: \Lambda \rightarrow \mathbb{R}$. The parameter vector $(x_{0},y_{0},z_{0}, \sigma_{0}, N_{0})\in \Theta$, the Poisson rate vector $\vec{\lambda} \in \Lambda$ and $\ell \in \mathbb{R}$. Note that we choose to optimize $\sigma_{x}$ and $\sigma_{y}$ directly and compute $z_{0}$ to simplify the computation of the Hessian. To get the Hessian, we need the chain-rule for Hessian matrices, which can be quickly computed in terms of the jacobian and hessian of $T$ and $W$.


\begin{equation*}
H_{\ell} = J_{\mu}^{T} H_{\ell} J_{\mu} + (J_{\ell}\otimes I_{n})H_{\mu}
\end{equation*}

where we have used $J_{\mu}$ to represent the jacobian of $T$ and $J_{\ell}$ for the jacobian of $W$. Similar notation is used for the corresponding Hessian matrices. 
In the 3D case, the Hessian matrix is not directly separable since $\mu \propto \lambda_{x}(x_{0},\sigma_{0},\sigma_{x})\lambda_{y}(y_{0},\sigma_{0},\sigma_{y})$. To see this, an abstract representation of the Hessian reads 


\subsection{Fisher information for 2D integrated gaussian}

For the 2D integrated gaussian point spread function, the Hessian only contains separable second order derivatives, so the Fisher information matrix takes on a convenient form

\begin{equation}
I_{ij}(\theta) = \underset{\theta}{\mathbb{E}}\left(\frac{\partial \ell}{\partial\theta_{i}}\frac{\partial\ell}{\partial\theta_{j}}\right) 
\end{equation}

For an arbitrary parameter then we have

\begin{align*}
\frac{\partial \ell}{\partial \theta_{i}} &= \frac{\partial}{\partial \theta_{i}} \sum_{k}  x_{k}\log x_{k} + \mu_{k}' - x_{k}\log\left(\mu_{k}'\right)\\
&= \sum_{k} \frac{\partial \mu_{k}'}{\partial\theta_{i}} \left(\frac{\mu_{k}'-x_{k}}{\mu_{k}'}\right)
\end{align*}

\begin{equation*}
I_{ij}(\theta) = \underset{\theta}{\mathbb{E}}\left(\sum_{k}\frac{\partial \mu_{k}'}{\partial\theta_{i}}\frac{\partial \mu_{k}'}{\partial\theta_{j}} \left(\frac{\mu_{k}'-x_{k}}{\mu_{k}'}\right)^{2}\right) = \sum_{k}\frac{1}{\mu_{k}'}\frac{\partial \mu_{k}'}{\partial\theta_{i}}\frac{\partial \mu_{k}'}{\partial\theta_{j}}
\end{equation*}

To compute the bound, it turns out all we need is the jacobian $\frac{\partial \mu_{k}'}{\partial\theta_{j}} $.



  %     }
  % doesn't help so it is only loaded if we are using BibLaTeX.
  %
  % Physics-related exmples.

  % Notes and footnotes are optional.
  % Reference: TM2017 page 34.
  % I have not implemented this yet.  Mark Senn 2002-06-03

  % A vita is optional for masters theses
  % and required for doctoral dissertations.
  % Reference: TM2017 page 13.
  \include{ap-vita}

  % Listing or including publications(s) is optional.
  %\include{ap-publications}

  % Print the index.
  % The index is optional.
  \pdfbookmark{INDEX}{index}
  \printindex

  % If \ZZshowcolophon is true, print the colophon.
  \pdfbookmark{COLOPHON}{colophon}
  \ifthen{\equal{true}{\ZZshowcolophon}}
    {\include{ap-colophon}}

% LaTeX won't read after the \end{document} command.
% You can put notes to yourself or LaTeX input not
% ready for use after "\end{document}" if you'd like.
\end{document}
