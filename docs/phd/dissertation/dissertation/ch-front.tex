\ProvidesFile{ch-front.tex}[2022-10-05 front matter chapter]
%
%  This is the ``front matter'' for the thesis.
%
%  REFERENCES
%
%    TCMOS17
%      The Chicago Manual of Style Online, 17th edition.
%      https://www.chicagomanualofstyle.org/home.html
%      retrieved on 2020-02-29
%
%    TEMPL
%      Thesis and Disertation Office Templates.
%      https://www.purdue.edu/gradschool/research/thesis/templates.html
%      retrieved on 2020-02-29
%
%    WNNCD
%    Webster's Ninth New Collegiate Dictionary.
%

%
%   Only Purdue University uses this page
%
%   Comment out \begin{statement} through \end{statement}
%   if you are not at Purdue University.
%
% Statement of Thesis/Dissertation Approval Page
% This page is REQUIRED.  The page should be numbered "2"
% and should NOT be listed in your TABLE OF CONTENTS.
\begin{statement}
  % Delete or add \entry commands as needed for all committe members.
  \entry{Dr. Gautam Vemuri, Chair}{Department of Physics}
  \entry{Dr. Jing Liu}{Department of Physics}
  \entry{Dr. Ruihua Cheng}{Department of Physics}
  \entry{Dr. Stephen Wassall}{Department of Physics}
  \entry{Dr. Horia Petrache}{Department of Physics}
  % There should be one \approvedby command containing the
  % "FORM 9 THESIS FORM HEAD NAME HERE" (from TEMPL, retrieved on 2020-03-01).
  \approvedby{Dr. Jing Liu}
\end{statement}

% Dedication page is optional.
% A name and often a message in tribute to a person or cause.
% References: WEB9 332.
\begin{dedication}
I dedicate this thesis to the Hochstetler family, who have encouraged my pursuit of a doctoral degree since its inception, in spite of the myriad of personal challenges and uncertainties it accompanied.  

\vspace{1in}
\textit{There is one thing we do know: that man is here for the sake of other men - above all for those upon whose smile and well-being our own happiness depends, and also for the countless unknown souls whose fate we are connected by a bond of sympathy}\\
Albert Einstein\\
\vspace{1in}
\textit{To deal with a 14-dimensional space, visualize a 3-dimensional space and say 'fourteen' to yourself very loudly. Everyone does it}\\ Geoffrey Hinton \\
\vspace{1in}
\textit{Information is the resolution of uncertainty} \\Claude Shannon
\vspace{1in}



\end{dedication}

% Acknowledgements page is optional but most theses include
% a brief statement of appreciation or recognition of special
% assistance.
\begin{acknowledgments}
Before I began my research career, I understood in theory, but perhaps didn’t fully grasp viscerally, Isaac Newton’s famous statement, “If I have seen further it is by standing on the shoulders of Giants.” In science, no statement is truer. Yes, it requires the personal elements of perseverance, work ethic, and creativity to succeed in obtaining a Ph.D. in particular and as a scientist in general. But such attributes are moot without the ideas and intellect of those thinkers that preceded us, and the personal buoying by mentors, friends, parents, and significant others in our own individual journeys. Certainly, the scope of the shoulders on which I have stood is incalculable. This section is a mere attempt to acknowledge it.

First, I would like to thank my adviser, Professor Vijay S. Pande. It is difficult to articulate the magnitude and the ways in which an outstanding adviser impacts their graduate students. Because of Vijay, I fundamentally think differently now than I did when I began research at Stanford in 2014. There are mentors, and then there are champions; Vijay is a champion who has enabled my varying interests through his intellectual guidance and forthrightness with making critical connections. Vijay has created a special culture in our lab, both driving the key scientific advances that raised the lab to prominence well before my time and, perhaps most pertinent to my own graduate career, had a keen eye for assembling the perfect group of students and postdoctoral researchers. One of my heroes, Joe Zawinul, founder of the supergroup Weather Report, described his band as “we never solo, we always solo.” The Pande Lab is a group of incredibly bright students who each work on their own creative ideas while constantly pushing each other, and collaborating with each other, such that everyone feels motivated and able to fulfill their potential as scientists. Both in lab and in individual meetings, Vijay also exemplified how to accomplish a rare feat: to go both deeply into many individual topics while maintaining breadth, synthesizing deep insights in disciplines as disparate as structural biology and machine learning to make key insights. As a fellow physicist by undergraduate training, Vijay was a critical role model in illustrating how I could progress from a more scientifically oriented background rooted in asking “why?” to a career at the interface of science and engineering to also ask “how?”

I am blessed to be enmeshed in a lab brimming with some of the most fiercely intelligent individuals I have ever encountered. I must give special thanks to Robert McGibbon for spending many hours during my first months in lab teaching me critical concepts from molecular simulation to machine learning that served as underpinnings for the remainder of my graduate career. I would like to thank co-authors and collaborators Amir Barati Farimani, Carlos X. Hernandez, Brooke Husic, Keri McKiernan, Bharath Ramsundar, Muneeb Sultan, and Zhenqin Wu, for their teamwork. I would like to thank Alex Ferris, Huanghao Mai, Debnil Sur, and Clare Zhu for being phenomenal students. While we never published or worked directly together, I would like to thank Steven Kearnes, Bowen Liu, Matthew Harrigan, Franklin Lee, Ariana Peck, Jade Shi, Nate Stanley, and many others for their friendship and counsel.

Three collaborations outside of lab are particularly important to note. As a former experimentalist, I am reluctant to believe any model until it makes a prediction verifiable in the laboratory. I thank Dr. Susruta Majumdar and Professor Gavril Pasternak of Memorial Sloan-Kettering Cancer Center in New York for their critical role in our discovery of novel opioid agonists. In addition, I would like to thank Dr. Bitna Yi, Dr. Michael Green, and Professor Mehrdad Shamloo for their fruitful collaboration on beta adrenergic agonists.

Finally, I would like to thank Dr. Alan Cheng for his friendship and for believing in and championing our efforts to translate our advances in deep learning to real-world drug discovery settings.

During my influential rotation with Professor Ron Dror, I learned indispensable concepts about molecular simulation and G Protein Coupled Receptors. During that fruitful rotation, I had the pleasure to also work with Professor Brian Kobilka, Professor Chris Garcia, (now Professor) Aashish Manglik, A.J. Venkatakrishnan, and Naomi Latoracca.

Before my rotation with Prof. Dror, I was fortunate to be a research intern in computational biophysics under Dr. Robert Abel and Dr. Lingle Wang at Schrodinger, Inc., a computational chemistry software firm. Dr. Abel gets a special thank you for first introducing me to the field of computational chemistry in 2005 (that year is not a typo). I thank him for his mentorship and his championship of my work for the past thirteen years that played such an indispensable role in helping me reach this stage.

In the Biophysics program, I would like to thank Prof. KC Huang, Kathleen Guan, and Amy Lin for creating such a phenomenal environment for learning and research.

Prior to pivoting my focus to developing computational methods in graduate school, I was an experimental researcher as an undergraduate in the lab of Dr. David Scheinberg at Sloan-Kettering. While I no longer am at the bench modifying carbon nanotubes for use as cancer diagnostics and therapeutics, Dr. Scheinberg will always be one of my main role models as a scientist. During my time in the Scheinberg Lab, Dr. J. Justin Mulvey had an outsized role in my development as a young scientist, and as a person. Dr. Mulvey invested countless hours, and trust, in training me to become an autonomous researcher as well as an upstanding adult. To Justin I will always owe an irrecuperable debt.

My parents, Bebe and Mark Feinberg, are the embodiment of lovingkindness. They are principally responsible for fostering my incipient love of science from a very young age, and for ensuring that my feet remained firmly placed on Earth as my mind wandered through the solar system in my earlier days as an aspiring theoretical physicist. I am who I am today primarily because of my parents. I am grateful for the creative influence of my older sibling, Lila, a playwright.

Finally, if I only had my wife, who is as beautiful as she is brilliant, as accommodating as she is compassionate, and is the main source of meaning in my life, it would have been enough.

About Danielle, I have written laconically, but said voluminously.
\end{acknowledgments}

% The preface is optional.
% References: TCMOS17 1.49, WEB9 927.
%\begin{preface}
%  This is the preface.
%\end{preface}

% The Table of Contents is required.
% The Table of Contents will be automatically created for you
% using information you supply in
%     \chapter
%     \section
%     \subsection
%     \subsubsection
%     commands.
\pdfbookmark{TABLE OF CONTENTS}{Contents}
\tableofcontents

% If your thesis has tables, a list of tables is required.
% The List of Tables will be automatically created for you using
% information you supply in
%     \begin{table} ... \end{table}
% environments.
\listoftables

% If your thesis has figures, a list of figures is required.
% The List of Figures will be automatically created for you using
% information you supply in
%     \begin{figure} ... \end{figure}
% environments.
\listoffigures

% If your thesis has listings, a list of listings is required.
% The List of Listings will be automatically created for you using
% information you supply in
%     \begin{ZZlisting} ... \end{ZZlisting}
% environments.
%\ZZlistoflistings

% If your thesis has protocols, you may want to do a list of protocols.
% The List of Protocols will be automatically created for you using
% information you supply in
%     \begin{protocol} ... \end{protocol}
% environments.
%\listofprotocols

% If your thesis has schemes, you may want to do a list of schemes.
% The List of Schemes will be automatically created for you using
% information you supply in
%     \begin{scheme} ... \end{scheme}
% environments.
%\listofschemes

% List of Symbols is optional.
\begin{symbols}
  $m$& mass\cr
  $v$& velocity\cr
\end{symbols}

% List of Abbreviations is optional.
\begin{abbreviations}
  abbr& abbreviation\cr
  bcf& billion cubic feet\cr
  BMOC& big man on campus\cr
\end{abbreviations}



% Abstract is required.
% Note that the information for the first paragraph of the output
% doesn't need to be input here...it is put in automatically from
% information you supplied earlier using \title, \author, \degree,
% and \majorprof.
% Reference: PU 17.
\begin{abstract}%

This dissertation introduces single molecule localization microscopy and covers work discussed in the following papers:
“BRD4 phosphorylation state regulates structure of chromatin nanodomains” [1] describes the role of the BRD4 phosphoswitch in the maintenance of chromatin nanodomains via super resolution microscopy and molecular dynamics simulation. We build on the notion that chromatin binding activity of BRD4 is regulated by phosphorylation by demonstrating that BRD4 phosphorylation regulated chromatin packing and mobility in mammalian nuclei.

“Denoising diffusion probabilistic models for blind deconvolution in single molecule localization microscopy” [2] describes an algorithm that leverages a novel paradigm for deep generative modeling using Gaussian diffusions in order to enhance the resolution of localization microscopy images. 

Single-molecule localization microscopy (SMLM) techniques, such as direct stochastic optical reconstruction microscopy (dSTORM), can be used to produce a pointillist representation of fluorescently-labeled biological structures at diffraction-unlimited precision. Direct STORM approaches leverage the deactivation of standard fluorescent tags, followed by spontaneous or photoinduced reactivation, allowing resolution of fluorophores at distances below the diffraction limit. This basic principle remains one of the method's primary limitations - standard SMLM fitting routines require tight control of activation and reactivation to maintain sparse emitters, presenting a tradeoff between imaging speed and labeling density. Here, I present two parallel projects, which aim to push the current state of the art in SMLM and apply SMLM to the study of gene regulation. The former represents a novel localization technique for dense SMLM, based on deep probabilistic modeling and photon statistics. In the latter, conventional dSTORM is adapted for live cell imaging of chromatin nanodomains, demonstrating that BRD4 protein concentrates in nucleosome depleted regions.

\end{abstract}
