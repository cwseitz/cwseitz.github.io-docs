\ProvidesFile{ch-front.tex}[2022-10-05 front matter chapter]
%
%  This is the ``front matter'' for the thesis.
%
%  REFERENCES
%
%    TCMOS17
%      The Chicago Manual of Style Online, 17th edition.
%      https://www.chicagomanualofstyle.org/home.html
%      retrieved on 2020-02-29
%
%    TEMPL
%      Thesis and Disertation Office Templates.
%      https://www.purdue.edu/gradschool/research/thesis/templates.html
%      retrieved on 2020-02-29
%
%    WNNCD
%    Webster's Ninth New Collegiate Dictionary.
%

%
%   Only Purdue University uses this page
%
%   Comment out \begin{statement} through \end{statement}
%   if you are not at Purdue University.
%
% Statement of Thesis/Dissertation Approval Page
% This page is REQUIRED.  The page should be numbered "2"
% and should NOT be listed in your TABLE OF CONTENTS.
\begin{statement}
  % Delete or add \entry commands as needed for all committe members.
  \entry{Dr. Gautam Vemuri, Chair}{Department of Physics}
  \entry{Dr. Jing Liu}{Department of Physics}
  \entry{Dr. Ruihua Cheng}{Department of Physics}
  \entry{Dr. Stephen Wassall}{Department of Physics}
  \entry{Dr. Horia Petrache}{Department of Physics}
  % There should be one \approvedby command containing the
  % "FORM 9 THESIS FORM HEAD NAME HERE" (from TEMPL, retrieved on 2020-03-01).
  \approvedby{TBD}
\end{statement}

% Dedication page is optional.
% A name and often a message in tribute to a person or cause.
% References: WEB9 332.
\begin{dedication}
I dedicate this thesis to Alexandra, who has steadily encouraged my pursuit of a doctoral degree. I am forever indebted for your patience, understanding, and proofreading.\\
\vspace{2in}
\textit{What we observe is not nature itself, but nature exposed to our method of questioning} \\Werner Heisenberg \\
\vspace{1in}
\textit{To deal with a 14-dimensional space, visualize a 3-dimensional space and say ``fourteen'' to yourself very loudly. Everyone does it}\\ Geoffrey Hinton \\
\vspace{1in}
\textit{Information is the resolution of uncertainty} \\Claude Shannon
\vspace{1in}

\end{dedication}

% Acknowledgements page is optional but most theses include
% a brief statement of appreciation or recognition of special
% assistance.
\begin{acknowledgments}

\end{acknowledgments}

% The preface is optional.
% References: TCMOS17 1.49, WEB9 927.
%\begin{preface}
%  This is the preface.
%\end{preface}

% The Table of Contents is required.
% The Table of Contents will be automatically created for you
% using information you supply in
%     \chapter
%     \section
%     \subsection
%     \subsubsection
%     commands.
\pdfbookmark{TABLE OF CONTENTS}{Contents}
\tableofcontents

% If your thesis has tables, a list of tables is required.
% The List of Tables will be automatically created for you using
% information you supply in
%     \begin{table} ... \end{table}
% environments.
\listoftables

% If your thesis has figures, a list of figures is required.
% The List of Figures will be automatically created for you using
% information you supply in
%     \begin{figure} ... \end{figure}
% environments.
\listoffigures

% If your thesis has listings, a list of listings is required.
% The List of Listings will be automatically created for you using
% information you supply in
%     \begin{ZZlisting} ... \end{ZZlisting}
% environments.
%\ZZlistoflistings

% If your thesis has protocols, you may want to do a list of protocols.
% The List of Protocols will be automatically created for you using
% information you supply in
%     \begin{protocol} ... \end{protocol}
% environments.
%\listofprotocols

% If your thesis has schemes, you may want to do a list of schemes.
% The List of Schemes will be automatically created for you using
% information you supply in
%     \begin{scheme} ... \end{scheme}
% environments.
%\listofschemes

% List of Symbols is optional.
\begin{symbols}
  $\bold{x}$& An image at base resolution\cr
  $\bold{y}$& An image at higher resolution\cr
  $\delta$& Pixel lateral width\cr
  $k$& Pixel index\cr
  $\theta$& A parameter\cr
  $u,v$& Cartesian coordinates in two-dimensions\cr
  $g_k$& Pixel-wise gain\cr
  $o_k$& Pixel-wise offset\cr
  $w_k$& Pixel-wise readout noise standard deviation\cr
  $\mu_k$& Pixel-wise expected value\cr
  $s_k$&  Pixel-wise  measured signal\cr
  $\xi_k$&  Pixel-wise measured readout noise\cr
  $O$& Point spread function\cr
  $\mathrm{erf}$& Error function\cr
  $\sigma_{\bold{x}}$& Gaussian PSF width\cr
  $\sigma_{\bold{y}}$& Kernel width for kernel density estimate\cr
  $\epsilon$& An image of pure Gaussian noise\cr
  $\beta$& Diffusion model noise variance\cr
  $\mathrm{SNR}$& Diffusion model signal to noise ratio\cr
  $\psi$& Diffusion model parameters\cr
  $\phi$& Augmentation network parameters\cr
  $\mathcal{L}$& An objective function\cr
  $D_{KL}$& KL-divergence\cr
  $p$& A discrete or continuous probability distribution\cr
  $\mathbb{E}_p$& Expectation with respect to a distribution $p$\cr
  $I$& Fisher information matrix\cr
  $\lambda$& Expected number of background counts per frame\cr
  $d$& Lateral dimension of a region of interest\cr
  $\zeta$& Photon detection probability\cr
  $\tau$& Delay time\cr
  $g^{(2)}(\tau)$& Second order coherence function\cr
  $B$& Expected number of signal-background coincidences per frame\cr
  $G^{(2)}(m)$& Measured number of signal-signal coincidences at lag time $m$\cr
  $N_{\mathrm{frames}}$& Number of frames\cr
  $N$& Number of active fluorescent emitters\cr
  $N^{*}$& Maximum a posteriori estimate of number of active emitters\cr
  $n$& Number of photon counts in a frame\cr
  $\ell$& Log-likelihood\cr
  $\hat{a}$& Ladder operator\cr
  $\hat{E}$& Electric field operator\cr
  $\rho$& Density matrix or number density for molecular dynamics\cr
  $L(r)$& Besag's L-function\cr
  $K(r)$& Ripley's K-function\cr
  $G(r)$& Nearest neighbor distribution function\cr
  $\lambda$& (Point Pattern Analysis) Intensity of a point process\cr
  $\gamma$& Friction tensor\cr
  $\xi$& A delta-correlated Gaussian noise\cr
  $k_{B}$& Boltzmann's constant\cr
  $T$& Temperature\cr
  $U$& Potential energy\cr
  $D$& Diffusion coefficient\cr
  $\epsilon$& (Molecular Dynamics) Energy\cr
  $r_{0}$& Harmonic bond equilibrium length\cr
  $R_{0}$& Binder potential equilibrium length\cr
  $\alpha$& Anomalous diffusion exponent\cr
  
  
\end{symbols}

% List of Abbreviations is optional.
%\begin{abbreviations}
%  abbr& abbreviation\cr
%  bcf& billion cubic feet\cr
%  BMOC& big man on campus\cr
%\end{abbreviations}



% Abstract is required.
% Note that the information for the first paragraph of the output
% doesn't need to be input here...it is put in automatically from
% information you supplied earlier using \title, \author, \degree,
% and \majorprof.
% Reference: PU 17.
\begin{abstract}%

Single-molecule localization microscopy (SMLM) techniques, such as direct stochastic optical reconstruction microscopy (dSTORM), can be used to produce a pointillist representation of fluorescently-labeled biological structures at diffraction-unlimited precision. Direct STORM approaches leverage the deactivation of standard fluorescent tags, followed by spontaneous or photoinduced reactivation, allowing resolution of fluorophores at distances below the diffraction limit. This dissertation introduces single molecule localization microscopy and covers its application as discussed in the following papers:

\textit{BRD4 phosphorylation regulates structure of chromatin nanodomains} [1] describes the role of the BRD4 phosphoswitch in the maintenance of chromatin nanodomains via super resolution microscopy and molecular dynamics simulation. We build on the notion that chromatin binding activity of BRD4 is regulated by phosphorylation by demonstrating that BRD4 phosphorylation regulated chromatin packing and mobility in mammalian nuclei.

\textit{Uncertainty-aware localization microscopy by variational diffusion} [2] describes a novel algorithm that leverages a diffusion model in order to model a posterior distribution on super-resolution localization microscopy images. 
Fast extraction of physically relevant information from images using deep neural networks has led to significant advances in fluorescence microscopy and its application to the study of biological systems. For example, the application of deep networks for kernel density (KD) estimation in single molecule localization microscopy (SMLM) has accelerated super-resolution imaging of densely-labeled structures in the cell. However, simple and interpretable uncertainty quantification is lacking in these applications, and remains a necessary modeling component in high-risk research. We propose a generative modeling framework for KD estimation in SMLM based on variational diffusion. This approach allows us to probe the structure of the posterior on KD estimates, creating an additional avenue toward quality control. We demonstrate that data augmentation with traditional SMLM architectures followed by a diffusion process permits simultaneous high-fidelity super-resolution with uncertainty estimation of regressed KDEs. 


\end{abstract}
