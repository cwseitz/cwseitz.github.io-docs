\ProvidesFile{appendix.tex}[Appendix]

\chapter{Appendix A}
\ix{physics//Physics appendix}

\subsection{Spatial coherence for an isolated emitter}

Photoswitching fluorescent molecules are described in the density matrix formalism

\begin{equation*}
\rho = \sum_{k}\xi_{k}\ket{\alpha_{k}}\bra{\alpha_{k}}\;\; \sum_{k}\xi_{k} = 1
\end{equation*}


where $\ket{\alpha_{k}}$ is a coherent state with amplitude $\alpha_{k}$ i.e., $\langle n\rangle = \bra{\alpha_{k}} n\ket{\alpha_{k}} = \lvert\alpha_{k}^{2}\rvert$. Typically $\xi_{k}$ and $\langle n_{k}\rangle$ are heterogeneous. We consider a simplified model consisting of a single mode field 

\begin{equation*}
E^{+}(r_{i}) = h(r_{i}-s_{0})\hat{a}_{n}
\end{equation*}

\begin{equation*}
g^{(2)}_{ij}(0) = \frac{\langle E^{-}(r_{i})E^{-}(r_{j})E^{+}(r_{i})E^{+}(r_{j}) \rangle}{\langle E^{-}(r_{i})E^{+}(r_{i})\rangle\langle E^{-}(r_{j})E^{+}(r_{j})\rangle} = \frac{\mathrm{Tr}(E^{-}(r_{i})E^{-}(r_{j})E^{+}(r_{i})E^{+}(r_{j})\rho)}{\mathrm{Tr}(E^{-}(r_{i})E^{+}(r_{i})\rho)\mathrm{Tr}(E^{-}(r_{j})E^{+}(r_{j})\rho)}
\end{equation*}

Terms related to point spread function will cancel. It is instructive to compute

\begin{align*}
\mathrm{Tr}(a^{\dagger}a^{\dagger}aa \left(\xi_{k}\ket{\alpha_{k}}\bra{\alpha_{k}}\right) &= \mathrm{Tr}\left(\xi_{k} e^{-\lvert\alpha\rvert^{2}}\sum_{n,m}^{\infty}\frac{\alpha^{n}}{n!}\ket{n}\bra{m}\right)\\
&= \mathrm{Tr}\left(\xi_{k} e^{-\lvert\alpha\rvert^{2}}\sum_{n}^{\infty}\frac{\lvert\alpha\rvert^{2n}}{n!}n(n-1)\right)\\
&= \mathrm{Tr}\left(\xi_{k} e^{-\lvert\alpha\rvert^{2}}\sum_{n=2}^{\infty}\frac{\lvert\alpha\rvert^{2n}}{(n-2)!}\right)\\
&= \xi_{k}\lvert\alpha_{k}\rvert^{4}
\end{align*}

Similarly,

\begin{align*}
\mathrm{Tr}(a^{\dagger}a \left(\xi \ket{\alpha}\bra{\alpha}\right)) &= \mathrm{Tr}\left(\xi e^{-\lvert\alpha\rvert^{2}}\sum_{n,m}^{\infty}\frac{\alpha^{n}(\alpha^{m})^{*}}{\sqrt{n!}\sqrt{m!}}a^{\dagger}a\ket{n}\bra{m} \right)\\
&= \xi e^{-\lvert\alpha\rvert^{2}}\sum_{n=0}^{\infty}\frac{(\lvert\alpha\rvert^{2})^{n}}{n!}n\\
&= \xi e^{-\lvert\alpha\rvert^{2}}\sum_{n=1}^{\infty}\frac{(\lvert\alpha\rvert^{2})^{n}}{(n-1)!}\\
&= \xi e^{-\lvert\alpha\rvert^{2}}\left(\lvert\alpha\rvert^{2} + \frac{\lvert\alpha\rvert^{4}}{1!} + \frac{\lvert\alpha\rvert^{6}}{2!}+...\right)\\
&= \xi e^{-\lvert\alpha\rvert^{2}}\lvert\alpha\rvert^{2}\left(1 + \frac{\lvert\alpha\rvert^{2}}{1!} + \frac{\lvert\alpha\rvert^{3}}{2!}+...\right)\\
&= \xi e^{-\lvert\alpha\rvert^{2}}e^{\lvert\alpha\rvert^{2}}\lvert\alpha\rvert^{2} = \xi\lvert\alpha\rvert^{2}
\end{align*}

\begin{align*}
\mathrm{Tr}(a a^{\dagger} \left(\xi \ket{\alpha}\bra{\alpha}\right)) &= \mathrm{Tr}\left(\xi e^{-\lvert\alpha\rvert^{2}}\sum_{n,m}^{\infty}\frac{\alpha^{n}(\alpha^{m})^{*}}{\sqrt{n!}\sqrt{m!}}a a^{\dagger}\ket{n}\bra{m} \right)\\
&= \xi e^{-\lvert\alpha\rvert^{2}}\sum_{n=0}^{\infty}\frac{(\lvert\alpha\rvert^{2})^{n}}{n!}(n+1)\\
&= \xi e^{-\lvert\alpha\rvert^{2}}\left(\sum_{n=1}^{\infty}\frac{(\lvert\alpha\rvert^{2})^{n}}{(n-1)!} + e^{\lvert\alpha\rvert^{2}}\right)\\
&= \xi e^{-\lvert\alpha\rvert^{2}}\left(\lvert\alpha\rvert^{2}e^{\lvert\alpha\rvert^{2}} + e^{\lvert\alpha\rvert^{2}}\right) = \xi(\lvert\alpha\rvert^{2} + 1)
\end{align*}

Putting it all together yields a simple expression for the two-point coherence function

\begin{equation*}
g^{(2)}_{ij}(0) = \frac{\sum_{k}\xi_{k}\lvert\alpha_{k}\rvert^{4}}{\left(\sum_{k}\xi_{k}\lvert\alpha_{k}\rvert^{2}\right)\left(\sum_{k}\xi_{k}\lvert\alpha_{k}\rvert^{2}\right)}
\end{equation*}

For example, if we have a two-level system consisting of a fluorescent state with amplitude $\alpha$ and the vacuum state, this becomes

\begin{equation*}
g^{(2)}_{ij}(0) = \frac{\xi\lvert\alpha\rvert^{4}}{\xi^{2}\lvert\alpha\rvert^{4}} = \frac{1}{\xi}
\end{equation*}

As $\xi\rightarrow 1$ (always on) we recover a coherent state. As $\xi\rightarrow 0$ we observe $g^{(2)}_{ij}(0) > 1$ i.e., bunching.

\subsection{Generalization to nonzero background}

\begin{equation*}
E_{0}^{+}\sim \sum_{j=1}^{M}\delta(s-s_{j})a_{j} \;\; E^{+}(r_{i}) = \int d^{2}s E_{0}^{+} = \sum_{n}h(r_{i}-s_{n})a_{n}
\end{equation*}

\begin{equation*}
\rho_{S} = \xi\ket{\alpha}\bra{\alpha} + (1-\xi)\ket{0}\bra{0}\;\;\rho_{B} = \ket{\beta}\bra{\beta}\;\;\rho = \rho_{S}\otimes\rho_{B}
\end{equation*}

\begin{equation*}
E(r_{i})^{+} = E_{S}(r_{i})^{+} + E_{B}(r_{i})^{+} = h(r_{i}-s_{n})a_{S} + a_{B}
\end{equation*}

\begin{align*}
G^{2}_{ij}(0) &= \langle(E_{S}^{\dagger} + E_{B}^{\dagger}) (E_{S}^{\dagger} + E_{B}^{\dagger})( E_{S} + E_{B}) (E_{S} + E_{B})\rangle \\
&= h_{i}^{2}h_{j}^{2}\langle a_{S}^{\dagger}a_{S}^{\dagger}a_{S}a_{S}\rangle + h_{i}^{2}\langle a_{S}^{\dagger}a_{B}^{\dagger}a_{S}a_{B}\rangle + h_{j}^{2}\langle a_{B}^{\dagger}a_{S}^{\dagger}a_{B}a_{S}\rangle  + \langle a_{B}^{\dagger}a_{B}^{\dagger}a_{B}a_{B}\rangle  \\
&= \xi(h_{i}^{2}h_{j}^{2}\lvert\alpha\rvert^{4}+ h_{i}^{2}\lvert\alpha\rvert^{2}\lvert\beta\rvert^{2} + h_{j}^{2}\lvert\alpha\rvert^{2}\lvert\beta\rvert^{2}\rangle  + \lvert\beta\rvert^{4} ) \\
&= \xi(h_{i}^{2}h_{j}^{2}\lvert\alpha\rvert^{4}+ \lvert\alpha\rvert^{2}\lvert\beta\rvert^{2}(h_{i}^{2} + h_{j}^{2})  + \lvert\beta\rvert^{4}) \\
\end{align*}

The normalized second order coherence function then reads

\begin{align*}
g^{2}_{ij}(0) &= \frac{\xi h_{i}^{2}h_{j}^{2}N_{0}^{2} + \xi N_{0}B_{0}(h_{i}^{2} + h_{j}^{2}) + B_{0}^{2}}{\xi^{2} h_{i}^{2}h_{j}^{2}N_{0}^{2} + \xi N_{0}B_{0}(h_{i}^{2}+h_{j}^{2}) +  B_{0}^{2}}
\end{align*}

Notice the PSF factor $h_{i}$ appears squared. This squared value can be seen as the probability of photon detection at a point $s_i$, while $h_{i}$ is the amplitude of the electric field. 

\subsection{Ergodicity of photoswitching}

In general, Markov jump processes are non-ergodic, meaning that their time averages and ensemble averages are not equal. The use of $\xi$ as a probability is only valid when the observation duration (exposure time) is much longer than the characteristic switching time. However, the use of $\xi$ above is quite convenient, so we look to determine how long our exposure must be for the emitter to be considered in equilibrium. For a two state process

\begin{equation*}
P(t) = e^{Wt}P(0)\rightarrow \dot{P}(t) = We^{Wt}P(0)
\end{equation*}

\begin{equation*}
e^{Wt} = I + W\frac{1-e^{-2\lambda t}}{2\lambda}
\end{equation*}

where $\lambda = (\lambda_1 + \lambda_2)/2$. Now,

\begin{equation*}
\dot{P}(t) = W\left(I + W\frac{1-e^{-2\lambda t}}{2\lambda}\right)P(0)
\end{equation*}

It can then be shown that the individual gradients are


\begin{equation*}
\begin{pmatrix}
\frac{\lambda_1 \lambda_2 \left(1-e^{-t (\lambda_1+\lambda_2)}\right)}{\lambda_1+\lambda_2}-\lambda_1 \left(1-\frac{\lambda_1 \left(1-e^{-t (\lambda_1+\lambda_2)}\right)}{\lambda_1+\lambda_2}\right) & \lambda_2 \left(1-\frac{\lambda_2 \left(1-e^{-t (\lambda_1+\lambda_2)}\right)}{\lambda_1+\lambda_2}\right)-\frac{\lambda_1 \lambda_2 \left(1-e^{-t (\lambda_1+\lambda_2)}\right)}{\lambda_1+\lambda_2} \\
\lambda_1 \left(1-\frac{\lambda_1 \left(1-e^{-t (\lambda_1+\lambda_2)}\right)}{\lambda_1+\lambda_2}\right)-\frac{\lambda_1 \lambda_2 \left(1-e^{-t (\lambda_1+\lambda_2)}\right)}{\lambda_1+\lambda_2} & \frac{\lambda_1 \lambda_2 \left(1-e^{-t (\lambda_1+\lambda_2)}\right)}{\lambda_1+\lambda_2}-\lambda_2 \left(1-\frac{\lambda_2 \left(1-e^{-t (\lambda_1+\lambda_2)}\right)}{\lambda_1+\lambda_2}\right)
\end{pmatrix}
\end{equation*}



\subsection{Details of the Gaussian PSF}\

We will derive the gradients for the integrated astigmatic Gaussian, since it is the more general case. As before, define $i_{0} = g_{k}\gamma\Delta t N_{0}$ such that $\mu_{k}' = i_{0}\lambda_{k}$

\begin{equation*}
J_{x_{0}} = \beta_{k}\lambda_{y}\frac{\partial \lambda_{x}}{\partial x_{0}} \;\; J_{y_{0}} = \beta_{k}\lambda_{x}\frac{\partial \lambda_{y}}{\partial y_{0}}\;\;\; J_{z_{0}}  = \frac{\partial \mu_{k}'}{\partial \sigma_{x}}\frac{\partial \sigma_{x}}{\partial z_{0}} + \frac{\partial \mu_{k}'}{\partial \sigma_{y}}\frac{\partial \sigma_{y}}{\partial z_{0}}
\end{equation*}

\begin{align*}
J_{x_{0}} &= \beta_{k}\lambda_{y}\frac{\partial \lambda_{x}}{\partial x_{0}} \\
&= \frac{\beta_{k}\lambda_{y}}{2}\frac{\partial}{\partial x_{0}}\left(\mathrm{erf}\left(\frac{x_{k}+\frac{1}{2}-x_{0}}{\sqrt{2}\sigma_{x}}\right) -\mathrm{erf}\left(\frac{x_{k}-\frac{1}{2}-x_{0}}{\sqrt{2}\sigma_{x}}\right)\right)\\
&= \frac{\beta_{k}\lambda_{y}}{\sqrt{2\pi}\sigma_{x}}\left(\mathrm{exp}\left(\frac{(x_{k}-\frac{1}{2}-x_{0})^{2}}{2\sigma_{x}^{2}}\right) -\mathrm{exp}\left(\frac{(x_{k}+\frac{1}{2}-x_{0})^{2}}{2\sigma_{x}^{2}}\right)\right)
\end{align*}

\begin{align*}
J_{y_{0}} &= \beta_{k}\lambda_{x}\frac{\partial \lambda_{y}}{\partial y_{0}} \\
&= \frac{\beta_{k}\lambda_{x}}{2}\frac{\partial}{\partial y_{0}}\left(\mathrm{erf}\left(\frac{y_{k}+\frac{1}{2}-y_{0}}{\sqrt{2}\sigma_{y}}\right) -\mathrm{erf}\left(\frac{y_{k}-\frac{1}{2}-y_{0}}{\sqrt{2}\sigma_{y}}\right)\right)\\
&= \frac{\beta_{k}\lambda_{x}}{\sqrt{2\pi}\sigma_{y}}\left(\mathrm{exp}\left(\frac{(y_{k}-\frac{1}{2}-y_{0})^{2}}{2\sigma_{y}^{2}}\right) -\mathrm{exp}\left(\frac{(y_{k}+\frac{1}{2}-y_{0})^{2}}{2\sigma_{y}^{2}}\right)\right)
\end{align*}

\begin{align*}
J_{\sigma_{x}} &= \beta_{k}\lambda_{y}\frac{\partial \lambda_{x}}{\partial \sigma_{x}} \\
&= \frac{\beta_{k}\lambda_{y}}{2}\frac{\partial}{\partial \sigma_{x}}\left(\mathrm{erf}\left(\frac{x_{k}+\frac{1}{2}-x_{0}}{\sqrt{2}\sigma_{x}}\right) -\mathrm{erf}\left(\frac{x_{k}-\frac{1}{2}-x_{0}}{\sqrt{2}\sigma_{x}}\right)\right)\\
&= \frac{\beta_{k}\lambda_{y}}{\sqrt{2\pi}}\left(\frac{\left(x-x_{0}-\frac{1}{2}\right) e^{-\frac{\left(x-x_{0}-\frac{1}{2}\right)^2}{2 \sigma_{x} ^2}}}{\sigma_{x} ^2}-\frac{ \left(x-x_{0}+\frac{1}{2}\right) e^{-\frac{\left(x-x_{0}+\frac{1}{2}\right)^2}{2 \sigma_{x} ^2}}}{\sigma_{x} ^2}\right)
\end{align*}

\begin{align*}
J_{\sigma_{y}} &= \beta_{k}\lambda_{x}\frac{\partial \lambda_{y}}{\partial \sigma_{y}} \\
&= \frac{\beta_{k}\lambda_{x}}{2}\frac{\partial}{\partial \sigma_{y}}\left(\mathrm{erf}\left(\frac{y_{k}+\frac{1}{2}-y_{0}}{\sqrt{2}\sigma_{y}}\right) -\mathrm{erf}\left(\frac{y_{k}-\frac{1}{2}-y_{0}}{\sqrt{2}\sigma_{y}}\right)\right)\\
&= \frac{\beta_{k}\lambda_{x}}{\sqrt{2\pi}}\left(\frac{\left(y-y_{0}-\frac{1}{2}\right) e^{-\frac{\left(y-y_{0}-\frac{1}{2}\right)^2}{2 \sigma_{y} ^2}}}{\sigma_{y} ^2}-\frac{ \left(y-y_{0}+\frac{1}{2}\right) e^{-\frac{\left(y-y_{0}+\frac{1}{2}\right)^2}{2 \sigma_{y} ^2}}}{\sigma_{y} ^2}\right)
\end{align*}

Luckily, computing the Hessian matrix for (2.9) is tractable, and is actually quite simple when one takes advantage of the chain rule for Hessian matrices. Looking at (2.9), the likelihood is a hierarchical function that maps a vector space $\Theta$ to a vector space $\Lambda$ to a scalar value. Formally, we define $T: \Theta \rightarrow \Lambda$ and $W: \Lambda \rightarrow \mathbb{R}$. The parameter vector $(x_{0},y_{0},z_{0}, \sigma_{0}, N_{0})\in \Theta$, the Poisson rate vector $\vec{\lambda} \in \Lambda$ and $\ell \in \mathbb{R}$. Note that we choose to optimize $\sigma_{x}$ and $\sigma_{y}$ directly and compute $z_{0}$ to simplify the computation of the Hessian. To get the Hessian, we need the chain-rule for Hessian matrices, which can be quickly computed in terms of the jacobian and hessian of $T$ and $W$.


\begin{equation*}
H_{\ell} = J_{\mu}^{T} H_{\ell} J_{\mu} + (J_{\ell}\otimes I_{n})H_{\mu}
\end{equation*}

where we have used $J_{\mu}$ to represent the jacobian of $T$ and $J_{\ell}$ for the jacobian of $W$. Similar notation is used for the corresponding Hessian matrices. 
In the 3D case, the Hessian matrix is not directly separable since $\mu \propto \lambda_{x}(x_{0},\sigma_{0},\sigma_{x})\lambda_{y}(y_{0},\sigma_{0},\sigma_{y})$. To see this, an abstract representation of the Hessian reads 


\subsection{Fisher information for 2D integrated gaussian}

For the 2D integrated gaussian point spread function, the Hessian only contains separable second order derivatives, so the Fisher information matrix takes on a convenient form

\begin{equation}
I_{ij}(\theta) = \underset{\theta}{\mathbb{E}}\left(\frac{\partial \ell}{\partial\theta_{i}}\frac{\partial\ell}{\partial\theta_{j}}\right) 
\end{equation}

For an arbitrary parameter then we have

\begin{align*}
\frac{\partial \ell}{\partial \theta_{i}} &= \frac{\partial}{\partial \theta_{i}} \sum_{k}  x_{k}\log x_{k} + \mu_{k}' - x_{k}\log\left(\mu_{k}'\right)\\
&= \sum_{k} \frac{\partial \mu_{k}'}{\partial\theta_{i}} \left(\frac{\mu_{k}'-x_{k}}{\mu_{k}'}\right)
\end{align*}

\begin{equation*}
I_{ij}(\theta) = \underset{\theta}{\mathbb{E}}\left(\sum_{k}\frac{\partial \mu_{k}'}{\partial\theta_{i}}\frac{\partial \mu_{k}'}{\partial\theta_{j}} \left(\frac{\mu_{k}'-x_{k}}{\mu_{k}'}\right)^{2}\right) = \sum_{k}\frac{1}{\mu_{k}'}\frac{\partial \mu_{k}'}{\partial\theta_{i}}\frac{\partial \mu_{k}'}{\partial\theta_{j}}
\end{equation*}

To compute the bound, it turns out all we need is the jacobian $\frac{\partial \mu_{k}'}{\partial\theta_{j}} $.


