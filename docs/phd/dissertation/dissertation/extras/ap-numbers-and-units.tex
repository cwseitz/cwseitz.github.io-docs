\ProvidesFile{ap-numbers-and-units.tex}[2022-10-05 numbers and units appendix]

%  Primary sources:
%      See https://www.bipm.org/utils/common/pdf/si-brochure/SI-Brochure-9-EN.pdf.
%      See https://www.nist.gov/pml/special-publication-811/nist-guide-si-chapter-4-two-classes-si-units-and-si-prefixes.
%
%  Notes:
%      See https://www.iso.org/standard/60241.html.

% Historic Vote Ties Kilogram and Other Units to Natural Constants
% NIST
% https://www.nist.gov/news-events/news/2018/11/historic-vote-ties-kilogram-and-other-units-natural-constants
% created 2018-11-16
% updated 2018-12-21
% last retrieved 2018-12-22

% https://www.bipm.org/utils/common/pdf/CGPM-2018/26th-CGPM-Resolutions.pdf
% Resolutions adopted
% 26^e CGPM
% Versailes
% 13--16 November 2018
% last retrieved 2018-12-30

% author Joseph Wright
% date 2018-05-17
% retrieved 2018-12-29
% url ftp://ftp.dante.de/tex-archive/macros/latex/exptl/siunitx/siunitx.pdf


\begin{VerbatimOut}{z.out}
\chapter{NUMBERS AND UNITS}
\end{VerbatimOut}

\MyIO


\begin{VerbatimOut}{z.out}

Note to self: scientific prefixes, scientific suffixes, tables.

The puthesis 2.0 and after documentclass uses the siunitx
package with some extra definitions in the puthesis.cls
file to do numbers and units.
\end{VerbatimOut}

\MyIO


\begin{VerbatimOut}{z.out}

\section{Number Examples}
\end{VerbatimOut}

\MyIO


\begin{VerbatimOut}{z.out}
\noindent\begin{tabular}{@{}lll@{}}
  \bfseries Input& \bfseries Output& \bfseries Comment\\
  \tabularspace
  \verb+\num{-0.12345}+& \num{-0.12345}& note the small space after the ``3''\\
  \verb+\num{-0.1234}+&
    \num{-0.1234}&
    note no space between the ``3'' and ``4''\\
  \verb+\num{-.123}+& \num{-.123}& the ``0.'' is inserted automatically\\
  \verb+\num{123}+& \num{123}\\
  \verb+\num{1234}+& \num{1234}\\
  \verb+\num{12345}+& \num{12345}& note the small space after the ``2''\\
  \verb+\num{2e4}+& \num{2e4}\\
  \verb+\num{e5}+& \num{e5}\\
  \verb+\num{2.34567e6}+&
    \num{2.34567e6}&
    note the small space after the ``5''\\
\end{tabular}
\end{VerbatimOut}

\MyIO


\begin{VerbatimOut}{z.out}

\section{Unit Examples}
\end{VerbatimOut}

\MyIO


\begin{VerbatimOut}{z.out}

See page~\pageref{se:Complete-List-of-Units}
for the complete list
of units defined by \PurdueThesisLogo.

\noindent\begin{tabular}{@{}lll@{}}
  \bfseries Input& \bfseries Output& \bfseries Comment\\
  \tabularspace
  \verb+\si{\kg}+& \si{\kg}& kilogram\\
  \verb+\si{\m}+& \si{\m}& meter\\
  \verb+\si{\kg\per\m\squared}+&
    \si{\kg\per\m\squared}&
    \(= \si{\kg}/\si{\m\squared}\)\\
\end{tabular}
\end{VerbatimOut}

\MyIO


\begin{VerbatimOut}{z.out}

\section{Combined Number and Unit Examples}
\end{VerbatimOut}

\MyIO


\begin{VerbatimOut}{z.out}
\begin{tabular}{@{}lll@{}}
  \bfseries Input& \bfseries Output& \bfseries Comment\\
  \tabularspace
  \verb+\SI{12}{\kg}+& \SI{12}{\kg}& 12 kilograms\\
  \verb+\SI{34}{\m}+&  \SI{34}{\m}& 34 meters\\
  % The next input line is too wide for the margins
  % so I'm splitting it into pieces.
  \verb+\SI{4.5e3}{\kg\per\m\squared}+&
    \SI{4.5e3}{\kg\per\m\squared}&
    \(= \num{4.5e3}\,\si{\kg}/\si{\m\squared}\)\\
\end{tabular}
\end{VerbatimOut}

\MyIO


\begin{VerbatimOut}{z.out}

How many seconds are in a non-leap year that does not have any leap seconds?
% I tried several things and couold not get \cancel to work with \per.
% Mark Senn    2019-12-29
\begin{align*}
           \frac{\SI{365}{\cancel\d}}{\si{\y}}
    \times \frac{\SI{24}{\cancel\h}}{\si{\cancel\d}}
    \times \frac{\SI{60}{\cancel\min}}{\si{\cancel\h}}         
    \times \frac{\SI{60}{\s}}{\si{\cancel\min}}         
    % From http://www.emerson.emory.edu/services/latex/latex_119.html
    %     Spacing in Math Mode
    %     In a math environment, LaTeX ignores the spaces you type
    %     and puts in the spacing that it thinks is best. LaTeX formats
    %     mathematics the way it's done in mathematics texts. If you
    %     want different spacing, LaTeX provides the following four
    %     commands for use in math mode:
    %         \; - a thick space
    %         \: - a medium space
    %         \, - a thin space
    %         \! - a negative thin space
    & = \num{31536000}\;\frac{\si{\s}}{\si{\y}}\\
    & = \SI{31536000}{\s\per\y}\\
    & \approx \SI{3e7}{\s\per\y}\\
    & \approx \text{30 million\,}\si{\s\per\y}\\
\end{align*}
\end{VerbatimOut}

\MyIO


\begin{VerbatimOut}{z.out}

\section{Binary Prefixes}
\end{VerbatimOut}

\MyIO


\begin{VerbatimOut}{z.out}

The \verb+\kibi+ \ldots \verb+\yobi+
commands are defined immediately after the \verb+\usepackage{siunitx}+ command
in the PurdueThesis.cls file.
\end{VerbatimOut}

\MyIO


\begin{VerbatimOut}{z.out}

\newcolumntype{m}{>{$}r<{$}}  % math mode version of "r" column type
\renewcommand{\t}[4]{\(2^{#1}\) bytes is a #2, \(10^{#3}\) bytes is a #4}
\begin{tabular}{@{}mllll@{}}
  \multicolumn{1}{l}{\bfseries Power}&
    \bfseries Prefix&
    \bfseries Symbol&
    \bfseries Command&
    \bfseries Comment\\
  \tabularspace
  10& kibi& \unit{\kibi\nounit}& \verb+\si{\kibi}+& \t{10}{KB}{3}{KiB}\\
  20& mebi& \unit{\mebi\nounit}& \verb+\si{\mebi}+& \t{20}{MB}{6}{MiB}\\
  30& gibi& \unit{\gibi\nounit}& \verb+\si{\gibi}+& \t{30}{GB}{9}{GiB}\\
  40& tebi& \unit{\tebi\nounit}& \verb+\si{\tebi}+& \t{40}{TB}{12}{TiB}\\
  50& pebi& \unit{\pebi\nounit}& \verb+\si{\pebi}+& \t{50}{PB}{15}{PiB}\\
  60& exbi& \unit{\exbi\nounit}& \verb+\si{\exbi}+& \t{60}{EB}{18}{EiB}\\
  70& zebi& \unit{\zebi\nounit}& \verb+\si{\zebi}+& \t{70}{ZB}{21}{ZiB}\\
  80& yobi& \unit{\yobi\nounit}& \verb+\si{\yobi}+& \t{80}{YB}{24}{YiB}\\
\end{tabular}
\end{VerbatimOut}

\MyIO


\begin{VerbatimOut}{z.out}

\section{Decimal Prefixes}
\end{VerbatimOut}

\MyIO

\begin{VerbatimOut}{z.out}

\newcolumntype{m}{>{$}r<{$}}  % math mode version of "r" column type
\begin{tabular}{@{}mllll@{}}
  \multicolumn{1}{l}{\bfseries Power}&
    \bfseries Prefix&
    \bfseries Symbol&
    \bfseries Command&
    \bfseries Comment\\
  \tabularspace
  -24& yocto& \unit{\yocto\nounit}& \verb+\si{\yocto}+\\
  -21& zepto& \unit{\zepto\nounit}& \verb+\si{\zepto}+\\
  -18& atto&  \unit{\atto\nounit}&  \verb+\si{\atto}+\\
  -15& femto& \unit{\femto\nounit}& \verb+\si{\femto}+\\
  -12& pico&  \unit{\pico\nounit}&  \verb+\si{\pico}+\\
   -9& nano&  \unit{\nano\nounit}&  \verb+\si{\nano}+\\
   -6& micro& \unit{\micro\nounit}& \verb+\si{\micro}+\\
   -3& milli& \unit{\milli\nounit}& \verb+\si{\milla}+\\
   -2& centi& \unit{\centi\nounit}& \verb+\si{\centi}+\\
   -1& deci&  \unit{\deci\nounit}&  \verb+\si{\deci}+\\
    1& deca&  \unit{\deca\nounit}&  \verb+\si{\deca}+\\
    1& deka&  \unit{\deka\nounit}&  \verb+\si{\deka}+& same as \verb+\si{\deca}+\\
    2& hecto& \unit{\hecto\nounit}& \verb+\si{\hecto}+\\
    3& kilo&  \unit{\kilo\nounit}&  \verb+\si{\kilo}+\\
    6& mega&  \unit{\mega\nounit}&  \verb+\si{\mega}+\\
    9& giga&  \unit{\giga\nounit}&  \verb+\si{\giga}+\\
   12& tera&  \unit{\tera\nounit}&  \verb+\si{\tera}+\\
   15& peta&  \unit{\peta\nounit}&  \verb+\si{\peta}+\\
   18& exa&   \unit{\exa\nounit}&   \verb+\si{\exa}+\\
   21& zetta& \unit{\zetta\nounit}& \verb+\si{\zetta}+\\
   24& yotta& \unit{\yotta\nounit}& \verb+\si{\yotta}+\\
\end{tabular}
\end{VerbatimOut}

\MyIO


\begin{VerbatimOut}{z.out}

\section{SI Units}
\end{VerbatimOut}

\MyIO


\begin{VerbatimOut}{z.out}

The International System of Units
(SI)
% !!! Doing
% !!!     \include{tipa}
% !!! in thesis.tex so \textprimstress works
% !!! apparently causes problems with math commands.
% !!! Figure out why the following doesn't work later.
% (%
%   SI,
%   abbreviated from the French Syst\`eme International
%   (d\textprimstress unit\'es)%
% )
is the modern form of the metric system.
There are seven SI base units:

\hspace{40pt}
\begin{tabular}{@{}lll@{}}
  \tabularspace
  \bfseries Name& \bfseries Unit Of&         \bfseries Symbol\\
  \tabularspace
  ampere&         electrical current&        \si{\ampere}\\
  candela&        luminous intensity&        \si{\candela}\\
  kelvin&         thermodynamic temperature& \si{\kelvin}\\
  kg&             mass&                      \si{\kilogram}\\
  meter&          length&                    \si{\meter}\\
  mole&           amount of substance&       \si{\mole}\\
  second&         time&                      \si{\second}\\
\end{tabular}
\end{VerbatimOut}

\MyIO


\begin{VerbatimOut}{z.out}

\section{Complete List of Units}
\label{se:Complete-List-of-Units}
\end{VerbatimOut}

\MyIO

\begin{VerbatimOut}{z.out}

{%
  \ZZbaselinestretch{1}
  \newcommand\vsp{\noalign{\vspace*{6pt}}}
  % From
  % https://tex.stackexchange.com/questions/31508/flushleft-with-p-option-in-tabular
  %     It's necessary to use the \arraybackslash in the last column,
  %     otherwise \\ would not end the table row.  You can use \newline
  %     to end lines in the last column cells (and the regular \\ in
  %     the other column cells).
  %     ...
  %     If you need it often, consider defining a new column type using
  %     array features, as I did here:
  %         \newcolumntype{P}[1]{>{\raggedright\arraybackslash}p{#1}}
  \newcolumntype{P}[1]{>{\raggedright\arraybackslash}p{#1}}%
% \begin{longtable}{@{}P{1.4in}P{1in}llP{1.8in}@{}}
% \begin{longtable}{@{}P{1in}P{1in}llP{1.8in}@{}}
% \begin{longtable}{@{}P{1.2in}P{1in}llP{1.8in}@{}}
% \begin{longtable}{@{}P{90.72pt}P{1in}llP{1.8in}@{}}  % 1.2in (86.72pt) + 4pt = 90.72pt
  \begin{longtable}{@{}P{1.4in}P{1in}llP{1.8in}@{}}% 1.2in (86.72pt) + 4pt = 90.72pt
      \caption{Units and Corresponding Symbols}\\
      \bfseries Name&
        \bfseries Unit Of&
        \bfseries Symbol&
        \bfseries Command&
        \bfseries Is equal to\\
      \vsp
    \endfirsthead
      \caption[]{~\emph{continued}}\\
      \bfseries Name&
        \bfseries Unit Of&
        \bfseries Symbol&
        \bfseries Command&
        \bfseries Is equal to\\
      \vsp
    \endhead
      \vsp
      % I don't know why the \hspace*{-7.5mm} was
      % needed to center this horizontally.
      \multicolumn{5}{@{}c@{}}{\hspace*{-7.5mm}\emph{continued on next page}}%
    \endfoot    
    \endlastfoot
    ampere&
      electrical current&
      \si{\A}&
      \verb+\si{\A}+&
      (SI base unit)\\
    \quad picoampere&
      \ditto&
      \si{\pA}&
      \verb+\si{\pA}+&
      \SI{e-12}{\A}\\ 
    \quad nanoampere&
      \ditto&
      \si{\nA}&
      \verb+\si{\nA}+&
      \SI{e-9}{\A}\\
    \quad microampere&
      \ditto&
      \si{\uA}&
      \verb+\si{\uA}+&
      \SI{e-6}{\A}\\
    \quad milliampere&
      \ditto&
      \si{\mA}&
      \verb+\si{\mA}+&
      \SI{e-3}{\A}\\
    \quad kiloampere&
      \ditto&
      \si{\kA}&
      \verb+\si{\kA}+&
      \SI{e3}{\A}\\
    \vsp
    % \aa ngstr\"om&
    %   length&
    %   \si{\AA}&
    %   \verb+\si{\AA}+&
    %   \SI{e-10}{\m}\\
    \vsp
    arcminute&
      plane angle&
      \si{\arcmin}&
      \verb+\si{\arcmin}+&
      % Changed
      %     \SI{1/60}{\degree}\\
      % to
      1/60\unit{\degree\nounit}\\
    arcsecond&
      plane angle&
      \si{\arcsec}&
      \verb+\si{\arcsec}+&
      % Changed
      %     \SI{1/60}{\arcmin}\\
      % to
      1/60\unit{\arcmin\nounit}\\
    \vsp
    astronomical unit&
      length&
      \si{\au}&
      \verb+\si{\au}+&
      mean earth to\newline sun distance\\
    \vsp
    % From
    %     siunitx - A comprehensive (SI) units package
    %     Joseph Wright
    %     Released 2021-08-04
    %     (this describes v3.0.24, last revised 2021-08-04)
    %     https://mirror.las.iastate.edu/tex-archive/macros/latex/contrib/siunitx/siunitx.pdf
    % page 51:
    %     ...the unit \atomicmassunit has similar deprecated status:
    %     this was listed as with experimentally-determined units
    %     in the 8th Edition of the si Brochure but is equivalent
    %     to the dalton, a unit which remains accepted.
    % atomic mass unit&
    %   mass&
    %   \si{\amu}&
    %   \verb+\si{\amu}+&
    %   \(1/12\) mass of\newline carbon-12 atom\\
    % \vsp
    bar&
      pressure&
      \si{\bar}&
      \verb+\si{\bar}+&
      \SI{e-5}{\Pa}\\
    \quad millibar&
      \ditto&
      \si{\mbar}&
      \verb+\si{\mbar}+&
      \SI{e-3}{\bar}\\
    \vsp
    barn&
      area&
      \si{\b}&
      \verb+\si{\b}+&
      \SI{e-28}{\m\squared}\\
    \vsp
    becquerel&
      radioactivity&
      \si{\Bq}&
      \verb+\si{\Bq}+&
      one radioactive\newline decay per second\\
    \vsp
    bel&
      sound intensity&
      \si{\B}&
      \verb+\si{\B}+&
      10 decibels\\
    \quad decibel&
      \ditto&
      \si{\dB}&
      \verb+\si{\dB}+&
      \SI{e-1}{\B}\\
    \vsp
    bohr&
      length&
      \si{\bohr}&
      \verb+\si{\bohr}+&
      distance between\newline nucleus and electron\newline in hydrogen atom\\
    \vsp
    bushel&
      quantity&
      \si{\bu}&
      \verb+\si{\bu}+&
      see \cite{wikipedia-bushel}\\
    \vsp
    candela&
      luminous intensity&
      \si{\cd}&
      \verb+\si{\cd}+&
      (SI base unit)\\
    \vsp
    coulomb&
      electrical charge&
      \si{\C}&
      \verb+\si{\C}+&
      \si{\A\per\s}\\
    \vsp
    dalton&
      mass&
      \si{\Da}&
      \verb+\si{\Da}+&
      another name for\newline atomic mass unit\\
    \vsp
    day&
      time&
      \si{\d}&
      \verb+\si{\d}+&
      \SI{86400}{\s}\\
    \vsp
    degree&
      plane angle&
      \si{\degree}&
      \verb+\si{\degree}+&
      1/360 of a cicle\\
    \vsp
    degree Celsius&
      temperature&
      \si{\celsius}&
      \verb+\si{\celsius}+&
      xxx\\
    \vsp
    electron mass&
      mass&
      \si{\em}&
      \verb+\si{\em}+&
      xxx\\
    \vsp
    electronvolt&
      energy&
      \si{\eV}&
      \verb+\si{\eV}+&
      xxx\\
    \quad millielectronvolt&
      \ditto&
      \si{\meV}&
      \verb+\si{meV}+&
      \SI{e-3}{\eV}\\
    \quad kiloelectronvolt&
      \ditto&
      \si{\keV}&
      \verb+\si{keV}+&
      \SI{e3}{\eV}\\
    \quad megaelectronvolt&
      \ditto&
      \si{\MeV}&
      \verb+\si{MeV}+&
      \SI{e6}{\eV}\\
    \quad gigaelectronvolt&
      \ditto&
      \si{\GeV}&
      \verb+\si{\GeV}+&
      \SI{e9}{\eV}\\
    \quad teraelectronvolt&
      \ditto&
      \si{\TeV}&
      \verb+\si{\TeV}+&
      \SI{e12}{\eV}\\
    \vsp
    elementary charge&
      electrical charge&
      \si{\ec}&
      \verb+\si{\ec}+&
      \href{https://en.wikipedia.org/wiki/Elementary_charge}{\SI{\approx 1.6e19}{\C}}\\
    \vsp
    farad&
      electrical capacitance&
      \si{\F}&
      \verb+\si{\F}+&
      \si{\s\tothe{4}\A\squared\per\m\squared\per\kg}\\
    \quad femtofarad&
      \ditto&
      \si{\fF}&
      \verb+\si{\fF}+&
      \SI{e-15}{\F}\\
    \quad picofarad&
      \ditto&
      \si{\pF}&
      \verb+\si{\pF}+&
      \SI{e-12}{\F}\\
    \vsp
    foot&
      length&
      \si{\ft}&
      \verb+\si{\ft}+&
      \SI{0.3048}{\m}\\  % not an SI unit
    \vsp
    % gauss: The gauss, symbol G, sometimes Gs, is the cgs unit of measurement of magnetic flux.
    gray&
      absorbed dose of ionizing radiation&
      \si{\Gy}&
      \verb+\si{\Gy}+&
      \si{\J\per\kg}\\
    \vsp
    hartree&
      energy used in molecular orbital calculations&
      \si{\hartree}&
      \verb+\si{\hartree}+&
      xxx\\
    \vsp
    hectare&
      area&
      \si{\ha}&
      \verb+\si{\ha}+&
      \SI{e4}{\m\squared}\\
    \vsp
    henry&
      electrical inductance&
      \si{\H}&
      \verb+\si{\H}+&
      \si{\kg\m\squared\per\s\squared\per\A\squared}\\
    \vsp
    hertz&
      frequency&
      \si{\Hz}&
      \verb+\si{\Hz}+&
      \si{\per\s}\\
    \quad millihertz&
      \ditto&
      \si{\mHz}&
      \verb+\si{\mHz}+&
      \SI{e-3}{\Hz}\\
    \quad kilohertz&
      \ditto&
      \si{\kHz}&
      \verb+\si{\kHz}+&
      \SI{e3}{\Hz}\\
    \quad megahertz&
      \ditto&
      \si{\MHz}&
      \verb+\si{\MHz}+&
      \SI{e6}{\Hz}\\
    \quad gigahertz&
      \ditto&
      \si{\GHz}&
      \verb+\si{\GHz}+&
      \SI{e9}{\Hz}\\
    \quad terahertz&
      \ditto&
      \si{\THz}&
      \verb+\si{\THz}+&
      \SI{e12}{\Hz}\\
    \vsp
    horsepower&
      power&
      \si{\hp}&
      \verb+\si{\hp}+&
      \SI{\approx 745.7}{\W}, {\bfseries IMPORTANT:\newline
        see \href{https://en.wikipedia.org/wiki/Horsepower#Mechanical_horsepower}{Horsepower}}\\
        % not an SI unit
    \vsp
    hour&
      time&
      \si{\h}&
      \verb+\si{\h}+&
      \SI{3600}{\s}\\
    \vsp
    inch&
      length&
      \si{\in}&
      \verb+\si{\in}+&
      \SI{25.4}{\mm}\\  % not an SI unit
    \vsp
    joule&
      work or energy&
      \si{\J}&
      \verb+\si{\J}+&
      \si{\kg\m\squared\per\s\squared}\\
    \quad microjoule&
      \ditto&
      \si{\uJ}&
      \verb+\si{\uJ}+&
      \SI{e-6}{\J}\\
    \quad millijoule&
      \ditto&
      \si{\mJ}&
      \verb+\si{\mJ}+&
      \SI{e-3}{\J}\\
    \quad kilojoule&
      \ditto&
      \si{\kJ}&
      \verb+\si{\kJ}+&
      \SI{e3}{\J}\\
    \quad megajoule&
      \ditto&
      \si{\MJ}&
      \verb+\si{\MJ}+&
      \SI{e6}{\J}\\
    \vsp
    katal&
      catalytic activity&
      \si{\kat}&
      \verb+\si{\kat}+&
      \si{\mol\per\s}\\
    \vsp
    kelvin&
      thermodynamic temperature&
      \si{\K}&
      \verb+\si{\K}+&
      (SI base unit)\\
    \vsp
    kilogram&
      mass&
      \si{\kg}&
      \verb+\si{\kg}+&
      (SI base unit)\\
    \quad femtogram&
      \ditto&
      \si{\fg}&
      \verb+\si{\fg}+&
      \SI{e-15}{\g}\\
    \quad picogram&
      \ditto&
      \si{\pg}&
      \verb+\si{\pg}+&
      \SI{e-12}{\g}\\
    \quad nanogram&
      \ditto&
      \si{\ng}&
      \verb+\si{\ng}+&
      \SI{e-9}{\g}\\
    \quad microgram&
      \ditto&
      \si{\ug}&
      \verb+\si{\ug}+&
      \SI{e-6}{\g}\\
    \quad milligram&
      \ditto&
      \si{\mg}&
      \verb+\si{\mg}+&
      \SI{e-3}{\g}\\
    \quad gram&
      \ditto&
      \si{\g}&
      \verb+\si{\g}+&
      \SI{e-3}{\kg}\\
    \vsp
    kilowatt hour&
      electrical energy&
      \si{\kWh}&
      \verb+\si{\kWh}+&
      \si{\kW\h}\\
    \vsp
    knot&
      speed&
      \si{\kn}&
      \verb+\si{\kn}+&
      \si{\M\per\h}\\
    \vsp
    liter&
      volume&
      \si{\L}&
      \verb+\si{\L}+&
      \SI{e-3}{m\cubed}\\
    \quad microliter&
      \ditto&
      \si{\uL}&
      \verb+\si{\uL}+&
      \SI{e-6}{\L}\\
    \quad milliliter&
      \ditto&
      \si{\mL}&
      \verb+\si{\mL}+&
      \SI{e-3}{\L}\\
    \quad hectoliter&
      \ditto&
      \si{\hL}&
      \verb+\si{\hL}+&
      \SI{e2}{\L}\\
    \vsp
    lumen&
      luminous flux&
      \si{\lm}&
      \verb+\si{\lm}+&
      \si{\cd\sr}\\
    \vsp
    lux&
      illumination&
      \si{\lx}&
      \verb+\si{\lx}+&
      \si{\lm\per\m\squared}\\
    \vsp
    meter&
      length&
      \si{\m}&
      \verb+\si{\m}+&
      (SI base unit)\\
    \quad picometer&
      \ditto&
      \si{\pm}&
      \verb+\si{\pm}+&
      \SI{e-12}{\m}\\
    \quad nanometer&
      \ditto&
      \si{\nm}&
      \verb+\si{\nm}+&
      \SI{e-9}{\m}\\
    \quad micrometer&
      \ditto&
      \si{\um}&
      \verb+\si{\um}+&
      \SI{e-6}{\m}\\
    \quad millimeter&
      \ditto&
      \si{\mm}&
      \verb+\si{\mm}+&
      \SI{e-3}{\m}\\
    \quad centimeter&
      \ditto&
      \si{\cm}&
      \verb+\si{\cm}+&
      \SI{e-2}{\m}\\
    \quad decimeter&
      \ditto&
      \si{\dm}&
      \verb+\si{\dm}+&
      \SI{e-1}{\m}\\
    \quad kilometer&
      \ditto&
      \si{\km}&
      \verb+\si{\km}+&
      \SI{e3}{\m}\\
    \vsp
    % mile: not an SI unit
    millimeter of mercury&
      pressure&
      \si{\mmHg}&
      \verb+\si{\mmHg}+&
      \href{https://en.wikipedia.org/wiki/Millimetre_of_mercury}{\SI{\approx 133}{\Pa}}\\
    \vsp
    minute&
      time&
      \si{\min}&
      \verb+\si{\min}+&
      \SI{60}{\s}\\
    \vsp
    mole&
      amount of substance&
      \si{\mol}&
      \verb+\si{\mol}+&
      (SI base unit)\\
    \quad femtomole&
      \ditto&
      \si{\fmol}&
      \verb+\si{\fmol}+&
      \SI{e-15}{\mol}\\
    \quad picomole&
      \ditto&
      \si{\pmol}&
      \verb+\si{\pmol}+&
      \SI{e-12}{\mol}\\
    \quad nanomole&
      \ditto&
      \si{\nmol}&
      \verb+\si{\nmol}+&
      \SI{e-9}{\mol}\\
    \quad micromole&
      \ditto&
      \si{\umol}&
      \verb+\si{\umol}+&
      \SI{e-6}{\mol}\\
    \quad millimole&
      \ditto&
      \si{\mmol}&
      \verb+\si{\mmol}+&
      \SI{e-3}{\mol}\\
    \quad kilomole&
      \ditto&
      \si{\kmol}&
      \verb+\si{\kmol}+&
      \SI{e3}{\mol}\\
    \vsp
    nautical mile&
      distance&
      \si{\M}&
      \verb+\si{\M}+&
      \SI{1852}{\m}\\
    \vsp
    neper&
      gain, loss, and relative values&
      \si{\Np}&
      \verb+\si{\Np}+&
      1\\
    \vsp
    newton&
      force&
      \si{\N}&
      \verb+\si{\N}+&
      \si{\kg\m\per\s\squared}\\
    \quad millinewton&
      \ditto&
      \si{\mN}&
      \verb+\si{\mN}+&
      \SI{e-3}{\N}\\
    \quad kilonewton&
      \ditto&
      \si{\kN}&
      \verb+\si{\kN}+&
      \SI{e3}{\N}\\
    \quad meganewton&
      \ditto&
      \si{\MN}&
      \verb+\si{\MN}+&
      \SI{e6}{\N}\\
    \vsp
    ohm&
      electrical resistance&
      \si{\ohm}&
      \verb+\si{\ohm}+&
      \si{\kg\m\squared\per\s\cubed\per\A\squared}\\
    \quad milliohm&
      \ditto&
      \si{\mohm}&
      \verb+\si{\mohm}+&
      \SI{e-3}{ohm}\\
    \quad kiloohm&
      \ditto&
      \si{\kohm}&
      \verb+\si{\kohm}+&
      \SI{e3}{ohm}\\
    \quad megaohm&
      \ditto&
      \si{\Mohm}&
      \verb+\si{\Mohm}+&
      \SI{e6}{ohm}\\
    \vsp
    pascal&
      pressure&
      \si{\Pa}&
      \verb+\si{\Pa}+&
      \si{\kg\per\m\per\s\squared}\\
    \qquad kilopascal&
      \ditto&
      \si{\kPa}&
      \verb+\si{\kPa}+&
      \SI{e3}{\Pa}\\
    \qquad megapascal&
      \ditto&
      \si{\MPa}&
      \verb+\si{\MPa}+&
      \SI{e6}{\Pa}\\
    \qquad gigapascal&
      \ditto&
      \si{\GPa}&
      \verb+\si{\GPa}+&
      \SI{e9}{\Pa}\\
    \vsp
    percent&
      hundredths&
      \si{\percent}&
      \verb+\si{\percent}+&
      \SI{e-2}{}\\
    \vsp
    pound&
      weight&
      \si{\lb}&
      \verb+\si{\lb}+&
      \SI{.45359237}{\kg}\\  % not an SI unit
    \vsp
    radian&
      plane angular measurement&
      \si{\rad}&
      \verb+\si{\rad}+&
      \(180/\pi\) \unit{\degree\nounit}\\
    \vsp
    reduced Planck constant&
      angular momentum&
      \si{\planckbar}&
      \verb+\si{\planckbar}+&
      \(\approx \SI{1.05e-34}{\J\s}\)\\
    \vsp
    second&
      time&
      \si{\s}&
      \verb+\si{\s}+&
      (SI base unit)\\
    \quad attosecond&
      \ditto&
      \si{\as}&
      \verb+\si{\as}+&
      \SI{e-18}{\s}\\
    \quad femtosecond&
      \ditto&
      \si{\fs}&
      \verb+\si{\fs}+&
      \SI{e-15}{\s}\\
    \quad picosecond&
      \ditto&
      \si{\ps}&
      \verb+\si{\ps}+&
      \SI{e-12}{\s}\\
    \quad nanosecond&
      \ditto&
      \si{\ns}&
      \verb+\si{\ns}+&
      \SI{e-9}{\s}\\
    \quad microsecond&
      \ditto&
      \si{\us}&
      \verb+\si{\us}+&
      \SI{e-6}{\s}\\
    \quad millisecond&
      \ditto&
      \si{\ms}&
      \verb+\si{\ms}+&
      \SI{e-3}{\s}\\
    \vsp
    siemens&
      conductance&
      \si{\S}&
      \verb+\si{\S}+&
      \si{\per\kg\per\m\squared\s\cubed\A\squared}\\
    \vsp
    sievert&
      dosage of ionizing radiation&
      \si{\Sv}&
      \verb+\si{\Sv}+&
      \si{\m\squared\per\s\squared}\\
    \vsp
    speed of light&
      speed&
      \si{\clight}&
      \verb+\si{\clight}+&
      \SI{299792458}{\m\per\s}\\
    \vsp
    standard deviation&
      amount of variation&
      \si{\SD}&
      \verb+\si{\SD}+&
      $\displaystyle \sqrt{\frac 1{N-1} \sum_{i=1}^N(x_i-\bar x)^2}$\\
    \vsp
    steradian&
      measure of solid angles&
      \si{\sr}&
      \verb+\si{\sr}+&
      \SI{1}{\m\squared\per\m\squared}\\
    \vsp
    tesla&
      magnetic flux density&
      \si{\T}&
      \verb+\si{\T}+&
      \si{\kg\per\s\squared\per\A}\\
    \vsp
    metric ton&
      mass&
      \si{\t}&
      \verb+\si{\t}+&
      \SI{e3}{\kg}\\
    \vsp
    volt&
      electrical potential difference&
      \si{\V}&
      \verb+\si{\V}+&
      \si{\kg\m\squared\per\s\cubed\per\A}\\
    \quad picovolt&
      \ditto&
      \si{\pV}&
      \verb+\si{\pV}+&
      \SI{e-12}{\V}\\
    \quad nanovolt&
      \ditto&
      \si{\nV}&
      \verb+\si{\nV}+&
      \SI{e-9}{\V}\\
    \quad microvolt&
      \ditto&
      \si{\uV}&
      \verb+\si{\uV}+&
      \SI{e-6}{\V}\\
    \quad millivolt&
      \ditto&
      \si{\mV}&
      \verb+\si{\mV}+&
      \SI{e-3}{\V}\\
    \quad kilovolt&
      \ditto&
      \si{\kV}&
      \verb+\si{\kV}+&
      \SI{e3}{\V}\\
    \vsp
    watt&
      power&
      \si{\W}&
      \verb+\si{\W}+&
      \si{\kg\m\squared\per\s\cubed}\\
    \quad microwatt&
      \ditto&
      \si{\uW}&
      \verb+\si{\uW}+&
      \SI{e-6}{\W}\\
    \quad milliwatt&
      \ditto&
      \si{\mW}&
      \verb+\si{\mW}+&
      \SI{e-3}{\W}\\
    \quad kilowatt&
      \ditto&
      \si{\kW}&
      \verb+\si{\kW}+&
      \SI{e3}{\W}\\
    \quad megawatt&
      \ditto&
      \si{\MW}&
      \verb+\si{\MW}+&
      \SI{e6}{\W}\\
    \quad gigawatt&
      \ditto&
      \si{\GW}&
      \verb+\si{\GW}+&
      \SI{e9}{\W}\\
    \vsp
    weber&
      magnetic flux&
      \si{\Wb}&
      \verb+\si{\Wb}+&
      \si{\kg\m\squared\per\s\squared\per\A}\\
    \vsp
    yard&
      length&
      \si{\yd}&
      \verb+\si{\yd}+&
      \SI{.9144}{\m}\\  % not an SI unit
    \vsp
    year&
      time&
      \si{\y}&
      \verb+\si{\y}+&
      \SI{\approx 365.25}{\d}\\  % not an SI unit
  \end{longtable}
}
\end{VerbatimOut}

\MyIO
\endinput

%   Non-SI units accepted for use with the International System of Units.
%   
%   % From
%   %     https://www.bipm.org/en/publications/si-brochure/section2-2-1.html
%   \begin{tabular}{@{}llll@{}}
%     \bfseries Symbol& \bfseries Command& \bfseries Name& \bfseries Unit of\\
%     m^{-1}& reciprocal meter& wavenumber\\
%     m^2& square meter& area\\
%     m^3& cubic meter& volume\\
%     m/s& meter per second& & speed, velocity\\
%     m/s^2& meter per second squared& acceleration\\
%   \end{tabular}
%   
%   In  addition  to  the  units  themselves,
%   siunitx
%   provides  pre-defined  macros  for  all
%   
%   
%   % xxx needs lots more work above and maybe below
%   
%   \section{angles}
%   
%   \ang{1}
%   \ang{1;2}
%   \ang{1;2;3}
%   \ang{;2}
%   \ang{;;3}
%   -\ang{;2}
%   
%   
%   \ang{10}    \\
%   \ang{12.3}  \\
%   \ang{4,5}   \\
%   \ang{1;2;3} \\
%   \ang{;;1}   \\
%   \ang{+10;;} \
%   
%       degrees
%       degrees,minutes
%       degrees,minutes,seconds
%   
%       \degree
%       \arcminute
%       \arcsecond
%   
%       \SI{3.1415}{\degree}
%   
%       \ang{-0;1;}
%   
%       list
%           \SIlist{10;20}{\meter}
%           \SIlist{10;20;30}{\meter}
%       
%   temperatures
%   
%       \degreeCelsius
%       \celsius
%       
%       range
%           \SIrange{1}{5}{\metre}
%           \SIrange{1}{5}{\milli\metre}
%       
%   
%   numbers
%   
%   123                \num{123}     \\
%   1234               \num{1234}    \\
%   12 345             \num{12345}   \\
%   0.123              \num{0.123}   \\
%   0.1234             \num{0.1234}  \\
%   0.123 45           \num{.12345}  \\
%   3.45 x 10-4        \num{3.45d-4} \\
%   -10^{10}           \num{-e10}
%                      \num{12345.67890}
%                      \num{1+-2i}
%                      \num{.3.45}
%   
%       number list
%           \numlist{10;20}
%           \numlist{10;20;30}
%       
%       number range
%           \numrange{10}{20}
%       \celsius
