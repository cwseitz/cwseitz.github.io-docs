\ProvidesFile{ch-test.tex}[2022-10-05 test chapter]

\begin{refsection}

\chapter{TEST ENVIRONMENTS AND PER-CHAPTER REFERENCES}

\verb+\cite[page v]{knuth2012}+ gives ``\cite[page v]{knuth2012}''.

\noindent
\verb+\cite[back cover]{lamport1994}+ gives ``\cite[back cover]{lamport1994}''.

\noindent
\verb+\cite{thesis2017}+ gives ``\cite{thesis2017}''.

\noindent
\verb+\cite{thesis2020}+ gives ``\cite{thesis2020}''.

This is an example of normal text.
This is an example of normal text.
This is an example of normal text.
This is an example of normal text.
This is an example of normal text.
This is an example of normal text.
This is an example of normal text.
This is an example of normal text.
This is an example of normal text.
This is an example of normal text.

\begin{definition}
  \ZZbaselinestretch{1.5}
  This is an example definition.
  This is an example definition.
  This is an example definition.
  This is an example definition.
  This is an example definition.
\end{definition}

This is an example of normal text.
This is an example of normal text.
This is an example of normal text.
This is an example of normal text.
This is an example of normal text.
This is an example of normal text.
This is an example of normal text.
This is an example of normal text.
This is an example of normal text.
This is an example of normal text.

\begin{observation}
  \ZZbaselinestretch{1.5}
  This is an example observation.
  This is an example observation.
  This is an example observation.
  This is an example observation.
  This is an example observation.
\end{observation}

This is an example of normal text.
This is an example of normal text.
This is an example of normal text.
This is an example of normal text.
This is an example of normal text.
This is an example of normal text.
This is an example of normal text.
This is an example of normal text.
This is an example of normal text.
This is an example of normal text.

\begin{proof}
  \ZZbaselinestretch{1.5}
  This is an example proof.
  This is an example proof.
  This is an example proof.
  This is an example proof.
  If \(a = b\) and \(b = c\) then \(a = c\).
\end{proof}

This is an example of normal text.
This is an example of normal text.
This is an example of normal text.
This is an example of normal text.
This is an example of normal text.
This is an example of normal text.
This is an example of normal text.
This is an example of normal text.
This is an example of normal text.
This is an example of normal text.

\begin{proposition}
  \ZZbaselinestretch{1.5}
  This is an example proposition.
  This is an example proposition.
  This is an example proposition.
  This is an example proposition.
  This is an example proposition.
\end{proposition}

This is an example of normal text.
This is an example of normal text.
This is an example of normal text.
This is an example of normal text.
This is an example of normal text.
This is an example of normal text.
This is an example of normal text.
This is an example of normal text.
This is an example of normal text.
This is an example of normal text.

\begin{theorem}
  \ZZbaselinestretch{1.5}
  This is an example theorem.
  This is an example theorem.
  This is an example theorem.
  This is an example theorem.
  This is an example theorem.
\end{theorem}

This is an example of normal text.
This is an example of normal text.
This is an example of normal text.
This is an example of normal text.
This is an example of normal text.
This is an example of normal text.
This is an example of normal text.
This is an example of normal text.
This is an example of normal text.
This is an example of normal text.

\begin{singlespace}
\def\sllnsez{[1] }
\PrintChapterBibliography
\end{singlespace}

\end{refsection}
