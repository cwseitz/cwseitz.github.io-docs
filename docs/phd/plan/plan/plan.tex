\documentclass{article}
\usepackage{amsmath, fullpage}
\usepackage{graphicx}

\begin{document}

\title{Progress Report}
\author{Clayton Seitz}
\maketitle
\thispagestyle{empty}


\section{Abstract}

Eukaryotic transcription is episodic, consisting of a series of transcriptional bursts, resulting in non-Poissonian gene expression. Bursty transcriptional dynamics are well-exemplified by the transient and amplified expression of pro-inflammatory GBP gene products, a group interferon-inducible GTPases that restrict the replication of intracellular pathogens [XXX]. Classical models of gene regulation explain this bursty behavior by invoking stochastic binding and unbinding of transcription factors, RNA polymerase and mediator proteins. However, recent evidence has emerged that transcription is often controlled by phase-separated aggregates of DNA, RNA, and proteins termed transcriptional condensates. Interestingly, super-enhancer-binding proteins MED1 and BRD4 undergo phase separation in close proximity to all the GBP family gene loci during the immune response [XXX]. The formation of these transcriptional condensates driven by MED1 and BRD4 alongside decreased entropy of the chromatin scaffold have been previously demonstrated with super-resolution microscopy. However, whether coordination of condensate formation and chromatin reorganization act to produce bursty transcriptional dynamics remains unclear. Therefore, here we utilize super-resolution fluorescence microscopy to demonstrate simultaneous (i) loss of disorder in chromatin structure (ii) formation of transcriptional condensates driven by MED1 and BRD4 and (iii) non-Poissonian gene expression. We also explore the properties of transcriptional condensates and their interaction with a Rouse-like chromatin scaffold with Monte Carlo simulations and analytical treatment.

\section{Single molecule localization microscopy}

Most detectors used for imaging have many elements (pixels) so that we can record an image projected onto the detector by a system of lenses. In fluorescence imaging, this is usually a relay consisting of an objective lens and a tube lens to focus the image onto the camera. Due to diffraction, any point emitter, such as a single fluorescent molecule, will be registered as a diffraction limited spot. The profile of that spot is often described as a Gaussian point spread function (Richardson and Wolf)

\begin{figure}
\centering
\includegraphics[width=12cm]{dSTORM.png}
\caption{Stochastic optical reconstruction microscope}
\end{figure}

\section{Langevin dynamics of a Rouse-like polymer embedded in a transcriptional condensate}

The formation of these transcriptional condensates driven by MED1 and BRD4 alongside decreased entropy of the chromatin scaffold have been previously demonstrated with super-resolution microscopy. Yet, the biophysical descriptions of the interaction between transcriptional condensates with the chromatin fiber and the mechanisms by which they evoke bursty gene expression remain mere speculation. 

Here, we intend to address these outstanding questions by modeling transcriptional condensate formation within a Rouse-like polymer model parameterized by constraints measured in single particle tracking experiments. These models serve to address the stability of transcriptional condensates as a function of their stoichiometry and the relationship between condensate formation and transcriptional bursting by integrating their associated Langevin dynamics with Monte Carlo simulations. 


The general properties of ranscriptional condensates can be investigated using the stochastic simulation algorithm [Gillespie].

 
\bibliographystyle{unsrt}
\bibliography{Dissertation-Transcription.bib}

 
\end{document}