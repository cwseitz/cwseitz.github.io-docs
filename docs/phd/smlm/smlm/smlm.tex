% Latex template: mahmoud.s.fahmy@students.kasralainy.edu.eg
% For more details: https://www.sharelatex.com/learn/Beamer

%\documentclass{beamer}					% Document class
% Aspect ratio
\documentclass[aspectratio=169]{beamer}

\setbeamertemplate{footline}[text line]{%
  \parbox{\linewidth}{\vspace*{-8pt}Single molecule localization microscopy \hfill\insertshortauthor\hfill\insertpagenumber}}
\setbeamertemplate{navigation symbols}{}

\usepackage[english]{babel}				% Set language
\usepackage[utf8x]{inputenc}			% Set encoding

\mode<presentation>						% Set options
{
  \usetheme{default}					% Set theme
  \usecolortheme{default} 				% Set colors
  \usefonttheme{default}  				% Set font theme
  \setbeamertemplate{caption}[numbered]	% Set caption to be numbered
}

% Uncomment this to have the outline at the beginning of each section highlighted.
%\AtBeginSection[]
%{
%  \begin{frame}{Outline}
%    \tableofcontents[currentsection]
%  \end{frame}
\usepackage{graphicx}					% For including figures
\usepackage{booktabs}					% For table rules
\usepackage{hyperref}	
\usepackage{tikz-network}				% For cross-referencing
\usepackage[absolute,overlay]{textpos}
\usepackage{bm}
\usepackage[font=small,labelfont=bf]{caption}				% For cross-referencing

\title{Single molecule localization microscopy}	% Presentation title
\author{Clayton W. Seitz}								% Presentation author
\date{\today}									% Today's date	

\begin{document}

% Title page
% This page includes the informations defined earlier including title, author/s, affiliation/s and the date
\begin{frame}
  \titlepage
\end{frame}

\begin{frame}{Direction stochastic optical reconstruction microscopy}
\end{frame}

\begin{frame}{Photoswitchable organic dyes}
\end{frame}

\begin{frame}{A model of photoswitching between ON and OFF states}
\end{frame}

\begin{frame}{Quantifying and optimizing single-molecule switching}
\end{frame}

\begin{frame}{PSF model and pixel-wise gain measurements}
\end{frame}

\begin{frame}{Pixel-wise readout noise measurements}
\end{frame}

\begin{frame}{The Fisher information metric and Cramer-Rao bound}
\end{frame}

\begin{frame}{Monte Carlo estimation of the Cramer-Rao Bound}
Start with $\theta_{\mathrm{MLE}}$, use MCMC to explore the parameter space. How to ensure $\theta_{\mathrm{MLE}}$ is a good starting point? Or should I start at the true parameters?
\end{frame}

\begin{frame}{Experimental determinants of localization accuracy}
\end{frame}


\end{document}