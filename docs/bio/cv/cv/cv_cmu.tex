\documentclass[margin, 10pt]{res} % Use the res.cls style, the font size can be changed to 11pt or 12pt here

\usepackage{helvet} % Default font is the helvetica postscript font
%\usepackage{newcent} % To change the default font to the new century schoolbook postscript font uncomment this line and comment the one above

\setlength{\textwidth}{5.1in} % Text width of the document

\begin{document}

%----------------------------------------------------------------------------------------
%	NAME AND ADDRESS SECTION
%----------------------------------------------------------------------------------------

\moveleft.5\hoffset\centerline{\large\bf Clayton W. Seitz, Ph.D.} % Your name at the top
 
\moveleft\hoffset\vbox{\hrule width\resumewidth height 1pt}\smallskip % Horizontal line after name; adjust line thickness by changing the '1pt'
 
\moveleft.5\hoffset\centerline{cwseitz@iu.edu} % Your address
\moveleft.5\hoffset\centerline{cwseitz.github.io} % Your address

%----------------------------------------------------------------------------------------

\begin{resume}


%\section{PERSONAL STATEMENT}  

%I research and model changes in chromatin structure after exposure to ionizing radiation and other chemical/biological perturbations

%I maintain a diverse background and specialize in single molecule localization microscopy, statistical/mathematical modeling, and developing machine learning models for computer vision in microscopy. Recently, I have developed deep generative models and analytical probabilistic methods to enhance bioimaging and test hypotheses in genomics/epigenetics.

%I am an optical microscopist and machine learning expert with experience with several imaging modalities including widefield, confocal, single molecule localization microscopy, and selective plane illumination. I also have extensive experience with various image processing techniques for quantification, segmentation, and object detection/tracking.

I am a microscopist with experience with several imaging modalities including widefield, confocal, single molecule localization microscopy, and selective plane illumination. I also have experience in cell biology research, particularly in the field of epigenetics.

%----------------------------------------------------------------------------------------
%	EDUCATION SECTION
%----------------------------------------------------------------------------------------

\section{EDUCATION}

\textbf{Doctor of Philosopy, Physics}\\
Purdue University\\
Thesis: \textit{Advancing super-resolution microscopy for quantitative in-vivo imaging of chromatin nanodomains}

\textbf{Master of Science, Biophysics}\\
University of Chicago\\
%Thesis: \textit{Stable cell assembly formation in excitatory-inhibitory neural networks}


\textbf{Bachelor of Science, Physics}, Magna Cum Laude\\
Indiana University\\
Minor: Mathematics 

\textbf{Bachelor of Science, Informatics}, Magna Cum Laude\\
Luddy School of Informatics, Computing, and Engineering, Indiana University\\
Concentration: Mathematics 
  
%----------------------------------------------------------------------------------------
%	PROFESSIONAL EXPERIENCE SECTION
%----------------------------------------------------------------------------------------
 
\section{EXPERIENCE}

\textbf{Graduate Researcher} \hfill 2021-Present \\
Indiana University, Indianapolis, IN

\begin{itemize} \itemsep -2pt % Reduce space between items


\item Developed novel microscopy systems for super-resolution imaging of living cells 

\item Developed image processing software and probabilistic models for high-dimensional imaging datasets and Bayesian methods for statistical inference tasks

\item Designed diffusion models/score-based generative models and general computer vision techniques (object detection, segmentation, etc.) in PyTorch for modeling image datasets in super-resolution fluorescence microscopy

\item Investigated the impact of point mutations of epigenetic proteins on the structure of nucleosome nanodomains and complement experimental data with molecular dynamics simulations


\end{itemize}

\textbf{Graduate Researcher} \hfill 2020-2021 \\
University of Chicago, Chicago, IL

\begin{itemize} \itemsep -2pt % Reduce space between items

\item Utilized fluorescence microscopy to measure temporal dynamics of calcium concentration in MIN6 cells

\item Performed Monte Carlo simulations of cellular networks to relate network architecture to calcium dynamics
 
\end{itemize}
 
\textbf{Research Assistant} \hfill 2019-2020\\
Indiana University, Indianapolis, IN
\begin{itemize} \itemsep -2pt

\item Developed a scientific package in Python for high-throughput object detection and tracking
\item Managed the package lifecycle and user training throughout the laboratory
\end{itemize}


\section{AWARDS}

{\sl NIH Graduate Training Fellowship} \hfill 2020 \\
University of Chicago, Chicago, IL

{\sl Travel Award and Lightning Talk Invitation} \hfill 2019 \\
Physical Sciences in Oncology - Minneapolis, MN

{\sl Hudson and Holland Scholarship for Diversity and Inclusion} \hfill 2013-2017 \\
Indiana University, Bloomington, IN 

{\sl Founders Scholar} \hfill 2013-2017 \\
Indiana University, Bloomington, IN 

{\sl Cigital Scholarship} \hfill 2016-2017 \\
Indiana University, Bloomington, IN 

\section{PUBLICATIONS}

\textbf{Clayton Seitz}\textsuperscript{\textdagger}, Donghong Fu\textsuperscript{\textdagger}, Mengyuan Liu, Hailan Ma, and Jing Liu. \textit{BRD4 phosphorylation regulates the structure of chromatin nanodomains}. Physical Review Letters (In Review). 2024

\textbf{Clayton Seitz} and Jing Liu. \textit{Counting fluorescent emitters with a single photon avalanche diode array}. Communications Physics (In Review). 2024

\textbf{Clayton Seitz} and Jing Liu. \textit{Uncertainty-aware localization microscopy by variational diffusion}. In Progress. 2024

Maelle Locatelli\textsuperscript{\textdagger}, Josh Lawrimore\textsuperscript{\textdagger}, Hua Lin\textsuperscript{\textdagger}, Sarvath Sanaullah, \textbf{Clayton Seitz}, Dave Segall, Paul Kefer, Salvador Moreno Naike, Benton Lietz, Rebecca Anderson, Julia Holmes, Chongli Yuan, George Holzwarth, Bloom Kerry, Jing Liu, Keith D Bonin, Pierre-Alexandre Vidi. \textit{DNA damage reduces heterogeneity and coherence of chromatin motions}. PNAS 12 July 2022; 119 (29): 1-11
\\
\\
Mengdi Zhang, \textbf{Clayton Seitz}, Garrick Chang, Fadil Iqbal, Hua Lin, and Jing Liu \textit{A guide for single-particle chromatin tracking in live cell nuclei}. Cell Biology International 15 January 2022; 46 (5): 683-700
\\
\\
Wenting Wu, Farooq Syed, Edward Simpson, Chih-Chun Lee, Jing Liu, Garrick Chang, Chuanpeng Dong, \textbf{Clayton Seitz}, Decio L. Eizirik, Raghavendra G. Mirmira, Yunlong Liu, Carmella Evans-Molina; \textit{Impact of Proinflammatory Cytokines on Alternative Splicing Patterns in Human Islets}. Diabetes 25 October 2021; 71 (1): 116–127

\textbf{Clayton Seitz}, Hailan Ma, and Jing Liu. \textit{Cytokine-induced transcriptional memory is evident in the kinetics of transcriptional bursts}. Biophysical Society Annual Conference 2022


\textbf{Clayton Seitz}, Hua Lin, Keith Bonin, Pierre-Alexandre Vidi, and Jing Liu. \textit{Quantifying the spatiotemporal dynamics of dUTP labeled chromatin during the DNA damage response}. Biophysical Society Annual Conference 2020


\section{TECHNICAL \\ SKILLS} 

ImageJ, CellProfiler, Linux, Bash, Python, R, PyTorch, C/C++, SQL, LaTeX, Git, Docker, SLURM\\

\end{resume}

\end{document}