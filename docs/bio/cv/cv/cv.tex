\documentclass[margin, 10pt]{res} % Use the res.cls style, the font size can be changed to 11pt or 12pt here

\usepackage{helvet} % Default font is the helvetica postscript font
%\usepackage{newcent} % To change the default font to the new century schoolbook postscript font uncomment this line and comment the one above

\setlength{\textwidth}{5.1in} % Text width of the document

\begin{document}

%----------------------------------------------------------------------------------------
%	NAME AND ADDRESS SECTION
%----------------------------------------------------------------------------------------

\moveleft.5\hoffset\centerline{\large\bf Clayton Seitz} % Your name at the top
 
\moveleft\hoffset\vbox{\hrule width\resumewidth height 1pt}\smallskip % Horizontal line after name; adjust line thickness by changing the '1pt'
 
\moveleft.5\hoffset\centerline{cwseitz@iu.edu} % Your address
\moveleft.5\hoffset\centerline{cwseitz.github.io} % Your address

%----------------------------------------------------------------------------------------

\begin{resume}

%----------------------------------------------------------------------------------------
%	OBJECTIVE SECTION
%----------------------------------------------------------------------------------------
 
\section{SUMMARY}  

Aspiring machine learning engineer in pharma or biotech. As a physicist, I am entering into this space having relevant background in Bayesian statistics, optimization, variational inference, information theory, and stochastic processes. I am also quite familiar with deep learning and modern deep learning frameworks. I have a passion for dissecting data by building statistical models.\\
\\
During my academic training, I have brought mathematical and statistical rigor, learned in physics and computer science, to biology. In brief, I have studied two examples of memory formation in biological sytems (i) cytokine-induced transcriptional memory in mammalian cells and (ii) inference methods for determining the structure and plasticity of neural microcircuits. I also have experience building specialized fluorescence microscopes for high-throughput imaging, which has led to a general interest in high-content screens of the tumor microenvironment during cancer immunotherapy. In the future, I envision that I will work as a project lead for a machine learning group focusing on drug discovery.


%----------------------------------------------------------------------------------------
%	EDUCATION SECTION
%----------------------------------------------------------------------------------------

\section{EDUCATION}

{\sl Doctor of Philosopy,} Physics\\
Purdue University, West Lafeyette, IN, 2024\\
Thesis: \textit{Untitled}

{\sl Master of Science,} Biophysics\\
University of Chicago, Chicago, IL, 2021\\
Thesis: \textit{Stable cell assembly formation in excitatory-inhibitory neuronal networks}


{\sl Bachelor of Science,} Magna Cum Laude, Physics \\
Purdue University, Indianapolis, IN, 2019\\
Minor: Mathematics 

{\sl Bachelor of Science,} Magna Cum Laude, Informatics \\
Luddy School of Informatics, Computing, and Engineering, Indiana University Bloomington, 2019\\
Concentration: Mathematics 
 

%----------------------------------------------------------------------------------------
%	COMPUTER SKILLS SECTION
%----------------------------------------------------------------------------------------

\section{COMPUTER \\ SKILLS} 

{\sl Languages \& Software:} 
Python, R, PyTorch, C, Git, LaTeX, Bash, Linux\\
 
%----------------------------------------------------------------------------------------
%	PROFESSIONAL EXPERIENCE SECTION
%----------------------------------------------------------------------------------------
 
\section{EXPERIENCE}

{\sl Research Technician} \hfill 2019-2021 \\
Indiana University - Purdue University, Indianapolis, IN

\begin{itemize} \itemsep -2pt % Reduce space between items

\item Develop an image processing software pipeline for high-throughput quantification of images in fluorescence microscopy

\item Utilize high performance computing clusters for image segmentation, single particle tracking, and image registration
 
\end{itemize}
 
{\sl Undergraduate Researcher} \hfill 2019-2020\\
Indiana University - Purdue University, Indianapolis, IN
\begin{itemize} \itemsep -2pt % Reduce space between items

\item Utilize time-correlated single photon counting (TCSPC) to characterize the sub-Poissonian emission of organic quantum dots dispersed in a thin film of poly-methyl methacrylate (PMMA)

\item Design and utilize a 3-color imaging protocol to perform single-molecule imaging of mRNA transcripts in human epithelial kidney and osteosarcoma cells 

\end{itemize} 


{\sl Undergraduate Tutor} \hfill 2018-2019\\
Indiana University - Purdue University, Indianapolis, IN
\begin{itemize} \itemsep -2pt % Reduce space between items

\item Tutored undergraduate students in introductory physics courses covering classical mechanics, classical electromagnetism, circuit analysis, and modern physics

\end{itemize} 


%----------------------------------------------------------------------------------------
% AWARDS SECTION
%---------------------------------------------------------------------------------------- 

\section{AWARDS}

{\sl NIH Graduate Training Fellowship} \hfill 2020 \\
University of Chicago, Chicago, IL

{\sl Travel Award and Lightning Talk Invitation} \hfill 2019 \\
Physical Sciences in Oncology - Minneapolis, MN

{\sl Hudson and Holland Scholarship for Diversity and Inclusion} \hfill 2013-2017 \\
Indiana University, Bloomington, IN 

{\sl Founders Scholar} \hfill 2013-2017 \\
Indiana University, Bloomington, IN 

{\sl Cigital Scholarship} \hfill 2016-2017 \\
Indiana University, Bloomington, IN 

\section{PUBLICATIONS}

Maelle Locatelli\textsuperscript{\textdagger}, Josh Lawrimore\textsuperscript{\textdagger}, Hua Lin\textsuperscript{\textdagger}, Sarvath Sanaullah, Clayton Seitz, Dave Segall, Paul Kefer, Salvador Moreno Naike, Benton Lietz, Rebecca Anderson, Julia Holmes, Chongli Yuan, George Holzwarth, Bloom Kerry, Jing Liu, Keith D Bonin, Pierre-Alexandre Vidi. \textit{DNA damage reduces heterogeneity and coherence of chromatin motions}. PNAS. 2022
\\
\\
Mengdi Zhang, Clayton Seitz, Garrick Chang, Fadil Iqbal, Hua Lin, and Jing Liu \textit{A guide for single-particle chromatin tracking in live cell nuclei}. Cell Biology International. January 2022.
\\
\\
Wenting Wu, Farooq Syed, Edward Simpson, Chih-Chun Lee, Jing Liu, Garrick Chang, Chuanpeng Dong, Clayton Seitz, Decio L. Eizirik, Raghavendra G. Mirmira, Yunlong Liu, Carmella Evans-Molina; \textit{Impact of Proinflammatory Cytokines on Alternative Splicing Patterns in Human Islets}. Diabetes 1 January 2022; 71 (1): 116–127


\end{resume}

\end{document}