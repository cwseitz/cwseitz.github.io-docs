\documentclass[margin, 10pt]{res} % Use the res.cls style, the font size can be changed to 11pt or 12pt here

\usepackage{helvet} % Default font is the helvetica postscript font
%\usepackage{newcent} % To change the default font to the new century schoolbook postscript font uncomment this line and comment the one above

\setlength{\textwidth}{5.1in} % Text width of the document

\begin{document}

%----------------------------------------------------------------------------------------
%	NAME AND ADDRESS SECTION
%----------------------------------------------------------------------------------------

\moveleft.5\hoffset\centerline{\large\bf Clayton Seitz} % Your name at the top
 
\moveleft\hoffset\vbox{\hrule width\resumewidth height 1pt}\smallskip % Horizontal line after name; adjust line thickness by changing the '1pt'
 
\moveleft.5\hoffset\centerline{cwseitz@iu.edu} % Your address
\moveleft.5\hoffset\centerline{cwseitz.github.io} % Your address

%----------------------------------------------------------------------------------------

\begin{resume}


\section{PERSONAL STATEMENT}  

I specialize in the application of probabilistic models, particularly deep generative models, biological discovery and computer vision. I have primarily applied these frameworks to advancing super-resolution fluorescence microscopy of so-called nucleosome nanodomains, and have published on this topic in high-impact scientific journals. 

%----------------------------------------------------------------------------------------
%	EDUCATION SECTION
%----------------------------------------------------------------------------------------

\section{EDUCATION}

\textbf{Doctor of Philosopy, Physics}\\
Indiana University, Indianapolis, IN\\
Thesis: \textit{Advancing super-resolution microscopy for quantitative in-vivo imaging of chromatin nanodomains}

\textbf{Master of Science, Biophysics}\\
University of Chicago, Chicago, IL, 2021\\
Thesis: \textit{Stable cell assembly formation in excitatory-inhibitory neural networks}


\textbf{Bachelor of Science, Physics}, Magna cum laude\\
Indiana University, Indianapolis, IN, 2019\\
Minor: Mathematics 

\textbf{Bachelor of Science, Informatics}, Magna cum laude\\
Luddy School of Informatics, Computing, and Engineering, Indiana University Bloomington, 2019\\
Concentration: Mathematics 
  
%----------------------------------------------------------------------------------------
%	PROFESSIONAL EXPERIENCE SECTION
%----------------------------------------------------------------------------------------
 
\section{RESEARCH EXPERIENCE}

\textbf{Graduate Research Assistant} \hfill 2022-Present \\
Indiana University, Indianapolis, IN

\begin{itemize} \itemsep -2pt % Reduce space between items

\item Design deep generative models for enhancing super-resolution fluorescence microscopy

\item Implement direct stochastic optical reconstruction microscopy (dSTORM) for super-resolution imaging of chromatin in livigng cell nuclei

\item Investigate the impact of mutations in epigenetic proteins on chromatin structure in living cell nuclei experimentally and complement experimental data with molecular dynamics simulations
 
\end{itemize}

\textbf{Graduate Trainee} \hfill 2020-2022 \\
University of Chicago, Chicago, IL

\begin{itemize} \itemsep -2pt % Reduce space between items

\item Utilize fluorescence microscopy to measure temporal dynamics of calcium concentration in single cells
\item Generate Monte Carlo simulations of spiking neural networks to relate network architecture to spiking dynamics
 
\end{itemize}
 
\textbf{Undergraduate Research Assistant} \hfill 2019-2020\\
Indiana University, Indianapolis, IN
\begin{itemize} \itemsep -2pt % Reduce space between items

\item Develop an image processing package in Python for processing large volumes of images generated by fluorescence microscopy

\item Utilize time-correlated single photon counting (TCSPC) to characterize the sub-Poissonian emission of organic quantum dots dispersed in a thin film of poly-methyl methacrylate (PMMA)


\end{itemize} 


%----------------------------------------------------------------------------------------
% AWARDS SECTION
%---------------------------------------------------------------------------------------- 

\section{AWARDS}

{\sl NIH Graduate Training Fellowship} \hfill 2020 \\
University of Chicago, Chicago, IL

{\sl Travel Award and Lightning Talk Invitation} \hfill 2019 \\
Physical Sciences in Oncology - Minneapolis, MN

{\sl Hudson and Holland Scholarship for Diversity and Inclusion} \hfill 2013-2017 \\
Indiana University, Bloomington, IN 

{\sl Founders Scholar} \hfill 2013-2017 \\
Indiana University, Bloomington, IN 

{\sl Cigital Scholarship} \hfill 2016-2017 \\
Indiana University, Bloomington, IN 

\section{PUBLICATIONS}

\textbf{Clayton Seitz}\textsuperscript{\textdagger}, Donghong Fu\textsuperscript{\textdagger}, Mengyuan Liu, Hailan Ma, and Jing Liu. \textit{BRD4 phosphorylation regulates the structure of chromatin nanodomains}. In Review. 2024

\textbf{Clayton Seitz} and Jing Liu. \textit{Uncertainty-aware localization microscopy by variational diffusion}. In Review. 2024

Maelle Locatelli\textsuperscript{\textdagger}, Josh Lawrimore\textsuperscript{\textdagger}, Hua Lin\textsuperscript{\textdagger}, Sarvath Sanaullah, \textbf{Clayton Seitz}, Dave Segall, Paul Kefer, Salvador Moreno Naike, Benton Lietz, Rebecca Anderson, Julia Holmes, Chongli Yuan, George Holzwarth, Bloom Kerry, Jing Liu, Keith D Bonin, Pierre-Alexandre Vidi. \textit{DNA damage reduces heterogeneity and coherence of chromatin motions}. PNAS 12 July 2022; 119 (29): 1-11
\\
\\
Mengdi Zhang, \textbf{Clayton Seitz}, Garrick Chang, Fadil Iqbal, Hua Lin, and Jing Liu \textit{A guide for single-particle chromatin tracking in live cell nuclei}. Cell Biology International 15 January 2022; 46 (5): 683-700
\\
\\
Wenting Wu, Farooq Syed, Edward Simpson, Chih-Chun Lee, Jing Liu, Garrick Chang, Chuanpeng Dong, \textbf{Clayton Seitz}, Decio L. Eizirik, Raghavendra G. Mirmira, Yunlong Liu, Carmella Evans-Molina; \textit{Impact of Proinflammatory Cytokines on Alternative Splicing Patterns in Human Islets}. Diabetes 25 October 2021; 71 (1): 116–127

\textbf{Clayton Seitz}, Hailan Ma, and Jing Liu. \textit{Bayesian analysis of GBP5 transcriptional bursts}. Biophysical Society Annual Conference 2022


\textbf{Clayton Seitz}, Hua Lin, Keith Bonin, Pierre-Alexandre Vidi, and Jing Liu. \textit{Quantifying the spatiotemporal dynamics of dUTP labeled chromatin during the DNA damage response}. Biophysical Society Annual Conference 2020


\section{TECHNICAL \\ SKILLS} 

{\sl Programming Languages \& Software:} 
Python, R, PyTorch, C/C++, Git, LaTeX, Bash, HPCs/SLURM\\

\end{resume}

\end{document}