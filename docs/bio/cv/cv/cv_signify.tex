\documentclass[margin, 10pt]{res} % Use the res.cls style, the font size can be changed to 11pt or 12pt here

\usepackage{helvet} % Default font is the helvetica postscript font
%\usepackage{newcent} % To change the default font to the new century schoolbook postscript font uncomment this line and comment the one above

\setlength{\textwidth}{5.1in} % Text width of the document

\begin{document}

%----------------------------------------------------------------------------------------
%	NAME AND ADDRESS SECTION
%----------------------------------------------------------------------------------------

\moveleft.5\hoffset\centerline{\large\bf Clayton W. Seitz, Ph.D.} % Your name at the top
 
\moveleft\hoffset\vbox{\hrule width\resumewidth height 1pt}\smallskip % Horizontal line after name; adjust line thickness by changing the '1pt'
 
\moveleft.5\hoffset\centerline{cwseitz@iu.edu} % Your address
\moveleft.5\hoffset\centerline{cwseitz.github.io} % Your address

%----------------------------------------------------------------------------------------

\begin{resume}


\section{PERSONAL STATEMENT}  

%I have a broad background in machine learning, statistics, and physics/mathematics. Recently, I have researched variational diffusion models. I also specialize in optimization problems and machine learning methods for time series analysis and generative modeling.

%I have a background in statistics, machine learning, and mathematics. Recently, I have researched variational inference with diffusion models. I specialize in Bayesian modeling and the application of machine learning methods for time series analysis and generative modeling.

%I have a background in statistics, machine learning, and mathematics. Recently, I have researched prediction models based on variational diffusion. I generally specialize in the development and application of machine learning methods to make predictions from data.

I have a background in computer vision, machine learning, and mathematical physics. Recently, I have researched novel image/video generation methods based on diffusion models. I also specialize in image/video processing, object detection, segmentation, and object tracking. These methods were applied to microscopy in biological research.

%I have strong expertise in AI/ML techniques and frameworks, such as deep learning, generative modeling, and large language models. I also have proficiency in tools like Python, TensorFlow, PyTorch, and others. I am familiar cloud infrastructure, big data, and best practices in AI/ML and data science communities. I can lead complex projects and deliver successful outcomes.

%I have a background in machine learning and mathematics. Recently, I have researched novel image sampling methods based on deep learning. I also have a background with spiking neural networks and machine learning algorithms for image/video processing, object detection, segmentation, and object tracking

%I am a biophysicist with experience developing AI/ML and statistical methods to solve biological problems. This experience is interdisciplinary and lies at the interface of biology, physics, and AI/ML. 

%I specialize in machine learning methods for computer vision, image processing, and generative modeling. 

%I am a physicist with a background in image science, optics, and optical sensing. I specialize in the development and application of advanced modeling of image formation for image processing, computer vision, time series analysis, and generative modeling. 

%I specialize in the development and application of machine learning methods for computer vision, large language models, time series analysis, and generative modeling. I maintain a strong background in functional and object-oriented programming with experience writing production quality software.

%----------------------------------------------------------------------------------------
%	EDUCATION SECTION
%----------------------------------------------------------------------------------------

\section{EDUCATION}

\textbf{Doctor of Philosopy, Physics}\hfill 2024 \\
Purdue University\\
%Thesis: \textit{Advancing super-resolution microscopy for quantitative in-vivo imaging of chromatin nanodomains}

\textbf{Master of Science, Physics} \hfill 2021\\
University of Chicago\\
%Thesis: \textit{Stable cell assembly formation in excitatory-inhibitory neural networks}


\textbf{Bachelor of Science, Physics}, Magna Cum Laude \hfill 2019\\
Indiana University\\
Minor: Mathematics 

\textbf{Bachelor of Science, Informatics (Math Focus)}, Magna Cum Laude \hfill 2019\\
Indiana University\\


%----------------------------------------------------------------------------------------
%	PROFESSIONAL EXPERIENCE SECTION
%----------------------------------------------------------------------------------------
 
\section{EXPERIENCE}

\textbf{Graduate Researcher} \hfill 2021-2024 \\
Purdue University, Indianapolis, IN

\begin{itemize} \itemsep -2pt % Reduce space between items

\item Designed diffusion models/score-based generative models and general computer vision techniques (object detection, segmentation, etc.) in PyTorch for modeling image datasets in super-resolution microscopy

%\item Developed analysis pipelines for high-throughput spatial biology assays such as multiplexed immunofluorescence and spatial transcriptomics

\item Applied deep neural networks for three dimensional reconstruction of objects

\item Explored text to video diffusion models for physically realistic video generation


%\item Solved optimization problems in the context of image analysis e.g., particle localization, super-resolution, and inference of particle dynamics


\item Applied general probabilistic models for high-dimensional imaging datasets and associated Bayesian methods for statistical inference tasks

%\item Developed Hidden Markov models of photoswitching dynamics and particle motion


\end{itemize}

\textbf{Graduate Researcher} \hfill 2020-2021 \\
University of Chicago, Chicago, IL

\begin{itemize} \itemsep -2pt % Reduce space between items

\item Investigated fundamental learning mechanisms in recurrent neural networks (RNNs) using dynamical models, mean-field theory, and time-series analysis. 

\item Designed and ran Monte Carlo simulations of spiking neural networks 
 
\end{itemize}
 
\textbf{Research Assistant} \hfill 2018-2020\\
Purdue University, Indianapolis, IN
\begin{itemize} \itemsep -2pt

\item Developed a scientific package in Python for high-throughput object detection and tracking
\item Managed the package lifecycle and user training throughout the laboratory

\end{itemize}

%\textbf{Information Systems Intern} \hfill 2016\\
%Liberty Mutual Insurance
%\begin{itemize} \itemsep -2pt
%\item Wrote Python scripts to automate attestation testing of web facing servers
%\item Assisted in configuration of web application firewalls (WAFs)
%\end{itemize} 



%----------------------------------------------------------------------------------------
% AWARDS SECTION
%---------------------------------------------------------------------------------------- 

\section{AWARDS}

{\sl NIH Graduate Training Fellowship} \hfill 2020 \\
University of Chicago, Chicago, IL

{\sl Travel Award and Lightning Talk Invitation} \hfill 2019 \\
Physical Sciences in Oncology - Minneapolis, MN

{\sl Hudson and Holland Scholarship for Diversity and Inclusion} \hfill 2013-2017 \\
Indiana University, Bloomington, IN 

{\sl Founders Scholar} \hfill 2013-2017 \\
Indiana University, Bloomington, IN 

{\sl Cigital Scholarship} \hfill 2016-2017 \\
Indiana University, Bloomington, IN 

\section{PUBLICATIONS}

\textbf{Clayton Seitz}\textsuperscript{\textdagger}, Donghong Fu\textsuperscript{\textdagger}, Mengyuan Liu, Hailan Ma, and Jing Liu. \textit{BRD4 phosphorylation regulates the structure of chromatin nanodomains}.\\ Physical Review Letters (In Review). https://doi.org/10.1101/2024.09.03.611057. 2024

\textbf{Clayton Seitz} and Jing Liu. \textit{Uncertainty-aware localization microscopy by variational diffusion}. In Review. 2024

\textbf{Clayton Seitz} and Jing Liu. \textit{Quantum enhanced localization microscopy with a single photon avalanche diode array}. In Review. 2024

Maelle Locatelli\textsuperscript{\textdagger}, Josh Lawrimore\textsuperscript{\textdagger}, Hua Lin\textsuperscript{\textdagger}, Sarvath Sanaullah, \textbf{Clayton Seitz}, Dave Segall, Paul Kefer, Salvador Moreno Naike, Benton Lietz, Rebecca Anderson, Julia Holmes, Chongli Yuan, George Holzwarth, Bloom Kerry, Jing Liu, Keith D Bonin, Pierre-Alexandre Vidi. \textit{DNA damage reduces heterogeneity and coherence of chromatin motions}. PNAS 12 July 2022; 119 (29): 1-11
\\
\\
Mengdi Zhang, \textbf{Clayton Seitz}, Garrick Chang, Fadil Iqbal, Hua Lin, and Jing Liu \textit{A guide for single-particle chromatin tracking in live cell nuclei}. Cell Biology International 15 January 2022; 46 (5): 683-700
\\
\\
Wenting Wu, Farooq Syed, Edward Simpson, Chih-Chun Lee, Jing Liu, Garrick Chang, Chuanpeng Dong, \textbf{Clayton Seitz}, Decio L. Eizirik, Raghavendra G. Mirmira, Yunlong Liu, Carmella Evans-Molina; \textit{Impact of Proinflammatory Cytokines on Alternative Splicing Patterns in Human Islets}. Diabetes 25 October 2021; 71 (1): 116–127

\textbf{Clayton Seitz}, Hailan Ma, and Jing Liu. \textit{Cytokine-induced transcriptional memory is evident in the kinetics of transcriptional bursts}. Biophysical Society Annual Conference 2022


\textbf{Clayton Seitz}, Hua Lin, Keith Bonin, Pierre-Alexandre Vidi, and Jing Liu. \textit{Quantifying the spatiotemporal dynamics of dUTP labeled chromatin during the DNA damage response}. Biophysical Society Annual Conference 2020


\section{SOFTWARE \\ SKILLS} 

{\sl Programming Languages \& Software:} 
Linux, Bash, Python, R, PyTorch, C/C++, SQL, LaTeX, Git, Docker, SLURM, AWS\\

\end{resume}

\end{document}