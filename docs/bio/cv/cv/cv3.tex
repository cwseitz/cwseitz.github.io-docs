\documentclass[margin, 10pt]{res} % Use the res.cls style, the font size can be changed to 11pt or 12pt here

\usepackage{helvet} % Default font is the helvetica postscript font
%\usepackage{newcent} % To change the default font to the new century schoolbook postscript font uncomment this line and comment the one above

\setlength{\textwidth}{5.1in} % Text width of the document

\begin{document}

%----------------------------------------------------------------------------------------
%	NAME AND ADDRESS SECTION
%----------------------------------------------------------------------------------------

\moveleft.5\hoffset\centerline{\large\bf Clayton W. Seitz, Ph.D.} % Your name at the top
 
\moveleft\hoffset\vbox{\hrule width\resumewidth height 1pt}\smallskip % Horizontal line after name; adjust line thickness by changing the '1pt'
 
\moveleft.5\hoffset\centerline{cwseitz@iu.edu} % Your address
\moveleft.5\hoffset\centerline{cwseitz.github.io} % Your address

%----------------------------------------------------------------------------------------

\begin{resume}

I am an optical microscopist with experience with several imaging modalities including widefield, confocal, single molecule localization microscopy, and selective plane illumination. I have extensive experience with experimental and simulation techniques for studying quantum properties of fluorescent emitters, including photon correlations.

%\section{PERSONAL STATEMENT}  

%I have a background in quantum computing and quantum information theory, with research experience in quantum %imaging. Furthermore, I am a capable programmer and have experience with machine learning methods.

%----------------------------------------------------------------------------------------
%	EDUCATION SECTION
%----------------------------------------------------------------------------------------

\section{EDUCATION}

\textbf{Doctor of Philosopy, Physics}\hfill 2024 \\
Purdue University\\
%Thesis: \textit{Advancing super-resolution microscopy for quantitative in-vivo imaging of chromatin nanodomains}

\textbf{Master of Science, Physics} \hfill 2021\\
University of Chicago\\
%Thesis: \textit{Stable cell assembly formation in excitatory-inhibitory neural networks}


\textbf{Bachelor of Science, Physics}, Magna Cum Laude \hfill 2019\\
Indiana University\\
Minor: Mathematics 

\textbf{Bachelor of Science, Informatics (Math Focus)}, Magna Cum Laude \hfill 2019\\
Indiana University\\


%----------------------------------------------------------------------------------------
%	PROFESSIONAL EXPERIENCE SECTION
%----------------------------------------------------------------------------------------
 
\section{EXPERIENCE}

\textbf{Graduate Researcher} \hfill 2022-Present \\
Purdue University, Indianapolis, IN

\begin{itemize} \itemsep -2pt % Reduce space between items

\item Conceptualized and implemented a novel quantum imaging strategy for fluorescence nanoscopy 

\item Developed general probabilistic models for quantum imaging experiments and associated Bayesian methods for statistical inference tasks

\item Engineered novel hardware and software systems for widefield quantum imaging and photonics applications

\end{itemize}

\textbf{Graduate Researcher} \hfill 2020-2022 \\
University of Chicago, Chicago, IL

\begin{itemize} \itemsep -2pt % Reduce space between items

\item Investigated fundamental learning mechanisms in recurrent neural networks (RNNs) using dynamical models, mean-field theory, and time-series analysis. 

\item Designed and ran Monte Carlo simulations of spiking neural networks 
 
\end{itemize}
 
\textbf{Research Assistant} \hfill 2019-2020\\
Purdue University, Indianapolis, IN
\begin{itemize} \itemsep -2pt

\item Developed a scientific package in Python for high-throughput object detection and tracking
\item Managed the package lifecycle and user training throughout the laboratory

\end{itemize}


%----------------------------------------------------------------------------------------
% AWARDS SECTION
%---------------------------------------------------------------------------------------- 

\section{AWARDS}

{\sl NIH Graduate Training Fellowship} \hfill 2020 \\
University of Chicago, Chicago, IL

{\sl Travel Award and Lightning Talk Invitation} \hfill 2019 \\
Physical Sciences in Oncology - Minneapolis, MN

{\sl Hudson and Holland Scholarship for Diversity and Inclusion} \hfill 2013-2017 \\
Indiana University, Bloomington, IN 

{\sl Founders Scholar} \hfill 2013-2017 \\
Indiana University, Bloomington, IN 

{\sl Cigital Scholarship} \hfill 2016-2017 \\
Indiana University, Bloomington, IN 

\section{PUBLICATIONS}

\textbf{Clayton Seitz} and Jing Liu. \textit{Quantum enhanced localization microscopy with a single photon avalanche diode array}. In Review. 2024

\textbf{Clayton Seitz}\textsuperscript{\textdagger}, Donghong Fu\textsuperscript{\textdagger}, Mengyuan Liu, Hailan Ma, and Jing Liu. \textit{BRD4 phosphorylation regulates the structure of chromatin nanodomains}.\\ https://doi.org/10.1101/2024.09.03.611057. 2024

\textbf{Clayton Seitz} and Jing Liu. \textit{Uncertainty-aware localization microscopy by variational diffusion}. In Review. 2024

Maelle Locatelli\textsuperscript{\textdagger}, Josh Lawrimore\textsuperscript{\textdagger}, Hua Lin\textsuperscript{\textdagger}, Sarvath Sanaullah, \textbf{Clayton Seitz}, Dave Segall, Paul Kefer, Salvador Moreno Naike, Benton Lietz, Rebecca Anderson, Julia Holmes, Chongli Yuan, George Holzwarth, Bloom Kerry, Jing Liu, Keith D Bonin, Pierre-Alexandre Vidi. \textit{DNA damage reduces heterogeneity and coherence of chromatin motions}. PNAS 12 July 2022; 119 (29): 1-11
\\
\\
Mengdi Zhang, \textbf{Clayton Seitz}, Garrick Chang, Fadil Iqbal, Hua Lin, and Jing Liu \textit{A guide for single-particle chromatin tracking in live cell nuclei}. Cell Biology International 15 January 2022; 46 (5): 683-700
\\
\\
Wenting Wu, Farooq Syed, Edward Simpson, Chih-Chun Lee, Jing Liu, Garrick Chang, Chuanpeng Dong, \textbf{Clayton Seitz}, Decio L. Eizirik, Raghavendra G. Mirmira, Yunlong Liu, Carmella Evans-Molina; \textit{Impact of Proinflammatory Cytokines on Alternative Splicing Patterns in Human Islets}. Diabetes 25 October 2021; 71 (1): 116–127

\textbf{Clayton Seitz}, Hailan Ma, and Jing Liu. \textit{Cytokine-induced transcriptional memory is evident in the kinetics of transcriptional bursts}. Biophysical Society Annual Conference 2022


\textbf{Clayton Seitz}, Hua Lin, Keith Bonin, Pierre-Alexandre Vidi, and Jing Liu. \textit{Quantifying the spatiotemporal dynamics of dUTP labeled chromatin during the DNA damage response}. Biophysical Society Annual Conference 2020


\section{SOFTWARE \\ SKILLS} 

{\sl Programming Languages \& Software:} 
Linux, Bash, Python, Qiskit, Julia, R, PyTorch, C/C++, SQL, LaTeX, Git, Docker, SLURM, AWS, CUDA\\

\end{resume}

\end{document}