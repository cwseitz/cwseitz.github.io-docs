\documentclass{article}
\input ../../preamble
\parindent = 0em
\parskip = 1ex

%\newcommand{\solution}[1]{}
\newcommand{\solution}[1]{\bigskip {\color{red} {\bf Solution}: #1}}

\begin{document}


\centerline{\bf TTIC 31230 Fundamentals of Deep Learning}

\bigskip

\centerline{\bf Problems for Graphical Models.}

\bigskip

\bigskip
{\bf Problem 1. Dynamic Programing for HMMs}  Assume we have an input sequence
$x_1,\ldots,x_T$ and a phoneme gold label $y_1,\ldots,y_T$ with $y_t
\in {\cal P}$.  This problem is simpler than CTC because the gold
label has the same length as the input sequence.

\medskip
In an HMM we assume a hidden state sequence $s_1,\ldots,s_T$
with $s_t \in {\cal S}$ where ${\cal S}$ is some finite sets of ``hidden states''.  Here will assume that then
some deep network has computed transition probabilities and emission probabilities.
$$P_{\mathrm{Trans}}(s_{t+1}\;|\; s_t)$$
$$P_{\mathrm{Emit}}(y_t\;|\;s_t)$$
We assume an initial state $s_{\mathrm{init}}$ and a stop state $s_{\mathrm{stop}}$ such that $s_1= s_{\mathrm{init}}$
(before emitting any phonemes). The length $T$ is determined
by when the hidden state becomes $s_{\mathrm{stop}}$ giving $s_{T+1} = s_{\mathrm{stop}}$. 

\medskip
For a given gold sequence $y_1,\ldots,y_T$ we define a ``forward tensor'' as
$$F[t,s] = P(y_1,\ldots,y_{t-1}\;\wedge s_t = s)$$
We have
\begin{eqnarray*}
  F[1,s_{\mathrm{init}}] & = & 1 \\
  F[1,s] & = & 0 \;\;\;\mbox{for $s  \not = s_{\mathrm{init}}$}
\end{eqnarray*}

\medskip
(a) Write a dynamic programming equation to compute $F[t,s]$ from $F[t-1,s']$ for various values of $s'$.

\solution{
  $$F[t,s] = \sum_{s'}\;F[t-1,s']P_{\mathrm{Emit}}(y_{t-1}|s')P_{\mathrm{Trans}}(s|s')$$
}


\medskip
(b) Express $P(y_1,\ldots,y_T)$ in terms of $F[t,s]$.

\solution{
  $$P(y_1,\ldots y_T) = F[T+1,s_{\mathrm{stop}}]$$
}

\medskip
(c) EM for HMMs involves computing a ``backward'' tensor
$$B[t,s] = P(y_t,\ldots,y_T\;|\;s_t=s).$$
Explain why, if the forward equations are written in a framework,
we do not need to also compute the backward tensor.

\solution{
  Once we have expressed the loss $- \ln P(y_1,\ldots,y_T)$ in a framework we can train the model by SGD using the framework's implementation of back-propagation.
  Nothing more is needed.
}

\bigskip
{\bf Problem 2. CTC for image labeling}

Suppose that the training data consists of pairs $(I,S)$ where $I$ is an image and $S$
is a set of object types occuring in the image.  For example $S$ might be $\{\mathrm{Person},\mathrm{Dog},\mathrm{Car}\}$.
To be concrete we can take ${\cal C}$ to be the set of image labels used in CIFAR 100 and take $S$ to be a subset
of ${\cal C}$ containing no more than five labels ($|S| \leq 5$).
We want to do SGD on a model defining $P_\Phi(S\;|\;I)$.

We will use a latent variable $z[X,Y]$ such that for pixel coordinates $(x,y)$ we have $z[x,y] \in {\cal C} \cup \{\bot\}$.
For a given $z[X,Y]$ define $S(z[X,Y])$ to be the set of classes appearing in $z[X,Y]$, i.e.,
$S(z[X,Y]) = \{c\;\exists\;x,y\;z(x,y) = c\}$. Here the ``semantic segmentation'' $Z[X,Y]$ is analogous to the phoneme
sequence $z[T]$ in CTC. Unlike the CTC model, the label
$S$ is a set rather than a sequence.

We assume a CNN (with convolutions of stride 1 to preserve spatial dimensions) followed by
a softmax at each pixel to get a probability
$P_\Phi(z[x,y] = c)$ for each pixel location $(x,y)$ and each $c \in {\cal C} \cup \{\bot\}$ and where
each pixel location has an independent probability distribution over classes. 
To simplify notation we can reshape the pixel locations into a linear sequence
and replace $z[X,Y]$ by $z[T]$ with $T = X \times Y$ so we have $z[1],z[1],\ldots,z[T]$.

Define
$$S_t = \{c \in {\cal C}\;\exists t'\leq t\;z[t'] = c\}$$
For $U \subseteq S$ define
$$F[U,t] = P(S_t = U)$$
Note that for $|S| \leq 5$ there are at most 32 possible values of $U$.
Give dynamic programming equations defining $F[U,0]$ and defining $F[U,t+1]$ in term of $F[U',t]$ for various $U'$.

\solution{
  
\begin{tabbing}
  {\color{red} $F[\emptyset,0] = 1$} \\
  For $U$ a nonempty subset of $S$ {\color{red} $F[U,0] = 0$} \\
  Fo\=r $t = 1,\dots,T$ \\
      \> Fo\=r $U \subseteq S$ \\
      \>     \> {\color{red} $F[U,t] = P(z[t] = \bot) F[U,t-1] + \sum_{c \in U} P(z[t] = c)(F[U\backslash c,t-1] + F[U,t-1])$}
\end{tabbing}
}


\bigskip
\bigskip

~{\bf Problem 3. Pseudolikelihood of a one dimensional spin glass.}
We let $\hat{x}$ be an assignment of a value to every node where the nodes are numbered from $1$ to $N_\mathrm{nodes}$ and for every node $i$ we have
$\hat{x}[i] \in \{0,1\}$. We define the score of $\hat{x}$ by
$$f(\hat{x}) = \sum_{i = 1}^{N-1} \bbone[\hat{x}[i] = \hat{x}[i+1]]$$
The probability distribution over assignments is defined by a softmax.  We let $\hat{x}[i:=v]$ be the assignment identical to $\hat{x}$ except that node $i$ is assigned the value $v$.
The expression $\hat{x}[i]=v$ is either true or false depending on whether no $i$ is assigned value $v$ in $\hat{x}$.  So these are quite different.
$$P_f(\hat{x}) = \softmax_{\hat{x}} f(\hat{x})$$
Pseudolikelihood is defined in terms of the softmax probability $P_f$ as follows.
\begin{eqnarray*}
  \tilde{P}_f(\hat{x}) & = & \prod_i P_f(\hat{x}[i] \;|\; \hat{x}\backslash i)
\end{eqnarray*}
What is the pseudolikelihood of the all ones assignment under the definition of $f$ given above?

\solution{
In a graphical model $P_f(\hat{x}[i] \;|\; \hat{x}/i)$ is determined by the neighbors of $i$ and we can consider only how a value is scored against it neighbors.  For $\hat{x}$ equal to all ones we have

\begin{eqnarray*}
f(\hat{x}) & = & N-1 \\
\\
f(\hat{x}[i := 0]) & = & \left\{\begin{array}{ll} N-3 & \mbox{for $1 < i < N$} \\ N-2 & \mbox{for $i=1$ or $i=N$} \end{array}\right.
\end{eqnarray*}
For $1 < i < N$ we get
\begin{eqnarray*}
Q_f(\hat{x}[i=1]\; |\;\hat{x}/i) & = & \frac{e^{N-1}}{e^{N-1} + e^{N-3}} \\
\\
& = & \frac{1}{1 + e^{-2}}
\end{eqnarray*}
and for $i = 1$ or $i = N$ we get
$$Q_f(\hat{x}[i=1]\; |\;\hat{x}/i) = \frac{1}{1 + e^{-1}}$$

This gives

$$\tilde{Q}(\hat{x}) = (1 + e^{-1})^{-2}(1 + e^{-2})^{-(N-2)}$$
}

\bigskip
{\bf Problem 4. Pseudolikelihood for images.} Consider a semantic segmentation $\hat{y}[i]$ on pixels $i$ with $\hat{y}[i]$
a semantic class label in $\{C_1,\ldots,C_K\}$.  We also assume a scoring function $s_\Phi$ on semantic segmentations defining
$$P_\Phi(\hat{y}) = \softmax_{\hat{y}}\;s_\Phi(\hat{y})$$
Pseudolikelihood is defined by
\begin{eqnarray*}
  \tilde{P}_\Phi(\hat{y}) & = & \prod_i P_\Phi(\hat{y}[i] \;|\; \hat{y}\backslash i)
\end{eqnarray*}
where $\hat{y}\backslash i$ assigns a class to every pixel other than $i$, and $\hat{y}[i:=c]$ is the semantic
segmentation identical to $\hat{y}$ except that pixel $i$ is labeled with semantic class $c$.  In a typical graphical model for images we have
$$P_\Phi(\hat{y}[i]\; |\; \hat{y}\backslash i) = P_\Phi(\hat{y}[i]\; | \;\hat{y}[N(i)])$$
where $\hat{y}[N(i)]$ is $\hat{y}$ restricted to those pixels which are neighbors of pixel $i$.

\medskip
(a) show
$$\frac{P_\Phi(\hat{y})}{\sum_c P_\Phi(\hat{y}[i:=c])} = \softmax_c s_\Phi(\hat{y}[i:=c])\;\;\;\mbox{evaluated at $c = y[i]$}$$

\solution{
\begin{eqnarray*}
  \frac{P_\Phi(\hat{y})}{\sum_c P_\Phi(\hat{y}[i:=c])} & = & \frac{\frac{1}{Z} e^{s_\Phi(\hat{y})}}{\sum_c \frac{1}{Z} e^{s_\Phi(\hat{y}[i:=c])}} \\
  \\
  & = & \frac{e^{s_\Phi(\hat{y})}}{\sum_c e^{s_\Phi(\hat{y}[i:=c])}} \\
  \\
  & = & \softmax_c\;s_\Phi(\hat{y}[i:=c]) \;\; \mbox{evaluated at $c = \hat{y}[i]$}
\end{eqnarray*}
}

\medskip
(b) How many scores need to be computed in the worst case for computing $P_\Phi(\hat{y})$.  Given the result of part (a), how many for computing $\tilde{P}_\Phi(\hat{y})$?

\solution{$K^N$ for $P_\Phi$ and $KN$ for $\tilde{P}_\Phi$.}

\medskip
(c) Consider a distribution on semantic segmentations where for each pixel the class assigned to that pixel is uniquely determined by the classes of its neighbors.
Can this distribution be defined by a softmax over scores?  Explain your answer.

\solution{
  No. Since $e^s > 0$ for any finite $s$, all semantic segmentations must have nonzero probability.
}

\medskip
(d) If $P_\Phi$ is a distribution defined in some other way such that the class of each pixel is completely determined by the other pixels,
given a simple expression for $\tilde{P}_\Phi(\hat{y})$ in the case where $\hat{y}$ has nonzero probability under $P_\Phi$.

\solution{We have $P_\Phi(\hat{y}|\hat{y}\backslash i) = 1$ which implies $\tilde{P}(\hat{y}) = 1$.}

\bigskip
~{\bf Problem 5. Pseudolikelihood for Monocular Distance Estimation. (25 points)} Here we are interested in labeling each pixed with a distance from the camera.
Each pixel $i$ is to be labeled with a real number $\hat{y}[i] > 0$ giving the distance in (say) meters from the camera to the point on the object displayed by that pixel.  We assume a neural network
that computes for each pixel $i$ an expected distance $\mu_i$ and a variance $\sigma_i > 0$.  For each pair of neighboring pixels $i$ and $j$
the neural network computes a real number $\lambda_{\tuple{i,j}} \geq 0$. For each assignment $\hat{y}$ of distances to pixels we then define the score $s(\hat{y})$ by
\begin{eqnarray*}
  s(\hat{y}) & = & \sum_{i\in \mathrm{nodes}}\;- (\hat{y}[i] - \mu_i)^2/\sigma_i^2\;\; + \sum_{\tuple{i,j} \in \mathrm{edges}}\;- \lambda_{\tuple{i,j}}|\hat{y}[i]-\hat{y}[j]|
\end{eqnarray*}

\medskip
(a) This scoring function determines a continuous softmax distribution defined by
$$p(\hat{y}) = \frac{1}{Z}e^{s(\hat{y})}$$
where $Z$ is an integral rather than a sum.  What is the dimension of the space to be integrated over in computing $Z$?

\solution{
 This is an integration over $\mathbb{R}^N$ where $N$ is the number of nodes --- an $N_\mathrm{nodes}$ dimentional space.
}

\medskip
(b) We now consider pseudolikelihood for this problem. Give an expression for the continuous conditional probability density on $\hat{y}[i]$ for the distance $\hat{y}[i]$
conditioned on the value of the neighbors $N(i)$ of node $i$.  This probability is written $p(\hat{y}[i]\;|\;\hat{y}[N(i)])$.  You answer should be given as a function of the values
$\hat{y}[j]$ for the nodes $j$ neighboring $i$ written $j \in N(i)$.   Write $Z$ as an integral but do not bother trying to solve it.  What is the dimention of the integral for this
conditional probability?

\solution{
  \begin{eqnarray*}
    p(\hat{y}[i]\;|\;\hat{y}[N(i)]) & = & \frac{1}{Z} \exp\left(- (\hat{y}[i] - \mu_i)^2/\sigma_i^2 +  \sum_{j \in N(i)}\;- \lambda_{\tuple{i,j}}|\hat{y}[i]-\hat{y}[j]|\right) \\
    \\
    Z & = & \int_0^\infty  \exp\left(- (x - \mu_i)^2/\sigma_i^2 +  \sum_{j \in N(i)}\;- \lambda_{\tuple{i,j}}|x-\hat{y}[j]|\right) dx
  \end{eqnarray*}
  This is an integral over a one dimensional space (a single real number).
}

\end{document}
