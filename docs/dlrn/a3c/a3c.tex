\input slide_preamble
\input preamble

\begin{document}

{\Huge

  \centerline{\bf TTIC 31230, Fundamentals of Deep Learning}
  \bigskip
  \centerline{David McAllester, Autumn 2020}
  \vfill
  \centerline{\bf Actor-Critic Algorithms and A3C}
  \vfill
  \vfill  

\slide{REINFORCE}
$$\nabla_\Phi \;E_{\pi_\Phi}\;R \;\;\; = \;\;\; E_{\pi_\Phi}\; \sum_{t,\;t' \geq t} \; \left(\nabla_\Phi\;\ln \pi_\Phi(a_t|s_t)\right) \;r_{t'}$$

\vfill
Sampling runs and computing the above sum over $t$ and $t'$ is Williams' REINFORCE algorithm.

\slide{The Variance Issue}

REINFORCE typically suffers from high variance of the gradient samples requiring very small learning rates and very long convergence times.

\begin{eqnarray*}
    \nabla_\Phi \; E_{\pi_\Phi}\; R  & = & \sum_{t,\;t'\geq t}\; E_{s_t,a_t,r_{t'}}\; \left(\nabla_\Phi\;\ln \pi_\Phi(a_{t}|s_{t})\right)\;\;r_{t'}
\end{eqnarray*}

\vfill
We will consider
\begin{itemize}
\item reducing variance due to $r_{t'}$ with Actor-Critic methods.
\item reducing variance due to  $s_t$ with Advantage Actor-Critic methods.
\item finally Asynchronous Advantge Actor-Critic Methods (A3C).
\end{itemize}

\slidetwo{Reducing the Variance over $r_{t'}$}{The Policy Gradient Theorem}

``Policy Gradient Methods for
Reinforcement Learning with Function
Approximation'' Sutton, McAllester, Singh, Mansour, 2000, cited by 2,841 as of March 2020.

\vfill
``Actor-Critic Algorithms'', Konda and Tsitsilas, 2000, cited by 776.

\vfill
These two papers both appeared at NeurIPS 2000 and are essentially identical.  The first is just easier to read.


\slide{Reducing the Variance over $r_{t'}$}

\begin{eqnarray*}
    \nabla_\Phi \; E_{\pi_\Phi}\; R  & = & \sum_{t,t'}\;E_{s_t,a_t,r_{t'}}\;\nabla_\Phi\; \ln \pi_\Phi(a_{t}|s_{t})\;\;{\color{red} r_{t'}} \\
    \\
    \\
    & = & \sum_{t}\;E_{s_t,a_t}\;\nabla_\Phi\; \ln \pi_\Phi(a_{t}|s_{t})\;\;{\color{red} \sum_{t' \geq t} E_{r_{t'}\;|\;s_t,a_t}\;r_{t'}} \\
  \\
  \\
  & = & E_{\pi_\Phi} \; \sum_t\; \left(\nabla_\Phi \;\ln \pi_\Phi(a_t|s_t)\right) \;\;{\color{red} Q^{\pi_\Phi}(s_t,a_t)} \\
  \\
  \\
 Q^\pi(s,a) & = & \expectsub{\pi}{\sum_t\; r_t\;|\;s_0=s,\;a_0=a}
 \end{eqnarray*}

\slideplain{Reducing the Variance over $r_{t'}$}
\begin{eqnarray*}
  \nabla_\Phi \;E_{\pi_\Phi}\;R   & = & \sum_t\;E_{s_t,a_t}\; \left(\nabla_\Phi \;\ln \pi_\Phi(a_t|s_t)\right) {\color{red} Q^{\pi_\Phi}(s_t,a_t)}
\end{eqnarray*}

\vfill
{\bf The point} is that we can now approximate $Q^{\pi_\Phi}$ with neural network $Q_\Phi$ where the networks $\pi_\Phi$ and $Q_\Phi$ can use different, perhaps overlapping, parts of $\Phi$.

\vfill
We reduced the variance at the cost of approximating the expected future reward.


\slide{The Actor-Critic Algorithm}
\begin{eqnarray*}
  \nabla_\Phi \;E_{\pi_\Phi}\;R   & \approx & E_{\pi_\Phi} \; \sum_t\; {\color{red} \left(\nabla_\Phi \;\ln \pi_\Phi(a_t|s_t)\right) \;\;Q_\Phi(s_t,a_t)}
\end{eqnarray*}

\vfill
\centerline{{\color{red} $\pi_\Phi$} is the ``actor'' and {\color{red} $Q_\Phi$} is the ``critic''}

\slide{The Actor-Critic Algorithm}
\begin{eqnarray*}
  \nabla_\Phi \;E_{\pi_\Phi}\;R   & \approx & E_{\pi_\Phi} \; \sum_t\; \left(\nabla_\Phi \;\ln \pi_\Phi(a_t|s_t)\right) \;\;Q_\Phi(s_t,a_t)
\end{eqnarray*}

We can sample an episode and then do

\begin{eqnarray*}
  \Phi  &\pluseq & \sum_t \eta_1\left(\nabla_{\Phi}\;\ln \pi_{\Phi}(a_i|s_i)\right)Q_\Phi(s_t,a_t)\\
  \Phi & \minuseq & \sum_t \eta_2 \;\nabla_{\Phi}\;\left(Q_\Phi(s_t,a_t) - \sum_{t' \geq t} r_{t'}\right)^2
\end{eqnarray*}

The two updates typically apply to different (but perhaps overlapping) subsets of the parameters $\Phi$.

\slide{Reducing the Variance over $s_t$}

Thoerem:
$$\nabla_\Phi \;E_{\pi_\Phi}\; R = \sum_t\;E_{s_t,a_t} \; \left(\nabla_\Phi \;\ln \pi_\Phi(a_t|s_t)\right)(Q^{\pi_\Phi}(s_t,a_t) - V^{\pi_\Phi}(s_t))$$

\vfill
$V^{\pi_\Phi}(s) = \expectsub{a \sim \pi_\Phi(a|s)}{Q^{\pi_\Phi}(s,a)}$

\vfill
$Q^{\pi_\Phi}(s,a) - V^{\pi_\Phi}(s)$ is the ``advantage'' of deterministically using $a$ rather than sampling an action.

\slide{Proof}

We have the following for any function $V(s)$ of states.

\begin{eqnarray*}
 & & E_{s_t,a_t} \;  \; \left(\nabla_\Phi \;\ln \pi_\Phi(a_t|s_t)\right) V(s_t) \\
 \\
 & = & E_{s_t} \;  \sum_{a_t} \left(\pi_\Phi(a_t|s_t) \;\nabla_\Phi \;\ln \pi_\Phi(a_t|s_t)\right) V(s_t) \\
 \\
 & = & E_{s_t} \;  \sum_{a_t} \left(\nabla_\Phi \pi_\Phi(a_t|s_t)\right) V(s_t) \\
 \\
 & = & E_{s_t} \;  V(s_t) \nabla_\Phi \sum_{a_t}\; \pi_\Phi(a_t|s_t) \;=\; 0
\end{eqnarray*}

\slide{Advantage-Actor-Critic Algorithm}
\begin{eqnarray*}
  \nabla_\Phi \;E_{\pi_\Phi}\;R   & \approx & E_{\pi_\Phi} \; \sum_t\; \left(\nabla_\Phi \;\ln \pi_\Phi(a_t|s_t)\right) \;\;(Q_\Phi(s_t,a_t) - V_\Phi(s_t))
\end{eqnarray*}

We can sample an episode and then do

\begin{eqnarray*}
  \Phi  &\pluseq & \sum_t \eta_1\left(\nabla_{\Phi}\;\ln \pi_{\Phi}(a_i|s_i)\right)(Q_\Phi(s_t,a_t) - V_\Phi(s_t))\\
  \Phi & \minuseq & \sum_t \eta_2 \;\nabla_{\Phi}\;\left(Q_\Phi(s_t,a_t) - \sum \!_{t' \geq t}\; r_{t'}\right)^2 \\
  \Phi & \minuseq & \sum_t \eta_3 \;\nabla_{\Phi}\;\left(V_\Phi(s_t) - Q_\Phi(s_t,a)\right)^2 \\
\end{eqnarray*}


\slidetwo{Asynchronous Methods for Deep RL (A3C)}{Mnih et al., Arxiv, 2016 (Deep Mind)}

\begin{quotation}
  \noindent $\tilde{\Phi} = \Phi$ (retrieve global $\Phi$)\newline
  \noindent using policy $\pi_{\tilde{\Phi}}$ compute $s_t,a_t,r_t,\ldots,s_{t+K},a_{t+K},r_{t+K}$
  $$R_i = \sum_{\delta=0}^D \gamma^{i+\delta} r_{(i+\delta)}$$
  \begin{eqnarray*}
  \Phi  &\pluseq & \eta \sum_{i=t}^{t+K-D} \left(\nabla_{\tilde{\Phi}}\;\ln \pi_{\tilde{\Phi}}(a_i|s_i)\right)\left(R_i - V_{\tilde{\Phi}}(s_i)\right)\\
  \Phi & \minuseq & \eta \sum_{i=t}^{t+K-D} \nabla_{\tilde{\Phi}}\;(V_{\tilde{\Phi}}(s_i) - R_i)^2
  \end{eqnarray*}
\end{quotation}

\slide{Issue: Policies must be Exploratory}

The optimal policy is deterministic --- $a(s) = \argmax_a Q(s,a)$.

\vfill
However, a deterministic policy never samples alternative actions.

\vfill
Typically one forces a random action some small fraction of the time.


\slide{Issue: Discounted Reward}

DQN and A3C use discounted reward on episodic or long term problems.

\vfill
Presumably this is because actions have near term consequences.

\vfill
This should be properly handled in the mathematics, perhaps in terms of the mixing time of
the Markov process defined by the policy.

\slide{Observation: Continuous Actions are Differentiable }

In problems like controlling an inverted pendulum, or robot control generally,
a continuous loss can be defined and the gradient of loss of with respect to a deterministic policy exists.

\slide{More Videos}

https://www.youtube.com/watch?v=g59nSURxYgk

https://www.youtube.com/watch?v=rAai4QzcYbs



\slide{END}

}
\end{document}


