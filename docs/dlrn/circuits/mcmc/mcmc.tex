\input slide_preamble
\input preamble

\begin{document}

{\Huge

  \centerline{\bf TTIC 31230, Fundamentals of Deep Learning}
  \bigskip
  \centerline{David McAllester, Autumn 2020}
  \vfill
  \vfill
  \centerline{\bf Monte-Carlo Markov Chain (MCMC) Sampling}
\vfill
\vfill
\vfill

\slide{Sampling From the Model}

For backpropagation through $P_s(\hat{\cal Y}) = \frac{1}{Z} e^{s(\hat{\cal Y})}$ we have

\vfill
\begin{eqnarray*}
    s^N.\mathrm{grad}[n,y] & = &   P_{\hat{\cal Y} \sim P_s}(\;\;{\color{red} \hat{\cal Y}}[n] = y\;\;) \\
    & & \;\;\;\;\;- \bbone[\;\;{\color{red} {\cal Y}}[n] = y\;\;] \\
    \\
    s^E.\mathrm{grad}[\tuple{n,m},y,y'] & = &  P_{\hat{\cal Y} \sim P_s}(\;\;{\color{red} \hat{\cal Y}}[n] = y \; \wedge \; {\color{red} \hat{\cal Y}}[m] = y'\;\;) \\
    & & \;\;\;\;\;- \bbone[\;\;{\color{red} {\cal Y}}[n] = y\; \wedge \; {\color{red} {\cal Y}}[m] = y'\;\;]
\end{eqnarray*}

\slide{MCMC Sampling}
The model marginals, such as the node marginals
 ${\color{red} P_s(\hat{\cal Y}[n]=y)}$, can be estimated by sampling $\hat{\cal Y}$ from $P_s(\hat{\cal Y})$.

\vfill
There are various ways to design a Markov process whose states are node labelings $\hat{\cal Y}$ and whose stationary distribution is $P_s$.

\vfill
Given such a process we can sample $\hat{\cal Y}$ from $P_s$ by running the process past its mixing time.

\vfill
We will consider Metropolis MCMC and the Gibbs MCMC.  But there are more (like Hamiltonian MCMC).

\slide{Metroplis MCMC}

We assume a neighor relation on node assignments and let $N(\hat{\cal Y})$ be the set of neighbors of assignment $\hat{\cal Y}$.

\vfill
For example, $N(\hat{\cal Y})$ can be taken to be the set of assignments $\hat{\cal Y}'$ that differ form $\hat{\cal Y}$ on exactly one node.

\vfill
For the correctness of Metropolis MCMC we need that all states have the same number of neighbors and that the neighbor relation is symmetric ---
$\hat{\cal Y}' \in N(\hat{\cal Y})$ if and only if $\hat{\cal Y} \in N(\hat{\cal Y}')$.

\slide{Metropolis MCMC}

Pick an initial state $\hat{\cal Y}_0$ and for $t \geq 0$ do

\vfill
\begin{quotation}

    \noindent \begin{enumerate}
    \item Pick a neighbor $\hat{\cal Y}' \in N(\hat{\cal Y}_t)$ uniformly at random.

    \vfill      
    \item If $s(\hat{\cal Y}') > s(\hat{\cal Y}_t)$ then {\color{red} $\hat{\cal Y}_{t+1} = \hat{\cal Y}'$}

    \vfill      
    \item If $s(\hat{\cal Y}') \leq s(\hat{\cal Y})$ then with probability $e^{-\Delta s} = e^{-(s(\hat{\cal Y}) - s(\hat{\cal Y}'))}$
   do  {\color{red} $\hat{\cal Y}_{t+1} = \hat{\cal Y}'$} and otherwise {\color{red} $\hat{\cal Y}_{t+1} = \hat{\cal Y}_t$} 
  \end{enumerate}  
\end{quotation}

\slide{The Metropolis Markov Chain}
We need to show that $P_s(\hat{\cal Y}) = \frac{1}{Z}e^{s(\hat{\cal Y})}$ is a stationary distribution of this process.

\vfill
Let $Q(\hat{\cal Y})$ be the distribution on states defined by drawing a state from $P_s$ and applying one stochastic transition of the Metropolis process.

\vfill
We must show that $Q(\hat{\cal Y}) = P_s(\hat{\cal Y})$.

\slide{Stationarity Condition}

\begin{eqnarray*}
Q(\hat{\cal Y}) & = & P_s(\hat{\cal Y}) + \mbox{flow-in} - \mbox{flow-out} \\
\\
& = & \left\{\;\begin{array}{l} P_s(\hat{\cal Y}) \\
\\
+ \sum_{\hat{\cal Y}' \in N(\hat{\cal Y})} P_s(\hat{\cal Y}'){P_{\mathrm{Trans}}(\hat{\cal Y}' \rightarrow \hat{\cal Y})} \\
\\
- \sum_{\hat{\cal Y}' \in N(\hat{\cal Y})} P_s(\hat{\cal Y}){P_{\mathrm{Trans}}(\hat{\cal Y} \rightarrow \hat{\cal Y}')} \end{array} \right.
\end{eqnarray*}


\slide{Detailed Balance}

Detailed balance means that for each pair of neighboring assignments $\hat{\cal Y}$, $\hat{\cal Y}'$ we have equal flows in both directions.

\vfill
$$P_s(\hat{\cal Y}')P_{\mathrm{Trans}}(\hat{\cal Y}' \rightarrow \hat{\cal Y}) = P_s(\hat{\cal Y})P_{\mathrm{Trans}}(\hat{\cal Y} \rightarrow \hat{\cal Y}')$$

\vfill
Without loss generality assume $s(\hat{\cal Y}') \geq s(\hat{\cal Y})$.

\vfill
Metropolis is defined by
{\huge $$P_{\mathrm{Trans}}(\hat{\cal Y}' \rightarrow \hat{\cal Y}) \;\; = \;\; \frac{1}{N}\;e^{-\Delta s}
\;\;= \;\;\frac{P_s(\hat{\cal Y})}{P_s(\hat{\cal Y}')}\;P_{\mathrm{Trans}}(\hat{\cal Y} \rightarrow \hat{\cal Y}')$$}

\slide{Gibbs Sampling}

The Metropolis algorithm wastes time by rejecting proposed moves.

\vfill
Gibbs sampling avoids this move rejection.

\vfill
In Gibbs sampling we select a node $n$ at random and change that node by drawing a new node value conditioned on the current values of the other nodes.

\vfill
We let {\color{red} $\hat{\cal Y} \backslash n$} be the assignment of labels given by $\hat{\cal Y}$ except that no label is assigned to node $n$.

\vfill
We let {\color{red} $\hat{\cal Y}[N(n)]$} be the assignment that $\hat{\cal Y}$ gives to the nodes (pixels) that are the neighbors of node $n$ (connected to $n$ by an edge.)

\slide{Gibbs Sampling}

Markov Blanket Property:
{\color{red} $$P_s(\hat{\cal Y}[n] \;|\;\hat{\cal Y} \backslash n) = P_s(\hat{\cal Y}[n] \;|\; \hat{\cal Y}[N(n)])$$}
\vfill
Gibbs Sampling, Repeat:

\begin{itemize}
\item   Select $n$ at random

\item {\color{red} draw $y$ from $P_s(\hat{\cal Y}[n]\;|\;\hat{\cal Y} \backslash n) = P_s(\hat{\cal Y}[n] \;|\;\hat{\cal Y}[N(n)])$}

\item $\hat{\cal Y}[n] = y$
\end{itemize}

\vfill
This algorithm does not require knowledge of $Z$.

\vfill
The stationary distribution is $P_s$.

\slide{END}

}

\end{document}
