\input slide_preamble
\input preamble

\begin{document}

{\Huge

  \centerline{\bf TTIC 31230, Fundamentals of Deep Learning}
  \bigskip
  \centerline{David McAllester, Autumn 2020}

  \vfill
  \centerline{\bf What About alpha-beta?}
  \vfill
  \vfill

\slide{Grand Unification}

AlphaZero unifies chess and go algorithms.

\vfill
This unification of intuition (go) and calculation (chess) is surprising.

\vfill
This unification grew out of go algorithms.

\vfill
But are the algorithmic insights of chess algorithms really irrelevant?

\slide{Chess Background}

The first min-max computer chess program was described by Claude Shannon in 1950.

\vfill
Alpha-beta pruning was invented by various people independently, including John McCarthy in the late 1950s.

\vfill
Alpha-beta has been the cornerstone of all chess algorithms until AlphaZero.


\slide{Alpha-Beta Pruning}

\begin{verbatim}
def MaxValue(s,alpha,beta):
   value = alpha
   for s2 in s.children():
     value = max(value, MinValue(s2,value,beta))
     if value >= beta: break()
   return value

def MinValue(s,alpha,beta):
   value = beta
   for s2 in s.children():
     value = min(value, MaxValue(s2,alpha,value))
     if value <= alpha: break()
   return value
\end{verbatim}

\slideplain{Strategies}


An optimal alpha-beta tree is the union of a root-player strategy and an opponent strategy.

\vfill
A strategy for the root player is a selection of a single action for each root-player move and a response for each possible action
of the opponent.

\vfill
A strategy for the opponent is a selection of a single action for each opponent move and a response for each possible action
of the root player.

\slide{Proposal}

Simulations should be divided into root-player strategy simulations and opponent strategy simulations.

\vfill
A root-player strategy simulation is optimistic for the root player and pessimistic for the opponent.

\vfill
An opponent strategy simulation is optimistic for the opponent player and pessimistic for the root-player.

\slide{Proposal}

$$U(s,a) =  \left\{\begin{array}{ll}\lambda_u \; \pi_\Phi(s,a) &\mbox{if $N(s,a) = 0$}
\\ \hat{\mu}(s,a) + \lambda_u\; \pi_\Phi(s,a)/N(s,a) & \mbox{otherwise} \end{array}\right. \;\;\;\;(1)$$

\vfill
$\lambda_u$ should be divided into $\lambda_u^+$ and $\lambda_u^-$ with $\lambda_u^+ > \lambda_u^-$.

\vfill
Simulations should be divided into two types --- optimistic and pessimistic.

\vfill
In optimistic simulations we use $\lambda_u^+$ for root-player moves and $\lambda_u^-$ for opponent moves.

\vfill
In pessimistic simulations we use $\lambda_u^-$ for root-player moves and $\lambda_u^+$ for opponent moves.

\slide{END}



}
\end{document}

\ignore{
\slideplain{Conspiracy Numbers}

Conspiracy Numbers for Min-Max search, McAllester, 1988

\vfill
Each node $s$ has a min-max value $V(s)$ determined by the leaf values.

\vfill
For any positive integer $N$ and potential value $V$ we define $L(N,V)$ to be the set of leaf nodes $s_1$ such that
there exist $N-1$ other leaf nodes $s_2$, $\ldots$, $s_N$ such that by changing the values of $s_1$, $\ldots$, $s_N$ the root node
can be changed to $V$.


\vfill
{\bf Algorithm:}


\vfill
Repeatedly select some $N$ and $V$ such that $L(N,V)$ is non empty and expand some leaf in $L(N,V)$.
}

\ignore{
\slide{Simulation}

To find an upper-confidence leaf for the root and value $U$:

\vfill
At a max node pick the child minimizing $N(s,U)$.

\vfill
At a min node select any child $s$ with $V(s) < U$.

\vfill
\slide{Refinement}

Let the static evaluator associate leaf nodes with values $U(s,N)$ and $L(s,N)$
}

