\documentclass[12pt]{article}
\usepackage{amsmath} % AMS Math Package
\usepackage{bm}
\usepackage{amsthm} % Theorem Formatting
\usepackage{amssymb}    % Math symbols such as \mathbb
\usepackage{graphicx} % Allows for eps images
\usepackage[dvips,letterpaper,margin=1in,bottom=0.7in]{geometry}
\usepackage{tensor}
\usepackage{amsmath}
\usepackage{siunitx}
\usepackage{physics}
\usepackage{amsmath, amssymb, graphics, setspace}

\newcommand{\mathsym}[1]{{}}
\newcommand{\unicode}[1]{{}}

\newcounter{mathematicapage}

\newtheorem{p}{Problem}
\usepackage{cancel}
\newtheorem*{lem}{Lemma}
\theoremstyle{definition}
\newtheorem*{dfn}{Definition}
 \newenvironment{s}{%\small%
        \begin{trivlist} \item \textbf{Solution}. }{%
            \hspace*{\fill} $\blacksquare$\end{trivlist}}%


\begin{document}

 {\noindent\Huge\bf  \\[0.5\baselineskip] {\fontfamily{cmr}\selectfont  Homework 9}         }\\[2\baselineskip] % Title
{ {\bf \fontfamily{cmr}\selectfont Quantum Mechanics}\\ {\textit{\fontfamily{cmr}\selectfont     \today}}}~~~~~~~~~~~~~~~~~~~~~~~~~~~~~~~~~~~~~~~~~~~~~~~~~~~~~~~~~~~~~~~~~~~~~~~~~~~~~    {\large \textsc{C Seitz}
\\[1.4\baselineskip] 

\begin{p}
4.1
\end{p}

\begin{s}

We just have to solve Schrodinger's equation for each individual particle

\begin{align*}
-\frac{\hbar^{2}}{2m}\nabla^{2}\psi(\bm{x}) = E\psi(\bm{x})
\end{align*}

which has eigenvalues

\begin{align*}
E = \frac{\hbar^{2}\pi^{2}}{2mL^{2}}(n_{x}^{2} + n_{y}^{2} + n_{z}^{2}) = E_{0}(n_{x}^{2} + n_{y}^{2} + n_{z}^{2})
\end{align*}

Denote $n_{ij}$ the $i$th dimension for particle $j$. For the three particle system: the lowest energy level has $n_{ij}=1$ and $E = 9E_{0}$. The second lowest energy levels for the whole system have one of $n_{ij} = 2$ and has energy $E = 12E_{0}$. The third lowest has two $n_{ij}=2$. The energy is then $E = 16E_{0}$. The degeneracies are 0, 9, and 36 respectively or 0, 72, 288 if we account for spin. 

For the three particle system: the lowest energy level has $n_{ij}=1$ and $E = 16E_{0}$. The second lowest energy level for the whole system has one of $n_{ij} = 2$ and has energy $E = 19E_{0}$. The third lowest has two $n_{ij}=2$. The energy is then $E = 22E_{0}$. The degeneracies are 0, 16, and 120 respectively or 0, 256, 1920 if we account for spin.

\end{s}

\begin{p}
4.2
\end{p}

\begin{s}

Two unique translation operators $\mathcal{T}$ and $\mathcal{T}'$ commute because

\begin{align*}
\mathcal{T}\mathcal{T'} = \exp\left(-\frac{i}{\hbar}(d\cdot p + d'\cdot p)\right)
\end{align*}

which is clearly the same if we swap the order, due to the exponential property. It is similar for rotations:

\begin{align*}
\mathcal{D}\mathcal{D'} = \exp\left(-\frac{i}{\hbar}(\phi n\cdot J + \phi' n'\cdot J')\right)
\end{align*}

For the translation operator and parity operator

\begin{align*}
\mathcal{T}\pi\ket{\bm{x}} &= \ket{\bm{-x}+d}\\
\pi\mathcal{T}\ket{\bm{x}} &= \ket{-\bm{x}-d}
\end{align*}

so they do not commute. However, parity and rotations commute


\end{s}

\begin{p}
4.3
\end{p}

\begin{s}
If two operators $A$ and $B$ anticommute, then

\begin{align*}
AB\ket{\psi} = -BA\ket{\psi}
\end{align*}

We are told they have common eigenvectors ket $\ket{\psi}$, so

\begin{align*}
ab\ket{\psi} = -ba\ket{\psi}
\end{align*}

In words, the product of eigenvalues changes sign when the order of operators flips. If we take one of the operators to be the parity operator and the other to be the momentum operator. Let $A = \pi$ and $B = \hat{p}:$

\begin{align*}
\pi\hat{p}\ket{\psi} &= -\hat{p}\pi\ket{\psi}\\
p\pi\ket{\psi} &= -\hat{p}\ket{-\psi} \\
p\ket{-\psi} &= p\ket{-\psi} 
\end{align*} 

because the momentum operator is odd under parity. The product of eigenvalues changed sign, but the eigenvalue of the parity operator is unity.


\end{s}

\begin{p}
4.4
\end{p}

\begin{s}

\begin{align*}
\mathcal{Y}_{l=0}^{j=1/2,m=1/2} = Y_{0}^{0}(\theta,\phi)\chi_{+}
\end{align*}

Rotating the spin gives

\begin{align*}
(\sigma\cdot r)\mathcal{Y}_{l=0}^{j=1/2,m=1/2} &= Y_{0}^{0}(\theta,\phi)(\sigma\cdot r)\chi_{+}\\
&= Y_{0}^{0}(\theta,\phi)r(\cos\theta\chi_{+} + \sin\theta e^{i\phi}\chi_{-})\\
&= r\left(\sqrt{3}Y_{1}^{0}\chi_{+} - \sqrt{\frac{2}{3}}Y_{1}^{1}\chi_{-}\right)\\
&= -r\mathcal{Y}_{l=1}^{j=1/2,m=1/2}
\end{align*}

which makes sense because, under space inversion, 

\begin{align*}
\pi^{-1}(\sigma\cdot r)\pi = -(\sigma\cdot r)
\end{align*}

\end{s}

\begin{p}
4.5
\end{p}

\begin{s}
We need to consider the first order corrections to the basis kets $\ket{nljm}$

\begin{align*}
\ket{nljm} = \ket{nljm^{0}} + \sum_{n'l'j'm'\neq nljm}\ket{n'l'j'm'^{0}}\frac{V_{j}}{E_{0}^{0}-E_{j}^{0}} + ...
\end{align*}

for

\begin{align*}
C_{n'l'j'm'} = \frac{V_{j}}{E_{0}^{0}-E_{j}^{0}} = \frac{\bra{nljm}V\ket{n'l'j'm'}}{E_{0}^{0}-E_{j}^{0}}
\end{align*}

The matrix element of the perturbation is 

\begin{align*}
\bra{nljm}V\ket{n'l'j'm'} = \lambda \bra{nljm}\delta^{(3)}(\bm{x})\bm{S}\cdot \bm{p}\ket{n'l'j'm'} + \lambda \bra{nljm}\bm{S}\cdot \bm{p}\delta^{(3)}(\bm{x})\ket{n'l'j'm'}
\end{align*}

which is zero unless $\bm{S}\cdot \bm{p}\ket{n'l'j'm'}$ and $\bm{S}\cdot \bm{p}\ket{nljm}$ are the same, meaning they have the same parity. 

\end{s}

\begin{p}
4.6
\end{p}

\begin{s}
The symmetric state $\ket{S}$ is a sum of sines and cosines in the classically allowed region and $\cosh (x)$ in the classically forbidden region. The antisymmetric state $\ket{A}$ is similar, but has $\sinh (x)$ in the classically forbidden region. 

\begin{align*}
\psi_S = \begin{cases}
  B\sin(kx)  \;\;a < |x| < a+b\\
  C\cosh(x) \;\; |x| < a
\end{cases}
\end{align*}

Continuity of the symmetric wavefunction at $x = a$ requires

\begin{align*}
B\sin(kx) = C\cosh(\kappa x)
\end{align*}

and continuity of the first derivative of the symmetric wavefunction at $x = a$ requires

\begin{align*}
kB\cos(kx) = \kappa C\sinh(\kappa x)
\end{align*}

\begin{align*}
\psi_A = \begin{cases}
  B\sin(kx)  \;\;a < |x| < a+b\\
  C\sinh(x) \;\; |x| < a
\end{cases}
\end{align*}

\begin{align*}
B\sin(kx) = C\sinh(\kappa x)
\end{align*}

and continuity of the first derivative of the symmetric wavefunction at $x = a$ requires

\begin{align*}
kB\cos(kx) = \kappa C\cosh(\kappa x)
\end{align*}


\end{s}

\end{document}