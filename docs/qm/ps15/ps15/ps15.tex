\documentclass[12pt]{article}
\usepackage{amsmath} % AMS Math Package
\usepackage{bm}
\usepackage{amsthm} % Theorem Formatting
\usepackage{amssymb}    % Math symbols such as \mathbb
\usepackage{graphicx} % Allows for eps images
\usepackage[dvips,letterpaper,margin=1in,bottom=0.7in]{geometry}
\usepackage{tensor}
\usepackage{amsmath}
\usepackage{siunitx}
\usepackage{physics}
\usepackage{amsmath, amssymb, graphics, setspace}

\newcommand{\mathsym}[1]{{}}
\newcommand{\unicode}[1]{{}}

\newcounter{mathematicapage}

\newtheorem{p}{Problem}
\usepackage{cancel}
\newtheorem*{lem}{Lemma}
\theoremstyle{definition}
\newtheorem*{dfn}{Definition}
 \newenvironment{s}{%\small%
        \begin{trivlist} \item \textbf{Solution}. }{%
            \hspace*{\fill} $\blacksquare$\end{trivlist}}%


\begin{document}

 {\noindent\Huge\bf  \\[0.5\baselineskip] {\fontfamily{cmr}\selectfont  Homework 5}         }\\[2\baselineskip] % Title
{ {\bf \fontfamily{cmr}\selectfont Quantum Mechanics}\\ {\textit{\fontfamily{cmr}\selectfont     \today}}}~~~~~~~~~~~~~~~~~~~~~~~~~~~~~~~~~~~~~~~~~~~~~~~~~~~~~~~~~~~~~~~~~~~~~~~~~~~~~    {\large \textsc{C Seitz}
\\[1.4\baselineskip] 


\begin{p}
Problem 4.4
\end{p}

\begin{s}
\begin{align*}
H &= e^{i\alpha}R_{z}(\frac{\pi}{2})R_{x}(\frac{\pi}{2})\\
&= \frac{e^{i\alpha}e^{i\pi/4}}{\sqrt{2}}
\begin{pmatrix}1 & 1 \\ 1 & -1 \end{pmatrix}
\end{align*}
Therefore, $\alpha = -\pi/4$.
\end{s}

\begin{p}
Problem 4.5
\end{p}

\begin{s}
\begin{align*}
(n\cdot\sigma)^{2} &= (n_{x}\sigma_{x} + n_{y}\sigma_{y} + n_{z}\sigma_{z})^{2}\\
&= n_{x}^{2}\sigma_{x}^{2} + n_{y}^{2}\sigma_{y}^{2} + n_{z}^{2}\sigma_{z}^{2}\\
&= (n_{x}^{2} + n_{y}^{2} + n_{z}^{2})I = I
\end{align*}
\end{s}

\begin{p}
Problem 4.7
\end{p}

\begin{s}

Simple matrix operations can confirm that $XYX = -Y$. It follows that

\begin{equation*}
e^{\frac{i\theta}{2}(XYX)} = e^{-\frac{i\theta}{2}Y}
\end{equation*}

Now recall that $U^{\dagger}e^{A}U = e^{U^{\dagger}A U}$, which can be proven via a series expansion. Of course $X$ is both unitary and hermitian, so we get that 

\begin{equation*}
X e^{\frac{i\theta}{2}Y} X = e^{-\frac{i\theta}{2}Y}
\end{equation*}

which is the desired result.

\end{s}

\begin{p}
Problem 4.16
\end{p}

\begin{s}
For the first circuit, the matrix representation is
\begin{equation*}
A = \begin{pmatrix}1 & 0 & 0 & 0\\
0 & 1 & 0 & 0\\
0 & 0 & h_{11} & h_{12}\\
0 & 0 & h_{21} & h_{22}
\end{pmatrix}
\end{equation*}
For the second circuit, the matrix representation is
\begin{equation*}
A = \begin{pmatrix} h_{11} & h_{12} & 0 & 0\\
h_{21} & h_{22} & 0 & 0\\
0 & 0 & 1 & 0\\
0 & 0 & 0 & 1
\end{pmatrix}
\end{equation*}
\end{s}

\begin{p}
Problem 4.17
\end{p}

\begin{s}

\end{s}


\begin{p}
Problem 4.19
\end{p}

\begin{s}

The density matrix is

\begin{equation*}
\rho = \sum_{i}p_{i}\ket{\psi_{i}}\bra{\psi_{i}}
\end{equation*}

We can see how the gate transforms the density matrix by just considering how it acts on the most general state $\ket{\psi_{i}} = \alpha_{i}\ket{00} + \beta_{i}\ket{01} + \gamma_{i}\ket{10} + \delta_{i}\ket{11}$.

\begin{equation*}
\mathrm{CNOT} = \ket{00}\bra{00} + \ket{01}\bra{01} + \ket{11}\bra{10} + \ket{10}\bra{11}
\end{equation*}

It is straightforward to see that

\begin{align*}
\mathrm{CNOT}\ket{\psi_{i}}\bra{\psi_{i}} &= \left(\alpha_{i}\ket{00} + \beta_{i}\ket{01} + \gamma_{i}\ket{11} + \delta_{i}\ket{10}\right)\\
&* \left(\alpha_{i}^{*}\bra{00} + \beta_{i}^{*}\bra{01} + \gamma_{i}^{*}\bra{10} + \delta_{i}^{*}\bra{11}\right)\\
&=
\begin{pmatrix}
|\alpha_{i}|^{2} & \alpha_{i}\beta_{i}^{*} & \alpha_{i}\gamma_{i}^{*} & \alpha_{i}\delta_{i}^{*} \\
\beta_{i}\alpha_{i}^{*} & |\beta_{i}|^{2} & \beta_{i}\gamma_{i}^{*} & \beta_{i}\delta_{i}^{*} \\
\delta_{i}\alpha_{i}^{*} & \delta_{i}\beta_{i}^{*} & \delta_{i}\gamma_{i}^{*} & |\delta_{i}|^{2} \\
\gamma_{i}\alpha_{i}^{*} & \gamma_{i}\beta_{i}^{*} & |\gamma_{i}|^{2} & \gamma_{i}\delta_{i}^{*} 
\end{pmatrix}
\end{align*}



\end{s}

\begin{p}
Problem 4.20
\end{p}

\begin{s}
Writing the circuit identity out in symbols, 

\begin{equation*}
H^{\otimes 2}*\mathrm{CNOT}*H^{\otimes 2} \left(\ket{\psi}\otimes \ket{\phi}\right) = \mathrm{CNOT}\left(\ket{\psi}\otimes \ket{\phi}\right)
\end{equation*}

where $\ket{\psi} = \alpha\ket{0} + \beta\ket{1}$ and $\ket{\phi} = \gamma\ket{0} + \delta\ket{1}$. We can start with the left hand side. After the first Hadamard gate, the qubits are

\begin{equation*}
\ket{\psi'} = \frac{\alpha+\beta}{\sqrt{2}}\ket{0} + \frac{\alpha-\beta}{\sqrt{2}}\ket{1}
\end{equation*}

\begin{equation*}
\ket{\phi'} = \frac{\gamma+\delta}{\sqrt{2}}\ket{0} + \frac{\gamma-\delta}{\sqrt{2}}\ket{1}
\end{equation*}

Now, we write

\begin{align*}
\ket{\psi'}\otimes\ket{\phi'} &= \frac{(\alpha+\beta)(\gamma+\delta)}{2}\ket{00} + \frac{(\alpha+\beta)(\gamma-\delta)}{2}\ket{01} \\
&+ \frac{(\alpha-\beta)(\gamma+\delta)}{2}\ket{10} + \frac{(\alpha-\beta)(\gamma-\delta)}{2}\ket{11}
\end{align*}

\begin{align*}
\mathrm{CNOT}(\ket{\psi'}\otimes\ket{\phi'}) &= \left(\ket{00}\bra{00} + \ket{01}\bra{01} + \ket{11}\bra{10} + \ket{10}\bra{11}\right)(\ket{\psi'}\otimes\ket{\phi'})\\
&= \frac{(\alpha+\beta)(\gamma+\delta)}{2}\ket{00} + \frac{(\alpha+\beta)(\gamma-\delta)}{2}\ket{01} \\
&+ \frac{(\alpha-\beta)(\gamma+\delta)}{2}\ket{11} + \frac{(\alpha-\beta)(\gamma-\delta)}{2}\ket{10}
\end{align*}

For the right hand side 

\begin{align*}
\mathrm{CNOT}\left(\ket{\psi}\otimes \ket{\phi}\right) &= \mathrm{CNOT}\left(\alpha\gamma\ket{00} + \alpha\delta\ket{01} + \beta\gamma\ket{10} + \beta\delta\ket{11}\right)\\
&= \alpha\gamma\ket{00} + \alpha\delta\ket{01} + \beta\gamma\ket{11} + \beta\delta\ket{10}
\end{align*}

Using the circuit identity, when the first qubit is control and the second qubit is target, we see that

\begin{align*}
\mathrm{CNOT}\ket{++}  &= H^{\otimes 2}*\mathrm{CNOT}*H^{\otimes 2}\ket{++}\\  
&=  H^{\otimes 2}*\mathrm{CNOT}\ket{00}\\
&= H^{\otimes 2}\ket{00}\\
&= \ket{++}
\end{align*}

\begin{align*}
\mathrm{CNOT}\ket{+-}  &= H^{\otimes 2}*\mathrm{CNOT}*H^{\otimes 2}\ket{+-}\\  
&=  H^{\otimes 2}*\mathrm{CNOT}\ket{01}\\
&= H^{\otimes 2}\ket{01}\\
&= \ket{+-}
\end{align*}

\begin{align*}
\mathrm{CNOT}\ket{-+}  &= H^{\otimes 2}*\mathrm{CNOT}*H^{\otimes 2}\ket{-+}\\  
&=  H^{\otimes 2}*\mathrm{CNOT}\ket{10}\\
&= H^{\otimes 2}\ket{10}\\
&= \ket{++}
\end{align*}

\begin{align*}
\mathrm{CNOT}\ket{--}  &= H^{\otimes 2}*\mathrm{CNOT}*H^{\otimes 2}\ket{--}\\  
&=  H^{\otimes 2}*\mathrm{CNOT}\ket{11}\\
&= H^{\otimes 2}\ket{10}\\
&= \ket{+-}
\end{align*}

So in this case, we can see that target qubit is unchanged while the control qubit is flipped in the last two cases. Therefore, what we call control and target are dependent on the basis we think of the device as operating in.

\end{s}

\begin{p}
Problem 4.24
\end{p}

\begin{s}

I first want to show how to write the CNOT gate in terms of tensor products, when the control and target qubits are adjacent and non-adjacent in significance, in a three qubit circuit. Let $\mathrm{CNOT}_{12}$ represent a gate where 1 is the control and 2 is the target. When the qubits are adjacent, it is straightforward:

\begin{align*}
\mathrm{CNOT}_{12} = \mathrm{CNOT}\otimes I_{3}
\end{align*}

However, $\mathrm{CNOT}_{13}$ must be written in a different way

\begin{align*}
\mathrm{CNOT}_{13} = \ket{0}\bra{0}\otimes I_{2}\otimes I_{3} + \ket{1}\bra{1}\otimes I_{2}\otimes X_{3}
\end{align*}


For the Tofolli gate, there are ten unique gates used. I will not write their matrix representation explicitly since they are all 8x8 matrices, but their tensor product representation in the order the are applied is

\begin{align*}
U_{1} &= T_1 \otimes S_2 \otimes I_3\\
U_{2} &= \mathrm{CNOT}_{12}\otimes I_{3}\\
U_{3} &= I_{1} \otimes T_{2}^{\dagger} \otimes I_3\\
U_{4} &= \mathrm{CNOT}_{12}\otimes H_{3}\\
U_{5} &= I_1 \otimes T_{2}^{\dagger} \otimes T_3\\
U_{6} &= \ket{0}\bra{0}\otimes I_{2}\otimes I_{3} + \ket{1}\bra{1}\otimes I_{2}\otimes X_{3}\\
U_{7} &= I_1 \otimes I_2 \otimes T_3^{\dagger}\\
U_{8} &= I_{1}\otimes \mathrm{CNOT}_{23}\\
U_{9} &= I_1 \otimes I_2 \otimes T_3^{\dagger}\\
U_{10} &= I_1 \otimes I_2 \otimes H_3
\end{align*}



\end{s}




\end{document}