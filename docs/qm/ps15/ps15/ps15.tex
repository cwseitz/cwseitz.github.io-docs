\documentclass[12pt]{article}
\usepackage{amsmath} % AMS Math Package
\usepackage{bm}
\usepackage{amsthm} % Theorem Formatting
\usepackage{amssymb}    % Math symbols such as \mathbb
\usepackage{graphicx} % Allows for eps images
\usepackage[dvips,letterpaper,margin=1in,bottom=0.7in]{geometry}
\usepackage{tensor}
\usepackage{amsmath}
\usepackage{siunitx}
\usepackage{physics}
\usepackage{amsmath, amssymb, graphics, setspace}

\newcommand{\mathsym}[1]{{}}
\newcommand{\unicode}[1]{{}}

\newcounter{mathematicapage}

\newtheorem{p}{Problem}
\usepackage{cancel}
\newtheorem*{lem}{Lemma}
\theoremstyle{definition}
\newtheorem*{dfn}{Definition}
 \newenvironment{s}{%\small%
        \begin{trivlist} \item \textbf{Solution}. }{%
            \hspace*{\fill} $\blacksquare$\end{trivlist}}%


\begin{document}

 {\noindent\Huge\bf  \\[0.5\baselineskip] {\fontfamily{cmr}\selectfont  Homework 5}         }\\[2\baselineskip] % Title
{ {\bf \fontfamily{cmr}\selectfont Quantum Mechanics}\\ {\textit{\fontfamily{cmr}\selectfont     \today}}}~~~~~~~~~~~~~~~~~~~~~~~~~~~~~~~~~~~~~~~~~~~~~~~~~~~~~~~~~~~~~~~~~~~~~~~~~~~~~    {\large \textsc{C Seitz}
\\[1.4\baselineskip] 


\begin{p}
Problem 4.4
\end{p}

\begin{s}
\begin{align*}
H &= e^{i\alpha}R_{z}(\frac{\pi}{2})R_{x}(\frac{\pi}{2})\\
&= \frac{e^{i\alpha}e^{i\pi/4}}{\sqrt{2}}
\begin{pmatrix}1 & 1 \\ 1 & -1 \end{pmatrix}
\end{align*}
Therefore, $\alpha = -\pi/4$.
\end{s}

\begin{p}
Problem 4.5
\end{p}

\begin{s}
\begin{align*}
(n\cdot\sigma)^{2} &= (n_{x}\sigma_{x} + n_{y}\sigma_{y} + n_{z}\sigma_{z})^{2}\\
&= n_{x}^{2}\sigma_{x}^{2} + n_{y}^{2}\sigma_{y}^{2} + n_{z}^{2}\sigma_{z}^{2}\\
&= (n_{x}^{2} + n_{y}^{2} + n_{z}^{2})I = I
\end{align*}
\end{s}

\begin{p}
Problem 4.7
\end{p}

\begin{s}

Simple matrix operations can confirm that $XYX = -Y$. It follows that

\begin{equation*}
e^{\frac{i\theta}{2}(XYX)} = e^{-\frac{i\theta}{2}Y}
\end{equation*}

Now recall that $U^{\dagger}e^{A}U = e^{U^{\dagger}A U}$, which can be proven via a series expansion. Of course $X$ is both unitary and hermitian, so we get that 

\begin{equation*}
X e^{\frac{i\theta}{2}Y} X = e^{-\frac{i\theta}{2}Y}
\end{equation*}

which is the desired result.

\end{s}

\begin{p}
Problem 4.16
\end{p}

\begin{s}
For the first circuit, the matrix representation is
\begin{equation*}
A = \begin{pmatrix}1 & 0 & 0 & 0\\
0 & 1 & 0 & 0\\
0 & 0 & h_{11} & h_{12}\\
0 & 0 & h_{21} & h_{22}
\end{pmatrix}
\end{equation*}
For the second circuit, the matrix representation is
\begin{equation*}
A = \begin{pmatrix} h_{11} & h_{12} & 0 & 0\\
h_{21} & h_{22} & 0 & 0\\
0 & 0 & 1 & 0\\
0 & 0 & 0 & 1
\end{pmatrix}
\end{equation*}
\end{s}

\begin{p}
Problem 4.17
\end{p}

\begin{s}

\end{s}

\begin{p}
Problem 4.18
\end{p}

\begin{s}
The controlled-Z gate acts in the following way on the basis kets

\begin{align*}
CZ\ket{00} &= \ket{00}\\
CZ\ket{01} &= \ket{01}\\
CZ\ket{10} &= \ket{10}\\
CZ\ket{11} &= -\ket{11}
\end{align*}

For $\ket{01}$ and $\ket{10}$, the result is the same, so the controlled-Z operator shown is the same at least with respect to those states. Then we see that the result on $\ket{11}$ is just a global phase, so we can safely conclude the same operator works, regardless of which qubit is the control. 

\end{s}


\begin{p}
Problem 4.19
\end{p}

\begin{s}

The density matrix is

\begin{equation*}
\rho = \sum_{i}p_{i}\ket{\psi_{i}}\bra{\psi_{i}}
\end{equation*}

We can see how the gate transforms the density matrix by just considering how it acts on the most general state $\ket{\psi_{i}} = \alpha_{i}\ket{00} + \beta_{i}\ket{01} + \gamma_{i}\ket{10} + \delta_{i}\ket{11}$.

\begin{equation*}
\mathrm{CNOT} = \ket{00}\bra{00} + \ket{01}\bra{01} + \ket{11}\bra{10} + \ket{10}\bra{11}
\end{equation*}

It is straightforward to see that

\begin{align*}
\mathrm{CNOT}\ket{\psi_{i}}\bra{\psi_{i}} &= \left(\alpha_{i}\ket{00} + \beta_{i}\ket{01} + \gamma_{i}\ket{11} + \delta_{i}\ket{10}\right)\\
&* \left(\alpha_{i}^{*}\bra{00} + \beta_{i}^{*}\bra{01} + \gamma_{i}^{*}\bra{10} + \delta_{i}^{*}\bra{11}\right)\\
&=
\begin{pmatrix}
|\alpha_{i}|^{2} & \alpha_{i}\beta_{i}^{*} & \alpha_{i}\gamma_{i}^{*} & \alpha_{i}\delta_{i}^{*} \\
\beta_{i}\alpha_{i}^{*} & |\beta_{i}|^{2} & \beta_{i}\gamma_{i}^{*} & \beta_{i}\delta_{i}^{*} \\
\delta_{i}\alpha_{i}^{*} & \delta_{i}\beta_{i}^{*} & \delta_{i}\gamma_{i}^{*} & |\delta_{i}|^{2} \\
\gamma_{i}\alpha_{i}^{*} & \gamma_{i}\beta_{i}^{*} & |\gamma_{i}|^{2} & \gamma_{i}\delta_{i}^{*} 
\end{pmatrix}
\end{align*}



\end{s}



\end{document}