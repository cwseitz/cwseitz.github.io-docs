\documentclass[12pt]{article}
\usepackage{amsmath} % AMS Math Package
\usepackage{bm}
\usepackage{amsthm} % Theorem Formatting
\usepackage{amssymb}    % Math symbols such as \mathbb
\usepackage{graphicx} % Allows for eps images
\usepackage[dvips,letterpaper,margin=1in,bottom=0.7in]{geometry}
\usepackage{tensor}
\usepackage{amsmath}
\usepackage{siunitx}
\usepackage{physics}
\usepackage{amsmath, amssymb, graphics, setspace}

\newcommand{\mathsym}[1]{{}}
\newcommand{\unicode}[1]{{}}

\newcounter{mathematicapage}

\newtheorem{p}{Problem}
\usepackage{cancel}
\newtheorem*{lem}{Lemma}
\theoremstyle{definition}
\newtheorem*{dfn}{Definition}
 \newenvironment{s}{%\small%
        \begin{trivlist} \item \textbf{Solution}. }{%
            \hspace*{\fill} $\blacksquare$\end{trivlist}}%


\begin{document}

 {\noindent\Huge\bf  \\[0.5\baselineskip] {\fontfamily{cmr}\selectfont  Homework 5}         }\\[2\baselineskip] % Title
{ {\bf \fontfamily{cmr}\selectfont Quantum Mechanics}\\ {\textit{\fontfamily{cmr}\selectfont     October 17th, 2022}}}~~~~~~~~~~~~~~~~~~~~~~~~~~~~~~~~~~~~~~~~~~~~~~~~~~~~~~~~~~~~~~~~~~~~~~~~~~~~~    {\large \textsc{C Seitz}
\\[1.4\baselineskip] 

\begin{p}
Problem 3.10 from Sakurai
\end{p}

\begin{s}

\begin{align*}
\exp(i(\bm{\sigma}\cdot\bm{\hat{n}})\theta) &=
\begin{pmatrix}
\cos\theta + in_{z}\sin\theta & (-in_{x}+n_{y})\sin\theta\\
(in_{x}-n_{y})\sin\theta & \cos\theta - in_{z}\sin\theta 
\end{pmatrix}\\
 &= \begin{pmatrix}
e^{-(i\alpha+\gamma)/2}\cos\frac{\beta}{2} & -e^{-(i\alpha-\gamma)/2}\sin\frac{\beta}{2} \\
e^{-(i\alpha-\gamma)/2}\sin\frac{\beta}{2} & e^{(i\alpha+\gamma)/2}\cos\frac{\beta}{2}\end{pmatrix}
\end{align*}

\end{s}

Equating the trace of these matrices gives

\begin{align*}
2\cos\theta = 2\cos(\frac{\alpha+\gamma}{2})\cos\frac{\beta}{2}
\end{align*}

So $\theta = \cos^{-1}(\cos(\frac{\alpha+\gamma}{2})\cos\frac{\beta}{2})$


\begin{p}

Problem 3.20 from Sakurai
\end{p}

\begin{s}
 
Recall that 

\begin{align*}
J_{\pm} = J_{x} \pm iJ_{y}
\end{align*}

and thus $J_{x} = (J_{+} + J_{-})/2$ and $J_{y} = \frac{J_{+}-J_{-}}{2i}$. We know that the matrix elements of $J_{\pm}$ are

\begin{align*}
\bra{j',m'}J_{\pm}\ket{j,m} = \sqrt{(j\mp m)(j\pm m + 1)}\hbar\delta_{jj'}\delta_{m,m'+1}
\end{align*}

where $j$ is our usual shorthand for $\hbar^{2}j(j+1)$ (the eigenvalue of $J^{2}$) and $m$ is short for $m\hbar$ (the eigenvalue of $J_{z}$). For a spin-1 system, $j=1$ and $m = -1, 0, 1$ which gives the eigenkets $\ket{1,-1},\ket{1,0},\ket{1,-1}$

\begin{equation*}
J_{+} =
\begin{pmatrix}
0&\sqrt{2}\hbar&0\\
0&0&\sqrt{2}\hbar\\
0&0&0
\end{pmatrix}\\ \;\;\;
J_{-} =
\begin{pmatrix}
0&0&0\\
\sqrt{2}\hbar&0&0\\
0&\sqrt{2}\hbar&0
\end{pmatrix}\\ 
\end{equation*}


\begin{equation*}
J_{x} =
\frac{\hbar}{\sqrt{2}}\begin{pmatrix}
0&1&0\\
1&0&1\\
0&1&0
\end{pmatrix}\\ \;\;\;
J_{y} =
\frac{\hbar}{\sqrt{2}i}\begin{pmatrix}
0&1&0\\
-1&0&1\\
0&-1&0
\end{pmatrix}\\ 
\end{equation*}

We can use Mathematica to find the eigenvectors of these two matrices

\begin{equation*}
\ket{J_{x}; +} =
\begin{pmatrix}
1/2\\
1/\sqrt{2}\\
1/2
\end{pmatrix}\\ \;\;\;
\ket{J_{x}; 0} =
\begin{pmatrix}
-1/\sqrt{2}\\
0\\
1/\sqrt{2}
\end{pmatrix}\\ \;\;\;
\ket{J_{x}; -1} =
\begin{pmatrix}
1/2\\
-1/\sqrt{2}\\
1/2
\end{pmatrix}\\ 
\end{equation*}

\begin{equation*}
\ket{J_{y}; +} =
\begin{pmatrix}
-1/2\\
-i\sqrt{2}\\
1/2
\end{pmatrix}\\ \;\;\;
\ket{J_{y}; 0} =
\begin{pmatrix}
1/\sqrt{2}\\
0\\
1/\sqrt{2}
\end{pmatrix}\\ \;\;\;
\ket{J_{y}; -1} =
\begin{pmatrix}
-1/2\\
i\sqrt{2}\\
1/2
\end{pmatrix}\\ 
\end{equation*}



\end{s}

\begin{p}
Problem 3.22 from Sakurai
\end{p}

\begin{s}

We are asked to derive

\begin{align*}
\bra{x}L_{z}\ket{\alpha} = -i\hbar\frac{\partial}{\partial\phi}\bra{x}\ket{\alpha}
\end{align*}

\begin{align*}
\bra{x}L_{z}\ket{\alpha} &= \bra{x}(xp_{y} - yp_{x})\ket{\alpha}\\
&= i\hbar\bra{x}y\frac{\partial}{\partial x}-x\frac{\partial}{\partial y}\ket{\alpha}
\end{align*}

We need to write these partial derivatives in spherical coordinates to complete the proof. I just looked up the coordinate transformation and made the substitution

\begin{align*}
\bra{x}L_{z}\ket{\alpha} &= i\hbar\bra{x}y\frac{\partial}{\partial x}-x\frac{\partial}{\partial y}\ket{\alpha}\\
&= i\hbar \left(y\frac{\partial}{\partial x}\bra{x}\ket{\alpha}-x\frac{\partial}{\partial y}\bra{x}\ket{\alpha}\right)\\
&= ir\hbar \sin\phi\sin\theta\left(\sin\theta\cos\phi\frac{\partial}{\partial r} + \frac{1}{r}\cos\theta\cos\phi\frac{\partial}{\partial \theta} - \frac{\sin\phi}{r\sin\theta}\frac{\partial}{\partial \phi}\right)\\
&- ir\hbar \cos\phi\sin\theta\left(\sin\theta\sin\phi\frac{\partial}{\partial r} + \frac{1}{r}\cos\theta\sin\phi\frac{\partial}{\partial \theta} + \frac{\cos\phi}{r\sin\theta}\frac{\partial}{\partial \phi}\right)\\
&= ir\hbar \sin\phi\sin\theta\left(- \frac{\sin\phi}{r\sin\theta}\frac{\partial}{\partial \theta}\right)- ir\hbar \cos\phi\sin\theta\left(\frac{\cos\phi}{r\sin\theta}\frac{\partial}{\partial \phi}\right)\\
&= i\hbar \left(-\sin^{2}\phi\frac{\partial}{\partial \phi}\right)- i\hbar \left(\cos^{2}\phi\frac{\partial}{\partial \phi}\right)\\
&= -i\hbar\frac{\partial}{\partial \phi}
\end{align*}

\end{s}

\begin{p}
Problem 3.23 from Sakurai
\end{p}

\begin{s}

We can write the wavefunction given in spherical coordinates

\begin{align*}
\psi(\bm{x}) = \bra{x}\ket{\alpha} = r\left(\cos\phi \sin\theta + \sin\phi\sin\theta + \cos\theta\right)f(r)
\end{align*}

If this is an eigenfunction of $L^{2}$, then we should be able to write it in terms of the spherical harmonics $Y_{l}^{m}(\theta,\phi)$. We can, and it is

\begin{align*}
\psi(\bm{x}) = \bra{x}\ket{\alpha} = \sqrt{\frac{8\pi}{3}}\left(\frac{Y_{1}^{-1}+Y_{1}^{1}}{2}+\frac{Y_{1}^{-1}-Y_{1}^{1}}{2i} + \frac{3}{\sqrt{2}}Y_{1}^{0}\right)rf(r)
\end{align*}

So it must be an eigenfunction of $L^{2}$. Recall that the spherical harmonics form an orthonormal basis, so 

\begin{align*}
\bra{l, m}\ket{l, m'} = \int (Y_{l}^{m})^{*}Y_{l}^{m'}d\bm{x} = \delta_{m,m'}
\end{align*}

The probability amplitudes are then just

\begin{align*}
\bra{1,-1}\ket{\alpha} &= \int (Y_{1}^{-1})^{*}\psi(\bm{x})d\bm{x} = \sqrt{\frac{8\pi}{3}}\left(\frac{1}{2}+\frac{1}{2i}\right)rf(r)\\
\bra{1,0}\ket{\alpha} &= \int (Y_{1}^{0})^{*}\psi(\bm{x})d\bm{x} = \sqrt{\frac{8\pi}{3}}\frac{3}{\sqrt{2}}rf(r)\\
\bra{1,1}\ket{\alpha} &= \int (Y_{1}^{1})^{*}\psi(\bm{x})d\bm{x} = \sqrt{\frac{8\pi}{3}}\left(\frac{1}{2}-\frac{1}{2i}\right)rf(r)\\
\end{align*}


\end{s}

\begin{p}
Problem 3.24 from Sakurai
\end{p}

\begin{s}
\begin{align*}
\bra{l,m}L_{x}\ket{l,m} &= \frac{1}{2}\bra{l,m}\left(L_{+}+L_{-}\right)\ket{l,m} = 0\\
\bra{l,m}L_{y}\ket{l,m} &= \frac{1}{2i}\bra{l,m}\left(L_{+}-L_{-}\right)\ket{l,m} = 0
\end{align*}
\begin{align*}
\bra{l,m}L_{x}^{2}\ket{l,m} &= \bra{l,m}L_{y}^{2}\ket{l,m}\\
&= \frac{1}{4}\bra{l,m}\left(L_{+}^{2}+L_{+}L_{-}+L_{-}L_{+}+L_{-}^{2}\right)\ket{l,m}\\
&= \frac{1}{4}\bra{l,m}\left(L_{+}L_{-}+L_{-}L_{+}\right)\ket{l,m}\\
&= \frac{1}{4}\left(\hbar^{2}l(l+1)-m^{2}\hbar^{2}\right) + \frac{1}{4}\left(\hbar^{2}l(l+1)-m^{2}\hbar^{2}\right) \\
&= \frac{1}{2}\left(\hbar^{2}l(l+1)-m^{2}\hbar^{2}\right)\\
\end{align*}


\end{s}
\begin{p}
Problem 3.38 from Sakurai
\end{p}

\begin{s}

We are asked to write $J_{y}$ when $j=1$. This was already done in Problem 2 above. The result was

\begin{equation*}
J_{y} =
\frac{\hbar}{\sqrt{2}i}\begin{pmatrix}
0&1&0\\
-1&0&1\\
0&-1&0
\end{pmatrix}\\ 
\end{equation*}

The problem suggests that we should think about the matrix exponential $\exp(-iJ_{y}\beta/\hbar)$, which is of course

\begin{align*}
\exp(-iJ_{y}\beta/\hbar) = 1 - \left(iJ_{y}\beta/\hbar\right) + \left(iJ_{y}\beta/\hbar\right)^{2} - \left(iJ_{y}\beta/\hbar\right)^{3} + ...
\end{align*}

Now notice that

\begin{equation*}
J_{y}^{2} =
\frac{-\hbar^{2}}{2}\begin{pmatrix}
0&1&0\\
-1&0&1\\
0&-1&0
\end{pmatrix}\begin{pmatrix}
0&1&0\\
-1&0&1\\
0&-1&0
\end{pmatrix}\\ = \frac{-\hbar^{2}}{2}\begin{pmatrix}
-1&0&1\\
0&-2&0\\
1&0&-1
\end{pmatrix}\\
\end{equation*}

and if we multiply this by $J_{y}$ we see that, when $j=1$, we have the property that $J_{y}^{3} = -\hbar^{2}J_{y}$. Therefore

\begin{align*}
\exp(-iJ_{y}\beta/\hbar) &= 1 - \left(iJ_{y}\beta/\hbar\right) + \left(iJ_{y}\beta/\hbar\right)^{2} - \left(iJ_{y}\beta/\hbar\right)^{3} + \left(iJ_{y}\beta/\hbar\right)^{4}+ ...\\
&= 1 - \left(iJ_{y}\beta/\hbar\right) + J_{y}^{2}\left(i\beta/\hbar\right)^{2}/2! - J_{y}^{3}\left(i\beta/\hbar\right)^{3}/3! + J_{y}^{4}\left(i\beta/\hbar\right)^{4}/4! + ...\\
&= 1 - \left(iJ_{y}\beta/\hbar\right) + J_{y}^{2}\left(i\beta/\hbar\right)^{2}/2! + \hbar^{2} J_{y}\left(i\beta/\hbar\right)^{3}/3! - \hbar^{2}J_{y}^{2}\left(i\beta/\hbar\right)^{4}/4!+ ...\\
&= 1 - i\frac{J_{y}}{\hbar}\sum_{n=0}^{\infty}(-1)^{n}\beta^{2n+1}/(2n+1)! - \frac{J_{y}^{2}}{\hbar^{2}}\sum_{m=1}^{\infty}(-1)^{m}\beta^{2m}/(2m)!\\
&=  1 - i\frac{J_{y}}{\hbar}\sin\beta + \frac{J_{y}^{2}}{\hbar^{2}}(\cos\beta-1)
\end{align*}

For the third part we just see that 

\begin{align*}
\exp(-iJ_{y}\beta/\hbar) &= 1 - i\frac{J_{y}}{\hbar}\sin\beta - \frac{J_{y}^{2}}{\hbar^{2}}(1-\cos\beta)\\
&= 1 - \frac{\sin\beta}{\sqrt{2}}\begin{pmatrix}
0&1&0\\
-1&0&1\\
0&-1&0
\end{pmatrix} + \frac{1-\cos\beta}{2}\begin{pmatrix}
-1&0&1\\
0&-2&0\\
1&0&-1
\end{pmatrix}\\
&= \begin{pmatrix}
\frac{1+\cos\beta}{2}&\frac{\sin\beta}{\sqrt{2}}&\frac{1-\cos\beta}{2}\\
-\frac{\sin\beta}{\sqrt{2}}&\cos\beta&\frac{\sin\beta}{\sqrt{2}}\\
\frac{1-\cos\beta}{2}&-\frac{\sin\beta}{\sqrt{2}}&\frac{1+\cos\beta}{2}
\end{pmatrix}
\end{align*}


\end{s}

\end{document}