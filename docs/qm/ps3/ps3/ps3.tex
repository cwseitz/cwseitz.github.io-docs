\documentclass[12pt]{article}
\usepackage{amsmath} % AMS Math Package
\usepackage{bm}
\usepackage{amsthm} % Theorem Formatting
\usepackage{amssymb}    % Math symbols such as \mathbb
\usepackage{graphicx} % Allows for eps images
\usepackage[dvips,letterpaper,margin=1in,bottom=0.7in]{geometry}
\usepackage{tensor}
\usepackage{amsmath}
\usepackage{siunitx}
\usepackage{physics}

\newtheorem{p}{Problem}
\usepackage{cancel}
\newtheorem*{lem}{Lemma}
\theoremstyle{definition}
\newtheorem*{dfn}{Definition}
 \newenvironment{s}{%\small%
        \begin{trivlist} \item \textbf{Solution}. }{%
            \hspace*{\fill} $\blacksquare$\end{trivlist}}%


\begin{document}

 {\noindent\Huge\bf  \\[0.5\baselineskip] {\fontfamily{cmr}\selectfont  Homework 3}         }\\[2\baselineskip] % Title
{ {\bf \fontfamily{cmr}\selectfont Quantum Mechanics}\\ {\textit{\fontfamily{cmr}\selectfont     Sept 15th, 2022}}}~~~~~~~~~~~~~~~~~~~~~~~~~~~~~~~~~~~~~~~~~~~~~~~~~~~~~~~~~~~~~~~~~~~~~~~~~~~~~    {\large \textsc{Clayton Seitz}
\\[1.4\baselineskip] 

\begin{p}
Problem 2.1 from Sakurai
\end{p}

\begin{s}
The Heisenberg equation of motion reads

\begin{align*}
\frac{dA}{dt} = \frac{1}{i\hbar}\left[A,H\right]
\end{align*}

For the spin precession problem, we have the Hamiltonian

\begin{align*}
H = -\left(\frac{eB}{mc}\right)S_{z} = \omega S_{z}
\end{align*}

For $A = S_{x},S_{y},S_{z}$, the time evolution is given by

\begin{align*}
\frac{dS_{x}}{dt} &= \frac{\omega}{i\hbar}\left[S_{x},S_{z}\right] = -\omega S_{y}\\
\frac{dS_{y}}{dt} &= \frac{\omega}{i\hbar}\left[S_{y},S_{z}\right] = \omega S_{x}\\
\frac{dS_{z}}{dt} &= \frac{\omega}{i\hbar}\left[S_{z},S_{z}\right] = 0
\end{align*}

The above system has a straightforward solution:

\begin{align*}
S_{x}(t) &= \cos(\omega t)\\
S_{y}(t) &= \sin(\omega t)\\
S_{z}(t) &= S_{z}(0)
\end{align*}


\end{s}

\begin{p}
Problem 2.3 from Sakurai
\end{p}

\begin{s}
We are given that $\vec{B} = B\hat{z}$ and that we are in the eigenstate $\ket{\psi(0)} = \ket{\bm{S}\cdot \bm{\hat{n}}}_{+}$, which reads

\begin{align*}
\ket{\psi(0)} &= \psi_{+}\ket{+} + \psi_{-}\ket{-}\\
&= \cos{\frac{\beta}{2}}\ket{+} + \sin{\frac{\beta}{2}}\ket{-}\\
\end{align*}

where we have set $\alpha=0$ since the ket is in the x-z plane. This state will evolve according to a Hamiltonian

\begin{align*}
H = -\left(\frac{eB}{m_{e}c}\right)S_{z} = \omega S_{z}
\end{align*}


Clearly the eigenkets of the Hamiltonian are the eigenkets of $S_{z}$

\begin{align*}
\ket{\psi(t)} &= \psi_{+}(0)\exp\left(\frac{-iE_{+}t}{\hbar}\right)\ket{+} + \psi_{-}(0)\exp\left(\frac{-iE_{-}t}{\hbar}\right)\ket{-}\\
&= \cos{\frac{\beta}{2}}\exp\left(\frac{-i\omega t}{2}\right)\ket{+} + \sin{\frac{\beta}{2}}\exp\left(\frac{i\omega t}{2}\right)\ket{-}
\end{align*}

where we have used $E_{+}$ and $E_{-}$ to denote the energies in the eigenstates of $S_{z}$. In general, the probability of measuring $\ket{+}_{x} = \frac{1}{\sqrt{2}}\ket{+} + \frac{1}{\sqrt{2}}\ket{-}$ is given by the inner product

\begin{align*}
|\bra{+;x}\ket{\psi(t)}|^{2} = 
\end{align*}





The Hamiltonian is time-independent, therefore, in the Schrodinger picture, we can write the following

\begin{align*}
\end{align*}

\end{s}

\begin{p}
Problem 2.9 from Sakurai
\end{p}

\begin{s}

\end{s}

\begin{p}
Problem 2.10 from Sakurai
\end{p}

\begin{s}

\end{s}

\begin{p}
Problem 2.12 from Sakurai
\end{p}

\begin{s}

\end{s}

\begin{p}
Problem 2.13 from Sakurai
\end{p}

\begin{s}

\end{s}

\end{document}