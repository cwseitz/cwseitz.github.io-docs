% Latex template: mahmoud.s.fahmy@students.kasralainy.edu.eg
% For more details: https://www.sharelatex.com/learn/Beamer

\documentclass[aspectratio=1610]{beamer}					% Document class

\setbeamertemplate{footline}[text line]{%
  \parbox{\linewidth}{\vspace*{-8pt}Bell's Inequality \hfill\insertshortauthor\hfill\insertpagenumber}}
\setbeamertemplate{navigation symbols}{}

\usepackage[english]{babel}				% Set language
\usepackage[utf8x]{inputenc}			% Set encoding

\mode<presentation>						% Set options
{
  \usetheme{default}					% Set theme
  \usecolortheme{default} 				% Set colors
  \usefonttheme{default}  				% Set font theme
  \setbeamertemplate{caption}[numbered]	% Set caption to be numbered
}

% Uncomment this to have the outline at the beginning of each section highlighted.
%\AtBeginSection[]
%{
%  \begin{frame}{Outline}
%    \tableofcontents[currentsection]
%  \end{frame}
\usepackage{graphicx}					% For including figures
\usepackage{booktabs}					% For table rules
\usepackage{hyperref}	
\usepackage{tikz-network}				% For cross-referencing
\usepackage[absolute,overlay]{textpos}
\usepackage{bm}
\usepackage[font=small,labelfont=bf]{caption}				% For cross-referencing
\usepackage{physics}

\title{Bell's Inequality}	% Presentation title
\author{Clayton W. Seitz}								% Presentation author
\date{\today}									% Today's date	

\begin{document}

% Title page
% This page includes the informations defined earlier including title, author/s, affiliation/s and the date
\begin{frame}
  \titlepage
\end{frame}

\begin{frame}{Bell's inequality (classical)}

\end{frame}

\begin{frame}{Tsirelson's inequality (quantum)}

\end{frame}

\begin{frame}{Expectation of QS and RS}

\begin{align*}
\langle QS \rangle &= \bra{\psi}QS\ket{\psi}\\
&= -\frac{1}{2^{3/2}}\left(\bra{01} - \bra{10}\right) \left(Z_{1}Z_{2}+Z_{1}X_{2}\right)\left(\ket{01} - \ket{10}\right)\\
&= -\frac{1}{2^{3/2}}\left(\bra{01}Z_{1}Z_{2}\ket{01} + \bra{10}Z_{1}Z_{2}\ket{10}\right)\\
&= \frac{2}{2^{3/2}} = \frac{1}{\sqrt{2}}
\end{align*}

\begin{align*}
\langle RS \rangle &= \bra{\psi}RS\ket{\psi}\\
&= -\frac{1}{2^{3/2}}\left(\bra{01} - \bra{10}\right) \left(X_{1}Z_{2}+X_{1}X_{2}\right)\left(\ket{01} - \ket{10}\right)\\
&= \frac{1}{2^{3/2}}\left(\bra{01}X_{1}X_{2}\ket{10} + \bra{10}X_{1}X_{2}\ket{01}\right)\\
&= \frac{2}{2^{3/2}} = \frac{1}{\sqrt{2}}
\end{align*}

\end{frame}

\begin{frame}{Expectation of QT and RT}

\begin{align*}
\langle QT \rangle &= \bra{\psi}QT\ket{\psi}\\
&= \frac{1}{2^{3/2}}\left(\bra{01} - \bra{10}\right) \left(Z_{1}Z_{2}-Z_{1}X_{2}\right)\left(\ket{01} - \ket{10}\right)\\
&= -\frac{1}{2^{3/2}}\left(\bra{01}Z_{1}Z_{2}\ket{01} + \bra{10}Z_{1}Z_{2}\ket{10}\right)\\
&= \frac{2}{2^{3/2}} = \frac{1}{\sqrt{2}}
\end{align*}

\begin{align*}
\langle RT \rangle &= \bra{\psi}RT\ket{\psi}\\
&= -\frac{1}{2^{3/2}}\left(\bra{01} - \bra{10}\right) \left(X_{1}Z_{2}+X_{1}X_{2}\right)\left(\ket{01} - \ket{10}\right)\\
&= \frac{1}{2^{3/2}}\left(\bra{01}X_{1}X_{2}\ket{10} + \bra{10}X_{1}X_{2}\ket{01}\right)\\
&= \frac{2}{2^{3/2}} = \frac{1}{\sqrt{2}}
\end{align*}

\end{frame}

\begin{frame}{Why do Bell states saturate the bound?}
\begin{align*}
S(\rho) = -\mathrm{Tr}(\rho\log\rho)
\end{align*}
\end{frame}

\end{document}