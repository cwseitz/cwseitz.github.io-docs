% Latex template: mahmoud.s.fahmy@students.kasralainy.edu.eg
% For more details: https://www.sharelatex.com/learn/Beamer

\documentclass[aspectratio=1610]{beamer}					% Document class

\setbeamertemplate{footline}[text line]{%
  \parbox{\linewidth}{\vspace*{-8pt}Bell's Inequality \hfill\insertshortauthor\hfill\insertpagenumber}}
\setbeamertemplate{navigation symbols}{}

\usepackage[english]{babel}				% Set language
\usepackage[utf8x]{inputenc}			% Set encoding

\mode<presentation>						% Set options
{
  \usetheme{default}					% Set theme
  \usecolortheme{default} 				% Set colors
  \usefonttheme{default}  				% Set font theme
  \setbeamertemplate{caption}[numbered]	% Set caption to be numbered
}

% Uncomment this to have the outline at the beginning of each section highlighted.
%\AtBeginSection[]
%{
%  \begin{frame}{Outline}
%    \tableofcontents[currentsection]
%  \end{frame}
\usepackage{graphicx}					% For including figures
\usepackage{booktabs}					% For table rules
\usepackage{hyperref}	
\usepackage{tikz-network}				% For cross-referencing
\usepackage[absolute,overlay]{textpos}
\usepackage{bm}
\usepackage[font=small,labelfont=bf]{caption}				% For cross-referencing
\usepackage{physics}

\title{Bell's Inequality}	% Presentation title
\author{Clayton W. Seitz}								% Presentation author
\date{\today}									% Today's date	

\begin{document}

% Title page
% This page includes the informations defined earlier including title, author/s, affiliation/s and the date
\begin{frame}
  \titlepage
\end{frame}

\begin{frame}{}

Basically I want to relate the expansion coefficients of the two-qubit pure state to the degree of entanglement using entanglement entropy. Then, I want to demonstrate that entangled states can violate Bell's inequality (but I'm not sure if this is exactly correct or under what conditions). Finally, I want to show maximally entangled states that saturate the Tsirelson bound.

\end{frame}

\begin{frame}{CHSH Inequality}

Define 4 spin operators along arbitrary directions $Q = \vec{q}\cdot\sigma, R = \vec{r}\cdot\sigma, S = \vec{s}\cdot\sigma, T = \vec{t}\cdot\sigma$.

\vspace{0.2in}

Alice: $Q,R$
Bob: $S, T$

\vspace{0.1in}
Combination of correlations between Alice and Bobs measurements are bounded according to the CHSH inequality

\begin{align*}
E(Q\otimes S) + E(R\otimes S) + E(R\otimes T) - E(Q\otimes T) \leq 2
\end{align*}

Let $\vec{q} = (0,0,1), \vec{r} = (1,0,0), \vec{s} = (-\frac{1}{\sqrt{2}},0,-\frac{1}{\sqrt{2}}), \vec{t} = (-\frac{1}{\sqrt{2}},0,\frac{1}{\sqrt{2}})$

\end{frame}

\begin{frame}{Calculating expectations}


\begin{align*}
\vec{q}\cdot\sigma\otimes\vec{s}\cdot\sigma = 
\begin{pmatrix}
\vec{s}\cdot\sigma & 0 \\
0 & -\vec{s}\cdot\sigma
\end{pmatrix} = 
\frac{1}{\sqrt{2}}\begin{pmatrix}
-1 & -1 & 0 & 0\\
-1 & 1 & 0 & 0\\
0 & 0 & 1 & 1 \\
0 & 0 & 1 & -1
\end{pmatrix}
\end{align*}

\begin{align*}
\vec{r}\cdot\sigma\otimes\vec{s}\cdot\sigma = 
\begin{pmatrix}
0 & \vec{s}\cdot\sigma\\
\vec{s}\cdot\sigma & 0
\end{pmatrix} = 
\frac{1}{\sqrt{2}}\begin{pmatrix}
0 & 0 & -1 & -1\\
0 & 0 & -1 & 1\\
-1 & -1 & 0 & 0 \\
-1 & 1 & 0 & 0
\end{pmatrix}
\end{align*}


\begin{align*}
\vec{r}\cdot\sigma\otimes\vec{t}\cdot\sigma = 
\begin{pmatrix}
0 & \vec{t}\cdot\sigma \\
\vec{t}\cdot\sigma & 0
\end{pmatrix} = 
\frac{1}{\sqrt{2}}\begin{pmatrix}
0 & 0 & 1 & -1\\
0 & 0 & -1 & -1\\
1 & -1 & 0 & 0 \\
-1 & -1 & 0 & 0
\end{pmatrix}
\end{align*}



\end{frame}

\begin{frame}{Calculating expectations}


\begin{align*}
\vec{q}\cdot\sigma\otimes\vec{t}\cdot\sigma = 
\begin{pmatrix}
\vec{t}\cdot\sigma & 0\\
0 & -\vec{t}\cdot\sigma
\end{pmatrix} = 
\frac{1}{\sqrt{2}}\begin{pmatrix}
1 & -1 & 0 & 0\\
-1 & -1 & 0 & 0\\
0 & 0 & -1 & 1 \\
0 & 0 & 1 & 1
\end{pmatrix}
\end{align*}


\begin{align*}
\langle \vec{q}\cdot\sigma\otimes\vec{s}\cdot\sigma\rangle &= \frac{1}{\sqrt{2}}\left(-\alpha^{*}(\alpha+\beta) + \beta^{*}(\beta-\alpha) + \gamma^{*}(\gamma+\delta) + \delta^{*}(\gamma-\delta)\right)\\
\langle \vec{r}\cdot\sigma\otimes\vec{s}\cdot\sigma\rangle &= \frac{1}{\sqrt{2}}\left(-\alpha^{*}(\gamma+\delta) + \beta^{*}(\delta-\gamma) - \gamma^{*}(\alpha + \beta) + \delta^{*}(\beta-\alpha)\right)\\
\langle \vec{r}\cdot\sigma\otimes\vec{t}\cdot\sigma\rangle &= \frac{1}{\sqrt{2}}\left(\alpha^{*}(\gamma-\delta) - \beta^{*}(\delta+\gamma) + \gamma^{*}(\alpha - \beta) - \delta^{*}(\beta+\alpha)\right)\\
\langle \vec{q}\cdot\sigma\otimes\vec{t}\cdot\sigma\rangle &= \frac{1}{\sqrt{2}}\left(\alpha^{*}(\alpha-\beta) - \beta^{*}(\beta+\alpha) + \gamma^{*}(\delta-\delta) + \delta^{*}(\gamma+\delta)\right)
\end{align*}


\end{frame}

\begin{frame}{Entanglement entropy of a two-qubit system}

Assume $\rho_{AB}$ is a pure state but not necessarily separable.

\begin{align*}
\rho_{AB} &=\begin{pmatrix}
|\alpha|^{2} & \alpha\beta^{*} & \alpha\gamma^{*} & \alpha\delta^{*} \\
\beta\alpha^{*} & |\beta|^{2} & \beta\gamma^{*} & \beta\delta^{*} \\
\gamma\alpha^{*} & \gamma\beta^{*} & |\gamma|^{2} & \gamma\delta^{*} \\
\delta\alpha^{*} & \delta\beta^{*} & \delta\gamma^{*} & |\delta|^{2}
\end{pmatrix}
\end{align*}



\end{frame}

\end{document}