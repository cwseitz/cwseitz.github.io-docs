% Latex template: mahmoud.s.fahmy@students.kasralainy.edu.eg
% For more details: https://www.sharelatex.com/learn/Beamer

\documentclass[aspectratio=1610]{beamer}					% Document class

\setbeamertemplate{footline}[text line]{%
  \parbox{\linewidth}{\vspace*{-8pt}Bell's Inequality \hfill\insertshortauthor\hfill\insertpagenumber}}
\setbeamertemplate{navigation symbols}{}

\usepackage[english]{babel}				% Set language
\usepackage[utf8x]{inputenc}			% Set encoding

\mode<presentation>						% Set options
{
  \usetheme{default}					% Set theme
  \usecolortheme{default} 				% Set colors
  \usefonttheme{default}  				% Set font theme
  \setbeamertemplate{caption}[numbered]	% Set caption to be numbered
}

% Uncomment this to have the outline at the beginning of each section highlighted.
%\AtBeginSection[]
%{
%  \begin{frame}{Outline}
%    \tableofcontents[currentsection]
%  \end{frame}
\usepackage{graphicx}					% For including figures
\usepackage{booktabs}					% For table rules
\usepackage{hyperref}	
\usepackage{tikz-network}				% For cross-referencing
\usepackage[absolute,overlay]{textpos}
\usepackage{bm}
\usepackage[font=small,labelfont=bf]{caption}				% For cross-referencing
\usepackage{physics}

\title{Bell's Inequality}	% Presentation title
\author{Clayton W. Seitz}								% Presentation author
\date{\today}									% Today's date	

\begin{document}

% Title page
% This page includes the informations defined earlier including title, author/s, affiliation/s and the date
\begin{frame}
  \titlepage
\end{frame}

\begin{frame}{Bell's Inequality}

\vspace{0.2in}

Alice: $Q,R$
Bob: $S, T$

\vspace{0.1in}
Classical observables distributed according to $P(Q,R,S,T)$. Combination of correlations between Alice and Bobs measurements are bounded according to the CHSH inequality

\begin{align*}
|E(QS) + E(RS) + E(RT) - E(QT)| \leq 2
\end{align*}

For the quantum version, define 4 spin operators along arbitrary directions $Q = \vec{q}\cdot\sigma, R = \vec{r}\cdot\sigma, S = \vec{s}\cdot\sigma, T = \vec{t}\cdot\sigma$.

Let $\vec{q} = (0,0,1), \vec{r} = (1,0,0), \vec{s} = (-\frac{1}{\sqrt{2}},0,-\frac{1}{\sqrt{2}}), \vec{t} = (-\frac{1}{\sqrt{2}},0,\frac{1}{\sqrt{2}})$

\begin{align*}
|\langle Q\otimes S\rangle + \langle R\otimes S\rangle  + \langle R\otimes T\rangle  - \langle Q\otimes T\rangle|  \leq 2\sqrt{2}
\end{align*}



\end{frame}

\begin{frame}{Calculating expectations}


\begin{align*}
\vec{q}\cdot\sigma\otimes\vec{s}\cdot\sigma = 
\begin{pmatrix}
\vec{s}\cdot\sigma & 0 \\
0 & -\vec{s}\cdot\sigma
\end{pmatrix} = 
\frac{1}{\sqrt{2}}\begin{pmatrix}
-1 & -1 & 0 & 0\\
-1 & 1 & 0 & 0\\
0 & 0 & 1 & 1 \\
0 & 0 & 1 & -1
\end{pmatrix}
\end{align*}

\begin{align*}
\vec{r}\cdot\sigma\otimes\vec{s}\cdot\sigma = 
\begin{pmatrix}
0 & \vec{s}\cdot\sigma\\
\vec{s}\cdot\sigma & 0
\end{pmatrix} = 
\frac{1}{\sqrt{2}}\begin{pmatrix}
0 & 0 & -1 & -1\\
0 & 0 & -1 & 1\\
-1 & -1 & 0 & 0 \\
-1 & 1 & 0 & 0
\end{pmatrix}
\end{align*}


\begin{align*}
\vec{r}\cdot\sigma\otimes\vec{t}\cdot\sigma = 
\begin{pmatrix}
0 & \vec{t}\cdot\sigma \\
\vec{t}\cdot\sigma & 0
\end{pmatrix} = 
\frac{1}{\sqrt{2}}\begin{pmatrix}
0 & 0 & 1 & -1\\
0 & 0 & -1 & -1\\
1 & -1 & 0 & 0 \\
-1 & -1 & 0 & 0
\end{pmatrix}
\end{align*}



\end{frame}

\begin{frame}{Calculating expectations}


\begin{align*}
\vec{q}\cdot\sigma\otimes\vec{t}\cdot\sigma = 
\begin{pmatrix}
\vec{t}\cdot\sigma & 0\\
0 & -\vec{t}\cdot\sigma
\end{pmatrix} = 
\frac{1}{\sqrt{2}}\begin{pmatrix}
1 & -1 & 0 & 0\\
-1 & -1 & 0 & 0\\
0 & 0 & -1 & 1 \\
0 & 0 & 1 & 1
\end{pmatrix}
\end{align*}


\begin{align*}
\langle \vec{q}\cdot\sigma\otimes\vec{s}\cdot\sigma\rangle &= \frac{1}{\sqrt{2}}\left(-\alpha^{*}(\alpha+\beta) + \beta^{*}(\beta-\alpha) + \gamma^{*}(\gamma+\delta) + \delta^{*}(\gamma-\delta)\right)\\
\langle \vec{r}\cdot\sigma\otimes\vec{s}\cdot\sigma\rangle &= \frac{1}{\sqrt{2}}\left(-\alpha^{*}(\gamma+\delta) + \beta^{*}(\delta-\gamma) - \gamma^{*}(\alpha + \beta) + \delta^{*}(\beta-\alpha)\right)\\
\langle \vec{r}\cdot\sigma\otimes\vec{t}\cdot\sigma\rangle &= \frac{1}{\sqrt{2}}\left(\alpha^{*}(\gamma-\delta) - \beta^{*}(\delta+\gamma) + \gamma^{*}(\alpha - \beta) - \delta^{*}(\beta+\alpha)\right)\\
\langle \vec{q}\cdot\sigma\otimes\vec{t}\cdot\sigma\rangle &= \frac{1}{\sqrt{2}}\left(\alpha^{*}(\alpha-\beta) - \beta^{*}(\beta+\alpha) + \gamma^{*}(\delta-\delta) + \delta^{*}(\gamma+\delta)\right)
\end{align*}


\end{frame}

\begin{frame}{Full density matrix}
\begin{align*}
\rho_{AB} &=\begin{pmatrix}
|\alpha|^{2} & \alpha\beta^{*} & \alpha\gamma^{*} & \alpha\delta^{*} \\
\beta\alpha^{*} & |\beta|^{2} & \beta\gamma^{*} & \beta\delta^{*} \\
\gamma\alpha^{*} & \gamma\beta^{*} & |\gamma|^{2} & \gamma\delta^{*} \\
\delta\alpha^{*} & \delta\beta^{*} & \delta\gamma^{*} & |\delta|^{2}
\end{pmatrix}
\end{align*}
\end{frame}

\begin{frame}{Partial traces}

\begin{align*}
\mathrm{Tr}_{A}(\rho_{AB}) &= \sum_{ijkl} \rho_{ij}^{kl} \mathrm{Tr}_{A}(\ket{i}\bra{k}) \otimes \ket{j}\bra{l}\\
&= \sum_{i} \left(\sum_{jl} \rho_{ij}^{il} \ket{j}\bra{l}\right)\\
&= (\rho_{00}^{00}+\rho_{10}^{10})\ket{0}\bra{0} + (\rho_{00}^{01}+\rho_{10}^{11})\ket{0}\bra{1} + (\rho_{01}^{00}+\rho_{11}^{10})\ket{1}\bra{0} + (\rho_{01}^{01}+\rho_{11}^{11})\ket{1}\bra{1}
\end{align*}

\begin{align*}
\mathrm{Tr}_{B}(\rho_{AB}) &= \sum_{ijkl} \rho_{ij}^{kl} \ket{i}\bra{k} \otimes \mathrm{Tr}_{B}(\ket{j}\bra{l})\\
&= \sum_{j} \left(\sum_{ik} \rho_{ij}^{kj} \ket{i}\bra{k}\right)\\
&= (\rho_{00}^{00}+\rho_{01}^{01})\ket{0}\bra{0} + (\rho_{00}^{10}+\rho_{01}^{11})\ket{0}\bra{1} + (\rho_{10}^{00}+\rho_{11}^{01})\ket{1}\bra{0} + (\rho_{10}^{10}+\rho_{11}^{11})\ket{1}\bra{1}
\end{align*}

\end{frame}

\begin{frame}{Reduced density matrices for an arbitrary state}

\begin{align*}
\mathrm{Tr}_{A}(\rho_{AB}) &=
\begin{pmatrix}
\rho_{00}^{00}+\rho_{10}^{10} & \rho_{00}^{01}+\rho_{10}^{11}\\
\rho_{01}^{00}+\rho_{11}^{10} & \rho_{01}^{01}+\rho_{11}^{11}
\end{pmatrix}
= \begin{pmatrix}
|\alpha|^{2} + |\gamma|^{2} & \alpha\beta^{*} + \gamma\delta^{*}\\
\beta\alpha^{*} + \delta\gamma^{*} & |\beta|^{2} + |\delta|^{2}
\end{pmatrix}
\end{align*}

\begin{align*}
\mathrm{Tr}_{B}(\rho_{AB}) &=
\begin{pmatrix}
\rho_{00}^{00}+\rho_{01}^{01} & \rho_{00}^{10}+\rho_{01}^{11}\\
\rho_{10}^{00}+\rho_{11}^{01} & \rho_{10}^{10}+\rho_{11}^{11}
\end{pmatrix}
= \begin{pmatrix}
|\alpha|^{2} + |\beta|^{2} & \alpha\gamma^{*} + \beta\delta^{*}\\
\gamma\alpha^{*} + \delta\beta^{*} & |\gamma|^{2} + |\delta|^{2}
\end{pmatrix}
\end{align*}

\end{frame}

\begin{frame}{Definition of entanglement entropy}

The Von Neumann entropy is

\begin{align*}
S(\rho) = -\mathrm{Tr}(\rho\log\rho) = -\sum_{x}\lambda_{x}\log\lambda_{x}
\end{align*}

for eigenvalues $\lambda_{x}$ of $\rho$. This tells us: do the reduced states $\rho_{A}$ and $\rho_{B}$ contain all the information in $\rho_{AB}$? Maybe analagous to the mutual information

\begin{align*}
I(X;Y) = H(X) + H(Y) - H(X,Y) \geq 0
\end{align*}

So a possible measurement of entanglement is

\begin{align*}
\Delta S = S(\rho) - S(\rho_{A}) - S(\rho_{B})
\end{align*}

\end{frame}

\begin{frame}{Entanglement entropy of $\ket{\phi^{+}}$}

\begin{align*}
S(\rho) = -\mathrm{Tr}(\rho\log\rho) = -\sum_{x}\lambda_{x}\log\lambda_{x}
\end{align*}

where $\lambda_{x}$ are the eigenvalues of $\rho$. Let $\ket{\phi^{+}} = \frac{\ket{00} + \ket{11}}{\sqrt{2}}$

\begin{align*}
\rho_{AB} = 
\begin{pmatrix}
1 & 0 & 0 & 1 \\
0 & 0 & 0 & 0 \\
0 & 0 & 0 & 0 \\
1 & 0 & 0 & 1
\end{pmatrix}
\end{align*}

which only has one nonzero eigenvalue $\lambda = 2$. Therefore $S(\rho) = 1$.

\begin{align*}
\rho_{A} = \rho_{B} = \frac{1}{2}\begin{pmatrix}
1 & 0\\
0 & 1\\
\end{pmatrix}
\;\;\; S(\rho_{A}) = S(\rho_{B}) = 0  \;\; \Delta S = 1
\end{align*}

\end{frame}

\begin{frame}{Entanglement entropy of $\ket{00}$}

Let $\psi = \ket{00}$

\begin{align*}
\rho_{AB} = 
\begin{pmatrix}
1 & 0 & 0 & 0 \\
0 & 0 & 0 & 0 \\
0 & 0 & 0 & 0 \\
0 & 0 & 0 & 0
\end{pmatrix}
\end{align*}

which only has one nonzero eigenvalue $\lambda = 1$. Therefore $S(\rho) = 0$.

\begin{align*}
\rho_{A} = \rho_{B} = \begin{pmatrix}
1 & 0\\
0 & 0\\
\end{pmatrix}
\;\;\; \lambda = 1
\end{align*}
Therefore,
\begin{align*}
S(\rho_{A}) = S(\rho_{B}) = 0  \;\; \Delta S = 0
\end{align*}

\end{frame}

\begin{frame}{Correlation functions for $\ket{\phi^{+}}$}

$\ket{\phi^{+}} = \frac{\ket{00} + \ket{11}}{\sqrt{2}}$

\begin{align*}
\langle \vec{q}\cdot\sigma\otimes\vec{s}\cdot\sigma\rangle &= \frac{1}{\sqrt{2}}\left(-\alpha^{*}(\alpha+\beta) + \beta^{*}(\beta-\alpha) + \gamma^{*}(\gamma+\delta) + \delta^{*}(\gamma-\delta)\right) = -\frac{1}{\sqrt{2}}\\
\langle \vec{r}\cdot\sigma\otimes\vec{s}\cdot\sigma\rangle &= \frac{1}{\sqrt{2}}\left(-\alpha^{*}(\gamma+\delta) + \beta^{*}(\delta-\gamma) - \gamma^{*}(\alpha + \beta) + \delta^{*}(\beta-\alpha)\right)= -\frac{1}{\sqrt{2}}\\
\langle \vec{r}\cdot\sigma\otimes\vec{t}\cdot\sigma\rangle &= \frac{1}{\sqrt{2}}\left(\alpha^{*}(\gamma-\delta) - \beta^{*}(\delta+\gamma) + \gamma^{*}(\alpha - \beta) - \delta^{*}(\beta+\alpha)\right)= -\frac{1}{\sqrt{2}}\\
\langle \vec{q}\cdot\sigma\otimes\vec{t}\cdot\sigma\rangle &= \frac{1}{\sqrt{2}}\left(\alpha^{*}(\alpha-\beta) - \beta^{*}(\beta+\alpha) + \gamma^{*}(\delta-\delta) + \delta^{*}(\gamma+\delta)\right)= \frac{1}{\sqrt{2}}
\end{align*}

\begin{align*}
|\langle Q\otimes S\rangle + \langle R\otimes S\rangle  + \langle R\otimes T\rangle  - \langle Q\otimes T\rangle|  = 2\sqrt{2}
\end{align*}


\end{frame}

\end{document}