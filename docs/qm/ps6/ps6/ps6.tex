\documentclass[12pt]{article}
\usepackage{amsmath} % AMS Math Package
\usepackage{bm}
\usepackage{amsthm} % Theorem Formatting
\usepackage{amssymb}    % Math symbols such as \mathbb
\usepackage{graphicx} % Allows for eps images
\usepackage[dvips,letterpaper,margin=1in,bottom=0.7in]{geometry}
\usepackage{tensor}
\usepackage{amsmath}
\usepackage{siunitx}
\usepackage{physics}
\usepackage{amsmath, amssymb, graphics, setspace}

\newcommand{\mathsym}[1]{{}}
\newcommand{\unicode}[1]{{}}

\newcounter{mathematicapage}

\newtheorem{p}{Problem}
\usepackage{cancel}
\newtheorem*{lem}{Lemma}
\theoremstyle{definition}
\newtheorem*{dfn}{Definition}
 \newenvironment{s}{%\small%
        \begin{trivlist} \item \textbf{Solution}. }{%
            \hspace*{\fill} $\blacksquare$\end{trivlist}}%


\begin{document}

 {\noindent\Huge\bf  \\[0.5\baselineskip] {\fontfamily{cmr}\selectfont  Homework 6}         }\\[2\baselineskip] % Title
{ {\bf \fontfamily{cmr}\selectfont Quantum Mechanics}\\ {\textit{\fontfamily{cmr}\selectfont     October 28th, 2022}}}~~~~~~~~~~~~~~~~~~~~~~~~~~~~~~~~~~~~~~~~~~~~~~~~~~~~~~~~~~~~~~~~~~~~~~~~~~~~~    {\large \textsc{C Seitz}
\\[1.4\baselineskip] 

\begin{p}
Problem 3.12 from Sakurai
\end{p}

\begin{s}

In general the ensemble average of an operator $[A]$ is defined as

\begin{align*}
[A] = \mathrm{Tr}(\rho A)
\end{align*}

where $\hat{\rho} = \sum_{i}w_{i}\rho_{i}$ and $\rho_{i} = \ket{\alpha_{i}}\bra{\alpha_{i}}$

\begin{align*}
\hat{\rho} &= a\ket{+}\bra{+} + (1-a)\ket{-;y}\bra{-;y}\\
&= \frac{1}{2}\begin{pmatrix}2a&0\\0&0 \end{pmatrix} + 
\frac{1-a}{2}\begin{pmatrix}1&-i\\i&-1 \end{pmatrix}\\
&= \frac{1}{2}\begin{pmatrix}a+1&-i(1-a)\\i(1-a)&a-1 \end{pmatrix}
\end{align*}

\begin{align*}
[\sigma_{x}] &= \mathrm{Tr}(\hat{\rho} \sigma_{x})\\
&= \mathrm{Tr}\left(\frac{1}{2}\begin{pmatrix}i(1-a)&a-1\\a+1&-i(1-a) \end{pmatrix}\right) = 0
\end{align*}

\begin{align*}
[\sigma_{y}] &= \mathrm{Tr}(\hat{\rho} \sigma_{x})\\
&= \mathrm{Tr}\left(\frac{1}{2}\begin{pmatrix}-(1-a)&i(a-1)\\i(a+1)&(1-a) \end{pmatrix}\right) = 0
\end{align*}

\begin{align*}
[\sigma_{z}] &= \mathrm{Tr}(\hat{\rho} \sigma_{z})\\
&= \mathrm{Tr}\left(\frac{1}{2}\begin{pmatrix}a+1&-i(1-a)\\i(1-a)&1-a \end{pmatrix}\right) = 1
\end{align*}


\end{s}

\begin{p}
Problem 3.13 from Sakurai
\end{p}

\begin{s}

The state vector in the $S_{z}$ basis has the form
\begin{align*}
\ket{\alpha} = c_{+}\ket{+} + c_{-}\ket{-}
\end{align*}

First note that

\begin{align*}
\langle S_{z}\rangle &= |c_{+}|^{2} - |c_{-}|^{2}\;\;\;|c_{+}|^{2} + |c_{-}|^{2} = 1
\end{align*}

Together, these equations tell us the magnitude of each complex component.

\begin{align*}
|c_{+}|^{2} = \frac{\langle S_{z}\rangle +1}{2}\;\;\; |c_{-}|^{2} = \frac{1-\langle S_{z}\rangle}{2}
\end{align*}


\begin{align*}
\langle S_{x}\rangle &= \bra{\alpha}(\ket{+}\bra{-}+\ket{-}\bra{+})(c_{+}\ket{+} + c_{-}\ket{-})\\
&= \bra{\alpha}(c_{-}\ket{+} + c_{+}\ket{-})\\
&= (c_{+}^{*}\bra{+} + c_{-}^{*}\bra{-})(c_{-}\ket{+} + c_{+}\ket{-})\\
&= c_{+}^{*}c_{-} + c_{-}^{*}c_{+}\\
&= |c_{+}||c_{-}|(e^{i(\theta-\phi)} + e^{i(\phi-\theta)})\\
&= 2|c_{+}||c_{-}|\cos(\theta-\phi)
\end{align*}

\begin{align*}
\langle S_{y}\rangle &= \bra{\alpha}((i\ket{+}\bra{-}-i\ket{-}\bra{+})(c_{+}\ket{+} + c_{-}\ket{-})\\
&= i\bra{\alpha}(c_{-}\ket{+} - c_{+}\ket{-})\\
&= i(c_{+}^{*}\bra{+} + c_{-}^{*}\bra{-})(c_{-}\ket{+} - c_{+}\ket{-})\\
&= c_{+}^{*}c_{-} - c_{-}^{*}c_{+}\\
&= |c_{+}||c_{-}|(e^{i(\theta-\phi)} - e^{i(\phi-\theta)})\\
&= 2i|c_{+}||c_{-}|\sin(\theta-\phi)
\end{align*}

So $\langle S_{x}\rangle$ gives us the phase difference of $c_{+}$ and $c_{-}$. Then the sign of $\langle S_{y} \rangle$ tells us whether $\theta$ or $\phi$ is larger, since sine is odd. This is all we can hope to extract from the expectation values, since multiplying by a global phase $e^{i\delta}\ket{\alpha}$ has no effect on the expectation values.


\end{s}

\begin{p}
Problem 3.14 from Sakurai
\end{p}

\begin{s}
\begin{align*}
\rho &= \sum_{i}w_{i}\ket{\psi_{i}}\bra{\psi_{i}}\\
&= \frac{1}{3}\left(\ket{\alpha}\bra{\alpha} + \ket{\beta}\bra{\beta} + \ket{2}\bra{2}\right)
\end{align*}
We can write this out explicitly in the subspace spanned by $\ket{0,1,2}$
\begin{align*}
\ket{\alpha}\bra{\alpha} = 
\frac{1}{2}\begin{pmatrix}
1 & 1 & 0\\
1 & 1 & 0\\
0 & 0 &0
\end{pmatrix}\;\;\ket{\beta}\bra{\beta} = 
\frac{1}{2}\begin{pmatrix}
0 & 0 & 0\\
0 & 1 & 1 \\
0 & 1 & 1 
\end{pmatrix}\;\; \ket{2}\bra{2} = 
\begin{pmatrix}
0 & 0 & 0\\
0 & 0 & 0 \\
0 & 0 & 1 
\end{pmatrix}
\end{align*}

\begin{align*}
\rho &= \frac{1}{6}\begin{pmatrix}
1 & 1 & 0\\
1 & 2 & 1\\
0 & 1 & 2
\end{pmatrix}
\end{align*}

Now recall that $H = \hbar\omega(N+\frac{1}{2})$ which is

\begin{align*}
H &= \left(\frac{\hbar\omega}{2}\right)\mathbb{I}_{3\times 3} + \frac{\hbar\omega}{2}\begin{pmatrix}
0 & 0 & 0\\
0 & 1 & 0\\
0 & 0 & 2
\end{pmatrix}
\end{align*}

\begin{align*}
[H] = \mathrm{Tr}(\rho H) = \frac{\hbar\omega}{2}\mathrm{Tr}(\rho N + \rho) = \frac{\hbar\omega}{2}\left(\mathrm{Tr}(\rho N) + \mathrm{Tr}(\rho)\right)
\end{align*}

\end{s}

\begin{p}
Problem 3.15 from Sakurai
\end{p}

\begin{s}
\begin{align*}
\rho(t_{0}) &= \sum_{i}w_{i}\ket{\psi_{i};t_{0}}\bra{\psi_{i};t_{0}}\\
\end{align*}

In the Schrodinger picture, the coefficients of the state vectors evolve. Therefore,

\begin{align*}
\rho(t) &= \sum_{i}w_{i}\;\mathcal{U}(t,t_{0})\ket{\psi_{i};t_{0}}\bra{\psi_{i};t_{0}}\mathcal{U}^{\dagger}(t,t_{0})\\
&= \mathcal{U}(t,t_{0})\left(\sum_{i}w_{i}\ket{\psi_{i};t_{0}}\bra{\psi_{i};t_{0}}\right)\mathcal{U}^{\dagger}(t,t_{0})\\
&= \mathcal{U}(t,t_{0})\rho(t_{0})\mathcal{U}^{\dagger}(t,t_{0})
\end{align*}

For a pure ensemble in state $\ket{\psi_{i}}$, the density matrix is 

\begin{equation*}
\rho(t_{0}) = \ket{\psi_{i};t_{0}}\bra{\psi_{i};t_{0}}
\end{equation*}
At a later time, the density matrix is
\begin{align*}
\rho(t) &= \mathcal{U}(t,t_{0})\rho(t_{0})\mathcal{U}^{\dagger}(t,t_{0})\\
&= \mathcal{U}(t,t_{0})\ket{\psi_{i};t_{0}}\bra{\psi_{i};t_{0}}\mathcal{U}^{\dagger}(t,t_{0})\\
&= \ket{\psi_{i};t_{0}}\bra{\psi_{i};t_{0}}
\end{align*}

\end{s}

\begin{p}
Problem 3.16 from Sakurai
\end{p}

\begin{s}
A 3x3 matrix has 9 parameters, but we only need 
\end{s}

\begin{p}
Problem 3.40 from Sakurai
\end{p}

\begin{s}
The singlet state is 

\begin{align*}
\ket{\psi} = \frac{1}{\sqrt{2}}(\ket{+-} + \ket{-+})
\end{align*}

The probability is $1/2$ to obtain $s_{1z} = \hbar/2$. The probability is $1/2$ to obtain $s_{1x} = \hbar/2$ since obtaining $s_{1x} = \hbar/2$ is equiprobable for the two states in the singlet superposition. If observer $B$ has determined that $s_{2z} = \hbar/2$, then observer $A$ must observe $s_{1z} = -\hbar/2$ since the measurement made by $B$ collapses $\ket{\psi}$ to $\ket{-+}$. Furthermore, if observer $B$ has measured $s_{2z} = \hbar/2$, then particle 1 must be in the $\ket{+}$ state which means $s_{1x} = \pm \hbar/2$ with equal probability.

\end{s}

\end{document}