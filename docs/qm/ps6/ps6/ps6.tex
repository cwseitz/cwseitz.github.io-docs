\documentclass[12pt]{article}
\usepackage{amsmath} % AMS Math Package
\usepackage{bm}
\usepackage{amsthm} % Theorem Formatting
\usepackage{amssymb}    % Math symbols such as \mathbb
\usepackage{graphicx} % Allows for eps images
\usepackage[dvips,letterpaper,margin=1in,bottom=0.7in]{geometry}
\usepackage{tensor}
\usepackage{amsmath}
\usepackage{siunitx}
\usepackage{physics}
\usepackage{amsmath, amssymb, graphics, setspace}

\newcommand{\mathsym}[1]{{}}
\newcommand{\unicode}[1]{{}}

\newcounter{mathematicapage}

\newtheorem{p}{Problem}
\usepackage{cancel}
\newtheorem*{lem}{Lemma}
\theoremstyle{definition}
\newtheorem*{dfn}{Definition}
 \newenvironment{s}{%\small%
        \begin{trivlist} \item \textbf{Solution}. }{%
            \hspace*{\fill} $\blacksquare$\end{trivlist}}%


\begin{document}

 {\noindent\Huge\bf  \\[0.5\baselineskip] {\fontfamily{cmr}\selectfont  Homework 6}         }\\[2\baselineskip] % Title
{ {\bf \fontfamily{cmr}\selectfont Quantum Mechanics}\\ {\textit{\fontfamily{cmr}\selectfont     October 28th, 2022}}}~~~~~~~~~~~~~~~~~~~~~~~~~~~~~~~~~~~~~~~~~~~~~~~~~~~~~~~~~~~~~~~~~~~~~~~~~~~~~    {\large \textsc{C Seitz}
\\[1.4\baselineskip] 

\begin{p}
Problem 3.12 from Sakurai
\end{p}

\begin{s}

In general the ensemble average of an operator $[A]$ is defined as

\begin{align*}
[A] = \sum_{i}w_{i}\bra{\alpha_{i}}A\ket{\alpha_{i}}
\end{align*}

where $\sum_{i}w_{i} = 1$

\begin{align*}
[\sigma_{x}] &= a\bra{+}\sigma_{x}\ket{+} + (1-a)\bra{-;y}\sigma_{x}\ket{-;y}\\
&= a\bra{+}(\ket{+}\bra{-}+\ket{-}\bra{+})\ket{+} + (1-a)\bra{-;y}(\ket{+}\bra{-}+\ket{-}\bra{+})\ket{-;y}\\
&= 0
\end{align*}

\begin{align*}
[\sigma_{y}] &= a\bra{+}\sigma_{y}\ket{+} + (1-a)\bra{-;y}\sigma_{y}\ket{-;y}\\
&= ai\bra{+}(\ket{+}\bra{-}-\ket{-}\bra{+})\ket{+} + i(1-a)\bra{-;y}(\ket{+}\bra{-}-\ket{-}\bra{+})\ket{-;y}\\
&= i(1-a)\bra{-;y}\left(-\frac{i}{\sqrt{2}}\ket{+} - \frac{1}{\sqrt{2}}\ket{-}\right)\\
&= -i(1-a)\bra{-;y}\ket{+;y} = 0
\end{align*}

\begin{align*}
[\sigma_{z}] &= a\bra{+}(\ket{-}\bra{-} - \ket{+}\bra{+})\ket{+} + i(1-a)\bra{-;y}(\ket{-}\bra{-} - \ket{+}\bra{+})\ket{-;y}\\
&= -a + i(1-a)\bra{-;y}\left(-\frac{i}{\sqrt{2}}\ket{+} - \frac{1}{\sqrt{2}}\ket{-}\right)\\
\end{align*}

\end{s}

\begin{p}
Problem 3.13 from Sakurai
\end{p}

\begin{s}
\end{s}

\begin{p}
Problem 3.14 from Sakurai
\end{p}

\begin{s}
\end{s}

\begin{p}
Problem 3.15 from Sakurai
\end{p}

\begin{s}
\end{s}

\begin{p}
Problem 3.16 from Sakurai
\end{p}

\begin{s}
\end{s}

\begin{p}
Problem 3.40 from Sakurai
\end{p}

\begin{s}
\end{s}

\end{document}