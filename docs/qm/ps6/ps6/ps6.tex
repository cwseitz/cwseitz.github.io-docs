\documentclass[12pt]{article}
\usepackage{amsmath} % AMS Math Package
\usepackage{bm}
\usepackage{amsthm} % Theorem Formatting
\usepackage{amssymb}    % Math symbols such as \mathbb
\usepackage{graphicx} % Allows for eps images
\usepackage[dvips,letterpaper,margin=1in,bottom=0.7in]{geometry}
\usepackage{tensor}
\usepackage{amsmath}
\usepackage{siunitx}
\usepackage{physics}
\usepackage{amsmath, amssymb, graphics, setspace}

\newcommand{\mathsym}[1]{{}}
\newcommand{\unicode}[1]{{}}

\newcounter{mathematicapage}

\newtheorem{p}{Problem}
\usepackage{cancel}
\newtheorem*{lem}{Lemma}
\theoremstyle{definition}
\newtheorem*{dfn}{Definition}
 \newenvironment{s}{%\small%
        \begin{trivlist} \item \textbf{Solution}. }{%
            \hspace*{\fill} $\blacksquare$\end{trivlist}}%


\begin{document}

 {\noindent\Huge\bf  \\[0.5\baselineskip] {\fontfamily{cmr}\selectfont  Homework 6}         }\\[2\baselineskip] % Title
{ {\bf \fontfamily{cmr}\selectfont Quantum Mechanics}\\ {\textit{\fontfamily{cmr}\selectfont     October 28th, 2022}}}~~~~~~~~~~~~~~~~~~~~~~~~~~~~~~~~~~~~~~~~~~~~~~~~~~~~~~~~~~~~~~~~~~~~~~~~~~~~~    {\large \textsc{C Seitz}
\\[1.4\baselineskip] 

\begin{p}
Problem 3.12 from Sakurai
\end{p}

\begin{s}

In general the ensemble average of an operator $[A]$ is defined as

\begin{align*}
[A] = \mathrm{Tr}(\rho A)
\end{align*}

where $\hat{\rho} = \sum_{i}w_{i}\rho_{i}$ and $\rho_{i} = \ket{\alpha_{i}}\bra{\alpha_{i}}$

\begin{align*}
\hat{\rho} &= a\ket{+}\bra{+} + (1-a)\ket{-;y}\bra{-;y}\\
&= \left(
\begin{array}{cc}
 \frac{1-a}{2}+a & \frac{1}{2} i (1-a) \\
 -\frac{1}{2} i (1-a) & \frac{1-a}{2} \\
\end{array}
\right)
\end{align*}

\begin{align*}
[S_{x}] &= \mathrm{Tr}(\hat{\rho} S_{x})\\
&= \frac{\hbar}{2}\mathrm{Tr}\left(\left(
\begin{array}{cc}
 \frac{1}{2} i (1-a) & \frac{1-a}{2}+a \\
 \frac{1-a}{2} & -\frac{1}{2} i (1-a) \\
\end{array}
\right)\right) = 0
\end{align*}

\begin{align*}
[S_{y}] &= \mathrm{Tr}(\hat{\rho} S_{x})\\
&= \frac{\hbar}{2}\mathrm{Tr}\left(\left(
\begin{array}{cc}
 \frac{a-1}{2} & -i \left(\frac{1-a}{2}+a\right) \\
 \frac{1}{2} i (1-a) & \frac{a-1}{2} \\
\end{array}
\right)\right) = \frac{\hbar}{2}(a-1)
\end{align*}

\begin{align*}
[S_{z}] &= \mathrm{Tr}(\hat{\rho} S_{z})\\
&= \frac{\hbar}{2}\mathrm{Tr}\left(\left(
\begin{array}{cc}
 \frac{1-a}{2}+a & -\frac{1}{2} i (1-a) \\
 -\frac{1}{2} i (1-a) & \frac{a-1}{2} \\
\end{array}
\right)\right) =  \frac{\hbar}{2}a
\end{align*}

When $a=1$, we get $[S_{x}] = 0, [S_{y}] = 0, [S_{z}] = \hbar/2$, which makes sense since it is then a pure ensemble in $\ket{+}$. When $a=0$, we get $[S_{x}] = 0, [S_{y}] = -\hbar/2, [S_{z}] = 0$, which makes sense because it is a pure ensemble in $\ket{-;y}$.

\end{s}

\begin{p}
Problem 3.13 from Sakurai
\end{p}

\begin{s}

The state vector in the $S_{z}$ basis has the form
\begin{align*}
\ket{\alpha} = c_{+}\ket{+} + c_{-}\ket{-}
\end{align*}

First note that

\begin{align*}
\langle S_{z}\rangle &= |c_{+}|^{2} - |c_{-}|^{2}\;\;\;|c_{+}|^{2} + |c_{-}|^{2} = 1
\end{align*}

Together, these equations tell us the magnitude of each complex component.

\begin{align*}
|c_{+}|^{2} = \frac{\langle S_{z}\rangle +1}{2}\;\;\; |c_{-}|^{2} = \frac{1-\langle S_{z}\rangle}{2}
\end{align*}


\begin{align*}
\langle S_{x}\rangle &= \bra{\alpha}(\ket{+}\bra{-}+\ket{-}\bra{+})(c_{+}\ket{+} + c_{-}\ket{-})\\
&= \bra{\alpha}(c_{-}\ket{+} + c_{+}\ket{-})\\
&= (c_{+}^{*}\bra{+} + c_{-}^{*}\bra{-})(c_{-}\ket{+} + c_{+}\ket{-})\\
&= c_{+}^{*}c_{-} + c_{-}^{*}c_{+}\\
&= |c_{+}||c_{-}|(e^{i(\theta-\phi)} + e^{i(\phi-\theta)})\\
&= 2|c_{+}||c_{-}|\cos(\theta-\phi)
\end{align*}

Let $\delta = \theta-\phi$, which means $\delta = \cos^{-1}\left(\frac{\langle S_{x}\rangle}{2|c_{+}||c_{-}|}\right)$

\begin{align*}
\langle S_{y}\rangle &= \bra{\alpha}((i\ket{+}\bra{-}-i\ket{-}\bra{+})(c_{+}\ket{+} + c_{-}\ket{-})\\
&= i\bra{\alpha}(c_{-}\ket{+} - c_{+}\ket{-})\\
&= i(c_{+}^{*}\bra{+} + c_{-}^{*}\bra{-})(c_{-}\ket{+} - c_{+}\ket{-})\\
&= c_{+}^{*}c_{-} - c_{-}^{*}c_{+}\\
&= |c_{+}||c_{-}|(e^{i(\theta-\phi)} - e^{i(\phi-\theta)})\\
&= 2i|c_{+}||c_{-}|\sin(\theta-\phi)\\
&= 2i|c_{+}||c_{-}|\sin(\delta)
\end{align*}

So $\langle S_{x}\rangle$ gives us the phase difference of $c_{+}$ and $c_{-}$. Then the sign of $\langle S_{y} \rangle$ tells us the sign of $\delta$, since sine is odd. This is all we can hope to extract from the expectation values, since multiplying by a global phase $e^{i\delta}\ket{\alpha}$ has no effect on the expectation values. To find $\rho$ using $[S_{x}],[S_{y}],[S_{z}]$, first note that

\begin{align*}
\mathrm{Tr}(\rho) = 
\mathrm{Tr}\left(\begin{pmatrix}a&b\\c&d\end{pmatrix}\right) = a+d = 0
\end{align*}

\begin{align*}
\mathrm{Tr}\left(\rho S_{x}\right) = 
\frac{\hbar}{2}\mathrm{Tr}\left(\begin{pmatrix}a&b\\c&d\end{pmatrix}\begin{pmatrix}0&1\\1&0\end{pmatrix}\right) = \frac{\hbar}{2}(c + b)
\end{align*}


\begin{align*}
\mathrm{Tr}\left(\rho S_{y}\right) = 
\frac{\hbar}{2}\mathrm{Tr}\left(\begin{pmatrix}a&b\\c&d\end{pmatrix}\begin{pmatrix}0&-i\\i&0\end{pmatrix}\right) = \frac{i\hbar}{2}(b-c)
\end{align*}


\begin{align*}
\mathrm{Tr}\left(\rho S_{z}\right) = 
\frac{\hbar}{2}\mathrm{Tr}\left(\begin{pmatrix}a&b\\c&d\end{pmatrix}\begin{pmatrix}1&0\\0&-1\end{pmatrix}\right) = \frac{\hbar}{2}(a-d)
\end{align*}

which gives us four equations for the four unknown elements of $\rho$.

\end{s}

\begin{p}
Problem 3.14 from Sakurai
\end{p}

\begin{s}
\begin{align*}
\hat{\rho} &= \sum_{i}w_{i}\ket{\psi_{i}}\bra{\psi_{i}}\\
&= \frac{1}{3}\left(\ket{\alpha}\bra{\alpha} + \ket{\beta}\bra{\beta} + \ket{2}\bra{2}\right)
\end{align*}
We can write this out explicitly in the subspace spanned by $\ket{0,1,2}$
\begin{align*}
\ket{\alpha}\bra{\alpha} = 
\frac{1}{2}\begin{pmatrix}
1 & 1 & 0\\
1 & 1 & 0\\
0 & 0 &0
\end{pmatrix}\;\;\ket{\beta}\bra{\beta} = 
\frac{1}{2}\begin{pmatrix}
0 & 0 & 0\\
0 & 1 & 1 \\
0 & 1 & 1 
\end{pmatrix}\;\; \ket{2}\bra{2} = 
\frac{1}{2}\begin{pmatrix}
0 & 0 & 0\\
0 & 0 & 0 \\
0 & 0 & 2
\end{pmatrix}
\end{align*}

\begin{align*}
\hat{\rho} &= \frac{1}{6}\begin{pmatrix}
1 & 1 & 0\\
1 & 2 & 1\\
0 & 1 & 3
\end{pmatrix}
\end{align*}

Now recall that $H = \hbar\omega(N+\frac{1}{2})$ which reads

\begin{align*}
H &= \hbar\omega\begin{pmatrix}
0 & 0 & 0\\
0 & 1 & 0\\
0 & 0 & 2
\end{pmatrix} + \left(\frac{\hbar\omega}{2}\right)\mathbb{I}_{3\times 3}
\end{align*}

\begin{align*}
[H] = \mathrm{Tr}(\rho H) &= \hbar\omega\mathrm{Tr}(\rho N + \rho/2) \\
&= \hbar\omega\left(\mathrm{Tr}(\rho N) + \mathrm{Tr}(\rho/2)\right)\\
&= \frac{11}{6}\hbar\omega
\end{align*}

\end{s}

\begin{p}
Problem 3.15 from Sakurai
\end{p}

\begin{s}
\begin{align*}
\rho(t_{0}) &= \sum_{i}w_{i}\ket{\psi_{i};t_{0}}\bra{\psi_{i};t_{0}}\\
\end{align*}
(a) In the Schrodinger picture, the coefficients of the state vectors evolve. Therefore,
\begin{align*}
\rho(t) &= \sum_{i}w_{i}\;\mathcal{U}(t,t_{0})\ket{\psi_{i};t_{0}}\bra{\psi_{i};t_{0}}\mathcal{U}^{\dagger}(t,t_{0})\\
&= \mathcal{U}(t,t_{0})\left(\sum_{i}w_{i}\ket{\psi_{i};t_{0}}\bra{\psi_{i};t_{0}}\right)\mathcal{U}^{\dagger}(t,t_{0})\\
&= \mathcal{U}(t,t_{0})\rho(t_{0})\mathcal{U}^{\dagger}(t,t_{0})
\end{align*}
(b) Note that this does not mean that the states do not evolve in time. Rather, they must evolve in the same way so that the ensemble remains pure. For example, a magnetic field along $z$ can cause precession of a state prepared in $\ket{S_{n};+}$, but every member of the ensemble evolves identically. We therefore need the more general property that $\rho^{2} = 1$. So we write
\begin{align*}
\rho^{2}(t) &= \mathcal{U}(t,t_{0})\rho(t_{0})\mathcal{U}^{\dagger}(t,t_{0})\mathcal{U}(t,t_{0})\rho(t_{0})\mathcal{U}^{\dagger}(t,t_{0})\\
&= \mathcal{U}(t,t_{0})\rho^{2}(t_{0})\mathcal{U}^{\dagger}(t,t_{0})\\
&= \mathcal{U}(t,t_{0})\mathcal{U}^{\dagger}(t,t_{0}) = 1
\end{align*}

\end{s}

\begin{p}
Problem 3.16 from Sakurai
\end{p}

\begin{s}

We can write a general form of the density matrix

\begin{align*}
\hat{\rho} = \begin{pmatrix}\alpha&a&b\\a*&\beta&c\\b*&c*&1-\alpha-\beta\end{pmatrix}\\
\end{align*}

which is due to the fact that the density matrix must have zero trace and must be Hermitian. This matrix has eight real parameters (two along the diagonal and 6 from the three complex numbers).

\end{s}

\begin{p}
Problem 3.40 from Sakurai
\end{p}

\begin{s}
The singlet state is 

\begin{align*}
\ket{\psi} = \frac{1}{\sqrt{2}}(\ket{+-} + \ket{-+})
\end{align*}

If $B$ doesn't make a measurement, $B$ will have no effect on $A$'s measurement. So the probability for $A$ to obtain $s_{1z} = \hbar/2$ is of course $1/2$. 

The probability that $A$ measures $s_{1x} = \hbar/2$ in this state is also $1/2$. This is because obtaining $s_{1x} = \hbar/2$ is equiprobable for the two states in the singlet superposition. 

If observer $B$ has determined that $s_{2z} = \hbar/2$, then observer $A$ must observe $s_{1z} = -\hbar/2$ since the measurement made by $B$ collapses $\ket{\psi}$ to $\ket{-+}$. Furthermore, if observer $B$ has measured $s_{2z} = \hbar/2$, then particle 1 must be in the $\ket{+}$ state (as stated before) which means $s_{1x} = \pm \hbar/2$ with equal probability.

\end{s}

\end{document}