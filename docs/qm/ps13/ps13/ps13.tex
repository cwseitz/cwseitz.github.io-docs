\documentclass[12pt]{article}
\usepackage{amsmath} % AMS Math Package
\usepackage{bm}
\usepackage{amsthm} % Theorem Formatting
\usepackage{amssymb}    % Math symbols such as \mathbb
\usepackage{graphicx} % Allows for eps images
\usepackage[dvips,letterpaper,margin=1in,bottom=0.7in]{geometry}
\usepackage{tensor}
\usepackage{amsmath}
\usepackage{siunitx}
\usepackage{physics}
\usepackage{amsmath, amssymb, graphics, setspace}

\newcommand{\mathsym}[1]{{}}
\newcommand{\unicode}[1]{{}}

\newcounter{mathematicapage}

\newtheorem{p}{Problem}
\usepackage{cancel}
\newtheorem*{lem}{Lemma}
\theoremstyle{definition}
\newtheorem*{dfn}{Definition}
 \newenvironment{s}{%\small%
        \begin{trivlist} \item \textbf{Solution}. }{%
            \hspace*{\fill} $\blacksquare$\end{trivlist}}%


\begin{document}

 {\noindent\Huge\bf  \\[0.5\baselineskip] {\fontfamily{cmr}\selectfont  Homework 3}         }\\[2\baselineskip] % Title
{ {\bf \fontfamily{cmr}\selectfont Quantum Mechanics}\\ {\textit{\fontfamily{cmr}\selectfont     \today}}}~~~~~~~~~~~~~~~~~~~~~~~~~~~~~~~~~~~~~~~~~~~~~~~~~~~~~~~~~~~~~~~~~~~~~~~~~~~~~    {\large \textsc{C Seitz}
\\[1.4\baselineskip] 


\begin{p}
Problem 2.48
\end{p}

\begin{s}
\end{s}

\begin{p}
Problem 2.49
\end{p}

\begin{s}
\end{s}

\begin{p}
Problem 2.50
\end{p}

\begin{s}
\end{s}

\begin{p}
Problem 2.51
\end{p}

\begin{s}
The Hadamard gate $H$ is unitary if $H^{\dagger} = H^{-1}$. It is easy to see that

\begin{equation*}
H^{\dagger} = H = \frac{1}{\sqrt{2}}
\begin{pmatrix}
1 & 1\\ 1& -1
\end{pmatrix}
\end{equation*}

It's inverse is 

\begin{equation*}
H^{-1} =-\frac{1}{\sqrt{2}}
\begin{pmatrix}
-1 & -1\\ -1& 1
\end{pmatrix} = H
\end{equation*}

\end{s}

\begin{p}
Problem 2.52
\end{p}

\begin{s}
\begin{equation*}
H^{2} =  \frac{1}{\sqrt{2}}
\begin{pmatrix}
1 & 1\\ 1& -1
\end{pmatrix} \frac{1}{\sqrt{2}}
\begin{pmatrix}
1 & 1\\ 1& -1
\end{pmatrix} = 
\begin{pmatrix}
1 & 0\\ 0& 1
\end{pmatrix}
\end{equation*}
\end{s}

\begin{p}
Problem 2.53
\end{p}

\begin{s}
Writing out the characteristic equation gives that the eigenvalues are $\lambda = \pm \sqrt{2}$. 
\end{s}

\begin{p}
Problem 2.54
\end{p}

\begin{s}
\end{s}

\begin{p}
Problem 2.55
\end{p}

\begin{s}
\end{s}

\begin{p}
Problem 2.56
\end{p}

\begin{s}
\end{s}

\begin{p}
Problem 2.57
\end{p}

\begin{s}
\end{s}

\begin{p}
Problem 2.58
\end{p}

\begin{s}
\end{s}

\begin{p}
Problem 2.59
\end{p}

\begin{s}
\end{s}

\begin{p}
Problem 2.60
\end{p}

\begin{s}
\end{s}
\begin{p}
Problem 2.61
\end{p}

\begin{s}
\end{s}


\end{document}