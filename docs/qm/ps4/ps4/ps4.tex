\documentclass[12pt]{article}
\usepackage{amsmath} % AMS Math Package
\usepackage{bm}
\usepackage{amsthm} % Theorem Formatting
\usepackage{amssymb}    % Math symbols such as \mathbb
\usepackage{graphicx} % Allows for eps images
\usepackage[dvips,letterpaper,margin=1in,bottom=0.7in]{geometry}
\usepackage{tensor}
\usepackage{amsmath}
\usepackage{siunitx}
\usepackage{physics}
\usepackage{amsmath, amssymb, graphics, setspace}

\newcommand{\mathsym}[1]{{}}
\newcommand{\unicode}[1]{{}}

\newcounter{mathematicapage}

\newtheorem{p}{Problem}
\usepackage{cancel}
\newtheorem*{lem}{Lemma}
\theoremstyle{definition}
\newtheorem*{dfn}{Definition}
 \newenvironment{s}{%\small%
        \begin{trivlist} \item \textbf{Solution}. }{%
            \hspace*{\fill} $\blacksquare$\end{trivlist}}%


\begin{document}

 {\noindent\Huge\bf  \\[0.5\baselineskip] {\fontfamily{cmr}\selectfont  Homework 4}         }\\[2\baselineskip] % Title
{ {\bf \fontfamily{cmr}\selectfont Quantum Mechanics}\\ {\textit{\fontfamily{cmr}\selectfont     Sept 22nd, 2022}}}~~~~~~~~~~~~~~~~~~~~~~~~~~~~~~~~~~~~~~~~~~~~~~~~~~~~~~~~~~~~~~~~~~~~~~~~~~~~~    {\large \textsc{Clayton Seitz}
\\[1.4\baselineskip] 

\begin{p}
Problem 2.14 from Sakurai
\end{p}

\begin{s}

We are given that the state vector is 

\begin{equation*}
\ket{\alpha; t_{0}} = \exp\left(\frac{-ipa}{\hbar}\right)\ket{0}
\end{equation*}

The Heisenberg equation of motion for $x(t)$ reads

\begin{align*}
x(t) = x(0)\cos(\omega t) + \frac{p(0)}{m}\sin(\omega t)
\end{align*}

Therefore

\begin{align*}
\langle x \rangle &= \bra{\alpha;t_{0}}x(t)\ket{\alpha;t_{0}}\\
&= \bra{\alpha;t_{0}}\left(x(0)\cos(\omega t) + \frac{p(0)}{m}\sin(\omega t)\right)\ket{\alpha;t_{0}}\\
&= \bra{0}\exp\left(\frac{ipa}{\hbar}\right)\left(x(0)\cos(\omega t) + \frac{p(0)}{m}\sin(\omega t)\right)\exp\left(\frac{-ipa}{\hbar}\right)\ket{0}\\
&= \bra{0}\exp\left(\frac{ipa}{\hbar}\right)x(0)\exp\left(-\frac{ipa}{\hbar}\right)\ket{0}\cos(\omega t)\\ 
&+ \frac{1}{m}\bra{0}\exp\left(\frac{ipa}{\hbar}\right)p(0)\exp\left(\frac{-ipa}{\hbar}\right)\ket{0}\sin(\omega t)
\end{align*}

We can simplify this last expression by using the Baker-Haussdorf lemma for arbitrary operators $G$ and $A$

\begin{align*}
\exp(iG\lambda)A\exp(-iG\lambda) = A + i\lambda[G,A] + ...
\end{align*}

\begin{align*}
\langle x \rangle &= \bra{0}\exp\left(\frac{ipa}{\hbar}\right)x(0)\exp\left(-\frac{ipa}{\hbar}\right)\ket{0}\cos(\omega t)\\ 
&+ \frac{1}{m}\bra{0}\exp\left(\frac{ipa}{\hbar}\right)p(0)\exp\left(\frac{-ipa}{\hbar}\right)\ket{0}\sin(\omega t)\\
&= \bra{0}\left(x + \frac{ia}{\hbar}[p,x]\right)\ket{0}\cos(\omega t)\\ 
&+ \frac{1}{m}\bra{0}\left(p + \frac{ia}{\hbar}[p,p]\right)\ket{0}\sin(\omega t)\\
&= \left(\bra{0}x\ket{0} + a\right)\cos(\omega t) = a\cos(\omega t)
\end{align*}

\end{s}

\begin{p}
Problem 2.15 from Sakurai
\end{p}

\begin{s}

First note that the state is being translated $x\rightarrow x + a$

\begin{equation*}
\ket{\alpha; t_{0}} = \exp\left(\frac{-ipa}{\hbar}\right)\ket{0} = \mathcal{J}(a)\ket{0}
\end{equation*}

where $\mathcal{J}$ is the translation operator. Using that

\begin{align*}
\bra{x}\ket{0} =\pi^{-1/4}x_{0}^{1/2} \exp\left(-\frac{1}{2}\left(\frac{x}{x_{0}}\right)^{2}\right)
\end{align*}

we expect to be able to show

\begin{align*}
\bra{x} \exp\left(\frac{-ipa}{\hbar}\right)\ket{0} &= \bra{x+a}\ket{0}\\
&= \pi^{-1/4}x_{0}^{1/2}\exp\left(-\frac{1}{2}\left(\frac{x+a}{x_{0}}\right)^{2}\right)
\end{align*}
where $x_{0} = \sqrt{\frac{\hbar}{m\omega}}$. The probability that $\ket{\alpha}$ is measured to be in the state $\ket{0}$ is

\begin{align*}
\bra{\alpha}\ket{0} &= \pi^{-1/2}x_{0}\int \exp\left(-\frac{1}{2}\left(\frac{x+a}{x_{0}}\right)^{2}\right)\exp\left(-\frac{1}{2}\left(\frac{x}{x_{0}}\right)^{2}\right)dx\\
\end{align*}

which is an integral of a product of Gaussians.
\end{s}

\begin{p}
Problem 2.16 from Sakurai
\end{p}

\begin{s}

We will assume the form of the annihilation and creation operators

\begin{align*}
a &= \sqrt{\frac{m\omega}{2\hbar}}\left(x + \frac{ip}{m\omega}\right)\\
a^{\dagger} &= \sqrt{\frac{m\omega}{2\hbar}}\left(x - \frac{ip}{m\omega}\right)
\end{align*}

Adding these equations gives and rearranging we can express $x$ as

\begin{align*}
x = \sqrt{\frac{\hbar}{2m\omega}}\left(a + a^{\dagger}\right)
\end{align*}

\begin{align*}
\bra{m}x\ket{n} &= \bra{m} \sqrt{\frac{\hbar}{2m\omega}}\left(a + a^{\dagger}\right)\ket{n}\\
&= \sqrt{\frac{\hbar}{2m\omega}} \left(\bra{m}a\ket{n} + \bra{m}a^{\dagger}\ket{n}\right)\\
&= \sqrt{\frac{\hbar}{2m\omega}} \left(\sqrt{n}\delta_{m,n-1} + \sqrt{n+1}\delta_{m,n+1}\right)\\
\end{align*}

Subtracting the creation operator from the annihalation operator allows us to write the momentum operator as

\begin{align*}
p &= i\sqrt{\frac{m\hbar \omega}{2}}\left(a^{\dagger}-a\right)
\end{align*}

\begin{align*}
\bra{m}p\ket{n} &= \bra{m}\left(i\sqrt{\frac{m\hbar \omega}{2}}\left(a^{\dagger}-a\right)\right)\ket{n}\\
&= \left(i\sqrt{\frac{m\hbar \omega}{2}}\left(\bra{m}a^{\dagger}\ket{n}-\bra{m}a\ket{n}\right)\right)\\
&= i\sqrt{\frac{m\hbar \omega}{2}}\left(\sqrt{n+1}\delta_{m,n+1}-\sqrt{n}\delta_{m,n-1}\right)\\
\end{align*}
\begin{align*}
\bra{m}\{x,p\}\ket{n} &= \bra{m}xp\ket{n} + \bra{m}px\ket{n}\\
&= \frac{i\hbar}{2}\bra{m}\left((a^{\dagger})^{2}-a^{2}\right)\ket{n} + \frac{i\hbar}{2}\bra{m}\left((a^{\dagger})^{2} + a^{\dagger}a - aa^{\dagger} - a^{2}\right)\ket{n}\\
&=  \frac{i\hbar}{2}(\sqrt{n+1}\sqrt{n+2}\delta_{m,n+2} + \sqrt{n}\sqrt{n-1}\delta_{m,n-2})
\end{align*}

since only the cross terms will survive.

\begin{align*}
\bra{m}x^{2}\ket{n} &= \frac{\hbar}{2m\omega}\bra{m}\left(a^{2} + aa^{\dagger} + a^{\dagger} a + (a^{\dagger})^{2}\right)\ket{n}\\
&= \frac{\hbar}{2m\omega}\left((2n+1)\delta_{mn} + \sqrt{m}\sqrt{n+1}\delta_{m-1,n+1} + \sqrt{m+1}\sqrt{n}\delta_{m+1,n-1}\right)
\end{align*}

\begin{align*}
\bra{m}p^{2}\ket{n} &= -\frac{m\hbar \omega}{2}\bra{m}\left(a^{2}- aa^{\dagger} - a^{\dagger}a +(a^{\dagger})^{2}\right)\ket{n}\\
&= -\frac{m\hbar \omega}{2}\left((2n+1)\delta_{mn} - \sqrt{m}\sqrt{n+1}\delta_{m-1,n+1} - \sqrt{m+1}\sqrt{n}\delta_{m+1,n-1}\right)
\end{align*}
\end{s}

To validate the virial theorem we write, 

\begin{align*}
\frac{1}{m}\langle p^{2}\rangle = m\omega^{2}\langle x^{2}\rangle
\end{align*}

From the above, we can see that this is satisfied for an energy eigenstate

\begin{align*}
\frac{1}{m}\langle p^{2}\rangle &= \frac{\hbar \omega}{2}(2n+1)\\
m\omega^{2}\langle x^{2}\rangle &= m\omega^{2}\frac{\hbar}{2m\omega}(2n+1) = \frac{\hbar \omega}{2}(2n+1)
\end{align*}

\begin{p}
Problem 2.28 from Sakurai
\end{p}

\begin{s}

First of all, the solution is not trivial since $x$ does not commute with the Hamiltonian since $[x,p^{2}] \neq 0$. At $t=t_{0}$ we are in the position eigenstate

\begin{align*}
\bra{x}\ket{\alpha; t_{0}} = \delta\left(x-\frac{L}{2}\right)
\end{align*}

Since this is the infinite square well, we have the following energy eigenstates, in the position representation

\begin{align*}
\bra{x}\ket{\alpha} = \sqrt{\frac{2}{L}}\sin\left(\frac{n\pi x}{L}\right)
\end{align*}

Of course $\ket{\alpha; t_{0}}$ is not an eigenstate of $H$, so this state will measurably evolve in time. The state $\ket{\alpha; t_{0}}$ in the energy basis is

\begin{align*}
\ket{\beta} &= \sum_{n}\ket{\epsilon_{n}}\bra{\epsilon_{n}}\ket{\alpha;t_{0}}\\
&=  \sqrt{\frac{2}{L}}\sum_{n}\sin\left(\frac{n\pi}{2}\right)\ket{\epsilon_{n}}
\end{align*}

From this, we can show the probability of measuring the particle in energy eigenstate $\ket{\epsilon_{n}}$

\begin{align*}
\bra{\epsilon_{m}}\ket{\beta} &= \sqrt{\frac{2}{L}}\sum_{n}\sin\left(\frac{n\pi}{2}\right)\bra{\epsilon_{m}}\ket{\epsilon_{n}}\\
&= \sqrt{\frac{2}{L}}\sum_{n}\sin\left(\frac{n\pi}{2}\right)\delta_{mn}\\
&= \sqrt{\frac{2}{L}}\sin\left(\frac{m\pi}{2}\right)\\
\end{align*}

and therefore $|\bra{\epsilon_{m}}\ket{\beta}|^{2} = \sqrt{\frac{2}{L}}\sin^{2}\left(\frac{m\pi}{2}\right)$. The relative probabilities with respect to the ground state are then given by

\begin{align*}
r_{m+1} = \sin^{2}\left(\frac{(m+1)\pi}{2}\right)\csc^{2}\left(\frac{m\pi}{2}\right)
\end{align*}


Since we know a representation for $\ket{\alpha;t_{0}}$ in the energy basis, we can determine the time evolution of the wavefunction $\bra{x}\ket{\alpha}$

\begin{align*}
\ket{\alpha;t} &= \mathcal{U}(t)\ket{\beta} \\
&= \sqrt{\frac{2}{L}}\sum_{n}\exp\left(\frac{-i\epsilon_{n}t}{\hbar}\right)\sin\left(\frac{n\pi}{2}\right)\ket{\epsilon_{n}}
\end{align*}

which has the position representation (wave function)

\begin{align*}
\bra{x}\ket{\alpha; t} &= \psi(x,t)\\
&= \sqrt{\frac{2}{L}}\sum_{n}\exp\left(\frac{-i\epsilon_{n}t}{\hbar}\right)\sin\left(\frac{n\pi}{2}\right)\bra{x}\ket{\epsilon_{n}}\\
&= \sqrt{\frac{2}{L}}\sum_{n}\exp\left(\frac{-i\epsilon_{n}t}{\hbar}\right)\sin\left(\frac{n\pi}{2}\right)\psi_{n}(x)
\end{align*}

where $\psi_{n}(x)$ are the energy eigenstates given above.

\end{s}

\begin{p}
Problem 2.29 from Sakurai
\end{p}

\begin{s}

For a delta-potential, Schrodinger's equation reads

\begin{align*}
-\frac{\hbar^{2}}{2m}\frac{d\psi^{2}}{dx^{2}} - \nu_{0}\delta(x)\psi(x) = E\psi(x)
\end{align*}

We then solve Schrodingers equation in two regions. Let the first region be $x<0$ 

\begin{align*}
\frac{d\psi_{I}^{2}}{dx^{2}} &= -\frac{2mE}{\hbar^{2}}\psi_{I}(x) = \kappa_{0}^{2}\psi_{I}(x)\\
\end{align*}

for $\kappa_{0} = \sqrt{-2mE}/\hbar$. If $E <0$, then the general solution is

\begin{align*}
\psi_{I}(x) &= A\exp(\kappa_{0} x) + B\exp(-\kappa_{0} x)\\
\end{align*}

We choose $B = 0$ on physical grounds. Now for the second region $x>0$, we have a similar situation, but this time we choose an exponential decay

\begin{align*}
\psi_{II}(x) &= A\exp(-\kappa_{0} x)\\
\end{align*}

We have chosen the constant to be the same as the first region, to preserve continuity of $\psi(x)$.

\begin{equation*}
\psi(x) = \begin{cases}
A\exp(\kappa_{0}x) \;\;x <0 \\
A\exp(-\kappa_{0}x) \;\; x\geq 0\\
\end{cases}
\end{equation*}

The constant $A$ is found by enforcing the normalization condition:

\begin{align*}
2A^{2}\int_{-\infty}^{0} \exp(2\kappa_{0}x)dx = 1
\end{align*}

and it is straightforward to show that $A = \sqrt{\kappa_{0}}$

The energy is $E = -\hbar^{2}\kappa_{0}^{2}/2m$. The usual trick for relating this to the strength of the potential $\nu_{0}$ is to integrate Schrodingers equation 

\begin{align*}
-\int_{-\epsilon}^{+\epsilon}\frac{\hbar^{2}}{2m}\frac{d\psi^{2}}{dx^{2}}dx - \int_{-\epsilon}^{+\epsilon}\nu_{0}\delta(x)\psi(x)dx = 0
\end{align*}

as $\epsilon \rightarrow 0$. Ignoring the normalization, because it will cancel

\begin{align*}
\bigg|_{-\epsilon}^{+\epsilon}\frac{d\psi}{dx} = -2\kappa_{0} = -\frac{2m\nu_{0}}{\hbar^{2}}\psi(0)
\end{align*}

So we find that $\kappa_{0} = \nu_{0}m/\hbar^{2}$ so $E = -m\nu_{0}^{2}/2\hbar^{2}$. This is unique so we just have one bound state. There are of course unbound states when $E > 0$, for which the solutions would be complex exponentials.



\end{s}

\begin{p}
Problem 2.32 from Sakurai
\end{p}

\begin{s}
Let us define 

\begin{align*}
\psi_{I} &= A\exp(\alpha x)\\
\psi_{II} &= B\exp(ikx) + C\exp(-ikx)\\
\psi_{III} &= D\exp(-\alpha x)\\
\end{align*}

Here $\alpha,k$ are constants. We can enforce continuity in the wavefunction itself at $x = -a$ and $x=+a$

\begin{align*}
A\exp(-\alpha a) &= B\exp(-ika) + C\exp(ika)\\
D\exp(-\alpha a) &= B\exp(ika) + C\exp(-ika)\\
\end{align*}

And we can also enforce continuity in the first-order derivative at these points

\begin{align*}
\alpha A\exp(-\alpha a) &= ikB\exp(-ika) -ikC\exp(ika)\\
-\alpha D\exp(-\alpha a) &= ikB\exp(ika) -ikC\exp(-ika)\\
\end{align*}

This system of four equations can be written in matrix form

\begin{align*}
\begin{pmatrix}
e^{-\alpha a} & e^{-ika} & e^{ika} & 0\\
0 & e^{ika} & e^{-ika} & e^{-\alpha a}\\
\alpha e^{-\alpha a} & ike^{-ika} & ike^{ika} & 0\\
0 & ike^{ika} & -ike^{-ika} & -\alpha e^{-\alpha a}\\
\end{pmatrix}\begin{pmatrix}A\\B\\C\\D\end{pmatrix} = 0
\end{align*}

According to Mathematica, the determinant is 

\begin{align*}
\mathcal{D} = \exp(-2a(ik+\alpha))\left(-\exp(4iak)(k-i\alpha)^{2} + (k+i\alpha)^{2}\right)
\end{align*}

If the determinant is zero, then a solution exists. The determinant $\mathcal{D}$ is zero when

\begin{align*}
\exp(-2iak)(k+i\alpha)^{2} = \exp(2iak)(k-i\alpha)^{2}
\end{align*}

Notice that we have just distributed the $\exp(-2ika)$ from the prefactor. If we let $z = \exp(-iak)(k+i\alpha)$ then the above equation just reads $z^{2} = (z^{*})^{2}$ or $z = \pm z^{*}$. 

Considering the purely real solution first, we make the substitutions

\begin{align*}
\exp(-iak) \rightarrow \frac{\sqrt{k^{2} + \alpha^{2}}}{k+i\alpha}\\
\exp(iak) \rightarrow \frac{\sqrt{k^{2} + \alpha^{2}}}{k-i\alpha}
\end{align*}


\begin{align*}
\begin{pmatrix}
e^{-\alpha a} & \frac{-i\sqrt{k^{2} + \alpha^{2}}}{k+i\alpha} & \frac{\sqrt{k^{2} + \alpha^{2}}}{k-i\alpha} & 0\\
0 & \frac{\sqrt{k^{2} + \alpha^{2}}}{k-i\alpha} & \frac{-i\sqrt{k^{2} + \alpha^{2}}}{k+i\alpha} & e^{-\alpha a}\\
\alpha e^{-\alpha a} & ik\frac{-i\sqrt{k^{2} + \alpha^{2}}}{k+i\alpha} & ik\frac{\sqrt{k^{2} + \alpha^{2}}}{k-i\alpha} & 0\\
0 & ik\frac{\sqrt{k^{2} + \alpha^{2}}}{k-i\alpha} & -ik\frac{-i\sqrt{k^{2} + \alpha^{2}}}{k+i\alpha} & -\alpha e^{-\alpha a}\\
\end{pmatrix}\begin{pmatrix}A\\B\\C\\D\end{pmatrix} = 0
\end{align*}

Now considering the purely imaginary solution, we make the substitutions

\begin{align*}
\exp(-iak) \rightarrow \frac{i\sqrt{k^{2} + \alpha^{2}}}{k+i\alpha}\\
\exp(iak) \rightarrow \frac{-i\sqrt{k^{2} + \alpha^{2}}}{k-i\alpha}
\end{align*}

\begin{align*}
\begin{pmatrix}
e^{-\alpha a} & \frac{i\sqrt{k^{2} + \alpha^{2}}}{k+i\alpha} & \frac{-i\sqrt{k^{2} + \alpha^{2}}}{k-i\alpha} & 0\\
0 & \frac{-i\sqrt{k^{2} + \alpha^{2}}}{k-i\alpha} & \frac{i\sqrt{k^{2} + \alpha^{2}}}{k+i\alpha} & e^{-\alpha a}\\
\alpha e^{-\alpha a} & ik\frac{i\sqrt{k^{2} + \alpha^{2}}}{k+i\alpha} & ik\frac{-i\sqrt{k^{2} + \alpha^{2}}}{k-i\alpha} & 0\\
0 & ik\frac{-i\sqrt{k^{2} + \alpha^{2}}}{k-i\alpha} & -ik\frac{i\sqrt{k^{2} + \alpha^{2}}}{k+i\alpha} & -\alpha e^{-\alpha a}\\
\end{pmatrix}\begin{pmatrix}A\\B\\C\\D\end{pmatrix} = 0
\end{align*}


\end{s}

\end{document}