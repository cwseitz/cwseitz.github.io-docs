\documentclass[12pt]{article}
\usepackage{amsmath} % AMS Math Package
\usepackage{bm}
\usepackage{amsthm} % Theorem Formatting
\usepackage{amssymb}    % Math symbols such as \mathbb
\usepackage{graphicx} % Allows for eps images
\usepackage[dvips,letterpaper,margin=1in,bottom=0.7in]{geometry}
\usepackage{tensor}
\usepackage{amsmath}
\usepackage{siunitx}
\usepackage{physics}
\usepackage{amsmath, amssymb, graphics, setspace}

\newcommand{\mathsym}[1]{{}}
\newcommand{\unicode}[1]{{}}

\newcounter{mathematicapage}

\newtheorem{p}{Problem}
\usepackage{cancel}
\newtheorem*{lem}{Lemma}
\theoremstyle{definition}
\newtheorem*{dfn}{Definition}
 \newenvironment{s}{%\small%
        \begin{trivlist} \item \textbf{Solution}. }{%
            \hspace*{\fill} $\blacksquare$\end{trivlist}}%


\begin{document}

 {\noindent\Huge\bf  \\[0.5\baselineskip] {\fontfamily{cmr}\selectfont  Homework 4}         }\\[2\baselineskip] % Title
{ {\bf \fontfamily{cmr}\selectfont Quantum Mechanics}\\ {\textit{\fontfamily{cmr}\selectfont     Sept 22nd, 2022}}}~~~~~~~~~~~~~~~~~~~~~~~~~~~~~~~~~~~~~~~~~~~~~~~~~~~~~~~~~~~~~~~~~~~~~~~~~~~~~    {\large \textsc{Clayton Seitz}
\\[1.4\baselineskip] 

\begin{p}
Problem 2.14 from Sakurai
\end{p}

\begin{s}

We are given that the state vector is 

\begin{equation*}
\ket{\alpha} = \exp\left(\frac{-ipa}{\hbar}\right)\ket{0}
\end{equation*}

The Heisenberg equation of motion reads

\begin{align*}
\frac{dx}{dt} = \frac{1}{i\hbar}\left[x,H\right] = 0
\end{align*}

Therefore $x = x_{0}$ for all $t \geq t_{0}$

\begin{align*}
\langle x \rangle &= \int x_{0} \bra{x}\ket{\alpha}\bra{\alpha}\ket{x}dx\\
&= \int x \exp\left(\frac{-ipa}{\hbar}\right)\bra{x}\ket{0}\exp\left(\frac{ipa}{\hbar}\right)\bra{0}\ket{x}dx\\
&= \int x_{0} \;|\bra{x}\ket{0}|^{2}\;dx\\
&= \int x_{0} \;|\bra{x}\ket{0}|^{2}\;dx\\
\end{align*}

We could write out $\bra{x}\ket{0}$, its complex conjugate, and do the integral. Instead recall the general expression for the matrix element of $x$

\begin{align*}
\bra{n'}x\ket{n} = \sqrt{\frac{\hbar}{2m\omega}}\left(\sqrt{n}\delta_{n',n-1} +\sqrt{n+1}\delta_{n',n+1}\right)
\end{align*}

which is zero when $n = n'$ which means that $\langle x \rangle = 0$

\end{s}

\begin{p}
Problem 2.15 from Sakurai
\end{p}

\begin{s}
\end{s}

\begin{p}
Problem 2.16 from Sakurai
\end{p}

\begin{s}
\end{s}

\begin{p}
Problem 2.28 from Sakurai
\end{p}

\begin{s}
\end{s}

\begin{p}
Problem 2.29 from Sakurai
\end{p}

\begin{s}
\end{s}

\begin{p}
Problem 2.32 from Sakurai
\end{p}

\begin{s}
\end{s}

\end{document}