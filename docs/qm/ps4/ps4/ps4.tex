\documentclass[12pt]{article}
\usepackage{amsmath} % AMS Math Package
\usepackage{bm}
\usepackage{amsthm} % Theorem Formatting
\usepackage{amssymb}    % Math symbols such as \mathbb
\usepackage{graphicx} % Allows for eps images
\usepackage[dvips,letterpaper,margin=1in,bottom=0.7in]{geometry}
\usepackage{tensor}
\usepackage{amsmath}
\usepackage{siunitx}
\usepackage{physics}
\usepackage{amsmath, amssymb, graphics, setspace}

\newcommand{\mathsym}[1]{{}}
\newcommand{\unicode}[1]{{}}

\newcounter{mathematicapage}

\newtheorem{p}{Problem}
\usepackage{cancel}
\newtheorem*{lem}{Lemma}
\theoremstyle{definition}
\newtheorem*{dfn}{Definition}
 \newenvironment{s}{%\small%
        \begin{trivlist} \item \textbf{Solution}. }{%
            \hspace*{\fill} $\blacksquare$\end{trivlist}}%


\begin{document}

 {\noindent\Huge\bf  \\[0.5\baselineskip] {\fontfamily{cmr}\selectfont  Homework 4}         }\\[2\baselineskip] % Title
{ {\bf \fontfamily{cmr}\selectfont Quantum Mechanics}\\ {\textit{\fontfamily{cmr}\selectfont     Sept 22nd, 2022}}}~~~~~~~~~~~~~~~~~~~~~~~~~~~~~~~~~~~~~~~~~~~~~~~~~~~~~~~~~~~~~~~~~~~~~~~~~~~~~    {\large \textsc{Clayton Seitz}
\\[1.4\baselineskip] 

\begin{p}
Problem 2.14 from Sakurai
\end{p}

\begin{s}

We are given that the state vector is 

\begin{equation*}
\ket{\alpha} = \exp\left(\frac{-ipa}{\hbar}\right)\ket{0}
\end{equation*}

The Heisenberg equation of motion reads

\begin{align*}
\frac{dx}{dt} = \frac{1}{i\hbar}\left[x,H\right] = 0
\end{align*}

Therefore $x = x_{0}$ for all $t \geq t_{0}$

\begin{align*}
\langle x \rangle &= \int x_{0} \bra{x}\ket{\alpha}\bra{\alpha}\ket{x}dx\\
&= \int x \exp\left(\frac{-ipa}{\hbar}\right)\bra{x}\ket{0}\exp\left(\frac{ipa}{\hbar}\right)\bra{0}\ket{x}dx\\
&= \int x_{0} \;|\bra{x}\ket{0}|^{2}\;dx\\
&= \int x_{0} \;|\bra{x}\ket{0}|^{2}\;dx\\
\end{align*}

We could write out $\bra{x}\ket{0}$, its complex conjugate, and do the integral. Instead recall the general expression for the matrix element of $x$

\begin{align*}
\bra{n'}x\ket{n} = \sqrt{\frac{\hbar}{2m\omega}}\left(\sqrt{n}\delta_{n',n-1} +\sqrt{n+1}\delta_{n',n+1}\right)
\end{align*}

which is zero when $n = n'$ which means that $\langle x \rangle = 0$

\end{s}

\begin{p}
Problem 2.15 from Sakurai
\end{p}

\begin{s}
We were given the state

\begin{align*}
\ket{\alpha} = \exp\left(\frac{-ipa}{\hbar}\right)\ket{0}
\end{align*}

\begin{align*}
\bra{x}\ket{\alpha} =\pi^{-1/4}x_{0}^{1/2} \exp\left(\frac{-ipa}{\hbar}\right)\exp\left(-\frac{1}{2}\left(\frac{x}{x_{0}}\right)^{2}\right)
\end{align*}
where $x_{0} = \sqrt{\frac{\hbar}{m\omega}}$. The Hamiltonian operator $\hat{H}$ is independent of time so we have the unitary time evolution operator 

\begin{align*}
\mathcal{U}(t) = \exp\left(-\frac{i\hat{H}t}{\hbar}\right)
\end{align*}

Assuming $\ket{\alpha}$ is expressed in the energy basis, this can be alternatively be written as the power series

\begin{align*}
\mathcal{U}(t) = \sum_{n=0}^{\infty} \frac{\hat{H}^{n}}{n!}\rightarrow \mathcal{U}(t)\ket{\alpha} = \sum_{n=0}^{\infty} \frac{\hat{H}^{n}}{n!}\ket{\alpha}
\end{align*}

\begin{align*}
\sum_{n=0}^{\infty} \frac{\alpha^{n}}{n!}\ket{\alpha} = \sum_{n}\exp\left(\frac{-i\alpha_{n}t}{\hbar}\right)\ket{\alpha_{n}}
\end{align*}


The probability that $\ket{\alpha}$ is measured to be in the state $\ket{0}$ is

\begin{align*}
\bra{0}\ket{\alpha}\bra{\alpha}\ket{0} = \exp\left(\frac{-ipa}{\hbar}\right)\bra{0}\ket{0} \exp\left(\frac{ipa}{\hbar}\right)\bra{0}\ket{0} = 1
\end{align*}

This probability does not change for $t > 0$. This is clear when we look at the state

\begin{align*}
\ket{\alpha;t} = \exp\left(-\frac{iE_{0}t}{\hbar}\right)\exp\left(\frac{-ipa}{\hbar}\right)\ket{0}
\end{align*}

The second exponential is just a complex number and is time independent. The first exponential is just a phase, which is not measurable directly. In other words, when we hit this state with the dual ket $\bra{0}$, the phase goes away and we are left with a time-independent probability density.

\end{s}

\begin{p}
Problem 2.16 from Sakurai
\end{p}

\begin{s}

We will assume the form of the annihilation and creation operators

\begin{align*}
a &= \sqrt{\frac{m\omega}{2\hbar}}\left(x + \frac{ip}{m\omega}\right)\\
a^{\dagger} &= \sqrt{\frac{m\omega}{2\hbar}}\left(x - \frac{ip}{m\omega}\right)
\end{align*}

Adding these equations gives and rearranging we can express $x$ as

\begin{align*}
x = \sqrt{\frac{\hbar}{2m\omega}}\left(a + a^{\dagger}\right)
\end{align*}

\begin{align*}
\bra{m}x\ket{n} &= \bra{m} \sqrt{\frac{\hbar}{2m\omega}}\left(a + a^{\dagger}\right)\ket{n}\\
&= \sqrt{\frac{\hbar}{2m\omega}} \left(\bra{m}a\ket{n} + \bra{m}a^{\dagger}\ket{n}\right)\\
&= \sqrt{\frac{\hbar}{2m\omega}} \left(\sqrt{n}\delta_{m,n-1} + \sqrt{n+1}\delta_{m,n+1}\right)\\
\end{align*}

Subtracting the creation operator from the annihalation operator allows us to write the momentum operator as

\begin{align*}
p &= i\sqrt{\frac{m\hbar \omega}{2}}\left(a^{\dagger}-a\right)
\end{align*}

\begin{align*}
\bra{m}p\ket{n} &= \bra{m}\left(i\sqrt{\frac{m\hbar \omega}{2}}\left(a^{\dagger}-a\right)\right)\ket{n}\\
&= \left(i\sqrt{\frac{m\hbar \omega}{2}}\left(\bra{m}a^{\dagger}\ket{n}-\bra{m}a\ket{n}\right)\right)\\
&= i\sqrt{\frac{m\hbar \omega}{2}}\left(\sqrt{n+1}\delta_{m,n+1}-\sqrt{n}\delta_{m,n-1}\right)\\
\end{align*}
\begin{align*}
\bra{m}\{x,p\}\ket{n} &= \bra{m}xp\ket{n} + \bra{m}px\ket{n}\\
&= \frac{i\hbar}{2}\bra{m}\left((a^{\dagger})^{2}-a^{2}\right)\ket{n} + \frac{i\hbar}{2}\bra{m}\left((a^{\dagger})^{2} + a^{\dagger}a - aa^{\dagger} - a^{2}\right)\ket{n}\\
&= \frac{i\hbar}{2}\left(\sqrt{n+1}\sqrt{n+2}\delta_{m,n+2} - \sqrt{n}\sqrt{n-1}\delta_{m,n-2}\right)\\
&+ \frac{i\hbar}{2}(\sqrt{n+1}\sqrt{n+2}\delta_{m,n+2} + \sqrt{n}\sqrt{n-1}\delta_{m,n-2})
\end{align*}
\begin{align*}
\bra{m}x^{2}\ket{n} &= \frac{\hbar}{2m\omega}\bra{m}\left(a^{2} + aa^{\dagger} + a^{\dagger} a + (a^{\dagger})^{2}\right)\ket{n}
\end{align*}
\begin{align*}
\bra{m}p^{2}\ket{n} = -\frac{m\hbar \omega}{2}\bra{m}\left((a^{\dagger})^{2} + a^{\dagger}a - aa^{\dagger}-a^{2}\right)\ket{n}
\end{align*}
\end{s}

\begin{p}
Problem 2.28 from Sakurai
\end{p}

\begin{s}

First of all, the solution is not trivial since $x$ does not commute with the Hamiltonian since $[x,p^{2}] \neq 0$. At $t=t_{0}$ we are in the position eigenstate

\begin{align*}
\bra{x}\ket{\alpha; t_{0}} = \delta\left(x-\frac{L}{2}\right)
\end{align*}

Since this is the infinite square well, we have the following energy eigenstates, in the position representation

\begin{align*}
\bra{x}\ket{\alpha} = \sqrt{\frac{2}{L}}\sin\left(\frac{n\pi x}{L}\right)
\end{align*}

Of course $\ket{\alpha; t_{0}}$ is not an eigenstate of $H$, so this state will measurably evolve in time. The state $\ket{\alpha; t_{0}}$ in the energy basis is

\begin{align*}
\ket{\beta} &= \sum_{n}\ket{\epsilon_{n}}\bra{\epsilon_{n}}\ket{\alpha;t_{0}}\\
&=  \sqrt{\frac{2}{L}}\sum_{n}\sin\left(\frac{n\pi}{2}\right)\ket{\epsilon_{n}}
\end{align*}

From this, we can show the probability of measuring the particle in energy eigenstate $\ket{\epsilon_{n}}$

\begin{align*}
\bra{\epsilon_{m}}\ket{\beta} &= \sqrt{\frac{2}{L}}\sum_{n}\sin\left(\frac{n\pi}{2}\right)\bra{\epsilon_{m}}\ket{\epsilon_{n}}\\
&= \sqrt{\frac{2}{L}}\sum_{n}\sin\left(\frac{n\pi}{2}\right)\delta_{mn}\\
&= \sqrt{\frac{2}{L}}\sin\left(\frac{m\pi}{2}\right)\\
\end{align*}

and therefore $|\bra{\epsilon_{m}}\ket{\beta}|^{2} = \sqrt{\frac{2}{L}}\sin^{2}\left(\frac{m\pi}{2}\right)$. The relative probabilities with respect to the ground state are then given by

\begin{align*}
r_{m+1} = \sin^{2}\left(\frac{(m+1)\pi}{2}\right)\csc^{2}\left(\frac{m\pi}{2}\right)
\end{align*}


Since we know a representation for $\ket{\alpha;t_{0}}$ in the energy basis, we can determine the time evolution of the wavefunction $\bra{x}\ket{\alpha}$

\begin{align*}
\ket{\alpha;t} &= \mathcal{U}(t)\ket{\beta} \\
&= \sqrt{\frac{2}{L}}\sum_{n}\exp\left(\frac{-i\epsilon_{n}t}{\hbar}\right)\sin\left(\frac{n\pi}{2}\right)\ket{\epsilon_{n}}
\end{align*}

which has the position representation (wave function)

\begin{align*}
\bra{x}\ket{\alpha; t} &= \psi(x,t)\\
&= \sqrt{\frac{2}{L}}\sum_{n}\exp\left(\frac{-i\epsilon_{n}t}{\hbar}\right)\sin\left(\frac{n\pi}{2}\right)\bra{x}\ket{\epsilon_{n}}\\
&= \sqrt{\frac{2}{L}}\sum_{n}\exp\left(\frac{-i\epsilon_{n}t}{\hbar}\right)\sin\left(\frac{n\pi}{2}\right)\psi_{n}(x)
\end{align*}

where $\psi_{n}(x)$ are the energy eigenstates given above.

\end{s}

\begin{p}
Problem 2.29 from Sakurai
\end{p}

\begin{s}
\end{s}

\begin{p}
Problem 2.32 from Sakurai
\end{p}

\begin{s}
\end{s}

\end{document}