\documentclass[12pt]{article}
\usepackage{amsmath} % AMS Math Package
\usepackage{bm}
\usepackage{amsthm} % Theorem Formatting
\usepackage{amssymb}    % Math symbols such as \mathbb
\usepackage{graphicx} % Allows for eps images
\usepackage[dvips,letterpaper,margin=1in,bottom=0.7in]{geometry}
\usepackage{tensor}
\usepackage{amsmath}
\usepackage{siunitx}
\usepackage{physics}

\newtheorem{p}{Problem}
\usepackage{cancel}
\newtheorem*{lem}{Lemma}
\theoremstyle{definition}
\newtheorem*{dfn}{Definition}
 \newenvironment{s}{%\small%
        \begin{trivlist} \item \textbf{Solution}. }{%
            \hspace*{\fill} $\blacksquare$\end{trivlist}}%


\begin{document}

{\noindent\Huge\bf  \\[0.5\baselineskip] {\fontfamily{cmr}\selectfont  Exam 1}         }\\[2\baselineskip] % Title
{ {\bf \fontfamily{cmr}\selectfont Quantum Mechanics}\\ {\textit{\fontfamily{cmr}\selectfont     October 14th, 2022}}}~~~~~~~~~~~~~~~~~~~~~~~~~~~~~~~~~~~~~~~~~~~~~~~~~~~~~~~~~~~~~~~~~~~~~~~~~~~~~    {\large \textsc{C Seitz}
\\[1.4\baselineskip] 

\begin{p}
\end{p}

\begin{s}
Both operators are Hermitian, so their eigenvalues are real. The eigenvectors of operator $A$ are

\begin{equation*}
\ket{a_{1}} = \frac{1}{2}\left(
\begin{array}{ccc}
 -1 \\i \sqrt{2} \\ 1 \\
\end{array}
\right)\;
\ket{a_{2}} = \frac{1}{\sqrt{2}}\left(
\begin{array}{ccc}
 1 \\ 0 \\ 1 \\
\end{array}
\right)\;
\ket{a_{3}} = \frac{1}{2}\left(
\begin{array}{ccc}
 -1 \\ -i \sqrt{2} \\ 1 \\
\end{array}
\right)
\end{equation*}

with eigenvalues $3\lambda, 2\lambda, \lambda$, in that order. Notice that $\ket{a_{1}}$ and $\ket{a_{3}}$ are not orthogonal in this basis. The eigenvectors of operator $B$ are

\begin{equation*}
\ket{b_{1}} = \frac{1}{2}\left(
\begin{array}{ccc}
 1 \\\sqrt{2} \\ 1 \\
\end{array}
\right)\;
\ket{b_{2}} = \frac{1}{\sqrt{2}}\left(
\begin{array}{ccc}
 -1 \\ 0 \\ 1 \\
\end{array}
\right)\;
\ket{b_{3}} = \frac{1}{2}\left(
\begin{array}{ccc}
 1 \\ -\sqrt{2} \\ 1 \\
\end{array}
\right)
\end{equation*}

with eigenvalues $5\lambda, 3\lambda, \lambda$, in that order. If the physicist sends particles in state $\ket{1}$ and we measure $A$, the probability we observe the particle to be in the state $\ket{a_{1}}$, $\ket{a_{2}}$, and $\ket{a_{3}}$ can be found by using the expressions for eigenkets of $A$ above. These probabilities are 

\begin{align*}
|\bra{a_{1}}\ket{1}|^{2} &= \frac{1}{4}\\
|\bra{a_{2}}\ket{1}|^{2} &= \frac{1}{2}\\
|\bra{a_{3}}\ket{1}|^{2} &= \frac{1}{4}
\end{align*}

The values of $A$ that correspond with each beam are given by the eigenvalues of $A$ written above. The relative intensities are just given by the relative probabilities. After the coffee spill, measuring $A$ filters out only the beam corresponding to state $\ket{a_{2}}$. Therefore, these are the only particles that enter the apparatus measuring $B$. So when we measure $B$, the number of beams we get depends on the inner products

\begin{align*}
|\bra{b_{1}}\ket{a_{2}}|^{2} &= \frac{1}{2}\\
|\bra{b_{2}}\ket{a_{2}}|^{2} &= 0\\
|\bra{b_{3}}\ket{a_{2}}|^{2} &= \frac{1}{2}
\end{align*}

so there are only two beams. The intensities are equal for these two beams. The values of $B$ that correspond with these two beams are $b_{1}$ and $b_{3}$.
\end{s}

\begin{p}
\end{p}

\begin{s}
We are given the commutation relations between the operators $C$ and $D$ and the Hamiltonian, which suggests we should use the Heisenberg picture. The Heisenberg equations of motion are

\begin{align*}
\frac{dC}{dt} &= -\frac{1}{i\hbar}[H,C] = \alpha D - \beta C\\
\frac{dD}{dt} &= -\frac{1}{i\hbar}[H,D] = -\alpha C - \beta D
\end{align*}

When $\beta = 0$, the system becomes

\begin{align*}
\frac{dC}{dt} &= -\frac{1}{i\hbar}[H,C] = \alpha D\\
\frac{dD}{dt} &= -\frac{1}{i\hbar}[H,D] = -\alpha C
\end{align*}

which has the solution

\begin{align*}
C(t) &= -c_{0}\cos(\alpha t)\\
D(t) &= d_{0}\sin(\alpha t)
\end{align*}

\begin{align*}
\langle C(t) \rangle &= \bra{\alpha}C(t)\ket{\alpha} \\
&= -\bra{\alpha}c_{0}\cos(\alpha t)\ket{\alpha}\\
&= -c_{0}\cos(\alpha t)
\end{align*}

since $\ket{\alpha}$ is presumed to be normalized.


\begin{align*}
\langle D(t) \rangle &= \bra{\alpha}D(t)\ket{\alpha} \\
&= -\bra{\alpha}d_{0}\sin(\alpha t)\ket{\alpha}\\
&= d_{0}\sin(\alpha t)
\end{align*}

since $\ket{\alpha}$ is again presumed to be normalized. The constants $\alpha$ and $\beta$ must then relate to the angular frequency of $C(t)$ and $D(t)$. Of course, when $\beta = 0$, only $\alpha$ determines the angular frequency of $C(t)$ and $D(t)$.

\end{s}

\begin{p}
\end{p}

\begin{s}

We are given the wavefunction

\begin{align*}
\psi(r,\phi) = Ae^{-br^{2}}
\end{align*}

We know that probability current is related to the gradient of the phase of the wavefunction. Regardless of what $A$ is (purely real, imaginary, or complex) it is a constant. The exponential is real if $b$ is real, so the phase of $\psi(r,\phi)$ is the same for all $(r,\phi)$ and therefore the probability current is zero everywhere. For the wavefunction

\begin{align*}
\psi(r,\phi) = Ae^{-br^{2}}e^{-im\phi}
\end{align*}

The phase of the wavefunction is clearly dependent on $\phi$, so there is a probability current.  

\begin{align*}
\rho(r,\phi) = |A|^{2}e^{-2br^{2}}
\end{align*}

We can always write the wavefunction in the form:

\begin{align*}
\psi(r,\phi) &= \sqrt{\rho(r,\phi)}\exp\left(\frac{iS(r,\phi)}{\hbar}\right)\\
&= Ae^{-br^{2}}e^{-im\phi}
\end{align*}

so $-im\phi = iS/\hbar$. The phase of the wavefunction is then

\begin{align*}
S(r,\phi) = -m\hbar\phi
\end{align*}

The probability flux is related to $S(r,\phi)$ by

\begin{align*}
\bm{j}(r,\phi) &= \frac{\rho(r,\phi)\nabla S}{m}\\
&= -\hbar\rho(r,\phi)\hat{\phi}
\end{align*}

However, $\rho(r,\phi)$ is static because of the continuity equation

\begin{align*}
\frac{\partial \rho}{\partial t} = -\nabla \cdot \bm{j} = 0
\end{align*}

since the divergence of $\bm$ is surely zero. In the limit $r\rightarrow\infty$, we can see that 

\begin{align*}
\underset{r\rightarrow\infty}{\mathrm{lim}}\;\bm{j}(r,\phi) &= \underset{r\rightarrow\infty}{\mathrm{lim}}\; -\hbar|A|^{2}e^{-2br^{2}} = 0
\end{align*}

which makes sense if the particle is localized to some region of space. This situation might correspond to a particle with spin.

\end{s}



\end{document}