\documentclass[12pt]{article}
\usepackage{amsmath} % AMS Math Package
\usepackage{bm}
\usepackage{amsthm} % Theorem Formatting
\usepackage{amssymb}    % Math symbols such as \mathbb
\usepackage{graphicx} % Allows for eps images
\usepackage[dvips,letterpaper,margin=1in,bottom=0.7in]{geometry}
\usepackage{tensor}
\usepackage{amsmath}
\usepackage{siunitx}
\usepackage{physics}
\usepackage{amsmath, amssymb, graphics, setspace}

\newcommand{\mathsym}[1]{{}}
\newcommand{\unicode}[1]{{}}

\newcounter{mathematicapage}

\newtheorem{p}{Problem}
\usepackage{cancel}
\newtheorem*{lem}{Lemma}
\theoremstyle{definition}
\newtheorem*{dfn}{Definition}
 \newenvironment{s}{%\small%
        \begin{trivlist} \item \textbf{Solution}. }{%
            \hspace*{\fill} $\blacksquare$\end{trivlist}}%


\begin{document}

 {\noindent\Huge\bf  \\[0.5\baselineskip] {\fontfamily{cmr}\selectfont  Homework 8}         }\\[2\baselineskip] % Title
{ {\bf \fontfamily{cmr}\selectfont Quantum Mechanics}\\ {\textit{\fontfamily{cmr}\selectfont     \today}}}~~~~~~~~~~~~~~~~~~~~~~~~~~~~~~~~~~~~~~~~~~~~~~~~~~~~~~~~~~~~~~~~~~~~~~~~~~~~~    {\large \textsc{C Seitz}
\\[1.4\baselineskip] 

\begin{p}
5.27
\end{p}

\begin{s}
\begin{equation*}
\frac{\bra{\tilde{0}}H\ket{\tilde{0}}}{\bra{\tilde{0}}\ket{\tilde{0}}} \geq E_{0}
\end{equation*}

The denominator is easy to compute

\begin{equation*}
2\int_{-\infty}^{0}e^{\beta x}dx = \frac{1}{\beta}
\end{equation*}

The numerator 

\begin{align*}
\bra{\tilde{0}}H\ket{\tilde{0}} &= \int_{-\infty}^{\infty}\psi^{*}(x)H\psi(x)dx\\
&= \int_{-\infty}^{0}\psi^{*}(x)H\psi(x)dx + \int_{0}^{\infty}\psi^{*}(x)H\psi(x)dx
\end{align*}

\begin{align*}
\int_{-\infty}^{0}\psi^{*}(x)H\psi(x)dx &= \int_{-\infty}^{0}-e^{\beta x}\frac{\hbar^{2}}{2m}\frac{\partial^{2}}{\partial x^{2}}e^{\beta x} + \frac{1}{2}m\omega^{2}x^{2}e^{2\beta x}dx\\
&= \int_{-\infty}^{0}e^{2\beta x}\left(\frac{1}{2}m\omega^{2}x^{2}-\frac{\hbar^{2}\beta^{2}}{2m}\right)dx\\
&= \Big|_{-\infty}^{0}\frac{1}{2}m\omega^{2}\frac{e^{2\beta x}(1 - 2\beta x + 2\beta^2 x^2)}{4 \beta^3} - e^{2\beta x}\frac{\hbar^{2}\beta}{4m}\\
&= \frac{1}{2}m\omega^{2}\frac{1}{4 \beta^3} - \frac{\hbar^{2}\beta}{4m}
\end{align*}

\begin{align*}
\int_{0}^{\infty}\psi^{*}(x)H\psi(x)dx &= \int_{0}^{\infty}-e^{-\beta x}\frac{\hbar^{2}}{2m}\frac{\partial^{2}}{\partial x^{2}}e^{-\beta x} + \frac{1}{2}m\omega^{2}x^{2}e^{-2\beta x}dx\\
&= \int_{0}^{\infty}e^{-2\beta x}\left(\frac{1}{2}m\omega^{2}x^{2}-\frac{\hbar^{2}\beta^{2}}{2m}\right)dx\\
&= \Big|_{0}^{\infty}\frac{1}{2}m\omega^{2}\frac{e^{-2\beta x}(1 + 2\beta x + 2\beta^2 x^2)}{4 \beta^3} - e^{-2\beta x}\frac{\hbar^{2}\beta}{4m}\\
&= \frac{1}{2}m\omega^{2}\frac{1}{4 \beta^3} - \frac{\hbar^{2}\beta}{4m}
\end{align*}

\begin{align*}
\bar{H} = \frac{\bra{\tilde{0}}H\ket{\tilde{0}}}{\bra{\tilde{0}}\ket{\tilde{0}}} = \frac{m\omega^{2}}{4\beta^{2}} - \frac{\hbar^{2}\beta^{2}}{2m}
\end{align*}

\begin{align*}
\frac{d\bar{H}}{d\beta} =  -\frac{m\omega^{2}}{4\beta} - \frac{\hbar^{2}\beta}{m} = 0
\end{align*}

\end{s}

\begin{p}
5.29
\end{p}

\begin{s}

We have the full time-dependent Hamiltonian

\begin{align*}
H(t) = H_{0} + F_{0}x\cos\omega t
\end{align*}

We need to find $\ket{\psi(t)}$, which amounts to finding the expansion coefficients $c_{n}(t)$. In the interaction picture, we have that

\begin{align*}
i\hbar\dot{c_{n}}(t) &= \sum_{m}V_{nm}e^{i\omega_{nm}t}c_{m}(t)
\end{align*}

for $\omega_{nm} = (E_{n}-E_{m})/\hbar$.

\begin{align*}
V_{nm} &= F_{0}\cos\omega t\bra{n}x\ket{m}\\
&= F_{0}\cos\omega t\sqrt{\frac{\hbar}{2m\omega_{0}}}\left(\sqrt{n+1}\delta_{m,n-1} + \sqrt{n}\delta_{m,n+1}\right)
\end{align*}

But the initial condition says that $\ket{\psi(0)} = \ket{0}$, so $n=0$ and the only term of the summation that survives has $m=1$. Therefore,

\begin{align*}
i\hbar\dot{c_{1}}(t) &= V_{10}e^{i\omega_{0}t}c_{0}(t)\\
&= \frac{F_{0}\cos\omega t}{2}\sqrt{\frac{\hbar}{2m\omega_{0}}}e^{i\omega_{0}t}c_{0}(t)
\end{align*}

Solving for $c_{1}(t)$, 

\begin{align*}
c_{1}(t) &= -\frac{i}{\hbar}F_{0}\sqrt{\frac{\hbar}{2m\omega_{0}}}\int_{0}^{t}e^{i\omega_{0}t}\cos\omega t dt\\
&= -\frac{i}{2\hbar}F_{0}\sqrt{\frac{\hbar}{2m\omega_{0}}}\left(\frac{e^{i(\omega_{0}+\omega)t}-1}{\omega_{0}+\omega}+\frac{e^{i(\omega_{0}-\omega)t}-1}{\omega_{0}-\omega}\right)
\end{align*}

Now, to compute $\langle x \rangle$, we can express the $x$ operator in the interaction picture (or, equivalently, convert the $\ket{\tilde{\psi(t)}}$ back to $\ket{\psi(t)}$).

\begin{align*}
\langle x \rangle &= \bra{\psi(t)} x \ket{\psi(t)} \\
&= \bra{\tilde{\psi(t)}}e^{iH_{0}t/\hbar} x e^{-iH_{0}t/\hbar}\ket{\tilde{\psi(t)}}\\
&= \sqrt{\frac{\hbar}{2m\omega_{0}}}\left(\bra{0}c_{0}^{*}e^{i\omega_{0}t/2} + \bra{1}e^{3i\omega_{0}t/2}c_{1}^{*}(t)\right)(a+a^{\dagger})\left(e^{-i\omega_{0}t/2}c_{0}\ket{0} + e^{-3i\omega_{0}t/2}c_{1}(t)\ket{1}\right)\\
&= \sqrt{\frac{\hbar}{2m\omega_{0}}}\left(c_{1}(t)e^{-i\omega_{0}t} + c_{1}^{*}(t)e^{i\omega_{0}t}\right)
\end{align*}

\end{s}

\begin{p}
5.30
\end{p}

\begin{s}
\end{s}

\begin{p}
5.32
\end{p}

\begin{s}
\end{s}

\begin{p}
5.35
\end{p}

\begin{s}
\end{s}

\begin{p}
5.36
\end{p}

\begin{s}
\end{s}

\end{document}