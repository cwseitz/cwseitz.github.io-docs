\documentclass[12pt]{article}
\usepackage{amsmath} % AMS Math Package
\usepackage{bm}
\usepackage{amsthm} % Theorem Formatting
\usepackage{amssymb}    % Math symbols such as \mathbb
\usepackage{graphicx} % Allows for eps images
\usepackage[dvips,letterpaper,margin=1in,bottom=0.7in]{geometry}
\usepackage{tensor}
\usepackage{amsmath}
\usepackage{siunitx}
\usepackage{physics}
\usepackage{amsmath, amssymb, graphics, setspace}

\newcommand{\mathsym}[1]{{}}
\newcommand{\unicode}[1]{{}}

\newcounter{mathematicapage}

\newtheorem{p}{Problem}
\usepackage{cancel}
\newtheorem*{lem}{Lemma}
\theoremstyle{definition}
\newtheorem*{dfn}{Definition}
 \newenvironment{s}{%\small%
        \begin{trivlist} \item \textbf{Solution}. }{%
            \hspace*{\fill} $\blacksquare$\end{trivlist}}%


\begin{document}

 {\noindent\Huge\bf  \\[0.5\baselineskip] {\fontfamily{cmr}\selectfont  Homework 8}         }\\[2\baselineskip] % Title
{ {\bf \fontfamily{cmr}\selectfont Quantum Mechanics}\\ {\textit{\fontfamily{cmr}\selectfont     \today}}}~~~~~~~~~~~~~~~~~~~~~~~~~~~~~~~~~~~~~~~~~~~~~~~~~~~~~~~~~~~~~~~~~~~~~~~~~~~~~    {\large \textsc{C Seitz}
\\[1.4\baselineskip] 

\begin{p}
5.27
\end{p}

\begin{s}
\begin{equation*}
\frac{\bra{\tilde{0}}H\ket{\tilde{0}}}{\bra{\tilde{0}}\ket{\tilde{0}}} \geq E_{0}
\end{equation*}

The denominator is easy to compute

\begin{equation*}
2\int_{-\infty}^{0}e^{\beta x}dx = \frac{1}{\beta}
\end{equation*}

The numerator 

\begin{align*}
\bra{\tilde{0}}H\ket{\tilde{0}} &= \int_{-\infty}^{\infty}\psi^{*}(x)H\psi(x)dx\\
&= 2\int_{0}^{\infty}\psi^{*}(x)H\psi(x)dx - \frac{\hbar^{2}}{2m}\;\underset{\epsilon \rightarrow 0}{\mathrm{lim}}\int_{-\epsilon}^{\epsilon}dx e^{-\beta|x|}\frac{\partial^{2}}{\partial x^{2}}e^{-\beta|x|}
\end{align*}

The first integral is relatively straighforward:

\begin{align*}
\int_{0}^{\infty}\psi^{*}(x)H\psi(x)dx &= \int_{0}^{\infty}-e^{-\beta x}\frac{\hbar^{2}}{2m}\frac{\partial^{2}}{\partial x^{2}}e^{-\beta x} + \frac{1}{2}m\omega^{2}x^{2}e^{-2\beta x}dx\\
&= \int_{0}^{\infty}e^{-2\beta x}\left(\frac{1}{2}m\omega^{2}x^{2}-\frac{\hbar^{2}\beta^{2}}{2m}\right)dx\\
&= \Big|_{0}^{\infty}\frac{1}{2}m\omega^{2}\frac{e^{-2\beta x}(1 + 2\beta x + 2\beta^2 x^2)}{4 \beta^3} - e^{-2\beta x}\frac{\hbar^{2}\beta}{4m}\\
&= \frac{m\omega^{2}}{8 \beta^3} - \frac{\hbar^{2}\beta}{4m}
\end{align*}

The second is 

\begin{align*}
\underset{\epsilon \rightarrow 0}{\mathrm{lim}}\int_{-\epsilon}^{\epsilon}dx e^{-\beta|x|}\frac{\partial^{2}}{\partial x^{2}}e^{-\beta|x|} = \Big|_{-\epsilon}^{+\epsilon}
\frac{\partial}{\partial x}e^{-\beta|x|} = -2\beta
\end{align*}

Putting it all together gives

\begin{align*}
\bar{H} = \frac{\bra{\tilde{0}}H\ket{\tilde{0}}}{\bra{\tilde{0}}\ket{\tilde{0}}} = \frac{m\omega^{2}}{4 \beta^2} + \frac{\hbar^{2}\beta}{2m}
\end{align*}

We then minimize $\bar{H}$ w.r.t. $\beta$

\begin{align*}
\frac{d\bar{H}}{d\beta} =  -\frac{m\omega^{2}}{4\beta} + \frac{\hbar^{2}\beta}{m} = 0
\end{align*}

and $\beta^{*} = \sqrt{m^{2}\beta^{2}/4\hbar^{2}}$

\end{s}

\begin{p}
5.29
\end{p}

\begin{s}

We have the full time-dependent Hamiltonian

\begin{align*}
H(t) = H_{0} + F_{0}x\cos\omega t
\end{align*}

We need to find $\ket{\psi(t)}$, which amounts to finding the expansion coefficients $c_{n}(t)$. In the interaction picture, we have that

\begin{align*}
i\hbar\dot{c_{n}}(t) &= \sum_{m}V_{nm}e^{i\omega_{nm}t}c_{m}(t)
\end{align*}

for $\omega_{nm} = (E_{n}-E_{m})/\hbar$.

\begin{align*}
V_{nm} &= F_{0}\cos\omega t\bra{n}x\ket{m}\\
&= F_{0}\cos\omega t\sqrt{\frac{\hbar}{2m\omega_{0}}}\left(\sqrt{n+1}\delta_{m,n-1} + \sqrt{n}\delta_{m,n+1}\right)
\end{align*}

But the initial condition says that $\ket{\psi(0)} = \ket{0}$, so $n=0$ and the only term of the summation that survives has $m=1$. Therefore,

\begin{align*}
i\hbar\dot{c_{1}}(t) &= V_{10}e^{i\omega_{0}t}c_{0}(t)\\
&= F_{0}\cos\omega t\sqrt{\frac{\hbar}{2m\omega_{0}}}e^{i\omega_{0}t}c_{0}(t)
\end{align*}

Solving for $c_{1}(t)$, 

\begin{align*}
c_{1}(t) &= -\frac{i}{\hbar}F_{0}\sqrt{\frac{\hbar}{2m\omega_{0}}}\int_{0}^{t}e^{i\omega_{0}t}\cos\omega t dt\\
&= -\frac{1}{2\hbar}F_{0}\sqrt{\frac{\hbar}{2m\omega_{0}}}\left(\frac{e^{i(\omega_{0}+\omega)t}-1}{\omega_{0}+\omega}+\frac{e^{i(\omega_{0}-\omega)t}-1}{\omega_{0}-\omega}\right)
\end{align*}

Now, to compute $\langle x \rangle$, we can express the $x$ operator in the interaction picture (or, equivalently, convert the $\ket{\tilde{\psi(t)}}$ back to $\ket{\psi(t)}$).

\begin{align*}
\langle x \rangle &= \bra{\psi(t)} x \ket{\psi(t)} \\
&= \bra{\tilde{\psi(t)}}e^{iH_{0}t/\hbar} x e^{-iH_{0}t/\hbar}\ket{\tilde{\psi(t)}}\\
&= \sqrt{\frac{\hbar}{2m\omega_{0}}}\left(\bra{0}c_{0}^{*}e^{i\omega_{0}t/2} + \bra{1}e^{3i\omega_{0}t/2}c_{1}^{*}(t)\right)(a+a^{\dagger})\left(e^{-i\omega_{0}t/2}c_{0}\ket{0} + e^{-3i\omega_{0}t/2}c_{1}(t)\ket{1}\right)\\
&= \sqrt{\frac{\hbar}{2m\omega_{0}}}\left(c_{1}(t)e^{-i\omega_{0}t} + c_{1}^{*}(t)e^{i\omega_{0}t}\right)\\
&= -\frac{F_{0}}{4m\omega_{0}}\left(\left(\frac{e^{i(\omega_{0}+\omega)t}-1}{\omega_{0}+\omega}+\frac{e^{i(\omega_{0}-\omega)t}-1}{\omega_{0}-\omega}\right)e^{-i\omega_{0}t}  \left(\frac{e^{i(\omega_{0}+\omega)t}-1}{\omega_{0}+\omega}+\frac{e^{i(\omega_{0}-\omega)t}-1}{\omega_{0}-\omega}\right)e^{i\omega_{0}t}\right)\\
&= -\frac{F_{0}}{m}\frac{\cos\omega t - \cos\omega_{0}t}{\omega_{0}^{2}-\omega^{2}}
\end{align*}

Time dependent perturbation theory clearly breaks down for $\omega \sim \omega_{0}$, as it can be seen there is a resonance there according to our expression for $\langle x\rangle$. The expectation value blows up, which cannot have physical significance.

\end{s}

\begin{p}
5.30
\end{p}

\begin{s}
The potential is 

\begin{equation*}
V(x,t) = -xF_{0}e^{-t/\tau}
\end{equation*}

This is very similar to the previous problem, just with a different time-dependence to the potential. Write,

\begin{align*}
c_{1}(t) &= \frac{i}{\hbar}F_{0}\sqrt{\frac{\hbar}{2m\omega_{0}}}\int_{0}^{t}e^{i\omega_{0}t}e^{-t/\tau} dt\\
&= \frac{i}{\hbar}F_{0}\sqrt{\frac{\hbar}{2m\omega_{0}}}\frac{\left(e^{(i\omega_{0}-1/\tau)t}-1\right)}{(i\omega_{0}-1/\tau)}
\end{align*}

The probability of finding the particle in the first excited state is thus

\begin{align*}
|c_{1}(t)|^{2} = \frac{F_{0}^{2}}{2m\omega_{0}}\left(\frac{1- 2e^{-t/\tau} \cos\omega_{0}t + e^{-2t/\tau}}{\omega_{0}^{2}+(1/\tau)^{2}}\right)
\end{align*}

Taking the limit $t\rightarrow\infty$, the exponentials vanish and we get something independent of time. This is expected since the force is transient. We cannot find higher states to first order because, as was shown in the previous problem, $c_{n}(0) = 0$ and $\dot{c_{n}}(t) = 0$ for all $n > 1$. 

\end{s}

\begin{p}
5.32
\end{p}

\begin{s}$c_{n}(t)$

We need to find $\ket{\psi(t)}$, which amounts to finding the expansion coefficients $c_{0}(t)$ and $c_{1}(t)$. In the interaction picture, we have that

\begin{align*}
i\hbar\dot{c_{n}}(t) &= \sum_{m}V_{nm}e^{i\omega_{nm}t}c_{m}(t)
\end{align*}

We were given $V(t)$, so we already know the $V_{nm}$. The differential equation $c_{1}(t)$ is

\begin{align*}
i\hbar\dot{c_{1}}(t) = \sum_{m}V_{nm}e^{i\omega_{nm}t}c_{m}(t) = \lambda e^{i\omega_{0}t}\cos\omega t 
\end{align*}

Integrating in time, we get

\begin{align*}
c_{1}(t) &= -\frac{i\lambda}{\hbar}\int_{0}^{t} e^{i\omega_{0}t}\cos\omega t \\
&= -\frac{i\lambda}{\hbar}\int_{0}^{t} e^{i\omega_{0}t}\left(e^{i\omega t} + e^{-i\omega t}\right)dt \\
&= -\frac{\lambda}{\hbar}\left(\frac{e^{i(\omega_{0}+\omega)t}-1}{\omega_{0}+\omega}+\frac{e^{i(\omega_{0}-\omega)t}-1}{\omega_{0}-\omega}\right)
\end{align*}

Thus the probability the system is found in the state $\ket{1}$ is

\begin{align*}
|c_{1}(t)|^{2} &= \frac{\lambda^{2}}{\hbar^{2}}\left(\frac{e^{-i(\omega_{0}+\omega)t}-1}{\omega_{0}+\omega}+\frac{e^{-i(\omega_{0}-\omega)t}-1}{\omega_{0}-\omega}\right)\left(\frac{e^{i(\omega_{0}+\omega)t}-1}{\omega_{0}+\omega}+\frac{e^{i(\omega_{0}-\omega)t}-1}{\omega_{0}-\omega}\right)\\
&= \frac{\lambda^{2}}{\hbar^{2}}\left(\frac{\sin^{2}(\omega_{0}+\omega)}{2(\omega_{0}+\omega)^{2}}+\frac{\sin^{2}(\omega_{0}-\omega)}{2(\omega_{0}-\omega)^{2}}+\frac{\cos\omega t(\cos\omega t - \cos\omega_{0}t)}{(\omega_{0}-\omega)^{2}}\right)
\end{align*}

\end{s}

\begin{p}
5.35
\end{p}

\begin{s}

The electric potential in the capacitor is

\begin{align*}
V(z,t) = \int_{0}^{z} E_{0}e^{-t/\tau} dz = -zeE_{0}e^{-t/\tau}
\end{align*}

We are asked to find the probability of finding the hydrogen atom in the states $\ket{21\pm 1}, \ket{210}$, given the initial state $\ket{100}$. As usual, we make use of the interaction picture

\begin{align*}
i\hbar\dot{c_{n}}(t) &= \sum_{m}V_{nm}e^{i\omega_{nm}t}c_{m}(t)
\end{align*}

But need to find the matrix element $V_{nm} = \bra{nlm}V\ket{n'l'm'}$. These matrix elements are constrained by the selection rules for transitions of the hydrogen atom. It is well-known that the matrix element $\bra{nlm}z\ket{n'l'm'}$ is nonzero only when $m'=m$ and $l' = l \pm 1$. Therefore, we don't need to calculate the probability of a transition to $\ket{21\pm 1}$, because it is impossible. However, a transition to state $\ket{210}$ is possible and we write

\begin{align*}
i\hbar\dot{c_{210}}(t) &= -eE_{0}e^{-t/\tau}\bra{100}z\ket{210}e^{i\omega t}\\
\end{align*}

That matrix element is well known. Call it $\gamma = \bra{100}z\ket{210} = (128\sqrt{2}/243)a_{0}$

\begin{align*}
c_{210}(t) &= \frac{i\gamma}{\hbar} eE_{0}\int_{0}^{t} e^{(i\omega-1/\tau)t}dt\\
&= \frac{i\gamma}{\hbar} eE_{0}\frac{e^{(i\omega-1/\tau)t}}{(i\omega-1/\tau)}
\end{align*}

and the probability follows from basically the same algebra as problem 3

\begin{align*}
|c_{210}(t)|^{2} &= \frac{\gamma^{2} e^{2}E_{0}^{2}}{\hbar^{2}}\left(\frac{1- 2e^{-t/\tau} \cos\omega t + e^{-2t/\tau}}{\omega^{2}+(1/\tau)^{2}}\right)
\end{align*}

where $\omega = (E_{210}-E_{100})/\hbar$. For the $2s$ state $\ket{200}$, the same selection rules apply, so the only differences are in the value of $\omega=0$ (due to the degeneracy) and the matrix element $\gamma$. 

\begin{align*}
|c_{210}(t)|^{2} &= \frac{(\gamma ')^{2} e^{2}E_{0}^{2}}{\hbar^{2}}\left(\frac{1- 2e^{-t/\tau}  + e^{-2t/\tau}}{1+(1/\tau)^{2}}\right)
\end{align*}

In the limit $\tau\rightarrow \infty$, the field is $E_{0}$, so we get that

\begin{align*}
|c_{210}(t)|^{2} &= \frac{2\gamma^{2} e^{2}E_{0}^{2}}{\omega^{2}\hbar^{2}}\left(1- \cos\omega t \right)
\end{align*}

\end{s}

\begin{p}
5.36
\end{p}

\begin{s}

We were given the Hamiltonian 

\begin{align*}
H = \beta \left(S_{1}\cdot S_{2}\right)
\end{align*}

where $\beta = 4\Delta/\hbar^{2}$. We would like to use Schrodinger's equation to write out the time evolution; however, it isn't immediately clear what the eigenkets of $H$ are. Recall,

\begin{align*}
S_{1}\cdot S_{2} = \frac{1}{2}\left(J^{2} - S_{1}^{2} - S_{2}^{2}\right)
\end{align*}

where $J = S_{1} + S_{2}$. Consider a test ket $\ket{\alpha}$

\begin{align*}
H\ket{\alpha} = \beta\left(S_{1}\cdot S_{2}\right)\ket{\alpha} = \frac{\beta}{2}\left(J^{2} - S_{1}^{2} - S_{2}^{2}\right)\ket{\alpha}
\end{align*}

There must exist simultaneous eigenkets of all three operators since $S_{1}$ and $S_{2}$ commute. The Hilbert space dimension is 4, so we need 4 kets. Those kets are the three triplet states and the singlet state:

\begin{align*}
\ket{1,1} &= \ket{++}\\ 
\ket{1,0} &= \frac{1}{\sqrt{2}}\left(\ket{+-} + \ket{-+}\right)\\
\ket{1,-1} &= \ket{--}\\
\ket{0,0} &= \frac{1}{\sqrt{2}}\left(\ket{+-} - \ket{-+}\right)\\ 
\end{align*}

Also note that the energies of the triplet states and singlet state are $\hbar^{2}/4$ and $-3\hbar^{2}/4$, respectively (which gives energy eigenvalues of $\Delta$ and $-3\Delta$). Now, to find the time evolution of the system prepared in $\ket{\psi(0)} = \ket{+-}$, we can write this state in that basis

\begin{align*}
\ket{\psi(0)} = \frac{1}{\sqrt{2}}\left(\ket{1,0} + \ket{0,0}\right)
\end{align*}

\begin{align*}
\ket{\psi(t)} &= e^{-i\hat{H}t/\hbar}\ket{\psi(0)}\\
&= \frac{1}{\sqrt{2}}\left(e^{-i\Delta t/\hbar}\ket{1,0} + e^{i3\Delta t/\hbar}\ket{0,0}\right)
\end{align*}

The probabilities immediately follow from this result. Those corresponding to $\ket{--}$ and $\ket{++}$ are zero. The others are:

\begin{align*}
|\bra{+-}\ket{\psi(t)}|^{2} &= \frac{1}{4}\left(e^{-i\Delta t/\hbar} + e^{i3\Delta t/\hbar}\right)\left(e^{i\Delta t/\hbar} + e^{-i3\Delta t/\hbar}\right)\\
&= \frac{1+ e^{-4i\Delta t / \hbar} + e^{4i\Delta t / \hbar}}{2}\\
&= \frac{1+\cos\frac{4\Delta t}{\hbar}}{2}
\end{align*}

\begin{align*}
|\bra{-+}\ket{\psi(t)}|^{2} &= \frac{1}{4}\left(e^{-i\Delta t/\hbar} - e^{i3\Delta t/\hbar}\right)\left(e^{i\Delta t/\hbar} - e^{-i3\Delta t/\hbar}\right)\\
&= \frac{1 - e^{-4i\Delta t / \hbar} - e^{4i\Delta t / \hbar}}{2}\\
&= \frac{1-\cos\frac{4\Delta t}{\hbar}}{2}
\end{align*}


To solve this problem using time-dependent perturbation theory, we will need the matrix element for our Hamiltonian. We should use the 



\end{s}

\end{document}