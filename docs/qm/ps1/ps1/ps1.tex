\documentclass[12pt]{article}
\usepackage{amsmath} % AMS Math Package
\usepackage{amsthm} % Theorem Formatting
\usepackage{amssymb}    % Math symbols such as \mathbb
\usepackage{graphicx} % Allows for eps images
\usepackage[dvips,letterpaper,margin=1in,bottom=0.7in]{geometry}
\usepackage{tensor}
\usepackage{amsmath}
\usepackage{siunitx}
\usepackage{physics}

\newtheorem{p}{Problem}
\usepackage{cancel}
\newtheorem*{lem}{Lemma}
\theoremstyle{definition}
\newtheorem*{dfn}{Definition}
 \newenvironment{s}{%\small%
        \begin{trivlist} \item \textbf{Solution}. }{%
            \hspace*{\fill} $\blacksquare$\end{trivlist}}%


\begin{document}

{\noindent\Huge\bf  \\[0.5\baselineskip] {\fontfamily{cmr}\selectfont  Homework 1}         }\\[2\baselineskip] % Title
{ {\bf \fontfamily{cmr}\selectfont Quantum Mechanics}\\ {\textit{\fontfamily{cmr}\selectfont     August 29th, 2022}}}~~~~~~~~~~~~~~~~~~~~~~~~~~~~~~~~~~~~~~~~~~~~~~~~~~~~~~~~~~~~~~~~~~~~~~~~~~~~~    {\large \textsc{Clayton Seitz}
\\[1.4\baselineskip] 

\begin{p}
For the spin 1/2 state $\ket{+}_{x}$, evaluate both sides of the inequality
\begin{align*}
\langle(\Delta A)^{2}\rangle\langle(\Delta B)^{2}\rangle \geq \frac{1}{4}|\langle[A,B]\rangle|^{2}
\end{align*}

for the operators $A=S_x$ and $B=S_y$, and show that the inequality is satisfied. Repeat for the operators $A=S_z$ and $B=S_y$
\end{p}

\begin{s} 

\vspace{0.2in}
Let $A=S_x$ and $B=S_y$. The variance $\langle(\Delta S_x)^{2}\rangle$ in state $\ket{+}_{x}$ must be zero since $\ket{+}_{x}$ is an eigenvector of $S_{x}$  

\begin{align*}
\langle(\Delta S_x)^{2}\rangle = \langle S_x^{2}\rangle - \langle S_{x}\rangle^{2} = 0
\end{align*}

Therefore, the LHS of the above inequality is zero. The commutator $[S_{x},S_{y}] = i\hbar S_z$ and 

\begin{align*}
\langle S_z \rangle &= \bra{+}_{x} S_z \ket{+}_{x} = 0\\
\end{align*}

Clearly the inequality is satisfied since both sides are zero. Now let $A=S_z$ and $B=S_y$. Since the state is prepared in $\ket{+}_x$, the variances $\langle(\Delta S_x)^{2}\rangle$ and $\langle(\Delta S_x)^{2}\rangle$ must be $1/4$ (this is just a fair coin toss).
\\
The commutator $[S_{z},S_{y}] = -i\hbar S_x$ and $\langle S_x \rangle = \frac{\hbar}{2}$. The inequality then reads 

\begin{align*}
\frac{1}{16} \geq \frac{\hbar^{2}}{16}
\end{align*}

which is satisfied given that $\hbar \approx 10^{-34} \;\mathrm{J\cdot s}$


\end{s}

\begin{p}
Suppose a 2×2 matrix X (not necessarily Hermitian, nor unitary) is written as
\end{p}

\begin{s} 


\begin{align*}
\mathrm{Tr}(X) &= \mathrm{Tr}(a_{0}) + \mathrm{Tr}\left(\sum_k a_k\sigma_k\right)\\
&= 2a_{0}
\end{align*}

\begin{align*}
\mathrm{Tr}(\sigma_{k}X) &= \mathrm{Tr}\left(\sigma_{k}a_{0} + \sigma_{k}\sum_j a_{j}\sigma_j\right)\\
&= \mathrm{Tr}\left(\sigma_{k}a_{0} + \sum_j a_{j}\sigma_{k}\sigma_j\right)\\
&= \mathrm{Tr}\left(\sum_j a_{j}\sigma_{k}\sigma_j\right)
\end{align*}

We can write out the equation $X = a_{0} + \bold{\sigma}\cdot a$ explicitly

\begin{align*}
X =
\begin{pmatrix}
a_0 + a_3 & a_1 - ia_3\\
a_1+ia_2 & a_0 - a_3
\end{pmatrix}
\end{align*}

Thus we have four equations involving $X_{ij}$'s and $a_k$ for $k = (1,2,3)$. We can manipulate those four equations to show that

\begin{align*}
a_0 &= \frac{X_{11}+X_{22}}{2}\\
a_1 &= \frac{X_{12}+X_{21}}{2}\\
a_2 &= \frac{X_{21}-X_{12}}{2}\\
a_3 &= \frac{X_{11}-X_{22}}{2}
\end{align*}


\end{s}

\begin{p}

\end{p}

\begin{s} 
\end{s}

\begin{p}

\end{p}

\begin{s} 
\end{s}

\begin{p}

\end{p}

\begin{s} 
\end{s}

\begin{p}

\end{p}

\begin{s} 
\end{s}


\end{document}