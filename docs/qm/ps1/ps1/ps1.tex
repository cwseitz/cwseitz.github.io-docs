\documentclass[12pt]{article}
\usepackage{amsmath} % AMS Math Package
\usepackage{amsthm} % Theorem Formatting
\usepackage{amssymb}    % Math symbols such as \mathbb
\usepackage{graphicx} % Allows for eps images
\usepackage[dvips,letterpaper,margin=1in,bottom=0.7in]{geometry}
\usepackage{tensor}
\usepackage{amsmath}
\usepackage{siunitx}
\usepackage{physics}

\newtheorem{p}{Problem}
\usepackage{cancel}
\newtheorem*{lem}{Lemma}
\theoremstyle{definition}
\newtheorem*{dfn}{Definition}
 \newenvironment{s}{%\small%
        \begin{trivlist} \item \textbf{Solution}. }{%
            \hspace*{\fill} $\blacksquare$\end{trivlist}}%


\begin{document}

{\noindent\Huge\bf  \\[0.5\baselineskip] {\fontfamily{cmr}\selectfont  Homework 1}         }\\[2\baselineskip] % Title
{ {\bf \fontfamily{cmr}\selectfont Quantum Mechanics}\\ {\textit{\fontfamily{cmr}\selectfont     August 29th, 2022}}}~~~~~~~~~~~~~~~~~~~~~~~~~~~~~~~~~~~~~~~~~~~~~~~~~~~~~~~~~~~~~~~~~~~~~~~~~~~~~    {\large \textsc{Clayton Seitz}
\\[1.4\baselineskip] 

\begin{p}
For the spin 1/2 state $\ket{+}_{x}$, evaluate both sides of the inequality
\begin{align*}
\langle(\Delta A)^{2}\rangle\langle(\Delta B)^{2}\rangle \geq \frac{1}{4}|\langle[A,B]\rangle|^{2}
\end{align*}

for the operators $A=S_x$ and $B=S_y$, and show that the inequality is satisfied. Repeat for the operators $A=S_z$ and $B=S_y$
\end{p}

\begin{s} 

\vspace{0.2in}
Let $A=S_x$ and $B=S_y$. The variance $\langle(\Delta S_x)^{2}\rangle$ in state $\ket{+}_{x}$ must be zero since $\ket{+}_{x}$ is an eigenvector of $S_{x}$  

\begin{align*}
\langle(\Delta S_x)^{2}\rangle = \langle S_x^{2}\rangle - \langle S_{x}\rangle^{2} = 0
\end{align*}

Therefore, the LHS of the above inequality is zero. The commutator $[S_{x},S_{y}] = i\hbar S_z$ and 

\begin{align*}
\langle S_z \rangle &= \bra{+}_{x} S_z \ket{+}_{x} = 0\\
\end{align*}

Clearly the inequality is satisfied since both sides are zero. Now let $A=S_z$ and $B=S_y$. Since the state is prepared in $\ket{+}_x$, the variances $\langle(\Delta S_x)^{2}\rangle$ and $\langle(\Delta S_x)^{2}\rangle$ must be $1/4$ (this is just a fair coin toss).
\\
The commutator $[S_{z},S_{y}] = -i\hbar S_x$ and $\langle S_x \rangle = \frac{\hbar}{2}$. The inequality then reads 

\begin{align*}
\frac{1}{16} \geq \frac{\hbar^{2}}{16}
\end{align*}

which is satisfied given that $\hbar \approx 10^{-34} \;\mathrm{J\cdot s}$


\end{s}

\begin{p}
Suppose a 2×2 matrix X (not necessarily Hermitian, nor unitary) is written as
\end{p}

\begin{s} 


\begin{align*}
\mathrm{Tr}(X) &= \mathrm{Tr}(a_{0}) + \mathrm{Tr}\left(\sum_k a_k\sigma_k\right)\\
&= 2a_{0}
\end{align*}

\begin{align*}
\mathrm{Tr}(\sigma_{k}X) &= \mathrm{Tr}\left(\sigma_{k}a_{0} + \sigma_{k}\sum_j a_{j}\sigma_j\right)\\
&= \mathrm{Tr}\left(\sigma_{k}a_{0} + \sum_j a_{j}\sigma_{k}\sigma_j\right)\\
&= \mathrm{Tr}\left(\sum_j a_{j}\sigma_{k}\sigma_j\right)
\end{align*}

We can write out the equation $X = a_{0} + \bold{\sigma}\cdot a$ explicitly

\begin{align*}
X =
\begin{pmatrix}
a_0 + a_3 & a_1 - ia_3\\
a_1+ia_2 & a_0 - a_3
\end{pmatrix}
\end{align*}

Thus we have four equations involving $X_{ij}$'s and $a_k$ for $k = (1,2,3)$. We can manipulate those four equations to show that

\begin{align*}
a_0 &= \frac{X_{11}+X_{22}}{2}\\
a_1 &= \frac{X_{12}+X_{21}}{2}\\
a_2 &= \frac{X_{21}-X_{12}}{2}\\
a_3 &= \frac{X_{11}-X_{22}}{2}
\end{align*}


\end{s}

\begin{p}

\end{p}

\begin{s} 
\begin{align*}
\sigma\cdot a' = \exp\left(\frac{i\sigma\cdot\hat{n}\phi}{2}\right)\sigma\cdot a \exp\left(-\frac{i\sigma\cdot\hat{n}\phi}{2}\right)
\end{align*}
\end{s}

\begin{p}

\end{p}

\begin{s} 

\begin{equation*}
A(\ket{i} + \ket{j}) = i\ket{i} + j\ket{j}
\end{equation*}

If we have degenerate eigenvalues i.e., $i=j$ then

\begin{equation*}
A(\ket{i} + \ket{j}) = i(\ket{i} + \ket{j})
\end{equation*}

and $\ket{i} + \ket{j}$ is also an eigenvector of $A$

\end{s}

\begin{p}

\end{p}

\begin{s} 
We will make use of the following representations of the spin operators
\begin{align*}
S_{x} &= \frac{\hbar}{2}\left(\ket{+}\bra{-} + \ket{-}\bra{+}\right)\\
S_{y} &= \frac{i\hbar}{2}\left(-\ket{+}\bra{-} + \ket{-}\bra{+}\right)\\
S_{z} &= \frac{\hbar}{2}\left(\ket{+}\bra{+} - \ket{-}\bra{-}\right)
\end{align*}

\begin{align*}
[S_{x},S_{y}] &= \frac{i\hbar^{2}}{2}\left(\ket{+}\bra{-} + \ket{-}\bra{+}\right)\left(-\ket{+}\bra{-} + \ket{-}\bra{+}\right)\\
&- \left(-\ket{+}\bra{-} + \ket{-}\bra{+}\right)\left(\ket{+}\bra{-} + \ket{-}\bra{+}\right)\\
&= \frac{i\hbar^{2}}{2}\left(\ket{+}\bra{+} - \ket{-}\bra{-}\right)\\
&= i\hbar S_{z}
\end{align*}

Flipping the order of the commutator always flips the sign of the result i.e. $[S_{i},S_{j}] = -[S_{j},S_{i}]$. Thus for $[S_y,S_x]$ we would get $-i\hbar S_{z}$.

\begin{align*}
[S_{y},S_{z}] &= \frac{i\hbar^{2}}{4}\left(-\ket{+}\bra{-} + \ket{-}\bra{+}\right)\left(\ket{+}\bra{+} - \ket{-}\bra{-}\right)\\
&- \left(\ket{+}\bra{+} - \ket{-}\bra{-}\right)\left(\ket{+}\bra{-} + \ket{-}\bra{+}\right)\\
&= \frac{i\hbar^{2}}{4}\left(\ket{+}\bra{-} + \ket{-}\bra{+}\right)\\
&= i\hbar S_{x}
\end{align*}

\begin{align*}
[S_{z},S_{x}] &= \frac{\hbar^{2}}{4}\left(\ket{+}\bra{+} - \ket{-}\bra{-}\right)\left(\ket{+}\bra{-} + \ket{-}\bra{+}\right)\\
&- \left(\ket{+}\bra{-} + \ket{-}\bra{+}\right)\left(\ket{+}\bra{+} - \ket{-}\bra{-}\right)\\
&= -\frac{\hbar^{2}}{4}\left(-\ket{+}\bra{-} + \ket{-}\bra{+}\right)\\
&= i\hbar S_{y}
\end{align*}

For the anticommutator relations, all we need to prove is that $S_{i}S_{j} = -S_{j}S_{i}$ when $i\neq j$. Of course, when $i=j$ we will always have $\{S_{i},S_{j}\} = 2S_{i}^{2} = \frac{\hbar^{2}}{2}$ since $S_{i}^{2} = I \;\;\forall i$
\end{s}

\begin{p}

\end{p}

\begin{s} 

We would like to find a representation for the state $\ket{\mathbf{S}\cdot\hat{n}; +}$ in the $S_{z}$ basis. We first write the operator $\mathbf{S}\cdot\hat{n}$ explicitly in this basis

\begin{align*}
\mathbf{S}\cdot\hat{n} &= \sin\beta\cos\alpha \; S_{x} +\sin\beta\sin\alpha\;S_{y} + \cos\beta\; S_{z}\\
&= \frac{\hbar}{2}\begin{pmatrix}
\cos\beta & \sin\beta\exp(-i\alpha)\\
\sin\beta\exp(i\alpha) & -\cos\beta
\end{pmatrix}
\end{align*}

As usual, we find the eigenvalues of this operator by solving the characteristic equation:

\begin{align*}
\mathrm{det}\left(\mathbf{S}\cdot\hat{n} - \lambda I\right) &= \left(\frac{\hbar}{2}\cos\beta - \lambda\right)\left(-\frac{\hbar}{2}\cos\beta - \lambda\right) - \frac{\hbar^{2}}{4}\sin^{2}\beta\\
&= \lambda^{2} - \frac{\hbar^{2}}{4} = 0
\end{align*}

Therefore $\lambda = \pm \frac{\hbar}{2}$ as expected. Let $\psi_{1}$ and $\psi_{2}$ represent the components of the eigenket $\ket{\mathbf{S}\cdot\hat{n}; +}$ of this operator. We then need to solve the following system for the components $\psi_{1}$ and $\psi_{2}$

\begin{align*}
\psi_{1}\cos\beta+\psi_{2}\sin\beta\exp(-i\alpha) &= \psi_{1}\\
\psi_{1}\sin\beta\exp(i\alpha) - \psi_{2}\cos\beta &= \psi_{2}
\end{align*}

The system does not have a real solution. But we can make a lucky guess that $\psi_{1} = \cos\frac{\beta}{2}$ and $\psi_{2} = \sin\frac{\beta}{2}\exp(i\alpha)$


\end{s}




\end{document}