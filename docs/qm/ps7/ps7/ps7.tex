\documentclass[12pt]{article}
\usepackage{amsmath} % AMS Math Package
\usepackage{bm}
\usepackage{amsthm} % Theorem Formatting
\usepackage{amssymb}    % Math symbols such as \mathbb
\usepackage{graphicx} % Allows for eps images
\usepackage[dvips,letterpaper,margin=1in,bottom=0.7in]{geometry}
\usepackage{tensor}
\usepackage{amsmath}
\usepackage{siunitx}
\usepackage{physics}
\usepackage{amsmath, amssymb, graphics, setspace}

\newcommand{\mathsym}[1]{{}}
\newcommand{\unicode}[1]{{}}

\newcounter{mathematicapage}

\newtheorem{p}{Problem}
\usepackage{cancel}
\newtheorem*{lem}{Lemma}
\theoremstyle{definition}
\newtheorem*{dfn}{Definition}
 \newenvironment{s}{%\small%
        \begin{trivlist} \item \textbf{Solution}. }{%
            \hspace*{\fill} $\blacksquare$\end{trivlist}}%


\begin{document}

 {\noindent\Huge\bf  \\[0.5\baselineskip] {\fontfamily{cmr}\selectfont  Homework 7}         }\\[2\baselineskip] % Title
{ {\bf \fontfamily{cmr}\selectfont Quantum Mechanics}\\ {\textit{\fontfamily{cmr}\selectfont     \today}}}~~~~~~~~~~~~~~~~~~~~~~~~~~~~~~~~~~~~~~~~~~~~~~~~~~~~~~~~~~~~~~~~~~~~~~~~~~~~~    {\large \textsc{C Seitz}
\\[1.4\baselineskip] 

\begin{p}
5.1
\end{p}

\begin{s}

We are concerned here with the new ground state ket $\ket{0}$ in the presence of $H_{1}$ and the new ground state energy shift $\Delta_{0}$. 

\begin{align*}
\ket{0} = \ket{0^{0}} + \sum_{k\neq 0}\ket{k^{0}}\frac{V_{k0}}{E_{0}^{0}-E_{k}^{0}} + ...
\end{align*}

\begin{align*}
\Delta_{0} = V_{00} + \sum_{k\neq 0}\frac{|V_{k0}|^{2}}{E_{0}^{0}-E_{k}^{0}} + ...
\end{align*}

\begin{align*}
V_{nk} = b\bra{n^{0}}x\ket{k^{0}} &= b\sqrt{\frac{\hbar}{2m\omega}}\left(\sqrt{k}\delta_{n,k-1} + \sqrt{k+1}\delta_{n,k+1}\right)
\end{align*}

The lowest nonvanishing order is then $V_{01}$. Therefore 

\begin{align*}
\Delta_{0} = -\frac{b^{2}\hbar}{2m\omega}\frac{1}{\hbar\omega} = -\frac{b^{2}}{2m\omega^{2}}
\end{align*}

To solve it exactly, notice that the potential is of the form

\begin{align*}
V(x) = ax^{2} + bx
\end{align*}

This function shifts to the left by $b/2$ and has a new minimum at $-b/a$. Also, any multiplicative constant $\epsilon$ of the potential can just be absorbed into $\omega$, such that $E_{n} = \hbar\omega\sqrt{\epsilon}(n+\frac{1}{2})$. Importantly, this implies that the energy differences for various $n$, remain the same. Therefore, we can conclude that 

\begin{align*}
\Delta = -\frac{b}{2a} = -\frac{b}{2m\omega^{2}}
\end{align*}

which is exactly what we got with second-order perturbation theory.


\end{s}

\begin{p}
5.2
\end{p}

\begin{s}
\end{s}

\begin{p}
5.5
\end{p}

\begin{s}
\end{s}

\begin{p}
5.7
\end{p}

\begin{s}
\end{s}

\begin{p}
5.12a
\end{p}

\begin{s}
\end{s}

\begin{p}
5.24
\end{p}

\begin{s}
\end{s}

\end{document}