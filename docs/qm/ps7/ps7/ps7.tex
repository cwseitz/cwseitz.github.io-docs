\documentclass[12pt]{article}
\usepackage{amsmath} % AMS Math Package
\usepackage{bm}
\usepackage{amsthm} % Theorem Formatting
\usepackage{amssymb}    % Math symbols such as \mathbb
\usepackage{graphicx} % Allows for eps images
\usepackage[dvips,letterpaper,margin=1in,bottom=0.7in]{geometry}
\usepackage{tensor}
\usepackage{amsmath}
\usepackage{siunitx}
\usepackage{physics}
\usepackage{amsmath, amssymb, graphics, setspace}

\newcommand{\mathsym}[1]{{}}
\newcommand{\unicode}[1]{{}}

\newcounter{mathematicapage}

\newtheorem{p}{Problem}
\usepackage{cancel}
\newtheorem*{lem}{Lemma}
\theoremstyle{definition}
\newtheorem*{dfn}{Definition}
 \newenvironment{s}{%\small%
        \begin{trivlist} \item \textbf{Solution}. }{%
            \hspace*{\fill} $\blacksquare$\end{trivlist}}%


\begin{document}

 {\noindent\Huge\bf  \\[0.5\baselineskip] {\fontfamily{cmr}\selectfont  Homework 7}         }\\[2\baselineskip] % Title
{ {\bf \fontfamily{cmr}\selectfont Quantum Mechanics}\\ {\textit{\fontfamily{cmr}\selectfont     \today}}}~~~~~~~~~~~~~~~~~~~~~~~~~~~~~~~~~~~~~~~~~~~~~~~~~~~~~~~~~~~~~~~~~~~~~~~~~~~~~    {\large \textsc{C Seitz}
\\[1.4\baselineskip] 

\begin{p}
5.1
\end{p}

\begin{s}

We are concerned here with the new ground state ket $\ket{0}$ and the new ground state energy shift $\Delta_{0}$ in the presence of perturbation $V=bx$. 

\begin{align*}
\ket{0} = \ket{0^{0}} + \sum_{j\neq 0}\ket{j^{0}}\frac{V_{j0}}{E_{0}^{0}-E_{j}^{0}} + ...
\end{align*}

\begin{align*}
\Delta_{0} = V_{00} + \sum_{j\neq 0}\frac{|V_{j0}|^{2}}{E_{0}^{0}-E_{j}^{0}} + ...
\end{align*}

\begin{align*}
V_{nk} = b\bra{i^{0}}x\ket{j^{0}} &= b\sqrt{\frac{\hbar}{2m\omega}}\left(\sqrt{j}\delta_{i,j-1} + \sqrt{j+1}\delta_{i,j+1}\right)
\end{align*}

The lowest nonvanishing order is then $V_{01}$. Therefore 

\begin{align*}
\Delta_{0} = -\frac{b^{2}\hbar}{2m\omega}\frac{1}{\hbar\omega} = -\frac{b^{2}}{2m\omega^{2}}
\end{align*}

To solve it exactly, notice that the potential is of the form

\begin{align*}
V_{1}(x) = ax^{2} + bx
\end{align*}

The new potential shifts to the left by $b/2$, has a new minimum at $-b/2a$. So it is really just the original problem, we just have to make a change of coordinates and shift the equilibrium point down by $-b/2m\omega^{2}$. Therefore,

\begin{align*}
\Delta = -\frac{b}{2a} = -\frac{b}{2m\omega^{2}}
\end{align*}

which is exactly what we got with perturbation theory.

\end{s}

\begin{p}
5.2
\end{p}

\begin{s}

In general, the first order shift in the energy levels $i$ is

\begin{align*}
\Delta_{i} = V_{ii} = \bra{i^{0}}V\ket{i^{0}}
\end{align*}

Furthermore, the perturbation Hamiltonian is 


\begin{align*}
V = \frac{V_{0}x}{L}
\end{align*}

So we just need to identify the matrix elements along the diagonal of this matrix:

\begin{align*}
V_{ii} = \bra{i^{0}}V\ket{i^{0}} &= \frac{V_{0}}{L}\bra{i^{0}}x\ket{i^{0}} \\
&= \frac{V_{0}}{L}\frac{2}{L}\int_{0}^{L} x \sin^{2}\left(\frac{n\pi x}{L}\right)dx\\
&= \frac{V_{0}}{L}
\end{align*}

\end{s}

\begin{p}
5.5
\end{p}

\begin{s}

Up to order $\lambda^{2}$, we have

\begin{align*}
\ket{i} &= \ket{i^{0}} + \lambda\sum_{j\neq i}\frac{V_{ij}}{E_{i}^{0}-E_{j}^{0}}\ket{j^{0}} \\
&+ \lambda^{2}\left(\sum_{j\neq i}\sum_{l\neq i}\frac{V_{jl}V_{li}\ket{j^{0}}}{(E_{i}^{0}-E_{j}^{0})(E_{i}^{0}-E_{l}^{0})}-\sum_{j\neq i}\frac{V_{ii}V_{ji}\ket{j^{0}}}{(E_{i}^{0}-E_{j}^{0})^{2}}) \right)
\end{align*}

Now recall that we chose the normalization $\bra{i^{0}}\ket{i} = 1$ (which can be seen from the above equation), so what we have to calculate is $|\bra{i^{0}}\ket{i}|^{2}/|\bra{i}\ket{i}|^{2}$

\begin{align*}
\bra{i}\ket{i} &= 1 + \lambda^{2}\left(\sum_{j\neq i}\frac{V_{ij}^{*}}{E_{i}^{0}-E_{j}^{0}}\bra{j^{0}}\right)\left(\sum_{j\neq i}\frac{V_{ij}}{E_{i}^{0}-E_{j}^{0}}\ket{j^{0}}\right)\\
&= 1 + \lambda^{2}\sum_{j\neq i}\frac{V_{ij}^{*}V_{ij}}{\left(E_{i}^{0}-E_{j}^{0}\right)^{2}} 
\end{align*} 

So the probability is just

\begin{align*}
|\bra{i^{0}}\ket{i}|^{2}/|\bra{i}\ket{i}|^{2} = \left(1 + 2\lambda^{2}\sum_{j\neq i}\frac{V_{ij}^{*}V_{ij}}{\left(E_{i}^{0}-E_{j}^{0}\right)^{2}} \right)^{-1} + \mathcal{O}(\lambda^{3})
\end{align*} 

\end{s}

\begin{p}
5.7
\end{p}

\begin{s}

We can write this Hamiltonian as 
\begin{align*}
H_{0} = H_{x} + H_{y}
\end{align*}

Given some state $\ket{n, m}$, where $H_{x}$ acts on $\ket{n}$ and $H_{y}$ acts on $\ket{m}$. 

\begin{align*}
H_{0}\ket{n, m} &= (H_{x}+H_{y})\ket{n,m}\\
&= (E_{x}^{n}+E_{y}^{m})\ket{n,m}\\
&= \hbar\omega\left(n+\frac{1}{2} + m + \frac{1}{2}\right)\\
&= \hbar\omega(n + m + 1)
\end{align*}

So the energies of the three lowest states are $\hbar\omega, 2\hbar\omega, 3\hbar\omega$. There is a degeneracy - two unique $\ket{n,m}$ have the same eigenvalue for the first excited state. For example $\ket{0,1}$ and $\ket{1,0}$ have the same energy. Now, we are given the perturbation

\begin{align*}
V = \delta m\omega^{2}xy
\end{align*}

and we need to find a general matrix representation in the unperturbed $\ket{n,m}$ basis. I will use a shorthand:

\begin{align*}
V_{ij} &= \delta m\omega^{2}\bra{i}xy\ket{j}\\
&= \delta m\omega^{2}\bra{i_{x}}x\ket{j_{x}}\bra{i_{y}}y\ket{j_{y}}\\
&= \delta\frac{\hbar\omega}{2}\left(\sqrt{j_{x}}\delta_{i_{x},j_{x}-1} + \sqrt{j_{x}+1}\delta_{i_{x},j_{x}+1}\right)\left(\sqrt{j_{y}}\delta_{i_{y},j_{y}-1} + \sqrt{j_{y}+1}\delta_{i_{y},j_{y}+1}\right)
\end{align*}

First, we consider the ground state. To first order,

\begin{align*}
\Delta_{i} &= V_{ii} = 0
\end{align*}


This can be seen from the above equation, where the matrix elements of $V_{ij}$ are always zero when $i=j$. For the first exicted state, there is a degeneracy, so we need to diagonalize $V$ in the $\ket{1,0}$, $\ket{0,1}$ subspace. The matrix is

\begin{align*}
V' &= \begin{pmatrix}
\bra{0,1}V\ket{0,1} & \bra{0,1}V\ket{1,0}\\
\bra{1,0}V\ket{0,1} & \bra{1,0}V\ket{1,0}
\end{pmatrix}\\
&= \delta\frac{\hbar\omega}{2}\begin{pmatrix}
0 & 1\\
1 & 0
\end{pmatrix}
\end{align*}

which is a familiar matrix, so we can immediately write

\begin{align*}
\Delta_{1}^{1} = \pm \delta\frac{\hbar\omega}{2}
\end{align*}

and the eigenvectors are $\frac{1}{\sqrt{2}}\left(\ket{0,1}\pm\ket{1,0}\right)$


\end{s}

\begin{p}
5.12a
\end{p}

\begin{s}

We need to find the eigenvectors of $V$ in the degenerate subspace. If they are all independent, then the three-fold degeneracy is lifted. I will suppress the quantum numbers $n$ and $m_{s}$ for brevity

\begin{align*}
V &= \begin{pmatrix}
\bra{1,-1}V\ket{1,-1} & \bra{1,-1}V\ket{1,0} & \bra{1,-1}V\ket{1,1}\\
\bra{1,0}V\ket{1,-1} & \bra{1,0}V\ket{1,0} & \bra{1,0}V\ket{1,1}\\
\bra{1,1}V\ket{1,-1} & \bra{1,1}V\ket{1,0}& \bra{1,1}V\ket{1,1}
\end{pmatrix}\\
\end{align*}

It is immediately obvious that the elements along the diagonal must be zero, by symmetry. Also, the matrix is Hermitian (which can be verified by inspecting the spherical harmonics $Y_{l}^{m}(\theta,\phi)$. Therefore, we just need to find $\bra{1,-1}V\ket{1,0}$, $\bra{1,-1}V\ket{1,1}$ and $\bra{1,0}V\ket{1,1}$. Write,

\begin{align*}
\bra{1,-1}x^{2}\ket{1,0} &= \int r^{2}\sin^{2}\theta\cos^{2}\phi Y_{1}^{-1}(\theta,\phi)Y_{1}^{0}(\theta,\phi)rdrd\theta d\phi\\
&\propto \int r^{2}\sin^{2}\theta\cos^{2}\phi \sin\theta e^{i\phi}\cos\theta rdrd\theta d\phi = 0
\end{align*}

due to the $\theta$ integral. It is similar for $\langle y^{2} \rangle$ - the integral over $\theta$ is again zero and $\bra{1,-1}V\ket{1,0} = 0$. Moving on,

\begin{align*}
\bra{1,0}x^{2}\ket{1,1} &= \int r^{2}\sin^{2}\theta\cos^{2}\phi Y_{1}^{0}(\theta,\phi)Y_{1}^{1}(\theta,\phi)rdrd\theta d\phi\\
&\propto \int r^{2}\sin^{2}\theta\cos^{2}\phi \sin\theta e^{-i\phi}\cos\theta rdrd\theta d\phi = 0
\end{align*}

So it is basically the same as before and $\bra{1,0}V\ket{1,1} = 0$. Finally, 

\begin{align*}
\bra{1,-1}x^{2}\ket{1,1} &= \int r^{2}\sin^{2}\theta\cos^{2}\phi Y_{1}^{-1}(\theta,\phi)Y_{1}^{1}(\theta,\phi)rdrd\theta d\phi\\
&\propto \int r^{3}\sin^{4}\theta\cos^{2}\phi drd\theta d\phi
\end{align*}

\begin{align*}
\bra{1,-1}y^{2}\ket{1,1} &= \int r^{2}\sin^{2}\theta\cos^{2}\phi Y_{1}^{-1}(\theta,\phi)Y_{1}^{1}(\theta,\phi)rdrd\theta d\phi\\
&\propto \int r^{3}\sin^{2}\theta\cos^{4}\phi drd\theta d\phi
\end{align*}

We are not asked to work this integral out in detail, but it is nonzero and we will ultimately get a matrix like

\begin{align*}
V &= \begin{pmatrix}
0 & 0 & \alpha\\
0 & 0 & 0\\
\alpha & 0& 0
\end{pmatrix}
\end{align*}

since $\alpha$ is real. Finally, 

\begin{align*}
\mathrm{det}(V-\lambda I) = \mathrm{det}\begin{pmatrix}
-\lambda & 0 & \alpha\\
0 & -\lambda & 0\\
\alpha & 0& -\lambda
\end{pmatrix} = -\lambda^{3} + \lambda\alpha^{2} = 0
\end{align*}

which has three solutions $\lambda = 0,\pm\alpha$. So the degeneracy is broken.

\end{s}

\begin{p}
5.24
\end{p}

\begin{s}

We have seen before that we can either express the state in the basis of $\bm{J}^{2},\bm{L}^{2},\bm{S}^{2},J_z$ or $\bm{L}^{2},\bm{S}^{2},L_z,S_z$. In the former, our perturbation Hamiltonian $V$ reads

\begin{align*}
V = \frac{A}{2\hbar^{2}}\left(\bm{J}^{2}-\bm{L}^{2}-\bm{S}^{2}\right) + \frac{B}{\hbar}\left(J_z + S_z\right)
\end{align*} 

The only hiccup of this representation is that $S_z$ is not diagonal in the $\ket{lsjm}$ basis, so we will have to specify those matrix elements explicitly. For an arbitrary state $\ket{lsjm}$

\begin{align*}
\bra{lsjm}V\ket{lsjm} &= \bra{lsjm}\frac{A}{2\hbar^{2}}\left(\bm{J}^{2}-\bm{L}^{2}-\bm{S}^{2}\right) + \frac{B}{\hbar}\left(J_z + S_z\right)\ket{lsjm}\\
&= \frac{A}{2}\left(j(j+1)-l(l+1)-s(s+1)\right) + Bm + B\bra{lsjm}S_{z}\ket{lsjm}
\end{align*} 

We can write these states alternatively as

\begin{align*}
\ket{j=l\pm \frac{1}{2},m} &= \pm\sqrt{\frac{l\pm m +\frac{1}{2}}{2l+1}}\ket{m_{l} = m-\frac{1}{2}, m_{s}=\frac{1}{2}} \\
&+ \sqrt{\frac{l\mp m +\frac{1}{2}}{2l+1}}\ket{m_{l} = m+\frac{1}{2}, m_{s}=-\frac{1}{2}}
\end{align*} 

which means that

\begin{align*}
\bra{j=l\pm \frac{1}{2},m}S_{z}\ket{j=l\pm \frac{1}{2},m} &= \frac{\hbar}{2}(|c_{+}|^{2}-|c_{-}|^{2})\\
&= \frac{\hbar}{2}\frac{1}{2l+1}\left(\left(l\pm m+\frac{1}{2}\right)-\left(l\mp m+\frac{1}{2}\right)\right) \\
&= \pm \frac{m\hbar}{2l+1}
\end{align*} 

So we can add in the contribution of $S_z$ to the diagonal matrix element:

\begin{align*}
\bra{lsjm}V\ket{lsjm} &= \bra{lsjm}\frac{A}{2\hbar^{2}}\left(\bm{J}^{2}-\bm{L}^{2}-\bm{S}^{2}\right) + \frac{B}{\hbar}\left(J_z + S_z\right)\ket{lsjm}\\
&= \frac{A}{2}\left(j(j+1)-l(l+1)-s(s+1)\right) + Bm \pm B\frac{m}{2l+1}\\
&= \frac{A}{2}\left(j(j+1)-l(l+1)-s(s+1)\right) + Bm\left(1 \pm \frac{1}{2l+1}\right)\\
\end{align*} 

Now, we need to determine which pairs $\ket{lsjm},\ket{l's'j'm'}$ will give off diagonal matrix elements i.e., $\bra{lsjm}S_{z}\ket{l's'j'm'} \neq 0$. Notice these expressions do not include $j$, so probably the matrix element is nonzero is when $m= m'$, $j\neq j'$ but $l=l'$. These considerations suggest the non-diagonal 2x2 matrices includes states

\begin{align*}
\ket{1, \frac{1}{2}, \frac{1}{2}, -\frac{1}{2}},
\ket{1, \frac{1}{2}, \frac{3}{2}, -\frac{1}{2}}
\end{align*} 

and 

\begin{align*}
\ket{1, \frac{1}{2}, \frac{1}{2}, \frac{1}{2}},
\ket{1, \frac{1}{2}, \frac{3}{2}, \frac{1}{2}},
\end{align*} 

and the rest of the states are

\begin{align*}
\ket{0, \frac{1}{2}, \frac{1}{2}, \frac{1}{2}},
\ket{0, \frac{1}{2}, \frac{1}{2}, -\frac{1}{2}}
\ket{1, \frac{1}{2}, \frac{3}{2}, \frac{3}{2}},
\ket{1, \frac{1}{2}, \frac{3}{2}, -\frac{3}{2}},
\end{align*}

The values of the off-diagonal matrix elements have to be $\frac{B}{\hbar}\bra{lsjm}S_z\ket{lsj'm}$.

\begin{align*}
\frac{B}{\hbar}\bra{j=l\mp \frac{1}{2},m}S_z\ket{j=l\pm \frac{1}{2},m} = \left(\mp\sqrt{\frac{l\mp m +\frac{1}{2}}{2l+1}}\bra{\alpha} 
+ \sqrt{\frac{l\pm m +\frac{1}{2}}{2l+1}}\bra{\beta}\right)\\
\left(\pm\frac{\hbar}{2}\sqrt{\frac{l\pm m +\frac{1}{2}}{2l+1}}\ket{\alpha} 
- \frac{\hbar}{2}\sqrt{\frac{l\mp m +\frac{1}{2}}{2l+1}}\ket{\beta}\right)\\
= \frac{B}{\hbar}\left(-\frac{\hbar}{2}\sqrt{\frac{l\mp m +\frac{1}{2}}{2l+1}}\sqrt{\frac{l\pm m +\frac{1}{2}}{2l+1}} - \frac{\hbar}{2}\sqrt{\frac{l\pm m +\frac{1}{2}}{2l+1}}\sqrt{\frac{l\pm m +\frac{1}{2}}{2l+1}}\right)\\
= -\frac{B}{2l+1}\sqrt{(l\mp m +\frac{1}{2})(l\pm m +\frac{1}{2})}
\end{align*} 

For the states that are already diagonal in the perturbation (the third group of states given above), the first order energy shift is just the corresponding diagonal matrix element. Using our expression above 

\begin{align*}
\Delta_{0,\frac{1}{2},\pm\frac{1}{2}} = \pm B
\end{align*}

where the $\pm$ comes from the sign of $m$. For the second pair, $j\neq s$ so this one includes an $A$ term:

\begin{align*}
\Delta_{0,\frac{3}{2},\pm\frac{3}{2}} = \frac{3A}{2} \pm 3B
\end{align*}

The other diagonal elements for the $l=1$ states are

\begin{align*}
\bra{1,\frac{1}{2},\pm\frac{1}{2}}V\ket{1,\frac{1}{2},\pm\frac{1}{2}} = -\frac{3A}{2} - \frac{B}{3}
\end{align*}

\begin{align*}
\bra{1,\frac{3}{2},\pm\frac{1}{2}}V\ket{1,\frac{3}{2},\pm\frac{1}{2}} = - \frac{B}{3}
\end{align*}

The off diagonal elements are

\begin{align*}
\bra{1,\frac{1}{2},\pm\frac{1}{2}}V\ket{1,\frac{3}{2},\pm\frac{1}{2}} = -\frac{B}{3}\sqrt{(\frac{3}{2}\mp \frac{1}{2})(\frac{3}{2}\pm \frac{1}{2})}
\end{align*}

\begin{align*}
\bra{1,\frac{3}{2},\pm\frac{1}{2}}V\ket{1,\frac{1}{2},\pm\frac{1}{2}} = -\frac{B}{3}\sqrt{(\frac{3}{2}\mp \frac{1}{2})(\frac{3}{2}\pm \frac{1}{2})}
\end{align*}

Then you just need to set $\pm$ to $+$, and solve the characteristic equation for the eigenvalues, set $\pm$ to $-$, and solve that characteristic equation as well. The two sets of eigenvalues are the first-order shifts in the energies for those states.

...





\end{s}

\end{document}