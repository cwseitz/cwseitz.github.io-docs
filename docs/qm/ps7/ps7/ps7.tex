\documentclass[12pt]{article}
\usepackage{amsmath} % AMS Math Package
\usepackage{bm}
\usepackage{amsthm} % Theorem Formatting
\usepackage{amssymb}    % Math symbols such as \mathbb
\usepackage{graphicx} % Allows for eps images
\usepackage[dvips,letterpaper,margin=1in,bottom=0.7in]{geometry}
\usepackage{tensor}
\usepackage{amsmath}
\usepackage{siunitx}
\usepackage{physics}
\usepackage{amsmath, amssymb, graphics, setspace}

\newcommand{\mathsym}[1]{{}}
\newcommand{\unicode}[1]{{}}

\newcounter{mathematicapage}

\newtheorem{p}{Problem}
\usepackage{cancel}
\newtheorem*{lem}{Lemma}
\theoremstyle{definition}
\newtheorem*{dfn}{Definition}
 \newenvironment{s}{%\small%
        \begin{trivlist} \item \textbf{Solution}. }{%
            \hspace*{\fill} $\blacksquare$\end{trivlist}}%


\begin{document}

 {\noindent\Huge\bf  \\[0.5\baselineskip] {\fontfamily{cmr}\selectfont  Homework 7}         }\\[2\baselineskip] % Title
{ {\bf \fontfamily{cmr}\selectfont Quantum Mechanics}\\ {\textit{\fontfamily{cmr}\selectfont     \today}}}~~~~~~~~~~~~~~~~~~~~~~~~~~~~~~~~~~~~~~~~~~~~~~~~~~~~~~~~~~~~~~~~~~~~~~~~~~~~~    {\large \textsc{C Seitz}
\\[1.4\baselineskip] 

\begin{p}
5.1
\end{p}

\begin{s}

We are concerned here with the new ground state ket $\ket{0}$ in the presence of $H_{1}$ and the new ground state energy shift $\Delta_{0}$. 

\begin{align*}
\ket{0} = \ket{0^{0}} + \sum_{k\neq 0}\ket{k^{0}}\frac{V_{k0}}{E_{0}^{0}-E_{k}^{0}} + ...
\end{align*}

\begin{align*}
\Delta_{0} = V_{00} + \sum_{k\neq 0}\frac{|V_{k0}|^{2}}{E_{0}^{0}-E_{k}^{0}} + ...
\end{align*}

\begin{align*}
V_{nk} = b\bra{n^{0}}x\ket{k^{0}} &= b\sqrt{\frac{\hbar}{2m\omega}}\left(\sqrt{k}\delta_{n,k-1} + \sqrt{k+1}\delta_{n,k+1}\right)
\end{align*}

The lowest nonvanishing order is then $V_{01}$. Therefore 

\begin{align*}
\Delta_{0} = -\frac{b^{2}\hbar}{2m\omega}\frac{1}{\hbar\omega} = -\frac{b^{2}}{2m\omega^{2}}
\end{align*}

To solve it exactly, notice that the potential is of the form

\begin{align*}
V_{1}(x) = ax^{2} + bx
\end{align*}

The new potential shifts to the left by $b/2$, has a new minimum at $-b/2a$, and the gradient has changed:

\begin{align*}
V'(x) = 2ax \rightarrow V_{1}'(x) = 2ax + b
\end{align*}

This change in the gradient will not change the energy differences w.r.t. the original problem (why?) so we have really just shifted the equilibrium point down by $-b/2m\omega^{2}$.

\begin{align*}
\Delta = -\frac{b}{2a} = -\frac{b}{2m\omega^{2}}
\end{align*}

which is exactly what we got with perturbation theory.


\end{s}

\begin{p}
5.2
\end{p}

\begin{s}
The perturbation Hamiltonian is 

\begin{align*}
H_{1} = \frac{Vx}{L}
\end{align*}

\begin{align*}
V_{nk} = \bra{n^{0}}H_{1}\ket{k^{0}} &= \frac{V}{L}\bra{n^{0}}x\ket{k^{0}} \\
&= \frac{V}{L}\int_{0}^{L} x \sin\left(\frac{n\pi x}{L}\right)\sin\left(\frac{k\pi x}{L}\right)dx
\end{align*}


\begin{align*}
\Delta_{0} = V_{00} + \sum_{k\neq 0}\frac{|V_{k0}|^{2}}{E_{0}^{0}-E_{k}^{0}} + ...
\end{align*}



\end{s}

\begin{p}
5.5
\end{p}

\begin{s}
\end{s}

\begin{p}
5.7
\end{p}

\begin{s}
\end{s}

\begin{p}
5.12a
\end{p}

\begin{s}
\end{s}

\begin{p}
5.24
\end{p}

\begin{s}
\end{s}

\end{document}