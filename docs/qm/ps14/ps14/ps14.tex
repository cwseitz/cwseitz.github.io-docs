\documentclass[12pt]{article}
\usepackage{amsmath} % AMS Math Package
\usepackage{bm}
\usepackage{amsthm} % Theorem Formatting
\usepackage{amssymb}    % Math symbols such as \mathbb
\usepackage{graphicx} % Allows for eps images
\usepackage[dvips,letterpaper,margin=1in,bottom=0.7in]{geometry}
\usepackage{tensor}
\usepackage{amsmath}
\usepackage{siunitx}
\usepackage{physics}
\usepackage{amsmath, amssymb, graphics, setspace}

\newcommand{\mathsym}[1]{{}}
\newcommand{\unicode}[1]{{}}

\newcounter{mathematicapage}

\newtheorem{p}{Problem}
\usepackage{cancel}
\newtheorem*{lem}{Lemma}
\theoremstyle{definition}
\newtheorem*{dfn}{Definition}
 \newenvironment{s}{%\small%
        \begin{trivlist} \item \textbf{Solution}. }{%
            \hspace*{\fill} $\blacksquare$\end{trivlist}}%


\begin{document}

 {\noindent\Huge\bf  \\[0.5\baselineskip] {\fontfamily{cmr}\selectfont  Homework 4}         }\\[2\baselineskip] % Title
{ {\bf \fontfamily{cmr}\selectfont Quantum Mechanics}\\ {\textit{\fontfamily{cmr}\selectfont     \today}}}~~~~~~~~~~~~~~~~~~~~~~~~~~~~~~~~~~~~~~~~~~~~~~~~~~~~~~~~~~~~~~~~~~~~~~~~~~~~~    {\large \textsc{C Seitz}
\\[1.4\baselineskip] 


\begin{p}
Problem 2.65
\end{p}

\begin{s}
Let us call these states $\ket{\alpha}$ and $\ket{\beta}$:


\begin{align*}
\ket{\alpha} &= \frac{\ket{0} + \ket{1}}{\sqrt{2}}\\
\ket{\beta} &= \frac{\ket{0} - \ket{1}}{\sqrt{2}}
\end{align*}


If we choose a non-orthogonal basis, such as

\begin{align*}
\ket{e_{1}} = \ket{0}\;\; \ket{e_{2}} = \frac{\ket{0} + \ket{1}}{\sqrt{2}}
\end{align*}

These states have the following representation in this new basis

\begin{align*}
\ket{\alpha'} &= \left(\ket{e_{1}}\bra{e_{1}} + \ket{e_{2}}\bra{e_{2}}\right)\ket{\alpha}\\
&= \frac{1}{\sqrt{2}}\ket{e_{1}} + \ket{e_{2}}
\end{align*}

\begin{align*}
\ket{\beta'} &= \left(\ket{e_{1}}\bra{e_{1}} + \ket{e_{2}}\bra{e_{2}}\right)\ket{\beta}\\
&= \frac{1}{\sqrt{2}}\ket{e_{1}}
\end{align*}

The norm is not preserved, because the change of basis matrix $\ket{e_{1}}\bra{e_{1}} + \ket{e_{2}}\bra{e_{2}}$ was not unitary. But it is clear that these states differ neither by a global or relative phase.

\end{s}

\begin{p}
Problem 2.66
\end{p}

\begin{s}

\begin{align*}
\bra{\alpha}X_{1}Z_{2}\ket{\alpha} &= \frac{1}{2}(\bra{00}+\bra{11}) X_{1}Z_{2} (\ket{00} + \ket{11})\\
&= \frac{1}{2}(\bra{00}+\bra{11})(\ket{10} - \ket{01}) = 0
\end{align*}

\end{s}

\begin{p}
Problem 2.71
\end{p}

\begin{s}

\begin{align*}
\mathrm{Tr} (\rho^{2}) &= \sum_{k}\bra{k}\left(\sum_{i}p_{i}\ket{\alpha_{i}}\bra{\alpha_{i}}\right)\left(\sum_{j}p_{j}\ket{\alpha_{j}}\bra{\alpha_{j}}\right)\ket{k}\\
&= \left(\sum_{ijk}p_{i}p_{j}\bra{k}\ket{\alpha_{i}}\bra{\alpha_{i}}\ket{\alpha_{j}}\bra{\alpha_{j}}\ket{k}\right)\\
&= \sum_{ij} p_{i}p_{j} |\bra{\alpha_{i}}\ket{\alpha_{j}}|^{2} \\
&= \sum_{i}p_{i}^{2} \leq 1
\end{align*}


if $\ket{\alpha_{i}}$ and $\ket{\alpha_{j}}$ are orthonormal.

\end{s}

\begin{p}
Problem 2.72
\end{p}

\begin{s}

The Pauli matrices form a valid basis for 2x2 matrices. The Bloch vector representation for $\rho = I/2$ is $\vec{r} = 0$. 

\begin{align*}
\mathrm{Tr}(\rho^{2}) &= \mathrm{Tr}\left(\frac{I + 2(\vec{r}\cdot\sigma) + (\vec{r}\cdot\sigma)^{2}}{4}\right)\\
&= \frac{1}{2} + \frac{||\vec{r}||^{2}}{2} = 1
\end{align*}

which occurs when $||\vec{r}||^{2} = 1$. This is just algebra once we notice that the trace of $\vec{r}\cdot\sigma$ is zero and the trace of $(\vec{r}\cdot\sigma)^{2} = 2(r_{x}^{2} + r_{y}^{2} + r_{z}^{2})$ (the cross terms cancel since the anticommutator $\{\sigma_{i},\sigma{j}\} = \delta_{ij}$)

\end{s}

\begin{p}
Problem 2.75
\end{p}

\begin{s}

The bell states are

\begin{align*}
\ket{\phi^{+}} &= \frac{1}{\sqrt{2}}\left(\ket{00} + \ket{11}\right)\\
\ket{\phi^{-}} &= \frac{1}{\sqrt{2}}\left(\ket{00} - \ket{11}\right)\\
\ket{\psi^{+}} &= \frac{1}{\sqrt{2}}\left(\ket{01} + \ket{10}\right)\\
\ket{\psi^{-}} &= \frac{1}{\sqrt{2}}\left(\ket{01} - \ket{10}\right)
\end{align*}

For each state, we can trace out either the first or second qubit. 

\begin{align*}
\mathrm{tr}_{1}(\ket{\phi^{+}}\bra{\phi^{+}}) &= \frac{\mathrm{tr}_{1}\ket{00}\bra{00} + \mathrm{tr}_{1}\ket{00}\bra{11} + \mathrm{tr}_{1}\ket{11}\bra{00} + \mathrm{tr}_{1}\ket{11}\bra{11}}{2}\\
&= \frac{\bra{0}\ket{0}\ket{0}\bra{0} + \bra{0}\ket{1}\ket{0}\bra{1} + \bra{1}\ket{0}\ket{1}\bra{0} + \bra{1}\ket{1}\ket{1}\bra{1}}{2}\\
&= \frac{\ket{0}\bra{0} + \ket{1}\bra{1}}{2}
\end{align*}

In fact we get this same result when we trace out either qubit for any of the Bell states. Applying either partial trace to the cross terms always gives zero. 

\end{s}

\begin{p}
Problem 2.79
\end{p}

\begin{s}

For $\ket{\psi} = \frac{\ket{00} + \ket{11}}{\sqrt{2}}$, we have

\begin{align*}
\ket{\psi} = \sum_{jk} a_{jk}\ket{j}\ket{k}\;\; a = \frac{1}{\sqrt{2}}\begin{pmatrix}1 & 0\\ 0 & 1\end{pmatrix}
\end{align*}

The matrix $a$ is already diagonal, so the unitary matrices in the SVD will be identity matrices. Therefore, 

\begin{align*}
\ket{i_{A}} = \ket{0}_{A}, \ket{1}_{A} \;\;\; \ket{i_{B}} = \ket{0}_{B}, \ket{1}_{B} \;\;\; \lambda_{i} = \frac{1}{\sqrt{2}}
\end{align*}


\begin{align*}
\ket{\psi} = \frac{1}{\sqrt{2}}\left(\ket{0}_{A}\ket{0}_{B} + \ket{1}_{A}\ket{1}_{B}\right)
\end{align*}

For $\ket{\psi} = \frac{\ket{00} + \ket{01} + \ket{10} + \ket{11}}{2}$, we have the SVD

\begin{align*}
a = \frac{1}{2}\begin{pmatrix}1 & 1\\ 1& 1\end{pmatrix} = \left(
\begin{array}{cc}
 -\frac{1}{\sqrt{2}} & -\frac{1}{\sqrt{2}} \\
 -\frac{1}{\sqrt{2}} & \frac{1}{\sqrt{2}} \\
\end{array}
\right)\left(
\begin{array}{cc}
 1 & 0 \\
 0 & 0 \\
\end{array}
\right)
\left(
\begin{array}{cc}
 -\frac{1}{\sqrt{2}} & -\frac{1}{\sqrt{2}} \\
 -\frac{1}{\sqrt{2}} & \frac{1}{\sqrt{2}} \\
\end{array}
\right)
\end{align*}

\begin{align*}
\ket{i_{A}} = -\frac{1}{\sqrt{2}}(\ket{0}+\ket{1}), \frac{1}{\sqrt{2}}(\ket{1} - \ket{0})
\end{align*}

\begin{align*}
\ket{i_{B}} = \frac{1}{\sqrt{2}}(\ket{1}-\ket{0}), \frac{1}{\sqrt{2}}(\ket{1} - \ket{0})
\end{align*}


\begin{align*}
\ket{\psi} = -\frac{1}{2}\left(\ket{0} + \ket{1}\right)\left(\ket{0} + \ket{1}\right)
\end{align*}


For $\ket{\psi} = \frac{\ket{00} + \ket{01} + \ket{10}}{\sqrt{3}}$, we have the SVD

\begin{align*}
a = \frac{1}{\sqrt{3}}\begin{pmatrix}1 & 1 \\ 1 & 0\end{pmatrix}
\end{align*}

\end{s}

\end{document}