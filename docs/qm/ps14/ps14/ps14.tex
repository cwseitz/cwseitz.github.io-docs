\documentclass[12pt]{article}
\usepackage{amsmath} % AMS Math Package
\usepackage{bm}
\usepackage{amsthm} % Theorem Formatting
\usepackage{amssymb}    % Math symbols such as \mathbb
\usepackage{graphicx} % Allows for eps images
\usepackage[dvips,letterpaper,margin=1in,bottom=0.7in]{geometry}
\usepackage{tensor}
\usepackage{amsmath}
\usepackage{siunitx}
\usepackage{physics}
\usepackage{amsmath, amssymb, graphics, setspace}

\newcommand{\mathsym}[1]{{}}
\newcommand{\unicode}[1]{{}}

\newcounter{mathematicapage}

\newtheorem{p}{Problem}
\usepackage{cancel}
\newtheorem*{lem}{Lemma}
\theoremstyle{definition}
\newtheorem*{dfn}{Definition}
 \newenvironment{s}{%\small%
        \begin{trivlist} \item \textbf{Solution}. }{%
            \hspace*{\fill} $\blacksquare$\end{trivlist}}%


\begin{document}

 {\noindent\Huge\bf  \\[0.5\baselineskip] {\fontfamily{cmr}\selectfont  Homework 4}         }\\[2\baselineskip] % Title
{ {\bf \fontfamily{cmr}\selectfont Quantum Mechanics}\\ {\textit{\fontfamily{cmr}\selectfont     \today}}}~~~~~~~~~~~~~~~~~~~~~~~~~~~~~~~~~~~~~~~~~~~~~~~~~~~~~~~~~~~~~~~~~~~~~~~~~~~~~    {\large \textsc{C Seitz}
\\[1.4\baselineskip] 


\begin{p}
Problem 2.65
\end{p}

\begin{s}
Let us call these states $\ket{\alpha}$ and $\ket{\beta}$:


\begin{align*}
\ket{\alpha} &= \frac{\ket{0} + \ket{1}}{\sqrt{2}}\\
\ket{\beta} &= \frac{\ket{0} - \ket{1}}{\sqrt{2}}
\end{align*}


If we choose a non-orthogonal basis, such as

\begin{align*}
\ket{e_{1}} = \ket{0}\;\; \ket{e_{2}} = \frac{\ket{0} + \ket{1}}{\sqrt{2}}
\end{align*}

These states have the following representation in this new basis

\begin{align*}
\ket{\alpha'} &= \left(\ket{e_{1}}\bra{e_{1}} + \ket{e_{2}}\bra{e_{2}}\right)\ket{\alpha}\\
&= \frac{1}{\sqrt{2}}\ket{e_{1}} + \ket{e_{2}}
\end{align*}

\begin{align*}
\ket{\beta'} &= \left(\ket{e_{1}}\bra{e_{1}} + \ket{e_{2}}\bra{e_{2}}\right)\ket{\beta}\\
&= \frac{1}{\sqrt{2}}\ket{e_{1}}
\end{align*}

The norm is not preserved, because the change of basis matrix $\ket{e_{1}}\bra{e_{1}} + \ket{e_{2}}\bra{e_{2}}$ was not unitary. But it is clear that these states differ neither by a global or relative phase.

\end{s}

\begin{p}
Problem 2.66
\end{p}

\begin{s}

\begin{align*}
\bra{\alpha}X_{1}Z_{2}\ket{\alpha} &= \frac{1}{2}(\bra{00}+\bra{11}) X_{1}Z_{2} (\ket{00} + \ket{11})\\
&= \frac{1}{2}(\bra{00}+\bra{11})(\ket{10} - \ket{01}) = 0
\end{align*}

\end{s}

\begin{p}
Problem 2.71
\end{p}

\begin{s}
\end{s}

\begin{p}
Problem 2.72
\end{p}

\begin{s}
\end{s}

\begin{p}
Problem 2.75
\end{p}

\begin{s}
\end{s}

\begin{p}
Problem 2.79
\end{p}

\begin{s}
\end{s}

\end{document}