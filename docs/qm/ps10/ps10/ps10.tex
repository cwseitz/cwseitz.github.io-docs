\documentclass[12pt]{article}
\usepackage{amsmath} % AMS Math Package
\usepackage{bm}
\usepackage{amsthm} % Theorem Formatting
\usepackage{amssymb}    % Math symbols such as \mathbb
\usepackage{graphicx} % Allows for eps images
\usepackage[dvips,letterpaper,margin=1in,bottom=0.7in]{geometry}
\usepackage{tensor}
\usepackage{amsmath}
\usepackage{siunitx}
\usepackage{physics}
\usepackage{amsmath, amssymb, graphics, setspace}

\newcommand{\mathsym}[1]{{}}
\newcommand{\unicode}[1]{{}}

\newcounter{mathematicapage}

\newtheorem{p}{Problem}
\usepackage{cancel}
\newtheorem*{lem}{Lemma}
\theoremstyle{definition}
\newtheorem*{dfn}{Definition}
 \newenvironment{s}{%\small%
        \begin{trivlist} \item \textbf{Solution}. }{%
            \hspace*{\fill} $\blacksquare$\end{trivlist}}%


\begin{document}

 {\noindent\Huge\bf  \\[0.5\baselineskip] {\fontfamily{cmr}\selectfont  Homework 10}         }\\[2\baselineskip] % Title
{ {\bf \fontfamily{cmr}\selectfont Quantum Mechanics}\\ {\textit{\fontfamily{cmr}\selectfont     \today}}}~~~~~~~~~~~~~~~~~~~~~~~~~~~~~~~~~~~~~~~~~~~~~~~~~~~~~~~~~~~~~~~~~~~~~~~~~~~~~    {\large \textsc{C Seitz}
\\[1.4\baselineskip] 

\begin{p}
4.7
\end{p}

\begin{s}

The wave function in three dimensions for a free particle ($V=0$), is

\begin{align*}
\psi(\bm{x},t) &= u(\bm{x})e^{-iE_{n}t/\hbar}\\
\psi^{*}(\bm{x},-t) &= u^{*}(\bm{x})e^{-iE_{n}t/\hbar}\\
\end{align*}

where $u(\bm{x}) = e^{i\vec{p}\cdot \vec{k}}$. Note that the phase remains unchanged under complex conjugation and time reversal. Now if we reverse the direction of momentum i.e. $\ket{p}\rightarrow \ket{p'}$ for $\vec{p}\cdot\vec{p'} = -1$,

\begin{align*}
\psi'(\bm{x},t) &= u'(\bm{x})e^{-iE_{n}t/\hbar}\\
\end{align*}

Notice that $u'(\bm{x}) = e^{-i\vec{p'}\cdot \vec{k}} = u^{*}(\bm{x})$. Therefore $\psi'(\bm{x},t) = \psi^{*}(\bm{x},-t)$


\begin{align*}
\chi_{+}(\hat{n}) = \begin{pmatrix}\cos\frac{\theta}{2} \\ \sin\frac{\theta}{2}e^{i\gamma}\end{pmatrix}
\end{align*}

It is also known that

\begin{align*}
\chi_{-}(\hat{n}) = \begin{pmatrix}-\sin\frac{\theta}{2}e^{-i\gamma} \\ \cos\frac{\theta}{2} \end{pmatrix}
\end{align*}

So we just need to show that the given transformation gives this result:

\begin{align*}
-i\sigma_{2}\chi^{*}(\hat{n}) = -i\begin{pmatrix}0&-i\\i&0\end{pmatrix}\begin{pmatrix}\cos\frac{\theta}{2} \\ \sin\frac{\theta}{2}e^{-i\gamma}\end{pmatrix} = \begin{pmatrix}-\sin\frac{\theta}{2}e^{-i\gamma} \\ \cos\frac{\theta}{2} \end{pmatrix}
\end{align*}

\end{s}

\begin{p}
4.8
\end{p}

\begin{s}

First note that 

\begin{align*}
H\Theta\ket{n} = \Theta H \ket{n} = E_{n}\Theta \ket{n}
\end{align*}

so $\ket{n}$ and $\Theta\ket{n}$ have the same energy. If the states are nondegenerate then $\ket{n}$ and $\Theta\ket{n}$ represent the same state. Their wavefunctions are then the same:


\begin{align*}
\bra{x'}\ket{n} = \bra{n}\ket{x'}^{*}
\end{align*}

which occurs if they are real, or have a phase difference independent of x. For this reason the wavefunction $\psi = e^{ip\cdot x/\hbar}$ does not violate time reversal invariance, because it is degenerate with $e^{ip\cdot x/\hbar}$.

\end{s}

\begin{p}
4.9
\end{p}

\begin{s}
\begin{align*}
\Theta\ket{\alpha} &= \int d^{3}\bm{p}\;\Theta\ket{\bm{p}}\bra{\bm{p}}\ket{\alpha}^{*}\\
&= \int d^{3}\bm{p}\;\ket{\bm{-p}}\bra{\bm{p}}\ket{\alpha}^{*}\\
&=  \int d^{3}\bm{p}\;\ket{\bm{p}}\bra{-\bm{p}}\ket{\alpha}^{*}\\
&= \phi^{*}(-p)
\end{align*}
\end{s}

\begin{p}
4.10
\end{p}

\begin{s}

We first try to prove that

\begin{align*}
\Theta\ket{j,m} = e^{i\delta}(-1)^{m}\ket{j,-m}
\end{align*}

Consider acting on $\ket{j,m}$ with $J_{z}$:

\begin{align*}
J_{z}\Theta\ket{j,m} = -\Theta J_{z}\ket{j,m} = -m \left(\Theta\ket{j,m}\right)
\end{align*}

so clearly $\Theta\ket{j,m}$ behaves like $\ket{j,-m}$. The $e^{i\delta}$ part is there because we can always include an arbitrary phase. 


\begin{align*}
\Theta \mathcal{D}(R) \Theta^{-1} &= \Theta\left(1-i\frac{J\cdot \hat{n}}{\hbar}\right) \Theta^{-1}\\
&= \left(1+i\frac{\Theta J\Theta^{-1}\cdot \hat{n}}{\hbar}\right) \\
&= \left(1-i\frac{J\cdot \hat{n}}{\hbar}\right) = \mathcal{D}(R)
\end{align*}

so the time reversed state is just $\mathcal{D}(R)\Theta\ket{j,m}$.


\end{s}


\begin{p}
4.11
\end{p}

\begin{s}

Since the Hamiltonian is time reversal invariant, then energy eigenkets transform as $\Theta\ket{\alpha} = e^{i\delta}\ket{\alpha}$, 

\begin{align*}
\langle \bm{L}\rangle &= \bra{\alpha}\bm{L}\ket{\alpha} \\
&= e^{i\delta}e^{-i\delta}\bra{\alpha}\Theta\bm{L}\Theta^{-1}\ket{\alpha}\\
&= -\bra{\alpha}\bm{L}\ket{\alpha}
\end{align*}

This is only satisfied when $\langle \bm{L}\rangle = 0$. We know that when the Hamiltonian is invariant under time-reversal, the eigenkets must be real.

\begin{align*}
\psi(r,\theta,\phi) &= \sum_{l,m}F_{lm}(r)Y_{l}^{m}(\theta,\phi)\\
&\rightarrow e^{i\delta}\sum_{l,m}F_{lm}^{*}(r)(Y_{l}^{m}(\theta,\phi))^{*}\\
&= e^{i\delta} \sum_{l,m}F_{lm}^{*}(r)(-1)^{m}Y_{l}^{m}(\theta,\phi)
\end{align*}

which means that $F_{lm}(r) = e^{-i\delta}(-1)^{m}F_{lm}^{*}(r)$

\end{s}

\begin{p}
4.12
\end{p}

\begin{s}

We were given the Hamiltonian:

\begin{align*}
H = AS_{z}^{2} + B(S_{x}^{2} - S_{y}^{2})
\end{align*}

This Hamiltonian must be invariant under time reversal because these are all scalar values which are invariant e.g, $\Theta S_{x}^{2}\Theta^{-1} = S_{x}^{2}\Theta\Theta^{-1} = S_{x}^{2}$ and $A$ and $B$ are real. The explicit matrix representation is

\begin{align*}
H = \hbar^{2}\begin{pmatrix}A&0&B\\0&0&0\\B&0&A\end{pmatrix}
\end{align*}

which according to Mathematica has the following eigenvectors


\begin{align*}
\ket{E_{1}} &= (\ket{+1}+\ket{-1})/\sqrt{2}\\
\ket{E_{-1}} &= (\ket{+1}-\ket{-1})/\sqrt{2}\\
\ket{E_{0}} &= \ket{0}\\
\end{align*}

with eigenvalues $\hbar^{2}(A+B), \hbar^{2}(A-B), 0$ respectively. These eigenvectors transform in the following way under time reversal:

\begin{align*}
\Theta\ket{E_{1}} &= (\Theta\ket{+1}+\Theta\ket{-1})/\sqrt{2} = -(\ket{+1}+\ket{-1})/\sqrt{2}\\
\Theta\ket{E_{-1}} &= (\Theta\ket{+1}-\Theta\ket{-1})/\sqrt{2} = -(\ket{+1}-\ket{-1})/\sqrt{2}\\
\Theta\ket{E_{0}} &= (-1)^{0}\ket{E_{0}} = \ket{E_{0}}\\
\end{align*}


\end{s}

\end{document}