\documentclass[12pt]{article}
\usepackage{amsmath} % AMS Math Package
\usepackage{bm}
\usepackage{amsthm} % Theorem Formatting
\usepackage{amssymb}    % Math symbols such as \mathbb
\usepackage{graphicx} % Allows for eps images
\usepackage[dvips,letterpaper,margin=1in,bottom=0.7in]{geometry}
\usepackage{tensor}
\usepackage{amsmath}
\usepackage{siunitx}
\usepackage{physics}
\usepackage{amsmath, amssymb, graphics, setspace}

\newcommand{\mathsym}[1]{{}}
\newcommand{\unicode}[1]{{}}

\newcounter{mathematicapage}

\newtheorem{p}{Problem}
\usepackage{cancel}
\newtheorem*{lem}{Lemma}
\theoremstyle{definition}
\newtheorem*{dfn}{Definition}
 \newenvironment{s}{%\small%
        \begin{trivlist} \item \textbf{Solution}. }{%
            \hspace*{\fill} $\blacksquare$\end{trivlist}}%


\begin{document}

 {\noindent\Huge\bf  \\[0.5\baselineskip] {\fontfamily{cmr}\selectfont  Homework 10}         }\\[2\baselineskip] % Title
{ {\bf \fontfamily{cmr}\selectfont Quantum Mechanics}\\ {\textit{\fontfamily{cmr}\selectfont     \today}}}~~~~~~~~~~~~~~~~~~~~~~~~~~~~~~~~~~~~~~~~~~~~~~~~~~~~~~~~~~~~~~~~~~~~~~~~~~~~~    {\large \textsc{C Seitz}
\\[1.4\baselineskip] 

\begin{p}
4.7
\end{p}

\begin{s}

The wave function in three dimensions for a free particle ($V=0$), is

\begin{align*}
\psi(\bm{x},t) &= u(\bm{x})e^{-iE_{n}t/\hbar}\\
\psi^{*}(\bm{x},-t) &= u^{*}(\bm{x})e^{-iE_{n}t/\hbar}\\
\end{align*}

where $u(\bm{x}) = e^{i\vec{p}\cdot \vec{k}}$. Note that the phase remains unchanged under complex conjugation and time reversal. Now if we reverse the direction of momentum i.e. $\ket{p}\rightarrow \ket{p'}$ for $\vec{p}\cdot\vec{p'} = -1$,

\begin{align*}
\psi'(\bm{x},t) &= u'(\bm{x})e^{-iE_{n}t/\hbar}\\
\end{align*}

Notice that $u'(\bm{x}) = e^{-i\vec{p'}\cdot \vec{k}} = u^{*}(\bm{x})$. Therefore $\psi'(\bm{x},t) = \psi^{*}(\bm{x},-t)$


\begin{align*}
\chi_{+}(\hat{n}) = \begin{pmatrix}\cos\frac{\theta}{2} \\ \sin\frac{\theta}{2}e^{i\gamma}\end{pmatrix}
\end{align*}

It is also known that

\begin{align*}
\chi_{-}(\hat{n}) = \begin{pmatrix}-\sin\frac{\theta}{2}e^{-i\gamma} \\ \cos\frac{\theta}{2} \end{pmatrix}
\end{align*}

So we just need to prove that the given transformation gives this result, which it does

\begin{align*}
-i\sigma_{2}\chi^{*}(\hat{n}) = -i\begin{pmatrix}0&-i\\i&0\end{pmatrix}\begin{pmatrix}\cos\frac{\theta}{2} \\ \sin\frac{\theta}{2}e^{-i\gamma}\end{pmatrix} = \begin{pmatrix}-\sin\frac{\theta}{2}e^{-i\gamma} \\ \cos\frac{\theta}{2} \end{pmatrix}
\end{align*}


\end{s}

\begin{p}
4.8
\end{p}

\begin{s}
\end{s}

\begin{p}
4.9
\end{p}

\begin{s}
\end{s}

\begin{p}
4.10
\end{p}

\begin{s}
\end{s}


\begin{p}
4.11
\end{p}

\begin{s}
\end{s}

\begin{p}
4.12
\end{p}

\begin{s}
\end{s}

\end{document}