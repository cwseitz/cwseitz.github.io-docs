% Latex template: mahmoud.s.fahmy@students.kasralainy.edu.eg
% For more details: https://www.sharelatex.com/learn/Beamer

\documentclass[aspectratio=1610]{beamer}					% Document class

\setbeamertemplate{footline}[text line]{%
  \parbox{\linewidth}{\vspace*{-8pt}Bell's Inequality \hfill\insertshortauthor\hfill\insertpagenumber}}
\setbeamertemplate{navigation symbols}{}

\usepackage[english]{babel}				% Set language
\usepackage[utf8x]{inputenc}			% Set encoding

\mode<presentation>						% Set options
{
  \usetheme{default}					% Set theme
  \usecolortheme{default} 				% Set colors
  \usefonttheme{default}  				% Set font theme
  \setbeamertemplate{caption}[numbered]	% Set caption to be numbered
}

% Uncomment this to have the outline at the beginning of each section highlighted.
%\AtBeginSection[]
%{
%  \begin{frame}{Outline}
%    \tableofcontents[currentsection]
%  \end{frame}
\usepackage{graphicx}					% For including figures
\usepackage{booktabs}					% For table rules
\usepackage{hyperref}	
\usepackage{tikz-network}				% For cross-referencing
\usepackage[absolute,overlay]{textpos}
\usepackage{bm}
\usepackage[font=small,labelfont=bf]{caption}				% For cross-referencing
\usepackage{physics}

\title{The Hidden Subgroup Problem}	% Presentation title
\author{Clayton W. Seitz}								% Presentation author
\date{\today}									% Today's date	

\begin{document}

% Title page
% This page includes the informations defined earlier including title, author/s, affiliation/s and the date
\begin{frame}
  \titlepage
\end{frame}



\begin{frame}{The Hidden Subgroup Problem}
Let $G$ be a group and $X$ a finite set and $f:G\rightarrow X$ a function that \emph{hides} a subgroup $H\leq G$. The problem is to determine $H$. A nice example for the Abelian version is Simon's problem.\\
\vspace{0.2in}

\textbf{Simon's problem}. Given a 2-1 function $f:\{0,1\}^{n}\rightarrow \{0,1\}^{n}$ such that there is a secret string $s\in\{0,1\}^{n}$ where $f(x) = f(y)$ if and only if $x\oplus y = s$.\\
\vspace{0.2in}
The function $f$ is a black box. Clasically you would solve the problem by drawing pairs $x,y$ and checking if $f(x) = f(y)$. If they match, you can obviously retrieve $s = x\oplus y$\\
\vspace{0.2in}
Clasically the problem scales as $\mathcal{O}(2^{n/2})$ but Simon designed a quantum algorithm that scales as $\mathcal{O}(n)$.

\end{frame}

\begin{frame}{Solution to Simon's problem}
Solution is very similar to the common solution to the HSP\\
\vspace{0.1in}
In the first register, we prepare a uniform superposition over all possible input strings $x$\\
\vspace{0.1in}
In the second register we use ancillary bits that will store $f(x)$\\
\begin{equation*}
\ket{\psi} = H^{\otimes n}\ket{0^{n}} = \frac{1}{2^{n/2}}\sum_{x\in\{0,1\}^{n}}\ket{x}
\end{equation*}

We assume we have some oracle function $U_{f}$ which will compute and store $f(x)$ in register 2
\begin{equation*}
O_{f}(\ket{\psi}\ket{0^{n}}) = \frac{1}{2^{n/2}}\sum_{x\in\{0,1\}^{n}}\ket{x}\ket{f(x)}
\end{equation*}



\end{frame}

\begin{frame}{Solution to Simon's problem}

Then we measure the second register\\
\vspace{0.1in}
This collapses the system to a superposition of the two inputs that map to our measured output $\ket{f(a)}$

\begin{equation*}
\frac{1}{2^{n/2}}\sum_{x\in\{0,1\}^{n}}\ket{x}\ket{f(x)} \rightarrow \left(\ket{a}+\ket{a\oplus s}\right)\otimes \ket{f(a)}
\end{equation*}
\vspace{0.1in}
Then, we can Fourier transform the first register

\begin{equation*}
\sum_{\gamma}\gamma\ket{\gamma}\otimes \ket{f(a)}
\end{equation*}

If we measure the first register we get $\gamma$ with probability $|\gamma|^{2}$.

\end{frame}


\end{document}