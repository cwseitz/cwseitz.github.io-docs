% Latex template: mahmoud.s.fahmy@students.kasralainy.edu.eg
% For more details: https://www.sharelatex.com/learn/Beamer

\documentclass[aspectratio=1610]{beamer}					% Document class

\setbeamertemplate{footline}[text line]{%
  \parbox{\linewidth}{\vspace*{-8pt}Bell's Inequality \hfill\insertshortauthor\hfill\insertpagenumber}}
\setbeamertemplate{navigation symbols}{}

\usepackage[english]{babel}				% Set language
\usepackage[utf8x]{inputenc}			% Set encoding

\mode<presentation>						% Set options
{
  \usetheme{default}					% Set theme
  \usecolortheme{default} 				% Set colors
  \usefonttheme{default}  				% Set font theme
  \setbeamertemplate{caption}[numbered]	% Set caption to be numbered
}

% Uncomment this to have the outline at the beginning of each section highlighted.
%\AtBeginSection[]
%{
%  \begin{frame}{Outline}
%    \tableofcontents[currentsection]
%  \end{frame}
\usepackage{graphicx}					% For including figures
\usepackage{booktabs}					% For table rules
\usepackage{hyperref}	
\usepackage{tikz-network}				% For cross-referencing
\usepackage[absolute,overlay]{textpos}
\usepackage{bm}
\usepackage[font=small,labelfont=bf]{caption}				% For cross-referencing
\usepackage{physics}

\title{The Quantum Fourier Transform}	% Presentation title
\author{Clayton W. Seitz}								% Presentation author
\date{\today}									% Today's date	

\begin{document}

% Title page
% This page includes the informations defined earlier including title, author/s, affiliation/s and the date
\begin{frame}
  \titlepage
\end{frame}

\begin{frame}{Introduction}

Dimension of $n$-qubit Hilbert space $N=2^{n}$\\
\vspace{0.2in}
The quantum fourier transform (QFT) transforms a quantum state $\ket{\psi}\rightarrow\ket{\phi}$ via the transformation of basis states:

\begin{equation*}
\mathrm{QFT}\ket{j} = \frac{1}{2^{n/2}}\sum_{k=1}^{2^{n}}e^{2\pi ijk/2^{n}}\ket{k}
\end{equation*}

Equivalently, on the state $\ket{\psi}=\sum_{j}\psi_{j}\ket{j}$ reads

\begin{equation*}
\mathrm{QFT}\ket{\psi} = \ket{\phi} = \frac{1}{2^{n/2}}\sum_{j=1}^{2^{n}}\psi_{j}\left(\sum_{k=1}^{2^{n}}e^{2\pi ijk/2^{n}}\ket{k}\right)
\end{equation*}

which turns out to be a unitary transformation

\end{frame}

\begin{frame}{Product representation of the QFT}

Computational basis ket $\ket{j} = \ket{j_{1}j_{2}...j_{n}}$\\
\vspace{0.1in}
Fourier basis ket $\ket{k} = \ket{k_{1}k_{2}...k_{n}}$\\
\vspace{0.1in}
Converting $k$ to binary: $k = \sum_{l}k_{l}2^{l}$ \\
\vspace{0.1in}
Also, note that $\ket{k} = \ket{k_{1}k_{2}...k_{n}} = \bigotimes_{l=1}^{n}\ket{k_{l}}$

\end{frame}

\begin{frame}{Product representation of the QFT}


\begin{align*}
\mathrm{QFT}\ket{j} &= \frac{1}{2^{n/2}}\sum_{k=0}^{2^{n}-1}e^{2\pi ijk/2^{n}}\ket{k}\\
&= \frac{1}{2^{n/2}}\sum_{k=0}^{2^{n}-1}e^{2\pi ij\sum_{l}k_{l}2^{-l}}\bigotimes_{l=1}^{n}\ket{k_{l}}\\
&= \frac{1}{2^{n/2}}\sum_{k=0}^{2^{n}-1}\bigotimes_{l=1}^{n}e^{2\pi ij k_{l}2^{-l}}\ket{k_{l}}\\
&= \frac{1}{2^{n/2}}\bigotimes_{l=1}^{n}\sum_{k_{l}=0}^{1}e^{2\pi ij k_{l}2^{-l}}\ket{k_{l}}\\
&= \frac{1}{2^{n/2}}\bigotimes_{l=1}^{n}\left(\ket{0}+e^{2\pi ij 2^{-l}}\ket{1}\right)\
\end{align*}

\end{frame}

\begin{frame}{Phase estimation of a unitary operator}

An important module in many quantum algorithms that uses QFT\\
\vspace{0.1in}
Consider an eigenvector $\ket{u}$ of a Unitary operator $U$. Its eigenvalue can be written as $u = e^{2\pi i\theta}$

\begin{equation*}
U\ket{u} = u\ket{u} = e^{2\pi i\theta}\ket{u}
\end{equation*}

\end{frame}

\begin{frame}{Basic group theory}
A \textbf{group} is a non-empty set and an operation that combines any two elements of the set to produce a third element of the set, in such a way that the operation is associative, an identity element exists and every element has an inverse\\
Example: the integers $\mathbb{Z}$ form a group\\
\vspace{0.2in}
A \textbf{subgroup} $H$ of a group $G$ also satisfies the basic axioms but is a subset\\
\vspace{0.2in}
A \textbf{coset} is a subset 
\end{frame}

\begin{frame}{Hidden subgroup problem (HSP)}
From Nielsen and Chuang: \\
\vspace{0.2in}
Let $f$ be a function from a finitely generated group $G$ to a finite set $X$ such that
$f$ is constant on the cosets of a subgroup $K$, and distinct on each coset. Given a
quantum black box for performing the unitary transform $U$ for $g \in G$, $h \in X$, and $\oplus$ an appropriately chosen binary operation on $X$, find a generating set for $K$.\\
\vspace{0.2in}
We can see how period finding is determining the subgroup over which $f$ is constant
\end{frame}


\end{document}