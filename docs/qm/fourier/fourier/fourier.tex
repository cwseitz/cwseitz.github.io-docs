% Latex template: mahmoud.s.fahmy@students.kasralainy.edu.eg
% For more details: https://www.sharelatex.com/learn/Beamer

\documentclass[aspectratio=1610]{beamer}					% Document class

\setbeamertemplate{footline}[text line]{%
  \parbox{\linewidth}{\vspace*{-8pt}Bell's Inequality \hfill\insertshortauthor\hfill\insertpagenumber}}
\setbeamertemplate{navigation symbols}{}

\usepackage[english]{babel}				% Set language
\usepackage[utf8x]{inputenc}			% Set encoding

\mode<presentation>						% Set options
{
  \usetheme{default}					% Set theme
  \usecolortheme{default} 				% Set colors
  \usefonttheme{default}  				% Set font theme
  \setbeamertemplate{caption}[numbered]	% Set caption to be numbered
}

% Uncomment this to have the outline at the beginning of each section highlighted.
%\AtBeginSection[]
%{
%  \begin{frame}{Outline}
%    \tableofcontents[currentsection]
%  \end{frame}
\usepackage{graphicx}					% For including figures
\usepackage{booktabs}					% For table rules
\usepackage{hyperref}	
\usepackage{tikz-network}				% For cross-referencing
\usepackage[absolute,overlay]{textpos}
\usepackage{bm}
\usepackage[font=small,labelfont=bf]{caption}				% For cross-referencing
\usepackage{physics}

\title{The Abelian Hidden Subgroup Problem}	% Presentation title
\author{Clayton W. Seitz}								% Presentation author
\date{\today}									% Today's date	

\begin{document}

% Title page
% This page includes the informations defined earlier including title, author/s, affiliation/s and the date
\begin{frame}
  \titlepage
\end{frame}

\begin{frame}{Introduction}

Dimension of $n$-qubit Hilbert space $N=2^{n}$\\
\vspace{0.2in}
The quantum fourier transform (QFT) transforms a quantum state $\ket{\psi}\rightarrow\ket{\phi}$ via the transformation of basis states:

\begin{equation*}
\mathrm{QFT}\ket{j} = \frac{1}{2^{n/2}}\sum_{k=1}^{2^{n}}e^{2\pi ijk/2^{n}}\ket{k}
\end{equation*}

Equivalently, on the state $\ket{\psi}=\sum_{j}\psi_{j}\ket{j}$ reads

\begin{equation*}
\mathrm{QFT}\ket{\psi} = \ket{\phi} = \frac{1}{2^{n/2}}\sum_{j=1}^{2^{n}}\psi_{j}\left(\sum_{k=1}^{2^{n}}e^{2\pi ijk/2^{n}}\ket{k}\right)
\end{equation*}

which turns out to be a unitary transformation

\end{frame}

\begin{frame}{Product representation of the QFT}

Computational basis ket $\ket{j} = \ket{j_{1}j_{2}...j_{n}}$\\
\vspace{0.1in}
Fourier basis ket $\ket{k} = \ket{k_{1}k_{2}...k_{n}}$\\
\vspace{0.1in}
Converting $k$ to binary: $k = \sum_{l}k_{l}2^{l}$ \\
\vspace{0.1in}
Also, note that $\ket{k} = \ket{k_{1}k_{2}...k_{n}} = \bigotimes_{l=1}^{n}\ket{k_{l}}$

\end{frame}

\begin{frame}{Product representation of the QFT}


\begin{align*}
\mathrm{QFT}\ket{j} &= \frac{1}{2^{n/2}}\sum_{k=0}^{2^{n}-1}e^{2\pi ijk/2^{n}}\ket{k}\\
&= \frac{1}{2^{n/2}}\sum_{k=0}^{2^{n}-1}e^{2\pi ij\sum_{l}k_{l}2^{-l}}\bigotimes_{l=1}^{n}\ket{k_{l}}\\
&= \frac{1}{2^{n/2}}\sum_{k=0}^{2^{n}-1}\bigotimes_{l=1}^{n}e^{2\pi ij k_{l}2^{-l}}\ket{k_{l}}\\
&= \frac{1}{2^{n/2}}\bigotimes_{l=1}^{n}\sum_{k_{l}=0}^{1}e^{2\pi ij k_{l}2^{-l}}\ket{k_{l}}\\
&= \frac{1}{2^{n/2}}\bigotimes_{l=1}^{n}\left(\ket{0}+e^{2\pi ij 2^{-l}}\ket{1}\right)\
\end{align*}

\end{frame}

\begin{frame}{Phase estimation of a unitary operator}

An important module in many quantum algorithms that uses QFT\\
\vspace{0.1in}
Consider an eigenvector $\ket{u}$ of a Unitary operator $U$. Its eigenvalue can be written as $u = e^{2\pi i\theta}$

\begin{equation*}
U\ket{u} = u\ket{u} = e^{2\pi i\theta}\ket{u}
\end{equation*}

\end{frame}

\begin{frame}{The Hidden Subgroup Problem}
Let $G$ be a group and $X$ a finite set and $f:G\rightarrow X$ a function that \emph{hides} a subgroup $H\leq G$. The problem is to determine a generating set for $H$\\
\vspace{0.2in}

\textbf{Simon's problem}. Given a 2-1 function $f:\{0,1\}^{n}\rightarrow \{0,1\}^{n}$ such that there is a secret string $s\in\{0,1\}^{n}$ where $f(x) = f(y)$ if and only if $x\oplus y = s$. Equivalently $f(x) = f(y) = f(x\oplus y)$ which gives the periodicity of $f$\\
\vspace{0.2in}
The function $f$ is a black box. Clasically you would solve the problem by drawing pairs $x,y$ and checking if $f(x) = f(y)$. If they match, you can obviously retrieve $s = x\oplus y$\\
\vspace{0.2in}
Clasically the problem scales as $\mathcal{O}(2^{n/2})$ but we Simon designed a quantum algorithm that scales as $\mathcal{O}(n)$.

\end{frame}

\begin{frame}{The standard solution to the HSP}
The first register in Simon's algorithm is a uniform superposition over all possible input strings $x$. The second register are ancillary bits that will store $f(x)$. We assume we have some oracle function $U_{f}$ which will compute and store $f(x)$ in the ancillary bits

\begin{equation*}
\ket{\psi} = H^{\otimes n}\ket{0^{n}} = \frac{1}{2^{n/2}}\sum_{x\in\{0,1\}^{n}}\ket{x}
\end{equation*}

As in the standard solution, the oracle function then does

\begin{equation*}
O_{f}(\ket{\psi}\ket{0^{m}}) = \frac{1}{2^{n/2}}\sum_{x\in\{0,1\}^{n}}\ket{x}\ket{f(x)}
\end{equation*}

Then we measure the second register which collapses the system to a superposition of the two inputs that map to our measured output $\ket{f(a)}$

\begin{equation*}
\frac{1}{2^{n/2}}\sum_{x\in\{0,1\}^{n}}\ket{x}\ket{f(x)} \rightarrow \left(\ket{a}+\ket{a\oplus s}\right)\otimes \ket{f(a)}
\end{equation*}

\end{frame}

\begin{frame}{The standard solution to the HSP}

Essentially when we measure the second register we end up with an equal superposition of $x$ and $x\oplus s$. But how do we use that superposition to actually find $s$?

\end{frame}

\end{document}