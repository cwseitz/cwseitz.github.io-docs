% Latex template: mahmoud.s.fahmy@students.kasralainy.edu.eg
% For more details: https://www.sharelatex.com/learn/Beamer

\documentclass[aspectratio=1610]{beamer}					% Document class

\setbeamertemplate{footline}[text line]{%
  \parbox{\linewidth}{\vspace*{-8pt}Bell's Inequality \hfill\insertshortauthor\hfill\insertpagenumber}}
\setbeamertemplate{navigation symbols}{}

\usepackage[english]{babel}				% Set language
\usepackage[utf8x]{inputenc}			% Set encoding

\mode<presentation>						% Set options
{
  \usetheme{default}					% Set theme
  \usecolortheme{default} 				% Set colors
  \usefonttheme{default}  				% Set font theme
  \setbeamertemplate{caption}[numbered]	% Set caption to be numbered
}

% Uncomment this to have the outline at the beginning of each section highlighted.
%\AtBeginSection[]
%{
%  \begin{frame}{Outline}
%    \tableofcontents[currentsection]
%  \end{frame}
\usepackage{graphicx}					% For including figures
\usepackage{booktabs}					% For table rules
\usepackage{hyperref}	
\usepackage{tikz-network}				% For cross-referencing
\usepackage[absolute,overlay]{textpos}
\usepackage{bm}
\usepackage[font=small,labelfont=bf]{caption}				% For cross-referencing
\usepackage{physics}

\title{The Quantum Fourier Transform}	% Presentation title
\author{Clayton W. Seitz}								% Presentation author
\date{\today}									% Today's date	

\begin{document}

% Title page
% This page includes the informations defined earlier including title, author/s, affiliation/s and the date
\begin{frame}
  \titlepage
\end{frame}

\begin{frame}{Introduction}

Classical discrete Fourier transform maps a vector $\vec{x}\in \mathbb{C}^{N}$ to another vector $\vec{y}\in\mathbb{C}^{N}$, with elements

\begin{equation*}
y_{k} = \frac{1}{\sqrt{N}}\sum_{n=0}^{N-1}x_{n}\omega_{N}^{-nk}
\end{equation*}

where $\omega_{N} = e^{2\pi i/N}$. $\vec{x}$ is expanded in a basis for $\mathbb{C}^{N}$\\
\vspace{0.1in}
The quantum fourier transform (QFT) does exactly the same thing but the vector is now interpreted as a quantum state $\ket{\psi}=\sum_{n}\psi_{n}\ket{n}$ in a Hilbert space $\mathcal{H}$. 

\begin{equation*}
\mathrm{QFT}: \ket{c_{n}} \rightarrow \frac{1}{\sqrt{N}}\sum_{n=0}^{N-1}c_{n}\omega_{N}^{-nk}
\end{equation*}


\end{frame}

\begin{frame}{The QFT as a unitary transformation}

\end{frame}

\end{document}