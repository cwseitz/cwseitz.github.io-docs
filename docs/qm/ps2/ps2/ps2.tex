\documentclass[12pt]{article}
\usepackage{amsmath} % AMS Math Package
\usepackage{bm}
\usepackage{amsthm} % Theorem Formatting
\usepackage{amssymb}    % Math symbols such as \mathbb
\usepackage{graphicx} % Allows for eps images
\usepackage[dvips,letterpaper,margin=1in,bottom=0.7in]{geometry}
\usepackage{tensor}
\usepackage{amsmath}
\usepackage{siunitx}
\usepackage{physics}

\newtheorem{p}{Problem}
\usepackage{cancel}
\newtheorem*{lem}{Lemma}
\theoremstyle{definition}
\newtheorem*{dfn}{Definition}
 \newenvironment{s}{%\small%
        \begin{trivlist} \item \textbf{Solution}. }{%
            \hspace*{\fill} $\blacksquare$\end{trivlist}}%


\begin{document}

 {\noindent\Huge\bf  \\[0.5\baselineskip] {\fontfamily{cmr}\selectfont  Homework 2}         }\\[2\baselineskip] % Title
{ {\bf \fontfamily{cmr}\selectfont Quantum Mechanics}\\ {\textit{\fontfamily{cmr}\selectfont     August 29th, 2022}}}~~~~~~~~~~~~~~~~~~~~~~~~~~~~~~~~~~~~~~~~~~~~~~~~~~~~~~~~~~~~~~~~~~~~~~~~~~~~~    {\large \textsc{Clayton Seitz}
\\[1.4\baselineskip] 

\begin{p}
Problem 1.12 from Sakurai
\end{p}

\begin{s} 

If we choose the representation such that $\ket{1} = \frac{1}{\sqrt{2}}\begin{pmatrix} 1\\1 \end{pmatrix}$ and $\ket{2} = \frac{1}{\sqrt{2}}\begin{pmatrix} 1\\-1 \end{pmatrix}$ then we can use the definition of the outer product to show that

\begin{equation*}
H = a\begin{pmatrix} 1&1\\1&-1\end{pmatrix}
\end{equation*}

The energy eigenvalues are then found by

\begin{align*}
\mathrm{det}(H-\lambda I) &= \mathrm{det}\begin{pmatrix} a-\lambda&a\\a&-a-\lambda\end{pmatrix}\\
&= (a-\lambda)(-a-\lambda) - a^{2}\\
&= \lambda^{2} - 2a^{2} = 0
\end{align*}

therefore $E_{\pm} = \pm a\sqrt{2}$. The $+$ eigenvector $\ket{\psi_{+}}$ is given by the system

\begin{equation*}
\begin{pmatrix} a-E_{+}&a\\a&-a-E_{+}\end{pmatrix}\begin{pmatrix}\psi_{1}^{+}\\\psi_{2}^{+}\end{pmatrix} = \begin{pmatrix} a-a\sqrt{2}&a\\a&-a-a\sqrt{2}\end{pmatrix}\begin{pmatrix}\psi_{1}^{+}\\\psi_{2}^{+}\end{pmatrix}  = 0
\end{equation*}

\begin{align*}
(1-\sqrt{2})\psi_{1}^{+} + \psi_{2}^{+} &= 0\\
\psi_{1}^{+} - (1+\sqrt{2})\psi_{2}^{+} &= 0
\end{align*}

The second equation is just the first multiplied by $(1-\sqrt{2})$ so we can choose $\psi_{1}^{+} = 1$ giving $\psi_{2}^{+} = \sqrt{2} - 1$

The eigenvector $\ket{\psi_{-}}$ is found similarly

\begin{equation*}
\begin{pmatrix} a-E_{-}&a\\a&-a-E_{-}\end{pmatrix}\begin{pmatrix}\psi_{1}^{-}\\\psi_{2}^{-}\end{pmatrix} = \begin{pmatrix} a+a\sqrt{2}&a\\a&-a+a\sqrt{2}\end{pmatrix}\begin{pmatrix}\psi_{1}^{-}\\\psi_{2}^{-}\end{pmatrix}  = 0
\end{equation*}

\begin{align*}
(1+\sqrt{2})\psi_{1}^{+} + \psi_{2}^{+} &= 0\\
\psi_{1}^{+} + (-1+\sqrt{2})\psi_{2}^{+} &= 0
\end{align*}

Similar to before, the second equation is $(-1+\sqrt{2})$ multiplied by the first, allowing us to set $\psi_{1}^{-} = 1$ and $\psi_{2}^{-} = -(1+\sqrt{2})$, giving a $\ket{\psi_{-}}$ that is orthogonal to $\ket{\psi_{+}}$

\end{s}

\begin{p}
Problem 1.13 from Sakurai
\end{p}

\begin{s} 

Writing $H$ out in matrix form gives

\begin{align*}
H &= \frac{H_{11}}{2}\begin{pmatrix} 1&1\\1&1\end{pmatrix}
+ \frac{H_{22}}{2}\begin{pmatrix} 1&-1\\-1&1\end{pmatrix}
+ H_{12}\begin{pmatrix} 1&0\\0&-1\end{pmatrix}\\
&= \frac{H_{11}+H_{22}}{2}I + \frac{H_{11}-H_{12}}{2}\sigma_{x} + H_{12}\sigma_{z} \\
&= a I + b\sigma_{x} + c\sigma_{z}
\end{align*}

where have made appropriate substitutions of constants for brevity. Now this implies, 

\begin{align*}
H\ket{\psi} &= \left(a I + b\sigma_{x} + c\sigma_{z}\right)\ket{\psi}\\
&= a\ket{\psi} + (b\sigma_{x} + 0\sigma_{y} + c\sigma_{z})\ket{\psi}\\
\end{align*}

Any $\ket{\psi}$ is an eigenvector under the identity operation, so what we are really after is an eigenvector of the operator $\bm{\sigma}\cdot \bm{a}$ for $\bm{a} = (b, 0, c)$. In other words, if $\ket{\psi}$ is an eigenvector of $\bm{\sigma}\cdot\bm{a}$ then it is also an eigenvector of $H$. It is useful to work with the unit vector in the direction of $\bm{a}$ which is $\hat{\bm{n}}=(b/\sqrt{b^{2}+c^{2}}, 0, c/\sqrt{b^{2}+c^{2}})$. We already know the eigenvectors of $\bm{\sigma}\cdot\hat{\bm{n}}$

\begin{align*}
\ket{\psi_{+}} &= \cos\frac{\beta}{2}\ket{+} + \exp(i\alpha)\sin\frac{\beta}{2}\ket{-}\\
\ket{\psi_{-}} &= -\sin\frac{\beta}{2}\ket{+} + \exp(i\alpha)\cos\frac{\beta}{2}\ket{-}
\end{align*}

where we take the definition that $\alpha$ is the polar angle and $\beta$ the azimuthal angle. Therefore

\begin{align*}
\alpha &= 0\\
\beta &= \arctan\left(\frac{n_{z}}{n_{x}}\right) = \arctan\left(\frac{c}{b}\right) = \arctan\left(\frac{2H_{12}}{H_{11}-H_{12}}\right)
\end{align*}


\end{s}

\begin{p}
Problem 1.15 from Sakurai
\end{p}

\begin{s} 
After the first measurement along $+\hat{z}$ , all of our atoms are prepared in the $\ket{+}$ state in the $S_{z}$ basis. At the next apparatus oriented along $\hat{n}$, more atoms will be filtered out since $\ket{+}$ is not an eigenket of the $\mathbf{S}\cdot \hat{n}$ operator. Recall that $\ket{+}_{n}$ is 

\begin{align*}
\ket{+}_{n} &= \cos\frac{\beta}{2}\ket{+} + \sin\frac{\beta}{2}\ket{-}
\end{align*}

The probability the state $\ket{+}$ survives is given by the inner product

\begin{align*}
|\bra{+}\ket{+}_{n}|^{2} &= |\bra{+}\cos\frac{\beta}{2}\ket{+} + \bra{+}\sin\frac{\beta}{2}\ket{-}|^{2}\\
&= \cos^{2}\frac{\beta}{2}
\end{align*}

After this, all atoms are in the $\ket{+}_{n}$ state. We then filter the atoms one more time with an apparatus along $-\hat{z}$. The fraction that survive this one is given by

\begin{align*}
|\bra{-}\ket{+}_{n}|^{2} &= |\bra{-}\cos\frac{\beta}{2}\ket{+} + \bra{-}\sin\frac{\beta}{2}\ket{-}|^{2}\\
&= \sin^{2}\frac{\beta}{2}
\end{align*}

Therefore the fraction output is $\cos^{2}\frac{\beta}{2}\sin^{2}\frac{\beta}{2}$. We can maximize this function by setting $\beta = \pi/2$

\end{s}

\begin{p}
Problem 1.16 from Sakurai
\end{p}

\begin{s} 

We have the observable

\begin{equation*}
O = \frac{1}{\sqrt{2}}\begin{pmatrix} 0&1&0\\1&0&1\\0&1&0\end{pmatrix}
\end{equation*}

\begin{align*}
\mathrm{det}(O-\lambda I) &= \mathrm{det}\begin{pmatrix} -\lambda&\frac{1}{\sqrt{2}}&0\\\frac{1}{\sqrt{2}}&-\lambda&\frac{1}{\sqrt{2}}\\0&\frac{1}{\sqrt{2}}&-\lambda\end{pmatrix}\\
&= -\lambda\left(\lambda^{2}-\frac{1}{2}\right) - \frac{1}{\sqrt{2}}\left(-\frac{\lambda}{\sqrt{2}}\right)\\
&= -\lambda^{3} + \lambda = 0
\end{align*}

Clearly our eigenvalues are $\lambda = 0,\pm 1$. There is no degeneracy.

\begin{equation*}
\begin{pmatrix} -\lambda&\frac{1}{\sqrt{2}}&0\\\frac{1}{\sqrt{2}}&-\lambda&\frac{1}{\sqrt{2}}\\0&\frac{1}{\sqrt{2}}&-\lambda\end{pmatrix}
\begin{pmatrix}\psi_{1}\\\psi_{2}\\\psi_{3}\end{pmatrix} = 0
\end{equation*}

For $\lambda = 0$, we have the system

\begin{align*}
\frac{1}{\sqrt{2}}\psi_{2} &= 0\\
\frac{1}{\sqrt{2}}\psi_{1} + \frac{1}{\sqrt{2}}\psi_{3} &= 0\\
\frac{1}{\sqrt{2}}\psi_{2} &= 0
\end{align*}

Therefore $\psi_{2} = 0$ and we can take $\psi_{1} = 1$ and $ \psi_{3} = -1$ For $\lambda = -1$, we have

\begin{align*}
\frac{1}{\sqrt{2}}\psi_{2} - \psi_{1} &= 0\\
\frac{1}{\sqrt{2}}\psi_{1} - \psi_{2} + \frac{1}{\sqrt{2}}\psi_{3} &= 0\\
\frac{1}{\sqrt{2}}\psi_{2} - \psi_{3} &= 0
\end{align*}

The second equation can be eliminated since it is just $-1/\sqrt{2}$ times the first plus $-1/\sqrt{2}$ times the second. We are free to set $\psi_{2} = 1$ which gives $\psi_{1} = \psi_{3} = \frac{1}{\sqrt{2}}$. For the second eigenvector we have the system

\begin{align*}
\frac{1}{\sqrt{2}}\psi_{2} + \psi_{1} &= 0\\
\frac{1}{\sqrt{2}}\psi_{1} + \psi_{2} + \frac{1}{\sqrt{2}}\psi_{3} &= 0\\
\frac{1}{\sqrt{2}}\psi_{2} + \psi_{3} &= 0
\end{align*}

Again, the second equation can be eliminated and $\psi_{2}=1$ and $\psi_{1}=\psi_{3} = -\frac{1}{\sqrt{2}}$

\vspace{0.1in}
A physical system where this is all relevant is the spin-1 system, which in general has three possible eigenstates. However, this observable $O$ only has two non-trivial eigenvectors and the observable is limited to a two dimensional subspace of the three dimensional space.

\end{s}



\begin{p}
Problem 1.23 from Sakurai
\end{p}

\begin{s} 
For the ground state, the position space wavefunction $\ket{\psi}$ is a solution to the eigenvalue equation  

\begin{align*}
H\ket{\psi} &= \left[\frac{\bm{p}^{2}}{2m} + \bm{V}(x)\right]\ket{\psi}\\
&= -\frac{\hbar^{2}}{2m}\frac{\partial^{2}\ket{\psi}}{\partial x^{2}} + V(x)\ket{\psi}\\
&= E\ket{\psi}
\end{align*}

We set the boundary conditions $\psi(0) = 0$ and $\psi(a) = 0$ since the wavefunction must vanish at the two walls. Since $V(x) = 0$ inside the well, Schrodinger's equation reduces to

\begin{align*}
H\ket{\psi} = -\frac{\hbar^{2}}{2m}\frac{\partial^{2}\ket{\psi}}{\partial x^{2}} = E\ket{\psi}
\end{align*}

We have the following solution

\begin{align*}
\ket{\psi} = A\sin\left(\frac{n\pi x}{a}\right)
\end{align*}

This solution comes from applying our boundary conditions $\psi(0) = 0$ and $\psi(a) = 0$. These require that the wavelength must satisfy $ka = n\pi$ which means that $k = \frac{n\pi}{a}$ for integer $n > 0$. Now, it is straightforward to show that

\begin{align*}
\bra{\psi}\ket{\psi} = \frac{2}{a}\int_{0}^{a}\sin^{2}\left(\frac{n\pi x}{a}\right)dx = 1
\end{align*}

Giving the eigenkets 

\begin{align*}
\ket{\psi} = \sqrt{\frac{2}{a}}\sin\left(\frac{n\pi x}{a}\right)
\end{align*}

We would now like to compute the variance of our position measurement in the ground state ($n=1$). In general, for an observable $O$ and wavefunction $\psi(x)$, we can compute the variance of $O$ as 

\begin{align*}
\langle (\Delta O)^{2}\rangle &= \langle O^{2} \rangle - \langle O\rangle^{2}\\
\langle O \rangle &=  \bra{\psi} O \ket{\psi} = \int_{-\infty}^{+\infty} \psi^{*}(x) O \psi(x)dx
\end{align*}

For $O=x$, we have

\begin{align*}
\langle (\Delta x)^{2}\rangle &= \langle x^{2} \rangle - \langle x\rangle^{2}\\
&= \bra{\psi} x^{2} \ket{\psi} - \left(\bra{\psi} x \ket{\psi}\right)^{2}\\
\end{align*}

We can immediately write the value of $\langle x \rangle^{2}= (\bra{\psi} x \ket{\psi})^{2}$ based on the symmetry of the wavefunction

\begin{align*}
\langle x \rangle^{2} &= \frac{a^{2}}{4}
\end{align*}

Let $\alpha = n\pi/a$. The term $\langle x^{2} \rangle$ is then given by the integral

\begin{align*}
I = \langle x^{2} \rangle &\propto \int_{0}^{a} x^{2} \sin^{2}\left(\alpha x\right)dx\\
&= \int_{0}^{a} x^{2}\cdot \frac{1-\cos(2\alpha x)}{2}dx\\
&= \frac{1}{2}\left[\frac{a^{3}}{3}- \int_{0}^{a}x^{2}\cos(2\alpha x)dx\right]\\
&= \frac{1}{2}\left[\frac{a^{3}}{3}- \left[a^{2}\cdot \frac{\sin(2\alpha a)}{2\alpha} - \frac{1}{\alpha}\int_{0}^{a} x\sin(2\alpha x)\right]\right]\\
&= \frac{1}{2}\left[\frac{a^{3}}{3}+ \frac{1}{\alpha}\left[\int_{0}^{a} x\sin(2\alpha x)\right]\right]\\
\end{align*}

The last integral can be evalulated as follows

\begin{align*}
\int_{0}^{a} x\sin(2\alpha x)dx &= -\bigg\rvert_{0}^{a} \frac{x}{2\alpha}\cos(2\alpha x) + \frac{1}{2\alpha}\int_{0}^{a} \cos(2\alpha x)dx\\
&= -\frac{a}{2\alpha}\cos(2\alpha a) + \frac{1}{2\alpha}\sin(2\alpha a)\\
&= -\frac{a}{2\alpha}\cos(2\alpha a)\\
&= -\frac{a^{2}}{n\pi}\cos(n\pi)\\
\end{align*}

Combining these results, we get

\begin{align*}
I &= \frac{1}{2}\left[\frac{a^{3}}{3} - \frac{a^{3}}{n^{2}\pi^{2}}\cos(n\pi)\right]\\
\end{align*}

Finally, bringing in the normalization factor $2/a$ gives

\begin{align*}
I = \langle x^{2} \rangle &= a^{2}\left[\frac{1}{3} - \frac{1}{2 n^{2}\pi^{2}}\cos(n\pi)\right]\\
\end{align*}

which gives the variance 

\begin{align*}
\langle (\Delta x)^{2}\rangle &= \langle x^{2} \rangle - \langle x\rangle^{2}\\
&= a^{2}\left[\frac{1}{3} - \frac{1}{2 n^{2}\pi^{2}}\cos(n\pi)\right] - \frac{a^{2}}{4}
\end{align*}

Now for $O = p = -i\hbar\frac{\partial}{\partial x}$, we have

\begin{align*}
\langle (\Delta p)^{2}\rangle &= \langle p^{2} \rangle - \langle p\rangle^{2}\\
&= \bra{\psi} p^{2} \ket{\psi} - \left(\bra{\psi} p \ket{\psi}\right)^{2}\\
\end{align*}

Since the potential in the well is zero, the first term is just

\begin{align*}
\langle p^{2} \rangle = 2mE_{n} = \frac{n^{2}\pi^{2}\hbar^{2}}{a^{2}}
\end{align*}

Using the same substitution as for position, the second terms reads

\begin{align*}
\bra{\psi} p \ket{\psi} &\propto -i\hbar \int_{0}^{a} 2\alpha\sin(\alpha x)\cos(\alpha x)dx\\
&= -i\hbar \int_{0}^{a} \sin(2\alpha x)dx\\
&= 0
\end{align*}

Summarizing, we have

\begin{align*}
\langle (\Delta p)^{2}\rangle &= \frac{n^{2}\pi^{2}\hbar^{2}}{a^{2}}
\end{align*}



\end{s}

\begin{p}
Problem 1.24 from Sakurai
\end{p}

\begin{s} 

\end{s}

\end{document}