\documentclass[12pt]{article}
\usepackage{amsmath} % AMS Math Package
\usepackage{bm}
\usepackage{amsthm} % Theorem Formatting
\usepackage{amssymb}    % Math symbols such as \mathbb
\usepackage{graphicx} % Allows for eps images
\usepackage[dvips,letterpaper,margin=1in,bottom=0.7in]{geometry}
\usepackage{tensor}
\usepackage{amsmath}
\usepackage{siunitx}
\usepackage{physics}

\newtheorem{p}{Problem}
\usepackage{cancel}
\newtheorem*{lem}{Lemma}
\theoremstyle{definition}
\newtheorem*{dfn}{Definition}
 \newenvironment{s}{%\small%
        \begin{trivlist} \item \textbf{Solution}. }{%
            \hspace*{\fill} $\blacksquare$\end{trivlist}}%


\begin{document}

{\noindent\Huge\bf  \\[0.5\baselineskip] {\fontfamily{cmr}\selectfont  Homework 2}         }\\[2\baselineskip] % Title
{ {\bf \fontfamily{cmr}\selectfont Quantum Mechanics}\\ {\textit{\fontfamily{cmr}\selectfont     August 29th, 2022}}}~~~~~~~~~~~~~~~~~~~~~~~~~~~~~~~~~~~~~~~~~~~~~~~~~~~~~~~~~~~~~~~~~~~~~~~~~~~~~    {\large \textsc{Clayton Seitz}
\\[1.4\baselineskip] 

\begin{p}
Problem 1.12 from Sakurai
\end{p}

\begin{s} 

If we choose the representation such that $\ket{1} = \frac{1}{\sqrt{2}}\begin{pmatrix} 1\\1 \end{pmatrix}$ and $\ket{2} = \frac{1}{\sqrt{2}}\begin{pmatrix} 1\\-1 \end{pmatrix}$ then we can use the definition of the outer product to show that

\begin{equation*}
H = a\begin{pmatrix} 1&1\\1&-1\end{pmatrix}
\end{equation*}

The energy eigenvalues are then found by

\begin{align*}
\mathrm{det}(H-\lambda I) &= \mathrm{det}\begin{pmatrix} a-\lambda&a\\a&-a-\lambda\end{pmatrix}\\
&= (a-\lambda)(-a-\lambda) - a^{2}\\
&= \lambda^{2} - 2a^{2} = 0
\end{align*}

therefore $E_{\pm} = \pm \sqrt{2a}$. The $+$ eigenvector $\ket{\psi_{1}}$ is given by the system

\begin{align*}
(\psi_{1}^{1} + \psi_{1}^{2}) &= \sqrt{\frac{2}{a}} \psi_{1}^{1}\\
(\psi_{1}^{1} - \psi_{1}^{2}) &= \sqrt{\frac{2}{a}} \psi_{1}^{2}
\end{align*}

The $-$ eigenvector $\ket{\psi_{2}}$ is given by the system

\begin{align*}
(\psi_{2}^{1} + \psi_{2}^{2}) &= -\sqrt{\frac{2}{a}} \psi_{2}^{1}\\
(\psi_{2}^{1} - \psi_{2}^{2}) &= -\sqrt{\frac{2}{a}} \psi_{2}^{2}
\end{align*}

\end{s}

\begin{p}
Problem 1.13 from Sakurai
\end{p}

\begin{s} 

Writing $H$ out in matrix form gives

\begin{align*}
H &= H_{11}\begin{pmatrix} 1&1\\1&1\end{pmatrix}
+ H_{12}\begin{pmatrix} 1&-1\\-1&1\end{pmatrix}
+ H_{22}\begin{pmatrix} 1&-1\\1&-1\end{pmatrix}
+ \begin{pmatrix} 1&1\\-1&-1\end{pmatrix}\\
&= \begin{pmatrix} H_{11}+H_{12}+H_{22}+1&H_{11}-H_{12}-H_{22}+1\\H_{11}-H_{12}+H_{22}-1&H_{11}+H_{12}-H_{22}-1\end{pmatrix}
\end{align*}

\begin{align*}
\mathrm{det}(H-\lambda I) &= \mathrm{det}\begin{pmatrix} H_{11}+H_{12}+H_{22}+1-\lambda &H_{11}-H_{12}-H_{22}+1\\H_{11}-H_{12}+H_{22}-1&H_{11}+H_{12}-H_{22}-1-\lambda\end{pmatrix}\\
\end{align*}




\end{s}

\begin{p}
Problem 1.15 from Sakurai
\end{p}

\begin{s} 
After the first measurement along $+\hat{z}$ , all of our atoms are prepared in the $\ket{+}$ state in the $S_{z}$ basis. At the next apparatus oriented along $\hat{n}$, more atoms will be filtered out since $\ket{+}$ is not an eigenket of the $\mathbf{S}\cdot \hat{n}$ operator. Recall that $\ket{+}_{n}$ is 

\begin{align*}
\ket{+}_{n} &= \cos\frac{\beta}{2}\ket{+} + \sin\frac{\beta}{2}\ket{-}
\end{align*}

The probability the state $\ket{+}$ survives is given by the inner product

\begin{align*}
|\bra{+}\ket{+}_{n}|^{2} &= |\bra{+}\cos\frac{\beta}{2}\ket{+} + \bra{+}\sin\frac{\beta}{2}\ket{-}|^{2}\\
&= \cos^{2}\frac{\beta}{2}
\end{align*}

After this, all atoms are in the $\ket{+}_{n}$ state. We then filter the atoms one more time with an apparatus along $-\hat{z}$. The fraction that survive this one is given by

\begin{align*}
|\bra{-}\ket{+}_{n}|^{2} &= |\bra{-}\cos\frac{\beta}{2}\ket{+} + \bra{-}\sin\frac{\beta}{2}\ket{-}|^{2}\\
&= \sin^{2}\frac{\beta}{2}
\end{align*}

Therefore the fraction output is $\cos^{2}\frac{\beta}{2}\sin^{2}\frac{\beta}{2}$. We can maximize this function by setting $\beta = \pi/2$

\end{s}

\begin{p}
Problem 1.16 from Sakurai
\end{p}

\begin{s} 

\end{s}

\begin{p}
Problem 1.23 from Sakurai
\end{p}

\begin{s} 

\end{s}

\begin{p}
Problem 1.24 from Sakurai
\end{p}

\begin{s} 

\end{s}



\end{document}