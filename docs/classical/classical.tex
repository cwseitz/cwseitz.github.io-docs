\input SlidePreamble
\input preamble

\newcommand{\solution}[1]{\bigskip {\bf Solution}: #1}

\begin{document}

{\Huge
  \centerline{\bf TTIC 31230, Fundamentals of Deep Learning}
  \bigskip
  \centerline{David McAllester, Autumn 2020}
  \vfill
  \centerline{\bf Stochastic Gradient Descent (SGD)}
  \vfill
  \centerline{\bf The Classical Convergence Theorem}
  \vfill
  \vfill

\slide{Vanilla SGD}

$$\Phi \;\minuseq\; \eta \hat{g}$$

\vfill
\begin{eqnarray*}
  \hat{g} & = & E_{(x,y) \sim \mathrm{Batch}}\;\nabla_\Phi\;\mathrm{loss}(\Phi,x,y) \\
  \\
  \\
  g & = & E_{(x,y) \sim \mathrm{Pop}}\;\nabla_\Phi\;\mathrm{loss}(\Phi,x,y) \\
\end{eqnarray*}

\vfill
\centerline{$\eta$ is the ``learning rate'' hyper-parameter (a parameter not in $\Phi$).}


\slide{Issues}

\vfill
\begin{itemize}
\item {\bf Gradient Estimation.} The accuracy of $\hat{g}$ as an estimate of $g$.

  \vfill
\item {\bf Gradient Drift (second order structure).} The fact that $g$ changes as the parameters change.

  \vfill
\item {\bf Convergence.} To converge to a local optimum the learning rate must be gradually reduced toward zero.

  \vfill
  \item {\bf Exploration.} Since deep models are non-convex we need to search over the parameter space.  SGD can behave like MCMC.
\end{itemize}

\slide{A One Dimensional Example}

Suppose that $y$ is a scalar, and consider

\begin{eqnarray*}
 \mathrm{loss}(\beta,y) & = & \frac{1}{2}(\beta - y)^2 \\
 \\
  g & = &  \nabla_\beta\;E_{y \sim \mathrm{Pop}}\;\frac{1}{2}(\beta - y)^2 \\
  \\
  & = & \beta - E_{y \sim \mathrm{Pop}} \; y \\
  \\
  \hat{g} & = &\beta - E_{y \sim \mathrm{Batch}} \;y
\end{eqnarray*}

\vfill
Even if $\beta$ is optimal, for a finite batch we will have $\hat{g} \not = 0$.

\slide{The Classical Convergence Theorem}

$$\Phi \;\minuseq \; \eta_t \nabla_\Phi\;\mathrm{loss}(\Phi,x_t,y_t)$$

\vfill
For ``sufficiently smooth'' non-negative loss with

\vfill
$$\eta_t \geq 0\;\;\;\;\;\;\;\;\lim_{t \rightarrow \infty} \;\eta_t = 0\;\;\;\;\;\;\;\;\sum_t \eta_t = \infty \;\;\;\;\;\;\sum_t \;\eta_t^2 < \infty$$

\vfill
we have that the training loss $E_{(x,y) \sim \mathrm{Train}}\; \mathrm{loss}(\Phi,x,t)$ converges to a limit and any limit point of the sequence $\Phi_t$
is a stationary point in the sense that {\huge  $\nabla_\Phi \; E_{(x,y) \sim \mathrm{Train}} \;\mathrm{loss}(\Phi,x,t) = 0$}.

\vfill
{\Large
\vfill
{\bf Rigor Police:} One can construct cases where $\Phi$ diverges to infinity, converges to a saddle point, or even converges to a limit cycle.

}

\slide{Physicist's Proof of the Convergence Theorem}

Since $\lim_{t \rightarrow 0} \;\eta_t = 0$ we will eventually get to arbitrarilly small learning rates.

\vfill
For sufficiently small learning rates any meaningful update of the parameters will be based on an arbitrarily large sample
of gradients at essentially the same parameter value.

\vfill
An arbitrarily large sample will become arbitrarily accurate as an estimate of the full gradient.

\vfill
But since $\sum_t \eta_t = \infty$, no matter how small the learning rate gets, we still can make arbitrarily large motions in parameter space.


\vfill
{\Large
\vfill
For a rigorous proof see Neuro-Dynamic Programming, Bertsekas and Tsitsiklis, 1996.}

\slidetwo{SGD as a form of MCMC}{Learning Rate as a Temperature Parameter}

\centerline{\includegraphics[height= 4in]{\images/AnnealingSGD}}
\centerline{\Large Gao Huang et. al., ICLR 2017}

\slide{END}

} \end{document}

