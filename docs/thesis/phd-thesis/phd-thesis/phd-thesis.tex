% ---- ETD Document Class and Useful Packages ---- %
\documentclass{ucetd}
\usepackage{subfigure,epsfig,amsfonts}
\usepackage{natbib}
\usepackage{amsmath}
\usepackage{amssymb}
\usepackage{amsthm}
\usepackage[toc,page]{appendix}
\usepackage[labelfont=bf]{caption}
\usepackage{rotating}
\usepackage[dvipsnames]{xcolor}
\usepackage{url}
 

%% Use these commands to set biographic information for the title page:
\title{Monte Carlo methods and deep variational inference for fast Bayesian analysis of high dimensional biological systems}
\author{Clayton W. Seitz}
\department{Department of Physics}
\division{Physical Sciences}
\degree{Doctor of Philosophy}
\date{Spring 20XX}

%% Use these commands to set a dedication and epigraph text

\epigraph{Epigraph}

\begin{document}
%% Basic setup commands
% If you don't want a title page comment out the next line and uncomment the line after it:
\maketitle
%\omittitle

% These lines can be commented out to disable the copyright/dedication/epigraph pages
\makecopyright
%\makededication


%% Make the various tables of contents
\tableofcontents
%\listoffigures
%\listoftables

%\acknowledgments
% Enter Acknowledgements here

\abstract



\clearpage

\mainmatter

\chapter{Primer on exact Bayesian methods and variational inference}

\section{Markov Chain Monte Carlo}

\textcolor{red}{The invention of fast digital computers gives us the ability to simulate random processes and perform inference}

\subsection{Metropolis-Hastings and Gibbs sampling}

\subsection{Langevin Monte Carlo} 

\textcolor{red}{The metropolis adjusted Langevin algorithm (MALA) aka Langevin Monte Carlo}

\subsection{Hamiltonian Monte Carlo}

\subsection{Stochastic Gradient Langevin Dynamics}

\section{Variational Inference}


\subsection{Neural networks represent probability distributions}

\subsection{Training criteria for neural networks}

\subsection{The evidence lower bound}

\chapter{Deconvolving immuogenic tumor substructure with variational inference}

\textcolor{red}{Single cell variational inference. Using deep generative modeling for batch correction, differential expression, imputation, etc. Establish how that latent structure relates to immunogenicity. In particular, I am interested in the relationship between tumor substructure i.e. heterogenity and how it relates to the degree of T-cell inflammation. We can be confident that certain malignant clusters are more immunogenic than others, which, to a first approximation, can be understood by measuring T-cell quantity in the sample. This \emph{could} then be supplemented by using T-cell signatures documented in the literature, differential expression, and clustering methods/mixture models to design fluorescent biomarkers. These will then be added to T-cell markers. In principle, this would give a more complete association of gene expression to the inflammaotry state, and would be a useful method in trying to understand changes during the evolution of cancer and following pharmacological treatment.}

\vspace{0.2in}
to 
\textcolor{red}{For the implementation, I will use Seurat for preprocessing and visualization, as these plotting functions are more visually appealing that Python functions. We can port the preprocessed data over to Python for the deep learning step. Malignant cells are isolated first using standard methods e.g, filtering, normalization, and UMAP clustering For training, I will try to place emphasis on physically inspired algorithms for optimization in variational methods}

\chapter{A Bayesian approach for inferring neuronal connectivity from Ca2+ imaging data and Monte Carlo simulations}

\chapter{Bayesian inference of the kinetic parameters of interferon-gamma induced transcription}

%% Format a LaTeX bibliography

\makebibliography





\end{document}


