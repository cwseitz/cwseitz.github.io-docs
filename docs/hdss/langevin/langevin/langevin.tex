% Latex template: mahmoud.s.fahmy@students.kasralainy.edu.eg
% For more details: https://www.sharelatex.com/learn/Beamer

\documentclass{beamer}					% Document class

\setbeamertemplate{footline}[text line]{%
  \parbox{\linewidth}{\vspace*{-8pt}Langevin Dynamics\hfill\insertshortauthor\hfill\insertpagenumber}}
\setbeamertemplate{navigation symbols}{}

\usepackage[english]{babel}				% Set language
\usepackage[utf8x]{inputenc}			% Set encoding
\usepackage{bm}

\mode<presentation>						% Set options
{
  \usetheme{default}					% Set theme
  \usecolortheme{default} 				% Set colors
  \usefonttheme{default}  				% Set font theme
  \setbeamertemplate{caption}[numbered]	% Set caption to be numbered
}

% Uncomment this to have the outline at the beginning of each section highlighted.
%\AtBeginSection[]
%{
%  \begin{frame}{Outline}
%    \tableofcontents[currentsection]
%  \end{frame}
%}

\usepackage{graphicx}					% For including figures
\usepackage{booktabs}					% For table rules
\usepackage{hyperref}					% For cross-referencing

\title{Langevin Dynamics}	% Presentation title
\author{Clayton W. Seitz}								% Presentation author
\date{\today}									% Today's date	

\begin{document}

% Title page
% This page includes the informations defined earlier including title, author/s, affiliation/s and the date
\begin{frame}
  \titlepage
\end{frame}

% Outline
% This page includes the outline (Table of content) of the presentation. All sections and subsections will appear in the outline by default.
\begin{frame}{Outline}
  \tableofcontents
\end{frame}

\begin{frame}{Langevin Dynamics}

Originally a reformulation of Einsteins theory of Brownian motion (BM) using stochastic differential equations (SDEs)

\begin{equation*}
\frac{dx}{dt} = \eta(t), \;\;\; \eta(t) \sim T(x,t|x',t')
\end{equation*}

For BM, $T(x,t|x',t') = \mathcal{N}(x',\sigma^{2})$ where $\langle \eta(t)\eta(t')\rangle = \delta(t-t')$. If we have many x's, and $\eta(t)$ is uncorrelated over the ensemble we may write 

\begin{equation*}
\langle \eta(t)\eta(t')\rangle = \sigma^{2}\delta_{ij}\delta(t-t')
\end{equation*}

\end{frame}

\begin{frame}{Application to Brownian Motion}

The solution to an SDE is a probability distribution $P(x,t)$ which obeys the Markov property

\begin{equation*}
P(x,t') = \int T(x,t|x',t')P(x',t')dx'
\end{equation*}


With some effort this can be transformed into the Fokker-Planck equation

\begin{equation*}
\frac{dP}{dt} = \frac{\sigma^{2}}{2}\frac{\partial^{2}P}{\partial x^{2}} = D\frac{\partial^{2}P}{\partial x^{2}}
\end{equation*}

which has a familiar non-stationary solution for $P(x,t)$ in BM:

\begin{equation*}
P(x,t) = \frac{1}{\sqrt{4\pi Dt}}\exp\left(-\frac{x^{2}}{4Dt}\right)
\end{equation*}

\end{frame}

\begin{frame}{Generalization to higher dimensions}

When $\langle \eta_{i}(t)\eta_{j}(t)\rangle_{t} = \delta_{ij}$, the one-dimensional solution applies. Otherwise, $\langle \eta_{i}(t)\eta_{j}(t)\rangle_{t} = D_{ij} = \bm{\Sigma}/2$

\begin{equation*}
\frac{d\bm{x}}{dt} = \sqrt{\bm{\Sigma}}\bm{\eta}(t) 
\end{equation*}

where $\bm{D} = \bm{\Sigma}/2$ becomes a \emph{diffusion tensor}. The Fokker-Planck equation for N-dimensional BM generalizes to 

\begin{equation*}
\frac{dP}{dt} = \sum_{i}\sum_{j}D_{ij}\frac{\partial P}{\partial x_{i}\partial x_{j}}
\end{equation*}

\end{frame}

\begin{frame}{Diffusion in a harmonic potential: Ornstein-Uhlenbeck}

A stochastic form of Newton's equation

\begin{equation*}
\frac{d\bm{v}}{dt} = -\lambda \bm{v} + \bm{\eta}(t) - \bm{k}\cdot \bm{x}
\end{equation*}

\end{frame}

\begin{frame}{Mean squared displacement}

A common quantity measured experimentally is the mean-squared displacement (MSD). This is essentially the variance of $T(x,t|x',t')$

\begin{align*}
\mathrm{MSD}(\tau) &= \langle \left(x(t+\tau) - x(t)\right)^{2}\rangle\\
&=  4D\tau
\end{align*}

\end{frame}

\begin{frame}
We would like to perform maximum likelihood estimation

\begin{equation*}
\bm{k}^{*} = \underset{\bm{k}}{\mathrm{argmin}} -\log P(\bm{k}|\{\bm{x}\}_{i=1}^{N})
\end{equation*}

\end{frame}


\section{References}

% Adding the option 'allowframebreaks' allows the contents of the slide to be expanded in more than one slide.
\begin{frame}[allowframebreaks]{References}
	\tiny\bibliography{references}
	\bibliographystyle{apalike}
\end{frame}

\end{document}