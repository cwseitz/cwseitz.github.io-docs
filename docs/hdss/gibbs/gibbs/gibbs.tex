

\documentclass{article}
\title{Gibbs Sampling}
\author{C.W. Seitz}
\date{\today}

\usepackage{graphicx}
\usepackage{subfigure,epsfig,amsfonts}
\usepackage{amsmath}
\usepackage{siunitx}

\begin{document}
\maketitle

\section{Introduction}

Gibbs sampling is a Monte-Carlo Markov-Chain (MCMC) method which was originally developed in the context of image restoration. Since then it has found a broad set of applications such as computer vision, the Boltzmann machine, etc. Here, I will introduce the image restoration algorithm in the original Gibbs sampling paper published by Geman and Geman in the 1980s. This serves as an introduction to important concepts such as energy-based modeling and Markov Random Fields, which are generally useful in machine learning applications. This note will conclude with a toy model that can be used to explore Gibbs sampling.  

\section{Markov Random Fields}

The Markov Random Field (MRF), sometimes referred to as an undirected graphical model or Markov network, is a graph which is used to represent a multivariate probability distribution. The nodes in the graph represent the variables while the edges represent a notion of probabilistic interaction between those variables. A key property of any multivariate distribution are the dependencies and independencies of the individual variables. 

\end{document}