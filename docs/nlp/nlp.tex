\input SlidePreamble
\input preamble
\def\entity#1{\vfill {[#1] =~}}

\begin{document}

{\Huge

  \centerline{\bf TTIC 31230, Fundamentals of Deep Learning}
  \bigskip
  \centerline{David McAllester, Autumn 2020}

\vfill
\centerline{\bf AGI: Natural Language}
  \vfill
  \vfill


\slide{Natural Language for Knowledge Representation}

Is natural language a reflection of a general learning or knowledge representation architecture?

\vfill
Will some Transformer-like architecture ultimately underly general intelligence?

\slide{Natural Language for Knowledge Representation}

I like to sample news stories to try understand how language represents reality.

\vfill
{\bf Example:} President Donald Trump on Tuesday insisted he was serious when he % 11
revealed that he had directed his administration to slow coronavirus % 10
testing in the United States, shattering the defenses of senior White % 11
House aides who argued Trump’s remarks were made in jest. %10

\slide{Natural Language for Knowledge Representation}

I take the position that reference, rather than compositional meaning, is the fundamental semantic phenomenon in language.

\vfill
Most of the phrases of a sentence refer to a particular entity or event.

\vfill
We can introduce constant symbols for the entities and events and break sentences down into simple
statements involving the entities.

\slide{Natural Language for Knowledge Representation}

{\huge


\vfill
President Donald Trump on Tuesday insisted he was serious when he % 11
revealed that he had directed his administration to slow coronavirus % 10
testing in the United States, shattering the defenses of senior White % 11
House aides who argued Trump’s remarks were made in jest. %10

\entity{the order} [Trump] directed his administration to reduce [{\large COVID} testing].

\entity{the revelation} [Trump] revealed [the order].

\entity{the claim} [the order] was in jest.

\entity{the assertion} [the aides] made [the claim].
}

\slide{Natural Language for Knowledge Representation}

{\huge
President Donald Trump on Tuesday insisted he was serious when he % 11
revealed that he had directed his administration to slow coronavirus % 10
testing in the United States, shattering the defenses of senior White % 11
House aides who argued Trump’s remarks were made in jest. %10

\entity{the aides} Senior White House aides

\entity{the defense} [the assertion] defended [Trump].

\entity{the insistence} on Tuesday [Trump] insisted [the order] was serious.

\entity{the shattering} [the insistence] shattered [the defense].
}

\slide{Subtle Semantics}

We would like to be able to represent the subtle semantics of words like ``insisted'', 
``serious'', ``revealed'',
``shattered'', ``defenses'' and ``argued''.

\vfill
A translation to first order logic would introduce predicate symbols for these words.

\vfill
However, their subtle meanings seem impossible to express in the rigid Boolean semantics of first order logic.

\slide{Neural Semantics}

In contrast to Boolean semantics, the meaning of words seems to require embeddings processed by neural networks.

\vfill
But as in logic, discrete ``statements'' involving words and entities seem important for representing reality.

\entity{the insistence} on Tuesday [Trump] insisted [the order] was serious.

\slide{Natural Language in Code and Mathemamtics}

Mnemonic names and natural language comments are essential in writing intelligible code or mathematical proofs.

\vfill
This supports the idea that natural language somehow underlies our understanding of what would seem to be purely logical constructs.


\vfill
Rather than logic providing the meaning of language, neural language processing may support our understanding of logic.

\vfill
Or maybe logic and neural language processing just support each other.
\slide{END}

}
\end{document}
