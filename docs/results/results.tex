\input SlidePreamble
\input preamble

\begin{document}

{\Huge

  \centerline{\bf TTIC 31230, Fundamentals of Deep Learning}
  \bigskip
  \centerline{David McAllester, Autumn 2020}

  \vfill
  \centerline{\bf AlphaZero Results}
  \vfill
  \vfill

\slide{Training Time}

Single 20 block dual-headed ResNet.

\vfill
4.9 million Go games of self-play

\vfill
0.4s thinking time per move

\vfill
About 8 years of Go thinking time in training was completed in three days

\vfill
About 1000 fold parallelism.

\slide{Elo Learning Curve for Go}

\centerline{\includegraphics[height = 5in]{\images/alphalearning1}}

\slide{Learning Curve for Predicting Human Go Moves}

\centerline{\includegraphics[height = 5in]{\images/alphalearning2}}

\slide{Ablation Studies}

\centerline{\includegraphics[height=3.0in]{\images/alphagoArchitecture2}}

\vfill
We evaluate 20 layer networks which are either traditional CNNs or Renest and are either two separate networks or one network with dual heads.

\slide{Ablation Studies}

\centerline{\includegraphics[height = 5in]{\images/alphaablation}}

\slide{Increasing Blocks and Training}

Increasing the number of Resnet blocks from 20 to 40.

\vfill
Increasing the number of training days from 3 to 40.

\vfill
Gives a Go Elo rating over 5000.

\slide{Final Go Elo Ratings}

\centerline{\includegraphics[height = 5in]{\images/alpha40}}

\slide{Is Chess a Draw?}

In 2007 Jonathan Schaeffer at the University of Alberta showed that checkers is a draw.

\vfill
Using alpha-beta and end-game dynamic programming, Schaeffer computed drawing strategies for each player.

\vfill
This was listed by Science Magazine as one of the top 10 breakthroughs of 2007.

\vfill
It is generally believed that chess is a draw.  It was even conjectured that Stockfish could not be defeated ...

\slide{AlphaZero vs. Stockfish in Chess}

From white Alpha won 25/50 and lost none.

\vfill
From black Alpha won 3/50 and lost none.

\vfill
AlphaZero evaluates 70 thousand positions per second.

\vfill
Stockfish evaluates 80 million positions per second.

\slide{END}



}
\end{document}

\ignore{
\slideplain{Conspiracy Numbers}

Conspiracy Numbers for Min-Max search, McAllester, 1988

\vfill
Each node $s$ has a min-max value $V(s)$ determined by the leaf values.

\vfill
For any positive integer $N$ and potential value $V$ we define $L(N,V)$ to be the set of leaf nodes $s_1$ such that
there exist $N-1$ other leaf nodes $s_2$, $\ldots$, $s_N$ such that by changing the values of $s_1$, $\ldots$, $s_N$ the root node
can be changed to $V$.


\vfill
{\bf Algorithm:}


\vfill
Repeatedly select some $N$ and $V$ such that $L(N,V)$ is non empty and expand some leaf in $L(N,V)$.
}

\ignore{
\slide{Simulation}

To find an upper-confidence leaf for the root and value $U$:

\vfill
At a max node pick the child minimizing $N(s,U)$.

\vfill
At a min node select any child $s$ with $V(s) < U$.

\vfill
\slide{Refinement}

Let the static evaluator associate leaf nodes with values $U(s,N)$ and $L(s,N)$
}

