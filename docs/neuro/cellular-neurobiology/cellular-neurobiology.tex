%
% This is a borrowed LaTeX template file for lecture notes for CS267,
% Applications of Parallel Computing, UCBerkeley EECS Department.
% Now being used for CMU's 10725 Fall 2012 Optimization course
% taught by Geoff Gordon and Ryan Tibshirani.  When preparing 
% LaTeX notes for this class, please use this template.
%
% To familiarize yourself with this template, the body contains
% some examples of its use.  Look them over.  Then you can
% run LaTeX on this file.  After you have LaTeXed this file then
% you can look over the result either by printing it out with
% dvips or using xdvi. "pdflatex template.tex" should also work.
%

\documentclass[twoside]{article}
\setlength{\oddsidemargin}{0.25 in}
\setlength{\evensidemargin}{-0.25 in}
\setlength{\topmargin}{-0.6 in}
\setlength{\textwidth}{6.5 in}
\setlength{\textheight}{8.5 in}
\setlength{\headsep}{0.75 in}
\setlength{\parindent}{0 in}
\setlength{\parskip}{0.1 in}

%
% ADD PACKAGES here:
%

\usepackage{amsmath,amsfonts,graphicx}

%
% The following commands set up the lecnum (lecture number)
% counter and make various numbering schemes work relative
% to the lecture number.
%
\newcounter{lecnum}
\renewcommand{\thepage}{\thelecnum-\arabic{page}}
\renewcommand{\thesection}{\thelecnum.\arabic{section}}
\renewcommand{\theequation}{\thelecnum.\arabic{equation}}
\renewcommand{\thefigure}{\thelecnum.\arabic{figure}}
\renewcommand{\thetable}{\thelecnum.\arabic{table}}

%
% The following macro is used to generate the header.
%
\newcommand{\lecture}[4]{
   \pagestyle{myheadings}
   \thispagestyle{plain}
   \newpage
   \setcounter{lecnum}{#1}
   \setcounter{page}{1}
   \noindent
   \begin{center}
   \framebox{
      \vbox{\vspace{2mm}
    \hbox to 6.28in { {\bf NURB-31800: Cellular Neurobiology
	\hfill Winter 2021} }
       \vspace{4mm}
       \hbox to 6.28in { {\Large \hfill Lecture #1: #2  \hfill} }
       \vspace{2mm}
       \hbox to 6.28in { {\it Lecturer: #3 \hfill Scribes: #4} }
      \vspace{2mm}}
   }
   \end{center}
   \markboth{Lecture #1: #2}{Lecture #1: #2}
   \vspace*{4mm}
}
%
% Convention for citations is authors' initials followed by the year.
% For example, to cite a paper by Leighton and Maggs you would type
% \cite{LM89}, and to cite a paper by Strassen you would type \cite{S69}.
% (To avoid bibliography problems, for now we redefine the \cite command.)
% Also commands that create a suitable format for the reference list.
\renewcommand{\cite}[1]{[#1]}
\def\beginrefs{\begin{list}%
        {[\arabic{equation}]}{\usecounter{equation}
         \setlength{\leftmargin}{2.0truecm}\setlength{\labelsep}{0.4truecm}%
         \setlength{\labelwidth}{1.6truecm}}}
\def\endrefs{\end{list}}
\def\bibentry#1{\item[\hbox{[#1]}]}

%Use this command for a figure; it puts a figure in wherever you want it.
%usage: \fig{NUMBER}{SPACE-IN-INCHES}{CAPTION}
\newcommand{\fig}[3]{
			\vspace{#2}
			\begin{center}
			Figure \thelecnum.#1:~#3
			\end{center}
	}
% Use these for theorems, lemmas, proofs, etc.
\newtheorem{theorem}{Theorem}[lecnum]
\newtheorem{lemma}[theorem]{Lemma}
\newtheorem{proposition}[theorem]{Proposition}
\newtheorem{claim}[theorem]{Claim}
\newtheorem{corollary}[theorem]{Corollary}
\newtheorem{definition}[theorem]{Definition}
\newenvironment{proof}{{\bf Proof:}}{\hfill\rule{2mm}{2mm}}

\newcommand\E{\mathbb{E}}

\begin{document}
\lecture{1}{January 11}{Ruth Eatock}{Clayton Seitz}

\section{Electrical properties of neurons}

This section covers the origins of electrical properties of
single neurons. The physical properties of membranes ultimately 
govern the way that electrical signals can be generated and transmitted
in the nervous system. 

Neuroscience research can be conducted at molecular/genetic scales
as well as cellular/system scales. Hyperpolarization refers to a more
negative polarization while depolarization refers to a more postive
polarization. A threshold of -50\si{mV} is common for neurons. Larger
currents will result in faster firing (spike frequency). 

\subsection{Passive membrane properties}

The electrical properties of the membrane can be represented by a 
resistance (conductance), a capacitance, and a DC battery that maintains
the resting potential. Ions cannot move across the bilayer due to its
hydrophobicity and the ion channels reduce the membrane resistance 
(increase conductance). 

Outside the cell the potential is zero (ground). Membrane current has 
capacitive and resistive components

\begin{equation}
I_{m} = IC + I_{R}
\end{equation}

\subsection{Current Conventions}

Positive current is a flow of positive charges out of the cell. For example, 
a current of potassium into the cell is a negative current. Keep in mind
the electrical equivalence of in efflux of positive and influx of negative
and vice-versa. The capacitive current is given by 

\begin{equation}
I_{C} = C\frac{dV}{dt}
\end{equation}

which has a solution

\begin{equation}
V_{m}(t) = V_{\infty}(1-e^{\frac{-t}{\tau}})
\end{equation}

where $V_{\infty}$ is the steady state voltage. At long times
we can apply to Ohm's Law. The membrane is a low-pass filter. 
Vision runs on gain rather than speed while neurons involved 
in hearing are faster. The membrane time constant nicely 
summarizes the response of a neuron to an injected current.

The cell membrane is the dielectric of a parallel plate 
capacitor while the "plates" are the ions in the solution. 

A capacitance of 10pF is a common value for the total 
capacitance. Physiological range of voltage is +/- 100mV
beyond which you get membrane break-down. Average resting
is between -40mV and -80mV. 

Capactive current flows only when voltage is changing. Parallel 
capacitors (one per unit area) just sum to give the total 
capacitance. Recall the basic relation

\begin{equation}
C = \epsilon\frac{A}{d}
\end{equation}

There is a bath of chloride, potassium, sodium, and calcium. 
each of which has an associated channel. 
There are other anions that are too large to pass through 
ion channels: proteins, saccharides, ATP/GTP, etc. Charges
move across the membrane via pumps and channels. A transporter
(pump) moves a few ions at a time so we generally neglect them.  
The sodium-potassium pump is a common one. ATPases burn ATP
to pump ions across the membrane. 






\end{document}







