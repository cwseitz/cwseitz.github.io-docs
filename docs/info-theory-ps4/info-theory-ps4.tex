\documentclass[12pt]{article}
\usepackage{amsmath} % AMS Math Package
\usepackage{amsthm} % Theorem Formatting
\usepackage{amssymb}    % Math symbols such as \mathbb
\usepackage{graphicx} % Allows for eps images
\usepackage[dvips,letterpaper,margin=1in,bottom=0.7in]{geometry}
\usepackage{amsmath}


\newtheorem{p}{Problem}[section]
\usepackage{cancel}
\newtheorem*{lem}{Lemma}
\theoremstyle{definition}
\newtheorem*{dfn}{Definition}
 \newenvironment{s}{%\small%
        \begin{trivlist} \item \textbf{Solution}. }{%
            \hspace*{\fill} $\blacksquare$\end{trivlist}}%


\begin{document}

{\noindent\Huge\bf  \\[0.5\baselineskip] {\fontfamily{cmr}\selectfont  Problem Set 4}         }\\[2\baselineskip] % Title
{ {\bf \fontfamily{cmr}\selectfont Information and Coding Theory}\\ {\textit{\fontfamily{cmr}\selectfont     March 14, 2021}}}~~~~~~~~~~~~~~~~~~~~~~~~~~~~~~~~~~~~~~~~~~~~~~~~~~~~~~~~~~~~~~~~~~~~~~~~~~~~~    {\large \textsc{Clayton Seitz}
\\[1.4\baselineskip] 

\begin{p}
This is the first problem
\end{p}

\begin{s}
\begin{align*}
\Delta(C) &= \min_{x_{1},x_{2}\in C} \Delta(x_{1},x_{2})\\
&= \min_{x_{1},x_{2}\in C} \Delta(0,x_{2}-x_{1})\\
&= \min_{x\in C} \mathbf{wt}(x)
\end{align*}

Since the code is linear, $x_{2}-x_{1} \in C$. Now, we consider the parity check matrix $H \in \mathbb{F}_{2}^{r\times n}$ where $n = 2^{r} - 1$. We will find the dimension, block length, and distance for such a code. First, the dimension of the code $\dim(C)$ is $r+1$ since the rank of $H$ is $r$. The block length is then $2^{r+1}$ and the distance is 3. Now, consider the Hamming code $C: \mathbb{F}_{2}^{k} \rightarrow \mathbb{F}_{2}^{n}$ which is formally defined as the set of $x$ in the null space in of the parity check matrix:

\begin{equation*}
C = \left\{x\in \mathbb{F}_{2}^{n} | Hx = 0\right\}
\end{equation*}

where $H \in \mathbb{F}_{2}^{k\times n}$ is the parity check matrix. We can also define the dual code $C^{\perp}$ to be the code with generator matrix $H^{T}$ and  parity check matrix $G^{T}$. 

To see why this is possible, we will use the fact that we have defined our code $C$ to be the vectors $x$ that lie in the null space of the parity matrix $H$. Now, the definition of our code requires that $H(x)=H(G(w))=0$ which means that the generator matrix $G$ is a matrix with columns equal to the basis vectors of the null space of $H$ i.e. $HG = 0$. This is equivalent to saying that the columns of $H^{T}$ form the basis of the null space of $G^{T}$:

\begin{equation*}
HG = 0 \iff G^{T}H^{T}=0
\end{equation*}

Therefore $H^{T}$ can be viewed as the generator matrix and $G^{T}$ the parity check matrix for the dual code $C^{\perp}$.

\end{s}



\end{document}